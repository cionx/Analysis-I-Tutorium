\documentclass[a4paper,10pt]{article}
%\documentclass[a4paper,10pt]{scrartcl}

\usepackage{../mystyle}

\title{Kompakte Teilmengen von $\R^n$}
\author{Jendrik Stelzner}
\date{\today}

\begin{document}
\maketitle

\tableofcontents





\section{Stetige Funktionen auf abgeschlossenen Intervallen}


Zur Motivation des Kompaktheitbegriffes wollen wir zunächst die folgende wichtige Aussage beweisen:


\begin{lem}\label{lem: stetig auf Intervall}
 Es seien $a,b \in \R$ mit $a < b$ und $f \colon [a,b] \to \R$ stetig. Dann ist die Abbildung $f$ beschränkt und nimmt auf $[a,b]$ ihr Maximum und Minimum an, d.h. es gibt $x_{\text{max}}, x_{\text{min}} \in [a,b]$ mit
 \[
  f(x_{\text{max}}) = \sup_{y \in [a,b]} f(y)
  \quad
  \text{und}
  \quad
  f(x_{\text{min}}) = \inf_{y \in [a,b]} f(y).
 \]
\end{lem}


Für offene Intervalle oder halboffene Intervalle gilt diese Aussage nicht. Man betrachte etwa die Abbildung $(0,1] \to \R, x \mapsto 1/x$, oder gar $(0,1] \to \R, x \mapsto \sin(1/x)/x$.


Herzstück des Beweises ist die Beobachtung, dass auf $[a,b]$ jede Folge eine konvergente Teilfolge besitzt.


\begin{lem}\label{lem: Intervall ist kompakt}
 Es seien $a,b \in \R$ mit $a < b$ und $(x_n)_{n \in \N}$ eine Folge auf $[a,b]$. Dann besitzt $(x_n)$ eine konvergente Teilfolge $(x_{n_j})_{j \in \N}$, und für den Grenzwert $x \coloneqq \lim_{j \to \infty} x_{n_j}$ gilt $x \in [a,b]$.
\end{lem}
\begin{proof}
 Da $a \leq x_n \leq b$ für alle $n \in \N$ ist die Folge $(x_n)$ beschränkt und besitzt daher nach Bolzano-Weierstraß eine konvergente Teilfolge $(x_{n_j})_{j \in \N}$. Da das abgeschlossene Intervall $[a,b]$ abgeschlossen ist, ist auch $x \coloneqq \lim_{j \to \infty} x_{n_j}$.
\end{proof}


\begin{proof}[Beweis von Lemma \ref{lem: stetig auf Intervall}]
 Wir zeigen zunächst, dass $f$ beschränkt ist: Angenommen, $f$ wäre nach oben unbeschränkt. Dann gibt es für alle $n \in \N$ ein $x_n \in [a,b]$ mit $f(x_n) \geq n$. Nach Lemma \ref{lem: Intervall ist kompakt} besitzt die Folge $(x_n)$ eine konvergente Folge $(x_{n_j})_{j \in \N}$. Da $f$ stetig ist, konvergiert auch die Folge $(f(x_{n_j}))_{j \in \N}$. Für alle $j \in \N$ ist aber $f(x_{n_j}) \geq n_j$, die Folge $f(x_{n_j})$ konvergiert also nicht. Dieser Widerspruch zeigt dass $f$ nach oben unbeschränkt seien muss. Analog ergibt sich, dass $f$ auch nach unten beschränkt ist. Also ist $f$ beschränkt.
 
 Es sei
 \[
  M \coloneqq \sup_{y \in [a,b]} f(y).
 \]
 Da $f$ nach oben beschränkt ist, ist $M < \infty$. Nach der $\varepsilon$-Charakterisierung des Supremums gibt es für alle $n \geq 1$ ein $x_n \in [a,b]$ mit
 \[
  M \geq f(x_n) \geq M - \frac{1}{n}.
 \]
 Nach Lemma \ref{lem: Intervall ist kompakt} besitzt die Folge $(x_n)$ eine konvergente Teilfolge $(x_{n_j})_{j \in \N}$. Es sei $x \coloneq \lim_{j \to \infty} x_{n_j}$. Da $f$ stetig ist, konvergiert auch die Folge $(f(x_{n_j}))$ und es gilt
 \[
  \lim_{j \to \infty} f(x_{n_j}) = f(x).
 \]
 Andererseits gilt für alle $j \in \N$
 \[
  M \geq f(x_{n_j}) \geq M - \frac{1}{n_j}.
 \]
 Also muss nach dem Sandwich-Lemma auch
 \[
  \lim_{j \to \infty} f(x_{n_j}) = M.
 \]
 Also ist $f(x) = M$. Das zeigt, dass $f$ auf $[a,b]$ sein Maximum annimmt. Analog ergibt sich, dass $f$ auf $[a,b]$ auch sein Minimum annimmt.
\end{proof}





\section{Folgenkompaktheit}


Zum Beweis von Lemma \ref{lem: stetig auf Intervall} haben wir Folgenstetigkeit und Lemma \ref{lem: Intervall ist kompakt} benötigt. Diese Beobachtung legt nahe, dass sich Lemma \ref{lem: stetig auf Intervall} auf beliebige Teilmengen $K \subseteq \R^n$ verallgemeinern lässt, für die eine zu Lemma \ref{lem: Intervall ist kompakt} analoge Aussage gilt.


\begin{defi}
 Es sei $X \subseteq \R^n$ und $(x_n)$ ein Folge auf $X$. Wir sagen \emph{$(x_n)$ konvergiert auf $X$}, falls die Folge $(x_n)$ konvergiert und $\lim_{n \to \infty} x_n \in X$.
\end{defi}


\begin{bsp}
 Die Folge $(1/n)_{n \geq 1}$ konvergiert auf $[0,1]$, nicht aber auch $(0,1)$.
\end{bsp}


\begin{defi}
 Eine Teilmenge $K \subseteq \R^n$ heißt \emph{kompakt}, falls jede Folge $(x_n)$ auf $K$ eine auf $K$ konvergente Teilfolge besitzt.
\end{defi}


\begin{bsp}
 Lemma \ref{lem: Intervall ist kompakt} zeigt, dass ein abgeschlossenes Intervall $[a,b]$ mit $a < b$ kompakt ist. Offene Intervalle hingegen sind niemals kompakt: Sind $a,b \in \R$ mit $a < b$, so ist $(a + (b-a)/n)_{n \geq 2}$ eine Folge auf $(a,b)$, die keine auf $(a,b)$ konvergente Teilfolge besitzt.
\end{bsp}


\begin{prop}
 Es sei $K \subseteq \R^n$ kompakt und $f \colon K \to \R$ stetig. Dann ist die Abbildung $f$ auf $K$ beschränkt und nimmt auf $K$ ihr Maximum und ihr Minimum an, d.h. es gibt $x_{\text{min}}, x_{\text{max}} \in K$ mit
 \[
  f(x_{\text{max}}) = \sup_{y \in K} f(y)
  \quad
  \text{und}
  \quad
  f(x_{\text{min}}) = \inf_{y \in K} f(y).
 \]
\end{prop}
\begin{proof}
 Nehme den Beweis von Lemma \ref{lem: stetig auf Intervall} und ersetze $[a,b]$ durch $K$ und die Verweise auf Lemma \ref{lem: Intervall ist kompakt} durch Folgenompaktheit.
\end{proof}


Da sich stetige Funktionen auf kompakten Mengen gutartig verhalten, sind kompakte Mengen von großer Bedeutung für die Analysis.





\section{Überdeckungskompaktheit}


Um zu verstehen, welche Teilmengen von $\R^n$ folgenkompakt sind, wollen wir einen weiteren Kompaktheitsbegriff einführen: Die Überdeckungskompaktheit. Wie sich herausstellt, sind Überdeckungskompaktheit und Folgenkompaktheit für Teilmengen von $\R^n$ äquivalent.


\begin{defi}
 Es sei $\{U_i\}_{i \in I}$ eine Kollektion von Teilmengen $U_i \subseteq \R^n$ und $X \subseteq \R^n$.
 
 Die Kollektion $\{U_i\}_{i \in I}$ heißt \emph{Überdeckung von $X$}, falls $X \subseteq \bigcup_{i \in I} U_i$. Sie heißt \emph{endlich}, bzw.\ \emph{abzählbar}, falls $I$ endlich, bzw.\ abzählbar ist.
 
 Eine \emph{Teilüberdeckung} ist dann eine Teilkollektion $\{U_j\}_{j \in J}$, also $J \subseteq I$, so dass bereits $\{U_j\}_{j \in J}$ eine Überdeckung von $X$ ist.
 
 Außerdem heißt die Überdeckung $\{U_i\}_{i \in I}$ von $X$ \emph{offen}, falls die $U_i \subseteq \R^n$ alle offen sind.
\end{defi}


\begin{bem}
 Offenbar ist eine Teilüberdeckung einer offenen Überdeckung ebenfalls offen.
\end{bem}


\begin{bsp}
 \begin{itemize}
  \item
   Die Kollektion $\{B_n(0) \mid n \in \N, n \geq 1\}$ bildet eine abzählbare, offene Überdeckung von $\R^n$.
  \item
   Die offenen Intervalle $\{(a,b) \mid a,b \in \R, a < b\}$ bilden eine offene Überdeckung von $\R$.
  \item
   Allgemein bilden die $\varepsilon$-Bälle $\{B_\varepsilon(x) \mid x \in \R^n, \varepsilon > 0\}$ eine offene Überdeckung von $\R^n$.
  \item
   Die Würfel $\{[-K,K]^n \mid K \in \N, K \geq 1\}$ bilden eine abzählbar Überdeckung von $\R^n$, die nicht offen ist.
  \item
   Die $\varepsilon$-Bälle mit rationalen Radius um Mittelpunkte mit rationalen Koeffizienten $\{B_q(x) \mid x \in \Q^n, q \in \Q, q > 0\}$ bilden eine offene Überdeckung von $\R^n$.
  \item
   Die Intervalle $\{(n,n+1) \mid n \in \Z\}$ bilden eine offene Überdeckung von $\R \setminus \Z = \bigcup_{n \in \N} (n,n+1)$, aber nicht von $\R$.
  \item
   Die offenen Intervalle $\{ (-1/n, 1+1/n) \mid n \in \N, n \geq 1\}$ bilden eine offene Überdeckung des abgeschlossenen Einheitsintervalls $[0,1]$; diese Überdeckung besitzt eine endliche (sogar einelementige) Teilüberdeckung.
 \end{itemize}
\end{bsp}


\begin{defi}
 Eine Teilmenge $K \subseteq \R^n$ heißt \emph{überdeckungskompakt}, falls jede offene Überdeckung von $K$ eine endliche Teilüberdeckung besitzt.
\end{defi}


\begin{lem}
 Es sei $K \subseteq \R^n$ überdeckungskompakt. Dann gilt:
 \begin{enumerate}
  \item
   $K$ ist abgeschlossen.
  \item
   $K$ ist beschränkt.
 \end{enumerate}
\end{lem}
\begin{proof}
 \begin{enumerate}
  \item
   Wir zeigen, dass $K$ abgeschlossen ist, indem wir zeigen, dass $K^c$ offen ist. Es sei hierfür $x \in K^c$ beliebig aber fest. Wir wollen zeigen, dass es ein $\varepsilon > 0$ gibt, so dass $B_\varepsilon(x) \subseteq K$. Für jedes $y \in K$ ist $x \neq y$ (da $x \notin K$), also $\varepsilon_y \coloneqq \|x-y\|/2 > 0$. Für jedes $y \in K$ sind dann $B_{\varepsilon_y}(x)$ und $B_{\varepsilon_y}(y)$ disjunkt.
   
   (Wir würden nun gerne die Bälle $B_{\varepsilon_y}(x)$ schneiden, um eine, hoffentlich offene, Menge zu erhalten, die $x$ enthält, aber disjunkt zu $K$ ist. Das Problem ist, dass der möglicherweise unendliche Schnitt $\bigcap_{y \in K} B_{\varepsilon_y}(x)$ nicht mehr notwendigerweise offen ist. Mithilfe der Überdeckungskompaktheit können wir diesen unendlichen Schnitt aber durch einen endlichen ersetzen:)
   
   Die $\varepsilon$-Bälle $\{B_{\varepsilon_y}(y) \mid y \in K\}$ bilden eine offene Überdeckung von $K$. Da $K$ überdeckungskompakt ist, besitzt diese Überdeckung eine endliche Teilüberdeckung; es gibt also $y_1, \dotsc, y_s \in K$ mit
   \begin{equation}\label{eqn: K endliche Überdeckung}
    K \subseteq B_{\varepsilon_{y_1}}(y_1) \cup \dotsb \cup B_{\varepsilon_{y_s}}(y_s).
   \end{equation}
   Wir setzen $\varepsilon \coloneqq \min_{i=1,\dotsc,s} \varepsilon_{y_s} > 0$. Da $B_{\varepsilon_y}(x)$ und $B_{\varepsilon_y}(y)$ für alle $y \in K$ disjunkt sind, folgt aus \eqref{eqn: K endliche Überdeckung}, dass auch $B_\varepsilon(x)$ und $K$ disjunkt sind; es ist nämlich
   \begin{align*}
    B_\varepsilon(x) \cap K
    &\subseteq B_\varepsilon(x) \cap \bigcup_{i=1}^s B_{\varepsilon_{y_i}}(y_i)
    = \bigcup_{i=1}^s (B_\varepsilon(x) \cap B_{\varepsilon_{y_i}}(y_i)) \\
    &\subseteq \bigcup_{i=1}^s \underbrace{(B_{\varepsilon_{y_i}}(x) \cap B_{\varepsilon_{y_i}(y_i)})}_{= \emptyset}
    = \emptyset.
   \end{align*}
   Es ist also $B_\varepsilon(x) \subseteq K^c$. Aus der Beliebigkeit von $x \in K^c$ folgt, dass $K^c$ offen ist, und somit $K$ abgeschlossen.
  \item
   Die offenen Bälle $\{B_r(0) \mid r > 0\}$ bilden eine offene Überdeckung von $\R^n$, und damit insbesondere auch von $K$. Da $K$ überdeckungskompakt ist, besitzt diese Überdeckung eine endliche Teilüberdeckung. Es gibt also Radien $r_1, \dotsc, r_M > 0$ mit $K \subseteq B_{r_1}(0) \cup \dotsb \cup B_{r_M}(0)$. Für den Radius $R = \max_{i=1,\dotsc,M} r_i$ so ist $B_{r_i}(0) \subseteq B_R(0)$ für alle $1 \leq i \leq s$, und somit auch $K \subseteq B_R(0)$. Also ist $K$ beschränkt.
  \qedhere
 \end{enumerate}
\end{proof}





\section{Äquivalenz der Kompaktheitsbegriffe}


\begin{prop}
 Eine Teilmenge $K \subseteq \R^n$ ist genau dann folgenkompakt, wenn sie überdeckungskompakt ist.
\end{prop}
\begin{proof}
 (Überdeckungskompaktheit $\Rightarrow$ Folgenkompaktheit) Angenommen $K$ ist überdeckungskompakt, aber nicht folgenkompakt. Dann gibt es eine Folge $(x_n)$ auf $K$, die keine auf $K$ konvergente Teilfolge besitzt.
 
 \begin{beh}
  Für jedes $x \in K$ gibt es ein $\varepsilon_x > 0$, so dass $\|x - x_n\| \geq \varepsilon$ für fast alle $n \in \N$.
 \end{beh}
 \begin{proof}[Beweis der Behauptung]
  Ansonsten gibt es ein $x \in K$, so dass es für jedes $\varepsilon > 0$ unendlich viele $n \in N$ mit $\|x - x_n\| < \varepsilon$ gibt. Insbesondere gibt es dann ein $n_1 \in \N$ mit $\|x - x_{n_1}\| < 1$. Dann gibt es auch ein $n_2 \in \N$ mit $n_2 > n_1$ und $\|x - x_{n_2}\| < 1/2$. Rekursiv ergibt sich, dass es für alle $j \geq 1$ ein $n_j \in \N$ gibt, so dass $n_j > n_{j-1}$ und $\|x - x_{n_j}\| < 1/j$. Es ist dann $n_j$ eine Teilfolge mit $x_{n_j} \to x$. Dies steht im Widerspruch zur Annahme, dass $(x_n)$ keine auf $K$ konvergente Teilfolge besitzt.
 \end{proof}
 
 Die $\varepsilon$-Bälle $\{B_{\varepsilon_x}(x) \mid x \in K\}$ bilden offenbar eine offene Überdeckung von $K$. Da $K$ überdeckungskompakt ist, besitzt diese offene Überdeckung eine endliche Teilüberdeckung. Es gibt also $x_1, \dotsc, x_s \in K$, so dass
 \begin{equation}\label{eqn: K endlich überdeckt}
  K \subseteq B_{\varepsilon_{x_1}}(x_1) \cup \dotsb \cup B_{\varepsilon_{x_s}}(x_s).
 \end{equation}
 Da jeder der $\varepsilon$-Bälle $B_{\varepsilon_x}(x)$ nur endlich viele Folgeglieder enthält, enthält auch $K$ wegen \eqref{eqn: K endlich überdeckt} nur endlich viele Folgeglieder, was offenbar nicht seien kann. Das zeigt, dass aus Überdeckungskompaktheit Folgenkompaktheit folgt.
 
 (Überdeckungskompaktheit $\Rightarrow$ Folgenkompaktheit) Entscheident für diese Implikation ist die folgende Eigenschaften überdeckungskompakter Mengen:
 
 \begin{beh}
  Es sei $(x_n)$ eine Folge auf $K$. Dann gibt es für jedes $\varepsilon > 0$ eine Teilfolge $n_j$, so dass $\|x_{n_j} - x_{n_{j'}}\| < \varepsilon$ für alle $j, j' \in \N$.
 \end{beh}
 \begin{proof}[Beweis der Behauptung]
  Die $(\varepsilon/2)$-Bälle $\{B_{\varepsilon/2}(y) \mid y \in K\}$ bilden offenbar eine offene Überdeckung von $K$. Wegen der Überdeckungskompaktheit von $K$ besitzt diese offene Überdeckung eine endliche Teilüberdeckung. Es gibt daher $y_1, \dotsc, y_s \in K$ mit
  \[
   K \subseteq B_{\varepsilon/2}(y_1) \cup \dotsb \cup B_{\varepsilon/2}(y_s).
  \]
  Es muss dann einer dieser $(\varepsilon/2)$-Bälle unendlich viele Folgeglieder enthalten; wir können o.B.d.A.\ davon ausgehen, dass $x_n \in B_{\varepsilon/2}(y_1)$ für unendlich viele $n \in \N$. Es gibt daher eine Teilfolge $n_j$ mit $x_{n_j} \in B_{\varepsilon/2}(y_1)$ für alle $j \in \N$. Inbesondere ist daher für alle $j, j' \in \N$
  \[
   \|x_{n_j} - x_{n_{j'}}\|
   \leq \|x_{n_j} - y_1\| + \|x_{n_{j'}} - y_1\|
   < \frac{\varepsilon}{2} + \frac{\varepsilon}{2}
   = \varepsilon.
   \qedhere
  \]
 \end{proof}
 
 Nach der Behauptung gibt es eine Teilfolge $n_{1,j}$, so dass $\|x_{n_{1,j}}-x_{n_{1,j'}}\| < 1$ für alle $j, j' \in \N$. Nach der Behauptung besitzt die Folge $(x_{n_{1,j}})_{j \in \N}$ eine Teilfolge $n_{2,j}$, so dass $\|x_{n_{2,j}} - x_{n_{2,j'}}\| < 1/2$ für alle $j, j' \in \N$. Rekursives Weiterführen ergibt für jedes $k \geq 2$ eine Teilfolge $n_{k,j}$ von $(x_{n_{k-1,j}})_{j \in \N}$, so dass
 \[
  \|x_{n_{k,j}} - x_{n_{k,j'}}\| < 1/k \quad \text{für alle $j, j' \in \N$}.
 \]
 
 Wir wollen diese Folgen nun zu der gewünschten Cauchy-Folge zusammenfassen; hierfür benutzen wir einen Standardtrick, den der \emph{Diagonalfolge}: Wir definieren die gewünschte Teilfolge $n_j$ durch $n_j \coloneqq n_{j,j}$ für alle $j \geq 1$, d.h.\ wir nehmen das erste Glied aus der Teilfolge $(x_{n_{1,j}})$, das zweite Glied aus der Folge $(x_{n_{2,j}})$, das dritte Glied aus der Folge $(x_{n_{3,j}})$, usw. Für die so erhaltene Teilfolge $n_j$ ist
 \[
  \| x_{n_j} - x_{n_{j'}} \| < 1/J \quad \text{für alle $J \in \N$, $J \geq 1$ und $j, j' \geq J$}.
 \]
 Also ist $(x_{n_j})_{j \in \N}$ eine Cauchy-Folge, und somit konvergent. Da $K$ überdeckungskompakt ist, ist $K$ auch abgeschlossen, und deshalb $\lim_{j \to \infty} x_{n_j} \in K$.
 
 Das zeigt, dass jede Folge auf $K$ eine auf $K$ konvergente Teilfolge besitzt, also dass $K$ folgenkompakt ist.
\end{proof}


Wegen der Äquivalenz von Folgen- und Überdeckungskompaktheit in $\R^n$ spricht man auch nur von Kompaktheit; eine Teilmenge $K \subseteq \R^n$ heißt also \emph{kompakt}, falls sie folgenkompakt, bzw.\ überdeckungskompakt ist.





\section{Der Satz von Heine-Borel}
Der Satz von Heine-Borel beantwortet die Frage, wie kompakte Teilmengen von $\R^n$ aussehen.


\begin{thrm}[Satz von Heine-Borel]
 Eine Teilmenge $K \subseteq \R^n$ ist genau dann kompakt, wenn $K$ abgeschlossen und beschränkt ist.
\end{thrm}
\begin{proof}
 Wir haben bereits gesehen, dass jede (überdeckungs)kompakte Teilmenge sowohl abgeschlossen als auch beschränkt ist.
 
 Es sei andererseits $K$ abgeschlossen und beschränkt. Wir wollen zeigen, dass $K$ (folgen)kompakt ist. Es sei hierfür $(x_n)_{n \in \N}$ eine Folge auf $K$. Da $K$ beschränkt ist, besitzt $K$ nach dem Satz von Bolzano-Weierstraß eine konvergente Teilfolge $(x_{n_j})_{j \in \N}$. Da $(x_{n_j})$ eine Folge auf $K$ ist, und $K$ abgeschlossen ist, ist auch $\lim_{j \to \infty} x_{n_j} \in K$. Also ist $K$ (folgen)kompakt.
\end{proof}


Der Satz von Heine-Borel ist sehr nützlich, um Teilmengen von $\R^n$ auf Kompaktheit zu untersuchen.


\begin{bsp}
 \begin{itemize}
  \item
   Die $n$-dimensionale Sphäre $S^n = \{x \in \R^{n+1} \mid \|x\| = 1\}$ ist offenbar beschränkt, und wir wissen bereits, dass sie auch abgeschlossen ist. Also ist $S^n$ kompakt.
  \item
   Wir wissen bereits, dass der Einheitswürfel $[0,1]^n \subseteq \R^n$ abgeschlossen ist. Er ist auch beschränkt, denn für alle $x = (x_1, \dotsc, x_n) \in [0,1]^n$ ist $\|x\| \leq \sum_{i=1}^n |x_i| \leq n$. Also ist der Einheitswürfel kompakt.
  \item
   Das offen Intervall $(0,1)$ ist zwar beschränkt, aber nicht abgeschlossen, und somit auch nicht kompakt.
  \item
   Der unendliche Zylinder $S^1 \times \R \subseteq \R^3$ ist zwar abgeschlossen, aber nicht beschränkt, und somit ebenfalls nicht kompakt.
  \item
   Ist $f \colon \R \to \R$ eine stetige Funktion, so ist der Graph
   \[
    G \coloneqq \{(x,f(x)) \mid x \in \R\}
   \]
   zwar abgeschlossen, aber nicht beschränkt, und somit auch nicht kompakt.
  \item
   Offene $\varepsilon$-Bälle sind zwar beschränkt, aber nicht abgeschlossen, und somit ebenfalls nicht kompakt.
  \end{itemize}
\end{bsp}





\section{Weitere Eigenschaften kompakter Mengen}
Wir wollen hier noch weitere Eigenschaften kompakter Mengen angeben und beweisen. Dabei geben wir jeweils mehrere Beweise an, um den Leser mit den verschiedenen Charakterisierungen kompakter Mengen vertraut zu machen.


\begin{lem}
 Endliche Vereinigungen kompakter Mengen sind kompakt, d.h.\ sind $K_1, \dotsc, K_r \subseteq \R^n$ kompakt, so ist auch $K_1 \cup \dotsb \cup K_r$ kompakt.
\end{lem}
\begin{proof}
 Wir setzen $K \coloneqq K_1 \cup \dotsb \cup K_r$. Wir wollen drei Beweise für diese Aussage geben.
 
 \emph{(Überdeckungskompaktheit)} Es sei $\{U_i\}_{i \in I}$ eine offen Überdeckung von $K$. Dass ist $\{U_i\}_{i \in I}$ auch eine offene Überdeckung von $K_j$ für alle $1 \leq j \leq r$. Da die $K_j$ (überdeckungs)kompakt sind, besitzt diese offene Überdeckung für jedes $1 \leq j \leq r$ eine endliche Teilüberdeckung von $K_j$, also $i_{j,1}, \dotsc, i_{j,s_j} \in I$ mit
 \[
  K_j \subseteq U_{i_{j,1}} \cup \dotsb \cup U_{i_{j,s_j}}.
 \]
 Zusammenfügen dieser endlichen Teilüberdeckungen ergibt
 \[
  K
  = \bigcup_{j=1}^r K_j
  \subseteq \bigcup_{j=1}^r \bigcup_{\ell=1}^{s_j} U_{i_{j,\ell}},
 \]
 also eine endliche Teilüberdeckung von $K$. Dass zeigt, dass jede offene Überdeckung von $K$ eine endliche Teilüberdeckung besitzt, also dass $K$ (über\-deck\-ungs)kom\-pakt ist.
 
 \emph{(Folgenkompaktheit)} Es sei $(x_n)$ eine Folge auf $K$. Eine der Mengen $K_i$ muss unendlich viele Folgeglieder enthalten (denn sonst würde $K$ nur endlich viele Folgeglieder enthalten), d.h. es gibt ein $1 \leq \ell \leq r$, so dass $x_n \in K_\ell$ für unendlich viele $n \in \N$. Wir können o.B.d.A.\ davon ausgehen, dass $\ell = 1$, also $x_n \in K_1$ für unendlich viele $n \in \N$. Es gibt also eine Teilfolge $(\tilde{n}_j)_{j \in \N}$ mit $x_{\tilde{n}_j} \in K_1$ für alle $j \in \N$. Da $K_1$ (folgen)kompakt ist, besitzt dies Folge $(x_{\tilde{n}_j})_{j \in \N}$ eine auf $K_1$ konvergente Teilfolge $n_j$. Da $\lim_{j \to \infty} x_{n_j} \in K_1 \subseteq K$ ist dies eine Teilfolge von $(x_n)$ die auf $K$ konvergiert. Dass zeigt, dass jede Folge auf $K$ eine auf $K$ konvergente Teilfolge besitzt, dass also $K$ (folgen)kompakt ist.
 
 \emph{(Heine-Borel)} Da die $K_i$ kompakt sind, sind die $K_i$ beschränkt. Es gibt es daher für jedes $1 \leq j \leq r$ ein $C_i > 0$ mit $\|x\| \leq C_i$ für alle $x \in K_i$. Für $C \coloneqq \max_{j=1,\dotsc,r} C_j$ ist deshalb $\|x\| \leq C$ für alle $x \in K$. Also ist $K$ beschränkt. Da die $K_i$ kompakt sind, sind sie auch abgeschlossen; da endliche Vereinigungen abgeschlossener Mengen wieder abgeschlossen sind, ist daher auch $K$ abgeschlossen. Da $K$ abgeschlossen und beschränkt ist, ist $K$ kompakt.
\end{proof}


\begin{bem}
 Die unendliche Vereinigung kompakter Mengen ist im Allgemeinen nicht kompakt. So sind etwa die Würfel $[-R, R]^n \subseteq \R^n$ für alle $R > 0$ kompakt, aber $\R^n = \bigcup_{k=1}^\infty [-k,k]^n$ ist nicht kompakt. (Das Problem hier ist, dass die unendliche Vereinigung beschränkter Mengen nicht mehr beschränkt seien muss.) Auch $(-1,1)^n = \bigcup_{n=2}^\infty [-1+1/n, 1-1/n]$ ist nicht kompakt. (Das Problem hier ist, dass die unendliche Vereinigung abgeschlossener Mengen nicht mehr abgeschlossen seien muss.)
\end{bem}


Wie wir bereits gesehen haben, sind nicht alle abgeschlossenen Mengen kompakt. Es gilt jedoch, dass abgeschlossener Teilmengen kompakter Mengen selber kompakt ist.


\begin{lem}\label{lem: abgeschlossen Teilmengen von kompakten Mengen}
 Ist $K \subseteq \R^m$ kompakt und $C \subseteq \R^m$ abgeschlossen mit $C \subseteq K$, so ist auch $C$ kompakt.
\end{lem}
\begin{proof}
 Wir wollen drei Beweise für das Lemma geben.
 
 \emph{(Folgenkompaktheit)} Es sei $(x_n)$ eine Folge auf $C$. Dann ist $(x_n)$ auch eine Folge auf $K$. Da $K$ (folgen)kompakt ist besitzt $(x_n)$ eine auf $K$ konvergente Teilfolge $(n_j)_{j \in \N}$. Da $C$ abgeschlossen ist, ist schon $\lim_{j \to \infty} x_{n_j} \in C$, also $(x_{n_j})_{j \in \N}$ schon auf $C$ konvergent. Das zeigt, dass jede Folge auf $C$ eine auf $C$ konvergente Teilfolge besitzt. Also ist $C$ (folgen)kompakt.
 
 \emph{(Überdeckungskompaktheit)} Es sei $\{U_i\}_{i \in I}$ eine offene Überdeckung von $C$. Da $C$ abgeschlossen ist, ist $C^c$ offen. Da $\{U_i\}_{i \in I}$ eine offene Überdeckung von $C$ ist, is $\{U_i\}_{i \in I} \cup \{C^c\}$ ist eine offen Überdeckung von $K$ (sogar eine offene Überdeckung von $\R^m$). Da $K$ (überdeckungs)kompakt ist besitzt diese offene Überdeckung eine endliche Teilüberdeckung; es gibt also $i_1, \dotsc, i_s \in I$, so dass
 \[
  K \subseteq U_{i_1} \cup \dotsb \cup U_{i_s} \cup C^c.
 \]
 Dann ist auch
 \[
  C \subseteq U_{i_1} \cup \dotsb \cup U_{i_s}.
 \]
 Das zeigt, dass jede offen Überdeckung von $C$ eine endliche Teilüberdeckung besitzt. Also ist $C$ (überdeckungs)kompakt.
 
 \emph{(Heine-Borel)} Da $K$ kompakt ist, ist $K$ beschränkt. Da $C \subseteq K$ ist daher auch $C$ beschränkt. Da $C$ abgeschlossen und beschränkt ist, ist $K$ kompakt.
\end{proof}


\begin{kor}
 Ist $\{K_i\}_{i \in i}$ eine nichtleere Kollektion kompakter Teilmengen $K_i \subseteq \R^n$, so ist auch $\bigcap_{i \in I} K_i$ kompakt.
\end{kor}
\begin{proof}
 Da die $K_i$ kompakt sind, sind sie auch abgeschlossen. Also ist auch $\bigcap_{i \in I} K_i$ abgeschlossen. Da außerdem $\bigcap_{i \in I} K_i \subseteq K_j$ für beliebiges $j \in I$ ist $\bigcap_{i \in I} K_i$ nach Lemma \ref{lem: abgeschlossen Teilmengen von kompakten Mengen} kompakt. (Hierfür benötigen wir, dass ein $j \in I$ existiert.)
\end{proof}


Eine weitere wichtige Aussage ist, dass stetige Funktionen Kompaktheit erhalten.


\begin{lem}
 Es sei $f \colon K \to \R^m$ stetig mit $K \subseteq \R^n$ kompakt. Dann ist auch das Bild $f(K) \subseteq \R^m$ kompakt.
\end{lem}
\begin{proof}
 Wir wollen zwei Beweise angeben für das Lemma angeben.
 
 \emph{(Überdeckungskompaktheit)} Sei $\{U_i\}_{i \in I}$ eine offene Überdeckung von $f(K)$. Da $f$ stetig ist, gibt es für alle $i \in I$ eine offene Menge $V_i \subseteq \R^n$ mit
 \[
  f^{-1}(U_i) = V_i \cap K,
  \quad
  \text{also}
  \quad
  f(V_i \cap K) \subseteq U_i.
 \]
 Die offenen Mengen $\{V_i\}_{i \in I}$ bilden offenbar eine offene Überdeckung von $K$. Da $K$ (überdeckungs)kompakt ist besitzt diese offene Überdeckung eine endliche Teilüberdeckung, es gibt also $i_1, \dotsc, i_s \in I$ mit
 \[
  K \subseteq V_{i_1} \cup \dotsb \cup V_{i_s};
 \]
 es gilt damit
 \[
  K
  = (V_{i_1} \cup \dotsb \cup V_{i_s}) \cap K
  = (V_{i_1} \cap K) \cup \dotsb \cup (V_{i_s} \cap K).
 \]
 Für das Bild $f(K)$ ergibt sich damit, dass
 \begin{align*}
  f(K)
  &= f((V_{i_1} \cap K) \cup \dotsb \cup (V_{i_s} \cap K)) \\
  &= f(V_{i_1} \cap K) \cup \dotsb \cup f(V_{i_s} \cap K)
  \subseteq U_{i_1} \cup \dotsb \cup U_{i_s}.
 \end{align*}
 Also besitzt die offene Überdeckung $\{U_i\}_{i \in I}$ von $K$ eine endliche Teilüberdeckung. Aus der Beliebigkeit der offenen Überdeckung folgt, dass $f(K)$ (überdeckungs)kompakt ist.
 
 \emph{(Folgenkompaktheit)} Es sei $(y_n)$ eine Folge auf $f(K)$. Für jedes $n \in \N$ gibt es dann ein $x_n \in K$ mit $f(x_n) = y_n$. Da $K$ (folgen)kompakt ist besitzt die Folge $(x_n)$ eine auf $K$ konvergente Teilfolge $(n_j)_{j \in \N}$. Es sei $x \coloneqq \lim_{j \to \infty} x_{n_j} \in K$. Aus der Stetigkeit von $f$ folgt, dass auch die Folge $(f(x_{n_j}))_{j \in \N} = (y_{n_j})_{j \in \N}$ konvergiert und
 \[
  f(x)
  = f\left( \lim_{j \to \infty} x_{n_j} \right)
  = \lim_{j \to \infty} f(x_{n_j})
  = \lim_{j \to \infty} y_{n_j}.
 \]
 Da $f(x) \in f(K)$ besitzt die Folge $(y_n)$ also eine auf $K$ konvergent Teilfolge.
\end{proof}


\begin{bem}
 Die Umkehrung der Aussage gilt im Allgemeinen nicht, d.h.\ die Urbilder kompakter Mengen und stetigen Abbildungen sind nicht unbedingt selber wieder kompakt: Die Abbildung
 \[
  f \colon \R \to \R, x \mapsto \frac{x}{1+|x|}
 \]
 ist stetig, und es ist
 \[
  |f(x)|
  = \left| \frac{x}{1+|x|} \right|
  = \frac{|x|}{1+|x|}
  < 1
  \quad
  \text{für alle $x \in \R$},
 \]
 also $f(\R) \subseteq (0,1)$. Für die kompakte Menge $[0,1] \subseteq \R$ ist deshalb $f^{-1}([0,1]) = \R$ nicht kompakt.
\end{bem}











\end{document}
