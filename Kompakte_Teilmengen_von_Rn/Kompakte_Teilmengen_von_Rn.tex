\documentclass[a4paper,10pt]{article}
%\documentclass[a4paper,10pt]{scrartcl}

\usepackage{../mystyle}

\title{Kompakte Teilmengen von $\R^n$}
\author{Jendrik Stelzner}
\date{\today}

\begin{document}
\maketitle


\begin{abstract}
 Wir wollen hier eine kurzen Überblick über den Kompaktheitsbegriff auf $\R^n$ geben. Dabei gibt es zwei Ansätze: Folgenkompaktheit und Überdeckungskompaktheit. Wir werden beide Begriffe vorstellen und dann ihre Äquivalenz zeigen. Anschließend wenden wir uns dem Satz von Heine-Borel zu, der erklärt, wie kompakte Teilmengen von $\R^n$ aussehen. Zum Schluß zeigen wir noch verschiedene Eigenschaften kompakter Teilmenengen von $\R^n$. Um zu zeigen, wie sich diese Eigenschaften aus den unterschiedlichen Charakterisierungen der Kompaktheit ergeben, geben wir dabei jeweils mehrere Beweise an.
\end{abstract}


\tableofcontents





\section{Stetige Funktionen auf abgeschlossenen Intervallen}


Zur Motivation des Kompaktheitbegriffes wollen wir zunächst die folgende wichtige Aussage beweisen:


\begin{lem}\label{lem: stetig auf Intervall}
 Es seien $a,b \in \R$ mit $a < b$ und $f \colon [a,b] \to \R$ stetig. Dann ist die Funktion $f$ beschränkt und nimmt auf $[a,b]$ ihr Maximum und Minimum an, d.h. es gibt $x_{\text{max}}, x_{\text{min}} \in [a,b]$ mit
 \[
  f(x_{\text{max}}) = \sup_{y \in [a,b]} f(y)
  \quad
  \text{und}
  \quad
  f(x_{\text{min}}) = \inf_{y \in [a,b]} f(y).
 \]
\end{lem}


Für offene Intervalle oder halboffene Intervalle gilt diese Aussage nicht. Man betrachte etwa die Abbildungen
\begin{gather*}
 (0,1] \to \R, x \mapsto \frac{1}{x}
\shortintertext{und}
 (0,1] \to \R, x \mapsto \frac{\sin \frac{1}{x}}{x}.
\end{gather*}


Herzstück des Beweises von Lemma \ref{lem: stetig auf Intervall} ist die Beobachtung, dass auf $[a,b]$ jede Folge eine konvergente Teilfolge besitzt.


\begin{lem}\label{lem: Intervall ist kompakt}
 Es seien $a,b \in \R$ mit $a < b$ und $(x_n)_{n \in \N}$ eine Folge auf $[a,b]$. Dann besitzt $(x_n)$ eine konvergente Teilfolge $n_j$ und für den Grenzwert gilt $\lim_{j \to \infty} x_{n_j} \in [a,b]$.
\end{lem}
\begin{proof}
 Da $a \leq x_n \leq b$ für alle $n \in \N$ ist die Folge $(x_n)$ beschränkt und besitzt daher nach Bolzano-Weierstraß eine konvergente Teilfolge $n_j$. Da das abgeschlossene Intervall $[a,b]$ abgeschlossen ist, ist auch $\lim_{j \to \infty} x_{n_j} \in [a,b]$.
\end{proof}


\begin{proof}[Beweis von Lemma \ref{lem: stetig auf Intervall}]
 Wir zeigen zunächst, dass $f$ beschränkt ist: Angenommen, $f$ wäre nach oben unbeschränkt. Dann gibt es für alle $n \in \N$ ein $x_n \in [a,b]$ mit $f(x_n) \geq n$. Nach Lemma \ref{lem: Intervall ist kompakt} besitzt die Folge $(x_n)$ eine konvergente Teilfolge $n_j$. Da $(x_{n_j})_{j \in \N}$ konvergiert folgt aus der Stetigkeit von $f$, dass auch die Folge $(f(x_{n_j}))_{j \in \N}$ konvergiert. Für alle $j \in \N$ ist aber $f(x_{n_j}) \geq n_j$, die Folge $f(x_{n_j})$ konvergiert also nicht. Dieser Widerspruch zeigt, dass $f$ nach oben unbeschränkt seien muss. Analog ergibt sich, dass $f$ auch nach unten beschränkt ist. Also ist $f$ beschränkt.
 
 Wir zeigen nun, dass $f$ sein Maximum annimmt. Hierfür sei
 \[
  M \coloneqq \sup_{y \in [a,b]} f(y).
 \]
 Da $f$ nach oben beschränkt ist, ist $M < \infty$. Nach der $\varepsilon$-Charakterisierung des Supremums gibt es für alle $n \geq 1$ ein $x_n \in [a,b]$ mit
 \[
  M \geq f(x_n) > M - \frac{1}{n}.
 \]
 Nach Lemma \ref{lem: Intervall ist kompakt} besitzt die Folge $(x_n)$ eine konvergente Teilfolge $n_j$. Wir schreiben $x \coloneqq \lim_{j \to \infty} x_{n_j}$. Da $f$ stetig ist, konvergiert auch die Folge $(f(x_{n_j}))$ und es gilt
 \[
  \lim_{j \to \infty} f(x_{n_j})
  = f\left( \lim_{j \to \infty} x_{n_j} \right)
  = f(x).
 \]
 Andererseits gilt für alle $j \in \N$
 \[
  M \geq f(x_{n_j}) > M - \frac{1}{n_j}.
 \]
 Also muss nach dem Sandwich-Lemma auch
 \[
  \lim_{j \to \infty} f(x_{n_j}) = M.
 \]
 Also ist $f(x) = M$. Das zeigt, dass $f$ auf $[a,b]$ sein Maximum annimmt. Analog ergibt sich, dass $f$ auf $[a,b]$ auch sein Minimum annimmt.
\end{proof}





\section{Folgenkompaktheit}


Zum Beweis von Lemma \ref{lem: stetig auf Intervall} haben wir haben wir das Zusammenspiel aus Folgenstetigkeit und Lemma \ref{lem: Intervall ist kompakt} benötigt. Das legt nahe, dass sich Lemma \ref{lem: stetig auf Intervall} auf beliebige Teilmengen $K \subseteq \R^n$ verallgemeinern lässt, für die eine zu Lemma \ref{lem: Intervall ist kompakt} analoge Aussage gilt.


\begin{defi}
 Es sei $X \subseteq \R^n$ und $(x_n)$ ein Folge auf $X$. Wir sagen \emph{$(x_n)$ konvergiert auf $X$}, falls die Folge $(x_n)$ konvergiert und $\lim_{n \to \infty} x_n \in X$.
\end{defi}


\begin{bsp}
 \begin{itemize}
  \item
   Die Folge $(1/n)_{n \geq 1}$ konvergiert auf $[0,1]$, nicht aber auch $(0,1)$.
  \item
   Ist $A \subseteq \R^m$ abgeschlossen und $(x_n)$ eine Folge auf $A$, die auf $\R^m$ konvergiert, so konvergiert $(x_n)$ auch schon auf $A$.
 \end{itemize}
\end{bsp}


\begin{defi}
 Eine Teilmenge $K \subseteq \R^n$ heißt \emph{folgenkompakt}, falls jede Folge $(x_n)$ auf $K$ eine auf $K$ konvergente Teilfolge besitzt.
\end{defi}


\begin{bsp}
 Lemma \ref{lem: Intervall ist kompakt} zeigt, dass ein abgeschlossenes Intervall $[a,b]$ mit $a < b$ folgenkompakt ist. Das offene Intervall $(a,b)$ hingegen ist nicht folgenkompakt, denn $(a + (b-a)/n)_{n \geq 2}$ eine Folge auf $(a,b)$, die keine auf $(a,b)$ konvergente Teilfolge besitzt.
\end{bsp}


\begin{prop}
 Es sei $K \subseteq \R^n$ folgenkompakt und $f \colon K \to \R$ stetig. Dann ist $f$ auf $K$ beschränkt und nimmt auf $K$ sein Maximum und Minimum an, d.h.\ es gibt $x_{\text{min}}, x_{\text{max}} \in K$ mit
 \[
  f(x_{\text{max}}) = \sup_{y \in K} f(y)
  \quad
  \text{und}
  \quad
  f(x_{\text{min}}) = \inf_{y \in K} f(y).
 \]
\end{prop}
\begin{proof}
 Nehme den Beweis von Lemma \ref{lem: stetig auf Intervall} und ersetze $[a,b]$ durch $K$ und die Verweise auf Lemma \ref{lem: Intervall ist kompakt} durch Folgenompaktheit.
\end{proof}


Da sich stetige Funktionen auf folgenkompakten Mengen gutartig verhalten, sind folgenkompakte Mengen von großer Bedeutung für die Analysis.





\section{Überdeckungskompaktheit}


Ein weiterer Kompaktheitsbegriff ist \emph{Überdeckungskompaktheit}. Dieser ist nützlich, da er sich gut in die Sprache der Umgebungen und offenen Mengen einfügt. Wie sich herausstellen wird, sind Überdeckungskompaktheit und Folgenkompaktheit für Teilmengen von $\R^n$ äquivalent.


\begin{defi}
 Es sei $\{U_i\}_{i \in I}$ eine Kollektion von Teilmengen $U_i \subseteq \R^n$ und $X \subseteq \R^n$.
 
 Die Kollektion $\{U_i\}_{i \in I}$ heißt \emph{Überdeckung von $X$}, falls $X \subseteq \bigcup_{i \in I} U_i$. Diese Überdeckung heißt dann \emph{endlich}, bzw.\ \emph{abzählbar}, falls $I$ endlich, bzw.\ abzählbar ist.
 
 Eine \emph{Teilüberdeckung} ist dann eine Teilkollektion $\{U_j\}_{j \in J}$, also $J \subseteq I$, so dass bereits $\{U_j\}_{j \in J}$ eine Überdeckung von $X$ ist.
 
 Außerdem heißt die Überdeckung $\{U_i\}_{i \in I}$ von $X$ \emph{offen}, falls die $U_i \subseteq \R^n$ alle offen sind.
\end{defi}


\begin{bem}
 Offenbar ist eine Teilüberdeckung einer offenen Überdeckung ebenfalls offen.
\end{bem}


\begin{bsp}
 \begin{itemize}
  \item
   Die Kollektion $\{B_k(0) \mid k \in \N, k \geq 1\}$ bildet eine abzählbare, offene Überdeckung von $\R^n$.
  \item
   Die offenen Intervalle $\{(a,b) \mid a,b \in \R, a < b\}$ bilden eine offene Überdeckung von $\R$.
  \item
   Allgemeiner bilden die $\varepsilon$-Bälle $\{B_\varepsilon(x) \mid x \in \R^n, \varepsilon > 0\}$ eine offene Überdeckung von $\R^n$.
  \item
   Die Würfel $\{[-K,K]^n \mid K \in \N, K \geq 1\}$ bilden eine abzählbar Überdeckung von $\R^n$, die nicht offen ist.
  \item
   Die $\varepsilon$-Bälle mit rationalen Radius um Mittelpunkte mit rationalen Koeffizienten $\{B_q(x) \mid x \in \Q^n, q \in \Q^+\}$ bilden eine abzählbare offene Überdeckung von $\R^n$.
  \item
   Die Intervalle $\{(k,k+1) \mid k \in \Z\}$ bilden eine abzählbare, offene Überdeckung von $\R \setminus \Z = \bigcup_{k \in \N} (k,k+1)$; sie sind aber keine Überdeckung von $\R$.
  \item
   Die offenen Intervalle $\{ (-1/k, 1+1/k) \mid k \in \N, k \geq 1\}$ bilden eine offene Überdeckung des abgeschlossenen Einheitsintervalls $[0,1]$; diese Überdeckung besitzt eine endliche (sogar einelementige) Teilüberdeckung.
 \end{itemize}
\end{bsp}


\begin{defi}
 Eine Teilmenge $K \subseteq \R^n$ heißt \emph{überdeckungskompakt}, falls jede offene Überdeckung von $K$ eine endliche Teilüberdeckung besitzt.
\end{defi}


\begin{bsp}
 \begin{itemize}
  \item
   Endliche Mengen sind überdeckungskompakt: Ist $M \subseteq \R^n$ endlich mit Punkten $M = \{m_1, \dotsc, m_s\}$ und $\{U_i\}_{i \in I}$ eine offene Überdeckung von $M$, so gibt es für jedes $1 \leq j \leq s$ ein $i_j \in I$ mit $m_j \in U_{i_j}$. Dann ist $\{U_{i_j}\}_{j=1,\dotsc,s}$ eine endliche Teilüberdeckung von $M$.
  \item
   $\R^n$ ist nicht überdeckungskompakt, denn $\{B_k(0) \mid k \in \N, k \geq 1\}$ ist eine offen Überdeckung von $\R^n$, die keine endliche Teilüberdeckung besitzt.
  \item
   Das offene Einheitsintervall $(0,1)$ ist nicht überdeckungskompakt, denn $\{(1/k, 1-1/k) \mid k \in \N, k \geq 3\}$ ist eine offene Überdeckung von $(0,1)$, die keine offene Teilüberdeckung besitzt.
 \end{itemize}
\end{bsp}


\begin{lem}\label{lem: Überdeckungskompakt abgeschlossen und beschränkt}
 Es sei $K \subseteq \R^n$ überdeckungskompakt. Dann gilt:
 \begin{enumerate}
  \item
   $K$ ist abgeschlossen.
  \item
   $K$ ist beschränkt.
 \end{enumerate}
\end{lem}
\begin{proof}
 \begin{enumerate}
  \item
   Wir zeigen, dass $K$ abgeschlossen ist, indem wir zeigen, dass $K^c$ offen ist. Es sei hierfür $x \in K^c$ beliebig aber fest. Wir wollen zeigen, dass es ein $\varepsilon > 0$ gibt, so dass $B_\varepsilon(x) \subseteq K^c$. Für jedes $y \in K$ ist $x \neq y$ (da $x \notin K$), also $\varepsilon_y \coloneqq \|x-y\|/2 > 0$. Für jedes $y \in K$ sind dann $B_{\varepsilon_y}(x)$ und $B_{\varepsilon_y}(y)$ disjunkt.
   
   (Wir würden nun gerne die Bälle $B_{\varepsilon_y}(x)$ schneiden, um eine, hoffentlich offene, Menge zu erhalten, die $x$ enthält, aber disjunkt zu $K$ ist. Das Problem ist, dass der möglicherweise unendliche Schnitt $\bigcap_{y \in K} B_{\varepsilon_y}(x)$ nicht mehr notwendigerweise offen ist. Mithilfe der Überdeckungskompaktheit können wir diesen unendlichen Schnitt aber durch einen endlichen ersetzen.)
   
   Die $\varepsilon$-Bälle $\{B_{\varepsilon_y}(y) \mid y \in K\}$ bilden eine offene Überdeckung von $K$. Da $K$ überdeckungskompakt ist, besitzt diese Überdeckung eine endliche Teilüberdeckung; es gibt also $y_1, \dotsc, y_s \in K$ mit
   \begin{equation}\label{eqn: K endliche Überdeckung}
    K \subseteq B_{\varepsilon_{y_1}}(y_1) \cup \dotsb \cup B_{\varepsilon_{y_s}}(y_s).
   \end{equation}
   Wir setzen $\varepsilon \coloneqq \min_{i=1,\dotsc,s} \varepsilon_{y_s} > 0$. Da $B_{\varepsilon_y}(x)$ und $B_{\varepsilon_y}(y)$ für alle $y \in K$ disjunkt sind, folgt aus \eqref{eqn: K endliche Überdeckung}, dass auch $B_\varepsilon(x)$ und $K$ disjunkt sind; es ist nämlich
   \begin{align*}
    B_\varepsilon(x) \cap K
    &\subseteq B_\varepsilon(x) \cap \bigcup_{i=1}^s B_{\varepsilon_{y_i}}(y_i) \\
    &= \bigcup_{i=1}^s (B_\varepsilon(x) \cap B_{\varepsilon_{y_i}}(y_i)) \\
    &\subseteq \bigcup_{i=1}^s \underbrace{(B_{\varepsilon_{y_i}}(x) \cap B_{\varepsilon_{y_i}(y_i)})}_{= \emptyset} \\
    &= \emptyset.
   \end{align*}
   Es ist also $B_\varepsilon(x) \subseteq K^c$. Aus der Beliebigkeit von $x \in K^c$ folgt, dass $K^c$ offen ist, und somit $K$ abgeschlossen.
  \item
   Die offenen Bälle $\{B_r(0) \mid r > 0\}$ bilden eine offene Überdeckung von $\R^n$, und damit insbesondere auch von $K$. Da $K$ überdeckungskompakt ist, besitzt diese offene Überdeckung eine endliche Teilüberdeckung. Es gibt also Radien $r_1, \dotsc, r_s > 0$ mit $K \subseteq B_{r_1}(0) \cup \dotsb \cup B_{r_s}(0)$. Für den größten dieser Radien $R = \max_{i=1,\dotsc,s} r_i$ ist $B_{r_i}(0) \subseteq B_R(0)$ für alle $1 \leq i \leq s$, und somit auch
   \[
    K \subseteq B_{r_1}(0) \cup \dotsb \cup B_{r_s}(0) \subseteq B_R(0).
   \]
   Also ist $K$ beschränkt.
  \qedhere
 \end{enumerate}
\end{proof}





\section{Äquivalenz der Kompaktheitsbegriffe}
Wie wir nun sehen werden, sind Folgenkompaktheit und Überdeckungskompaktheit äquivalent, und führen daher zum gleichen Kompaktheitsbegriff. Diese beiden Charakterisierungen kompakter Mengen (über Folgenkompaktheit und Überdeckungskompaktheit) erlauben es uns dann, kompakte Mengen in verschiedenen Settings zu benutzen, ohne größere Übersetzungsschmerzen zwischen den Settings erleiden zu müssen.


\begin{prop}
 Eine Teilmenge $K \subseteq \R^n$ ist genau dann folgenkompakt, wenn sie überdeckungskompakt ist.
\end{prop}
\begin{proof}
 (Überdeckungskompaktheit $\Rightarrow$ Folgenkompaktheit) Angenommen $K$ ist überdeckungskompakt, aber nicht folgenkompakt. Dann gibt es eine Folge $(x_n)$ auf $K$, die keine auf $K$ konvergente Teilfolge besitzt.
 
 \begin{beh}
  Für jedes $x \in K$ gibt es ein $\varepsilon_x > 0$, so dass $\|x - x_n\| \geq \varepsilon_x$ für fast alle $n \in \N$.
 \end{beh}
 \begin{proof}[Beweis der Behauptung]
  Ansonsten gibt es ein $x \in K$, so dass es für jedes $\varepsilon > 0$ unendlich viele $n \in \N$ mit $\|x - x_n\| < \varepsilon$ gibt. Insbesondere gibt es dann ein $n_1 \in \N$ mit $\|x - x_{n_1}\| < 1$. Dann gibt es auch ein $n_2 \in \N$ mit $n_2 > n_1$ und $\|x - x_{n_2}\| < 1/2$. Rekursiv ergibt sich, dass es für alle $j \geq 1$ ein $n_j \in \N$ gibt, so dass $n_j > n_{j-1}$ und $\|x - x_{n_j}\| < 1/j$. Es ist dann $n_j$ eine Teilfolge mit $x_{n_j} \to x$ für $j \to \infty$. Dies steht im Widerspruch zur Annahme, dass $(x_n)$ keine auf $K$ konvergente Teilfolge besitzt.
 \end{proof}
 
 Die $\varepsilon$-Bälle $\{B_{\varepsilon_x}(x) \mid x \in K\}$ bilden offenbar eine offene Überdeckung von $K$. Da $K$ überdeckungskompakt ist, besitzt diese offene Überdeckung eine endliche Teilüberdeckung. Es gibt also $x_1, \dotsc, x_s \in K$, so dass
 \begin{equation}
  K \subseteq B_{\varepsilon_{x_1}}(x_1) \cup \dotsb \cup B_{\varepsilon_{x_s}}(x_s).
 \end{equation}
 Da jeder der $\varepsilon$-Bälle $B_{\varepsilon_x}(x)$ nur endlich viele Folgeglieder enthält, enthält damit auch $K$ nur endlich viele Folgeglieder; das kann offenbar nicht sein.
 
 (Folgenkompaktheit $\Rightarrow$ Überdeckungskompakt) Wir beginnen den Beweis dieser Implikation mit der folgenden Eigenschaft folgenkompakte Mengen:
 
 \begin{beh}
  $K$ ist \emph{total beschränkt}, d.h.\ für jedes $\varepsilon > 0$ lässt sich $K$ mit endlich vielen $\varepsilon$-Bällen überdecken.
 \end{beh}
 \begin{proof}[Beweis der Behauptung]
  Angenommen $K$ ist nicht total beschränkt. Dann gibt es ein $\varepsilon > 0$, so dass sich $K$ nicht mit endlich vielen $\varepsilon$-Bällen überdecken lässt. Wir können dann eine Folge $(x_n)$ auf $K$ konstruieren, die keine konvergente Teilfolge besitzt:
  
  Wir beginnen mit einem beliebigen Startpunkt $x_0 \in K$. Da $K$ sich nicht mit endlich vielen $\varepsilon$-Bällen überdecken lässt, ist $K \nsubseteq B_\varepsilon(x_0)$. Es gibt daher ein $x_1 \in K$ mit $x_1 \notin B_\varepsilon(x)$, also $\|x_1 - x_0\| \geq \varepsilon$. Da $K$ sich nicht mit endlich vielen $\varepsilon$-Bällen überdecken lässt, ist auch $K \nsubseteq B_\varepsilon(x_0) \cup B_\varepsilon(x_1)$, es gibt also $x_2 \in K$ mit $x_2 \notin B_\varepsilon(x_0) \cup B_\varepsilon(x_1)$ und somit $\|x_2 - x_i\| \geq \varepsilon$ für $i = 0, 1$. Dieses Prinzip können wir weiterführen: Haben wir schon $x_0, \dotsc, x_n$ definiert, so gibt es immer noch ein $x_{n+1} \in K$ mit $x_{n+1} \notin \bigcup_{i=0}^n B_\varepsilon(x_i)$, also $\|x_{n+1} - x_i\| \geq \varepsilon$ für alle $0 \leq i \leq n$.
  
  Wir erhalten somit eine Folge $(x_n)$ auf $K$, so dass $\|x_n - x_{n'}\| \geq \varepsilon$ für alle $n, n' \in \N$. Inbesondere ist keine Teilfolge von $(x_n)$ eine Cauchy-Folge, und somit auch keine Teilfolge konvergent.
 \end{proof}
 
 
 Es sei nun $\{U_i\}_{i \in I}$ eine offene Überdeckung von $K$. Wir nennen ein Teilmenge $A \subseteq K$ \emph{überdeckbar}, falls $A$ eine endliche Teilüberdeckung von $\{U_i\}_{i \in I}$ besitzt. Offenbar gilt für überdeckbare Mengen $A_1, \dotsc, A_s \subseteq A$, dass auch $A_1 \cup \dotsb \cup A_s$ überdeckbar ist. Wir wollen zeigen, dass schon $K$ überdeckbar ist. Hierfür nehmen wir an, dass $K$ nicht überdeckbar sei.
 
 Da $K$ total beschränkt ist, gibt es $y^{(1)}_1, \dotsc, y^{(1)}_{s_1} \in K$ mit
 \[
  K \subseteq B_1\left(y^{(1)}_1\right) \cup \dotsb \cup B_1\left(y^{(1)}_{s_1}\right),
 \]
 und damit
 \[
  K = \left(K \cap B_1\left(y^{(1)}_1\right)\right) \cup \dotsb \cup \left(K \cap B_1\left(y^{(1)}_{s_1}\right)\right).
 \]
 Da $K$ nicht überdeckbar ist, muss bereits $K \cap B_1(y^{(1)}_i)$ für ein $1 \leq i_1 \leq s_1$ nicht überdeckbar sein (denn sonst wäre $K$ als endliche Vereinigung überdeckbarer Mengen ebenfalls überdeckbar). Wir setzen $x_1 \coloneqq y_{i_1}$.
 
 Da $K$ total beschränkt ist, gibt es auch $y^{(2)}_1, \dotsc, y^{(2)}_{s_2} \in K$ mit
 \[
  K \subseteq B_{1/2}\left(y^{(2)}_1\right) \cup \dotsb \cup B_{1/2}\left(y^{(2)}_{s_2}\right),
 \]
 und somit
 \begin{align*}
   &\,K \cap B_1(x_1) \\
  =&\, \left(K \cap B_1(x_1) \cap B_{1/2}\left(y^{(2)}_1\right)\right) \cup \dotsb \cup \left(K \cap B_1(x_1) \cap B_{1/2}\left(y^{(2)}_{s_2}\right)\right).
 \end{align*}
 Da $K \cap B_1(x_1)$ nicht überdeckbar ist, gibt es ein $1 \leq i_2 \leq s_2$, sodass bereits $K \cap B_1(x_1) \cap B_{1/2}(y^{(2)}_i)$ nicht überdeckbar ist. Wir setzen $x_2 \coloneqq y^{(2)}_{i_2}$.
 
 Rekursiv erhalten wir so eine Folge $(x_n)_{n \geq 1}$ auf $K$, so dass
 \[
  K \cap B_1(x_1) \cap \dotsb \cap B_{1/n}(x_n)
  \quad \text{für alle $n \geq 1$ nicht überdeckbar ist}.
 \]
 Da $K$ folgenkompakt ist, besitzt die Folge $(x_n)$ eine auf $K$ konvergente Teilfolge $n_j$. Wir setzen $x^* \coloneqq \lim_{j \to \infty} x_{n_j} \in K$. Da $\{U_i\}_{i \in I}$ eine Überdeckung von $K$ ist gibt es ein $i_* \in I$ mit $x^* \in U_{i_*}$. Da $U_{i_*}$ offen ist, gibt es ein $\varepsilon > 0$ mit $B_{\varepsilon}(x^*) \subseteq U_{i_*}$. Es sei $J \in \N$, $J \geq 1$ mit
 \[
  1/{n_J} < \frac{\varepsilon}{2}
  \quad
  \text{und}
  \quad
  x_{n_J} \in B_\frac{\varepsilon}{2}(x^*).
 \]
 (Ein solches $J$ existiert, da $1/{n_j} \to 0$ und $x_{n_j} \to x^*$ für $j \to \infty$). Es ist dann $B_{1/n_J}(x_{n_J}) \subseteq B_{\varepsilon}(x^*)$, denn für alle $x \in B_{1/n_J}(x_{n_J})$ ist
 \[
  \|x - x^*\|
  \leq \|x - x_{n_J}\| + \|x_{n_J} - x^*\|
  < 1/n_J + \frac{\varepsilon}{2}
  < \frac{\varepsilon}{2} + \frac{\varepsilon}{2}
  = \varepsilon.
 \]
 Damit ist dann auch
 \[
  K \cap B_1(x_1) \cap \dotsb \cap B_{n_J}(x_{n_J})
  \subseteq B_{n_J}(x_{n_J})
  \subseteq B_{\varepsilon}(x^*)
  \subseteq U_{i_*},
 \]
 was im Widerspruch dazu steht, dass $K \cap B_1(x_1) \cap \dotsb \cap B_{n_J}(x_{n_J})$ nicht überdeckbar ist.
  
 Dieser Widerspruch zeigt, dass bereits $K$ überdeckbar ist. Aus der Beliebigkeit der offenen Überdeckung $\{U_i\}_{i \in I}$ von $K$ folgt, dass $K$ überdeckungskompakt ist.
\end{proof}


Wegen dieser Äquivalenz von Folgen- und Überdeckungskompaktheit spricht man auch nur von Kompaktheit: Eine Teilmenge $K \subseteq \R^n$ heißt also \emph{kompakt}, falls sie folgenkompakt, bzw.\ überdeckungskompakt ist. Kompakte Mengen sind von großer Nützlichkeit in der Mathematik.





\section{Der Satz von Heine-Borel}
Der Satz von Heine-Borel beantwortet die Frage, wie kompakte Teilmengen von $\R^n$ aussehen. Die nötigen Vorbereitungen zum Beweis des Satzes haben wir bereits hinter uns gebracht.


\begin{thrm}[Satz von Heine-Borel]
 Eine Teilmenge $K \subseteq \R^n$ ist genau dann kompakt, wenn $K$ abgeschlossen und beschränkt ist.
\end{thrm}
\begin{proof}
 Wir wir in Lemma \ref{lem: Überdeckungskompakt abgeschlossen und beschränkt} bereits gesehen haben, sind (über\-deck\-ungs)kom\-pak\-te Teilmengen von $\R^n$ sowohl abgeschlossen als auch beschränkt.
 
 Es sei andererseits $K$ abgeschlossen und beschränkt. Wir wollen zeigen, dass $K$ (folgen)kompakt ist. Es sei hierfür $(x_n)_{n \in \N}$ eine Folge auf $K$. Da $K$ beschränkt ist, besitzt $K$ nach dem Satz von Bolzano-Weierstraß eine konvergente Teilfolge $n_j$. Da $(x_{n_j})$ eine Folge auf $K$ ist, und $K$ abgeschlossen ist, ist auch $\lim_{j \to \infty} x_{n_j} \in K$. Also ist $K$ (folgen)kompakt.
\end{proof}


Der Satz von Heine-Borel ist sehr nützlich, um Teilmengen von $\R^n$ auf Kompaktheit zu untersuchen.


\begin{bsp}
 \begin{itemize}
  \item
   Die $n$-dimensionale Sphäre $S^n = \{x \in \R^{n+1} \mid \|x\| = 1\}$ ist offenbar beschränkt, und wir wissen bereits, dass sie auch abgeschlossen ist. Also ist $S^n$ kompakt.
  \item
   Wir wissen bereits, dass der Einheitswürfel $[0,1]^n \subseteq \R^n$ abgeschlossen ist. Er ist auch beschränkt, denn für alle $x = (x_1, \dotsc, x_n) \in [0,1]^n$ ist $\|x\| \leq \sum_{i=1}^n |x_i| \leq n$, bzw.\ sogar
   \[
    \|x\|
    = \sqrt{x_1^2 + \dotsb + x_n^2}
    \leq \sqrt{n}.
   \]
   Also ist der Einheitswürfel kompakt.
  \item
   Für alle $a,b \in \R$ mit $a < b$ ist das abgeschlossene Intervall $[a,b]$ abgeschlossen und beschränkt, und somit kompakt.
  \item
   Das offene Intervall $(0,1)$ ist zwar beschränkt, aber nicht abgeschlossen, und somit auch nicht kompakt.
  \item
   Der unendliche Zylinder $S^1 \times \R \subseteq \R^3$ ist zwar abgeschlossen, aber nicht beschränkt, und somit ebenfalls nicht kompakt.
  \item
   Ist $f \colon \R \to \R$ eine stetige Funktion, so ist der Graph
   \[
    G \coloneqq \{(x,f(x)) \mid x \in \R\}
   \]
   zwar abgeschlossen, aber nicht beschränkt, und somit auch nicht kompakt.
  \item
   Offene $\varepsilon$-Bälle sind zwar beschränkt, aber nicht abgeschlossen, und somit ebenfalls nicht kompakt.
  \item
   Die obere Halbebene
   \[
    \mathbb{H} \coloneqq \{(x,y) \in \R^2 \mid y > 0\}
   \]
   ist weder abgeschlossen noch beschränkt und somit nicht kompakt. Die abgeschlossene obere Halbebene
   \[
    \overline{\mathbb{H}} \coloneqq \{(x,y) \in \R^2 \mid y \geq 0\}
   \]
   ist zwar abgeschlossen, aber nicht beschränkt, und somit nicht kompakt.
 \end{itemize}
\end{bsp}





\section{Weitere Eigenschaften kompakter Mengen}
Wir wollen hier noch weitere Eigenschaften kompakter Mengen angeben und beweisen. Dabei geben wir jeweils mehrere Beweise an, um den Leser mit den verschiedenen Charakterisierungen kompakter Mengen vertraut zu machen.


Wie wir bereits in Lemma \ref{lem: Überdeckungskompakt abgeschlossen und beschränkt} gesehen haben, sind (überdeckungs)kompakte Teilmengen $K \subseteq \R^n$ abgeschlossen und beschränkt; nach dem Satz von Heine-Borel ist dies gerade die charakterisierende Eigenschaft kompakter Mengen. Der Vollständigkeit halber wollen wir diese beiden Eigenschaften kompakter Mengen auch noch einmal durch Folgenkompaktheit beweisen.


\begin{proof}
 Es sei $K \subseteq \R^m$ (folgen)kompakt.
 
 Um zu zeigen, dass $K$ abgeschlossen ist, werden wir zeigen, dass $K$ unter Grenzwerten von Folgen abgeschlossen ist. Hierfür sei $(x_n)$ eine Folge auf $K$, die auf $\R^m$ konvergiert. Für den Grenzwert $x \coloneqq \lim_{n \to \infty} x_n$ wollen wir zeigen, dass $x \in K$. Da $K$ (folgen)kompakt ist besitzt $(x_n)$ eine auf $K$ konvergente Teilfolge $n_j$. Für $x^* \coloneqq \lim_{j \to \infty} x_{n_j}$ ist also $x^* \in K$. Da $x_n \to x$ für $n \to \infty$ ist aber auch $x_{n_j} \to x$ für $j \to \infty$. Also ist
 \[
  x = \lim_{j \to \infty} x_{n_j} = x^* \in K.
 \]
 
 Die Beschränktheit von $K$ zeigen wir per Widerspruch. Wäre $K$ nicht beschränkt, so gebe es für alle $n \in \N$ ein $x_n \in K$ mit $\|x_n\| \geq n$. Da $K$ (folgen)kompakt ist, hat die Folge $(x_n)$ eine auf $K$ konvergente Teilfolge $n_j$. Da die Folge $(x_{n_j})$ konvergiert, folgt aus der Stetigkeit der Norm, dass auch die Folge der Normen $(\|x_{n_j}\|)$ konvergiert. Da $\|x_{n_j}\| \geq n_j$ für alle $j \in \N$ konvergiert die Folge der Normen aber nicht.
\end{proof}


\begin{lem}
 Endliche Vereinigungen kompakter Mengen sind kompakt, d.h.\ sind $K_1, \dotsc, K_r \subseteq \R^n$ kompakt, so ist auch $K_1 \cup \dotsb \cup K_r$ kompakt.
\end{lem}
\begin{proof}
 Wir setzen $K \coloneqq K_1 \cup \dotsb \cup K_r$. Wir wollen drei Beweise für diese Aussage geben.
 
 \emph{(Überdeckungskompaktheit)} Es sei $\{U_i\}_{i \in I}$ eine offen Überdeckung von $K$. Dann ist $\{U_i\}_{i \in I}$ auch eine offene Überdeckung von $K_j$ für alle $1 \leq j \leq r$. Da die $K_j$ (überdeckungs)kompakt sind, besitzt diese offene Überdeckung für jedes $1 \leq j \leq r$ eine endliche Teilüberdeckung von $K_j$, es gibt also $i_{j,1}, \dotsc, i_{j,s_j} \in I$ mit
 \[
  K_j \subseteq U_{i_{j,1}} \cup \dotsb \cup U_{i_{j,s_j}}.
 \]
 Zusammenfügen dieser endlichen Teilüberdeckungen ergibt
 \[
  K
  = \bigcup_{j=1}^r K_j
  \subseteq \bigcup_{j=1}^r \bigcup_{\ell=1}^{s_j} U_{i_{j,\ell}},
 \]
 also eine endliche Teilüberdeckung von $K$. Das zeigt, dass jede offene Überdeckung von $K$ eine endliche Teilüberdeckung besitzt, also dass $K$ (über\-deck\-ungs)kom\-pakt ist.
 
 \emph{(Folgenkompaktheit)} Es sei $(x_n)$ eine Folge auf $K$. Eine der Mengen $K_j$ muss unendlich viele Folgeglieder enthalten (denn sonst würde $K$ nur endlich viele Folgeglieder enthalten), d.h. es gibt ein $1 \leq j \leq r$, so dass $x_n \in K_j$ für unendlich viele $n \in \N$. Wir können o.B.d.A.\ davon ausgehen, dass $j = 1$, also $x_n \in K_1$ für unendlich viele $n \in \N$. Es gibt also eine Teilfolge $\tilde{n}_j$ mit $x_{\tilde{n}_j} \in K_1$ für alle $j \in \N$. Da $K_1$ (folgen)kompakt ist, besitzt die Folge $(x_{\tilde{n}_j})_{j \in \N}$ eine auf $K_1$ konvergente Teilfolge $n_j$. Da $\lim_{j \to \infty} x_{n_j} \in K_1 \subseteq K$ ist dies eine Teilfolge von $(x_n)$, die auf $K$ konvergiert. Dass zeigt, dass jede Folge auf $K$ eine auf $K$ konvergente Teilfolge besitzt, dass also $K$ (folgen)kompakt ist.
 
 \emph{(Heine-Borel)} Da die $K_j$ kompakt sind, sind die $K_j$ beschränkt. Es gibt daher für alle $1 \leq j \leq r$ ein $C_j > 0$ mit $\|x\| \leq C_j$ für alle $x \in K_j$. Für $C \coloneqq \max_{j=1,\dotsc,r} C_j$ ist deshalb $\|x\| \leq C$ für alle $x \in K$. Also ist $K$ beschränkt. Da die $K_j$ kompakt sind, sind sie auch abgeschlossen; da endliche Vereinigungen abgeschlossener Mengen wieder abgeschlossen sind, ist daher auch $K$ abgeschlossen. Da $K$ abgeschlossen und beschränkt ist, ist $K$ kompakt.
\end{proof}


\begin{bem}
 Die unendliche Vereinigung kompakter Mengen ist im Allgemeinen nicht kompakt. So sind etwa die Würfel $[-R, R]^n \subseteq \R^n$ für alle $R > 0$ kompakt, aber $\R^n = \bigcup_{k=1}^\infty [-k,k]^n$ ist nicht kompakt. (Das Problem hier ist, dass die unendliche Vereinigung beschränkter Mengen nicht mehr beschränkt seien muss.) Auch $(-1,1)^n = \bigcup_{n=2}^\infty [-1+1/n, 1-1/n]$ ist nicht kompakt. (Das Problem hier ist, dass die unendliche Vereinigung abgeschlossener Mengen nicht mehr abgeschlossen seien muss.)
\end{bem}


Wie wir bereits gesehen haben, sind nicht alle abgeschlossenen Mengen kompakt. Es gilt jedoch, dass abgeschlossener Teilmengen kompakter Mengen ebenfalls kompakt ist.


\begin{lem}\label{lem: abgeschlossen Teilmengen von kompakten Mengen}
 Ist $K \subseteq \R^n$ kompakt und $C \subseteq \R^n$ abgeschlossen mit $C \subseteq K$, so ist auch $C$ kompakt.
\end{lem}
\begin{proof}
 Wir wollen drei Beweise für das Lemma geben.
 
 \emph{(Folgenkompaktheit)} Es sei $(x_n)$ eine Folge auf $C$. Dann ist $(x_n)$ auch eine Folge auf $K$. Da $K$ (folgen)kompakt ist besitzt $(x_n)$ eine auf $K$ konvergente Teilfolge $n_j$. Da $C$ abgeschlossen ist, ist schon $\lim_{j \to \infty} x_{n_j} \in C$, also $(x_{n_j})_{j \in \N}$ schon auf $C$ konvergent. Das zeigt, dass jede Folge auf $C$ eine auf $C$ konvergente Teilfolge besitzt. Also ist $C$ (folgen)kompakt.
 
 \emph{(Überdeckungskompaktheit)} Es sei $\{U_i\}_{i \in I}$ eine offene Überdeckung von $C$. Da $C$ abgeschlossen ist, ist $C^c$ offen. Da $\{U_i\}_{i \in I}$ eine offene Überdeckung von $C$ ist, is $\{U_i\}_{i \in I} \cup \{C^c\}$ ist eine offen Überdeckung von $K$ (sogar eine offene Überdeckung von $\R^n$). Da $K$ (überdeckungs)kompakt ist besitzt diese offene Überdeckung eine endliche Teilüberdeckung; es gibt also $i_1, \dotsc, i_s \in I$, so dass
 \[
  K \subseteq U_{i_1} \cup \dotsb \cup U_{i_s} \cup C^c.
 \]
 Da $C$ und $C^c$ disjunkt sind, ist damit
 \[
  C \subseteq U_{i_1} \cup \dotsb \cup U_{i_s}.
 \]
 Das zeigt, dass jede offene Überdeckung von $C$ eine endliche Teilüberdeckung besitzt. Also ist $C$ (überdeckungs)kompakt.
 
 \emph{(Heine-Borel)} Da $K$ kompakt ist, ist $K$ beschränkt. Da $C \subseteq K$ ist daher auch $C$ beschränkt. Da $C$ abgeschlossen und beschränkt ist, ist $K$ kompakt.
\end{proof}


\begin{kor}
 Ist $\{K_i\}_{i \in i}$ eine nichtleere Kollektion von kompakten Teilmengen $K_i \subseteq \R^n$, so ist auch $\bigcap_{i \in I} K_i$ kompakt.
\end{kor}
\begin{proof}
 Da die $K_i$ kompakt sind, sind sie auch abgeschlossen. Also ist auch $\bigcap_{i \in I} K_i$ abgeschlossen. Da außerdem $\bigcap_{i \in I} K_i \subseteq K_j$ für beliebiges $j \in I$ ist $\bigcap_{i \in I} K_i$ nach Lemma \ref{lem: abgeschlossen Teilmengen von kompakten Mengen} kompakt. (Hierfür benötigen wir, dass ein $j \in I$ existiert.)
\end{proof}


Eine weitere wichtige Aussage ist, dass stetige Funktionen Kompaktheit erhalten.


\begin{lem}
 Es sei $f \colon K \to \R^m$ stetig mit $K \subseteq \R^n$ kompakt. Dann ist auch das Bild $f(K) \subseteq \R^m$ kompakt.
\end{lem}
\begin{proof}
 Wir wollen zwei Beweise für dieses Lemma angeben.
 
 \emph{(Überdeckungskompaktheit)} Es sei $\{U_i\}_{i \in I}$ eine offene Überdeckung des Bildes $f(K)$. Da $f$ stetig ist, gibt es für alle $i \in I$ eine offene Menge $V_i \subseteq \R^n$ mit
 \[
  f^{-1}(U_i) = V_i \cap K,
  \quad
  \text{also}
  \quad
  f(V_i \cap K) \subseteq U_i.
 \]
 Die offenen Mengen $\{V_i\}_{i \in I}$ bilden offenbar eine offene Überdeckung von $K$. Da $K$ (überdeckungs)kompakt ist besitzt diese offene Überdeckung eine endliche Teilüberdeckung, es gibt also $i_1, \dotsc, i_s \in I$ mit
 \[
  K \subseteq V_{i_1} \cup \dotsb \cup V_{i_s}.
 \]
 Es gilt damit
 \[
  K
  = (V_{i_1} \cup \dotsb \cup V_{i_s}) \cap K
  = (V_{i_1} \cap K) \cup \dotsb \cup (V_{i_s} \cap K).
 \]
 Für das Bild $f(K)$ ergibt sich damit, dass
 \begin{align*}
  f(K)
  &= f((V_{i_1} \cap K) \cup \dotsb \cup (V_{i_s} \cap K)) \\
  &= f(V_{i_1} \cap K) \cup \dotsb \cup f(V_{i_s} \cap K)
  \subseteq U_{i_1} \cup \dotsb \cup U_{i_s}.
 \end{align*}
 Also besitzt die offene Überdeckung $\{U_i\}_{i \in I}$ von $f(K)$ eine endliche Teilüberdeckung. Aus der Beliebigkeit der offenen Überdeckung folgt, dass $f(K)$ (überdeckungs)kompakt ist.
 
 \emph{(Folgenkompaktheit)} Es sei $(y_n)$ eine Folge auf $f(K)$. Für jedes $n \in \N$ gibt es ein $x_n \in K$ mit $f(x_n) = y_n$. Da $K$ (folgen)kompakt ist besitzt die Folge $(x_n)$ eine auf $K$ konvergente Teilfolge $n_j$. Es sei $x \coloneqq \lim_{j \to \infty} x_{n_j} \in K$. Aus der Stetigkeit von $f$ folgt, dass auch die Folge $(f(x_{n_j}))_{j \in \N} = (y_{n_j})_{j \in \N}$ konvergiert und
 \[
  f(x)
  = f\left( \lim_{j \to \infty} x_{n_j} \right)
  = \lim_{j \to \infty} f(x_{n_j})
  = \lim_{j \to \infty} y_{n_j}.
 \]
 Da $f(x) \in f(K)$ besitzt die Folge $(y_n)$ also eine auf $K$ konvergent Teilfolge.
\end{proof}


Bilder kompakter Mengen (unter stetigen Abbildungen) sind also kompakt.


\begin{bem}
 Die Umkehrung der Aussage gilt im Allgemeinen nicht, d.h.\ die Urbilder kompakter Mengen unter stetigen Abbildungen sind nicht zwangsweise kompakt. So ist etwa die Abbildung
 \[
  f \colon \R \to \R, x \mapsto \frac{x}{1+|x|}
 \]
 stetig, und es gilt
 \[
  |f(x)|
  = \left| \frac{x}{1+|x|} \right|
  = \frac{|x|}{1+|x|}
  < 1
  \quad
  \text{für alle $x \in \R$},
 \]
 also $f(\R) \subseteq (-1,1)$. Für die kompakte Teilmenge $[-1,1] \subseteq \R$ ist deshalb das Urbild $f^{-1}([-1,1]) = \R$ nicht kompakt.
\end{bem}


Wir wissen bereits, dass stetige, reellwertige Funktionen auf kompakten Mengen beschränkt sind. Man kann sich überlegen, dass die kompakte Teilmengen von $\R^n$ dadurch bereits charakterisiert sind.


\begin{lem}
 Es sei $K \subseteq \R^m$, so dass jede stetige Funktion $K \to \R$ beschränkt ist. Dann ist $K$ kompakt.
\end{lem}
\begin{proof}
 Wir zeigen, dass $K$ abgeschlossen und beschränkt ist.
 
 Angenommen $K$ wäre nicht abgeschlossen. Dann gebe es eine Folge $(x_n)$ auf $K$, die gegen ein $x \in \R^m$ mit $x \notin K$ konvergiert. Dann ist die Abbildung
 \[
  f \colon K \to \R, y \mapsto \frac{1}{\|x-y\|}
 \]
 wohldefiniert und stetig. Da $x_n \to x$ ist allerdings $f(x_n) = 1/\|x-x_n\| \to \infty$, also $f$ auf $K$ unbeschränkt. Das widerspricht den Annahmen an $K$. Also ist $K$ abgeschlossen unter Grenzwerten von Folgen und somit abgeschlossen.
 
 Dass $K$ beschränkt ist, ergibt sich direkt daraus, dass die Norm $\|\cdot\|$ stetig ist, und somit auf $K$ beschränkt.
\end{proof}













\end{document}
