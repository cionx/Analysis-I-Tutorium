\documentclass[a4paper,10pt]{article}
%\documentclass[a4paper,10pt]{scrartcl}

\usepackage{../mystyle}

\title{Kompakte Teilmengen von $\R^n$}
\author{Jendrik Stelzner}
\date{\today}

\begin{document}
\maketitle

\tableofcontents





\section{Stetige Funktionen auf abgeschlossenen Intervallen}


Zur Motivation des Kompaktheitbegriffes wollen wir zunächst die folgende wichtige Aussage beweisen:


\begin{lem}\label{lem: stetig auf Intervall}
 Es seien $a,b \in \R$ mit $a < b$ und $f \colon [a,b] \to \R$ stetig. Dann ist die Abbildung $f$ beschränkt und nimmt auf $[a,b]$ ihr Maximum und Minimum an, d.h. es gibt $x_{\text{max}}, x_{\text{min}} \in [a,b]$ mit
 \[
  f(x_{\text{max}}) = \sup_{y \in [a,b]} f(y)
  \quad
  \text{und}
  \quad
  f(x_{\text{min}}) = \inf_{y \in [a,b]} f(y).
 \]
\end{lem}


Für offene Intervalle oder halboffene Intervalle gilt diese Aussage nicht. Man betrachte etwa die Abbildung $(0,1] \to \R, x \mapsto 1/x$, oder gar $(0,1] \to \R, x \mapsto \sin(1/x)/x$.


Herzstück des Beweises ist die Beobachtung, dass auf $[a,b]$ jede Folge eine konvergente Teilfolge besitzt.


\begin{lem}\label{lem: Intervall ist kompakt}
 Es seien $a,b \in \R$ mit $a < b$ und $(x_n)_{n \in \N}$ eine Folge auf $[a,b]$. Dann besitzt $(x_n)$ eine konvergente Teilfolge $(x_{n_j})_{j \in \N}$, und für den Grenzwert $x \coloneqq \lim_{j \to \infty} x_{n_j}$ gilt $x \in [a,b]$.
\end{lem}
\begin{proof}
 Da $a \leq x_n \leq b$ für alle $n \in \N$ ist die Folge $(x_n)$ beschränkt und besitzt daher nach Bolzano-Weierstraß eine konvergente Teilfolge $(x_{n_j})_{j \in \N}$. Es sei $x \coloneqq \lim_{j \to \infty} x_{n_j}$. Da $a \leq x_{n_j} \leq b$ für alle $j \in \N$ ist auch $a \leq x \leq b$, also $x \in [a,b]$.
\end{proof}


\begin{proof}[Beweis von Lemma \ref{lem: stetig auf Intervall}]
 Wir zeigen zunächst, dass $f$ beschränkt ist: Angenommen, $f$ wäre nach oben unbeschränkt. Dann gibt es für alle $n \in \N$ ein $x_n \in [a,b]$ mit $f(x_n) \geq n$. Nach Lemma \ref{lem: Intervall ist kompakt} besitzt die Folge $(x_n)$ eine konvergente Folge $(x_{n_j})_{j \in \N}$. Da $f$ stetig ist, konvergiert auch die Folge $(f(x_{n_j}))_{j \in \N}$. Für alle $j \in \N$ ist aber $f(x_{n_j}) \geq n_j$, die Folge $f(x_{n_j})$ konvergiert also nicht. Dieser Widerspruch zeigt dass $f$ nach oben unbeschränkt seien muss. Analog ergibt sich, dass $f$ auch nach unten beschränkt ist. Also ist $f$ beschränkt.
 
 Es sei
 \[
  M \coloneqq \sup_{y \in [a,b]} f(y).
 \]
 Da $f$ nach oben beschränkt ist, ist $M < \infty$. Nach der $\varepsilon$-Charakterisierung des Supremums gibt es für alle $n \geq 1$ ein $x_n \in [a,b]$ mit
 \[
  M \geq f(x_n) \geq M - \frac{1}{n}.
 \]
 Nach Lemma \ref{lem: Intervall ist kompakt} besitzt die Folge $(x_n)$ eine konvergente Teilfolge $(x_{n_j})_{j \in \N}$. Es sei $x \coloneq \lim_{j \to \infty} x_{n_j}$. Da $f$ stetig ist, konvergiert auch die Folge $(f(x_{n_j}))$ und es gilt
 \[
  \lim_{j \to \infty} f(x_{n_j}) = f(x).
 \]
 Andererseits gilt für alle $j \in \N$
 \[
  M \geq f(x_{n_j}) \geq M - \frac{1}{n_j}.
 \]
 Also muss nach dem Sandwich-Lemma auch
 \[
  \lim_{j \to \infty} f(x_{n_j}) = M.
 \]
 Also ist $f(x) = M$. Das zeigt, dass $f$ auf $[a,b]$ sein Maximum annimmt. Analog ergibt sich, dass $f$ auf $[a,b]$ auch sein Minimum annimmt.
\end{proof}





\section{Folgenkompaktheit}


Zum Beweis von Lemma \ref{lem: stetig auf Intervall} haben wir Folgenstetigkeit und Lemma \ref{lem: Intervall ist kompakt} benötigt. Diese Beobachtung legt nahe, dass sich Lemma \ref{lem: stetig auf Intervall} auf beliebige Teilmengen $K \subseteq \R^n$ verallgemeinern lässt, für die eine zu Lemma \ref{lem: Intervall ist kompakt} analoge Aussage gilt.


\begin{defi}
 Es sei $X \subseteq \R^n$ und $(x_n)$ ein Folge auf $X$. Wir sagen \emph{$(x_n)$ konvergiert auf $X$}, falls die Folge $(x_n)$ konvergiert und $\lim_{n \to \infty} x_n \in X$.
\end{defi}


\begin{bsp}
 Die Folge $(1/n)_{n \geq 1}$ konvergiert auf $[0,1]$, nicht aber auch $(0,1)$.
\end{bsp}



\begin{defi}
 Eine Teilmenge $K \subseteq \R^n$ heißt \emph{kompakt}, falls jede Folge $(x_n)$ auf $K$ eine auf $K$ konvergente Teilfolge besitzt.
\end{defi}


\begin{bsp}
 Lemma \ref{lem: Intervall ist kompakt} zeigt, dass ein Intervall $[a,b]$ mit $a < b$ kompakt ist. Offene Intervalle hingegen sind niemals kompakt: Sind $a,b \in \R$ mit $a < b$, so ist $(a + (b-a)/(n+2))_{n \in \N}$ eine Folge auf $(a,b)$, die keine auf $(a,b)$ konvergente Teilfolge besitzt.
\end{bsp}


\begin{prop}
 Es sei $K \subseteq \R^n$ kompakt und $f \colon K \to \R$ stetig. Dann ist die Abbildung $f$ auf $K$ beschränkt und nimmt auf $K$ ihr Maximum und ihr Minimum an, d.h. es gibt $x_{\text{min}}, x_{\text{max}} \in K$ mit
 \[
  f(x_{\text{max}}) = \sup_{y \in K} f(y)
  \quad
  \text{und}
  \quad
  f(x_{\text{min}}) = \inf_{y \in K} f(y).
 \]
\end{prop}
\begin{proof}
 Nehme den Beweis von Lemma \ref{lem: stetig auf Intervall} und ersetze $[a,b]$ durch $K$ und die Verweise auf Lemma \ref{lem: Intervall ist kompakt} durch Kompaktheit.
\end{proof}


Da sich stetige Funktionen auf kompakten Mengen gutartig verhalten, sind kompakte Mengen von großer Bedeutung für die Analysis.





\section{Überdeckungskompaktheit}


Ein weiterer Kompaktheitsbegriff ist der der Überdeckungskompaktheit. Wie sich herausstellt, sind Überdeckungskompaktheit und Folgenkompaktheit für Teilmengen von $\R^n$ äquivalent.


\begin{defi}
 Es sei $\{U_i\}_{i \in I}$ eine Kollektion von Teilmengen $U_i \subseteq \R^n$ und $X \subseteq \R^n$.
 
 Die Kollektion $\{U_i\}_{i \in I}$ heißt \emph{Überdeckung von $X$}, falls $X \subseteq \bigcup_{i \in I} U_i$. Sie heißt \emph{endlich}, bzw.\ \emph{abzählbar}, falls $I$ endlich, bzw.\ abzählbar ist.
 
 Eine \emph{Teilüberdeckung} ist dann eine Teilkollektion $\{U_j\}_{j \in J}$, also $J \subseteq I$, so dass bereits $\{U_j\}_{j \in J}$ eine Überdeckung von $X$ ist.
 
 Außerdem heißt die Überdeckung $\{U_i\}_{i \in I}$ von $X$ \emph{offen}, falls die $U_i \subseteq \R^n$ alle offen sind.
\end{defi}


\begin{bem}
 Offenbar ist eine Teilüberdeckung einer offenen Überdeckung ebenfalls offen.
\end{bem}



\begin{bsp}
 \begin{itemize}
  \item
   Die Kollektion $\{B_n(0) \mid n \in \N, n \geq 1\}$ bildet eine abzählbare, offene Überdeckung von $\R^n$.
  \item
   Die offenen Intervalle $\{(a,b) \mid a,b \in \R, a < b\}$ bilden eine offene Überdeckung von $\R$.
  \item
   Allgemein bilden die $\varepsilon$-Bälle $\{B_\varepsilon(x) \mid x \in \R^n, \varepsilon > 0\}$ eine offene Überdeckung von $\R^n$.
  \item
   Die Würfel $\{[-K,K]^n \mid K \in \N, K \geq 1\}$ bilden eine abzählbar Überdeckung von $\R^n$, die nicht offen ist.
  \item
   Die $\varepsilon$-Bälle mit rationalen Radius um Mittelpunkte mit rationalen Koeffizienten $\{B_q(x) \mid x \in \Q^n, q \in \Q, q > 0\}$ bilden eine offene Überdeckung von $\R^n$.
  \item
   Die Intervalle $\{(n,n+1) \mid n \in \Z\}$ bilden eine offene Überdeckung von $\R \setminus \Z = \bigcup_{n \in \N} (n,n+1)$, aber nicht von $\R$.
  \item
   Die offenen Intervalle $\{ (-1/n, 1+1/n) \mid n \in \N, n \geq 1\}$ bilden eine offene Überdeckung des abgeschlossenen Einheitsintervalls $[0,1]$; diese Überdeckung besitzt eine endliche (sogar einelementige) Teilüberdeckung.
 \end{itemize}
\end{bsp}


\begin{defi}
 Eine Teilmenge $K \subseteq \R^n$ heißt \emph{überdeckungskompakt}, falls jede offene Überdeckung von $K$ eine endliche Teilüberdeckung besitzt.
\end{defi}


\begin{lem}
 Es sei $K \subseteq \R^n$ überdeckungskompakt. Dann gilt:
 \begin{enumerate}
  \item
   $K$ ist abgeschlossen.
  \item
   $K$ ist beschränkt.
 \end{enumerate}
\end{lem}
\begin{proof}
 \begin{enumerate}
  \item
   Wir zeigen, dass $K$ abgeschlossen ist, indem wir zeigen, dass $K^c$ offen ist. Es sei hierfür $x \in K^c$ beliebig aber fest. Wir wollen zeigen, dass es ein $\varepsilon > 0$ gibt, so dass $B_\varepsilon(x) \subseteq K$. Für jedes $y \in K$ ist $x \neq y$ (da $x \notin K$), also $\varepsilon_y \coloneqq \|x-y\|/2 > 0$. Für jedes $y \in K$ sind dann $B_{\varepsilon_y}(x)$ und $B_{\varepsilon_y}(y)$ disjunkt.
   
   (Wir würden nun gerne die Bälle $B_{\varepsilon_y}(x)$ schneiden, um eine, hoffentlich offene, Menge zu erhalten, die $x$ enthält, aber disjunkt zu $K$ ist. Das Problem ist, dass der möglicherweise unendliche Schnitt $\bigcap_{y \in K} B_{\varepsilon_y}(x)$ nicht mehr notwendigerweise offen ist. Mithilfe der Überdeckungskompaktheit können wir diesen unendlichen Schnitt aber durch einen endlichen ersetzen:)
   
   Die $\varepsilon$-Bälle $\{B_{\varepsilon_y}(y) \mid y \in K\}$ bilden eine offene Überdeckung von $K$. Da $K$ überdeckungskompakt ist, besitzt diese Überdeckung eine endliche Teilüberdeckung; es gibt also $y_1, \dotsc, y_s \in K$ mit
   \begin{equation}\label{eqn: K endliche Überdeckung}
    K \subseteq B_{\varepsilon_{y_1}}(y_1) \cup \dotsb \cup B_{\varepsilon_{y_s}}(y_s).
   \end{equation}
   Wir setzen $\varepsilon \coloneqq \min_{i=1,\dotsc,s} \varepsilon_{y_s} > 0$. Da $B_{\varepsilon_y}(x)$ und $B_{\varepsilon_y}(y)$ für alle $y \in K$ disjunkt sind, folgt aus \eqref{eqn: K endliche Überdeckung}, dass auch $B_\varepsilon(x)$ und $K$ disjunkt sind; es ist nämlich
   \begin{align*}
    B_\varepsilon(x) \cap K
    &\subseteq B_\varepsilon(x) \cap \bigcup_{i=1}^s B_{\varepsilon_{y_i}}(y_i)
    = \bigcup_{i=1}^s (B_\varepsilon(x) \cap B_{\varepsilon_{y_i}}(y_i)) \\
    &\subseteq \bigcup_{i=1}^s \underbrace{(B_{\varepsilon_{y_i}}(x) \cap B_{\varepsilon_{y_i}(y_i)})}_{= \emptyset}
    = \emptyset.
   \end{align*}
   Es ist also $B_\varepsilon(x) \subseteq K^c$. Aus der Beliebigkeit von $x \in K^c$ folgt, dass $K^c$ offen ist, und somit $K$ abgeschlossen.
  \item
   Die offenen Bälle $\{B_r(0) \mid r > 0\}$ bilden eine offene Überdeckung von $\R^n$, und damit insbesondere auch von $K$. Da $K$ überdeckungskompakt ist, besitzt diese Überdeckung eine endliche Teilüberdeckung. Es gibt also Radien $r_1, \dotsc, r_M > 0$ mit $K \subseteq B_{r_1}(0) \cup \dotsb \cup B_{r_M}(0)$. Für den Radius $R = \max_{i=1,\dotsc,M} r_i$ so ist $B_{r_i}(0) \subseteq B_R(0)$ für alle $1 \leq i \leq s$, und somit auch $K \subseteq B_R(0)$. Also ist $K$ beschränkt.
  \qedhere
 \end{enumerate}
\end{proof}




\begin{prop}
 Es sei $K \subseteq \R^n$. Dann sind äquivalent:
 \begin{enumerate}
  \item
   $K$ ist folgenkompakt.
  \item
   $K$ ist überdeckungskompakt.
 \end{enumerate}
\end{prop}
\begin{proof}
 Angenommen $K$ ist überdeckungskompakt, aber nicht folgenkompakt. Dann gibt es eine Folge $(x_n)$ auf $K$, die keine auf $K$ konvergente Teilfolge besitzt.
 
 \begin{beh}
  Für jedes $x \in K$ gibt es ein $\varepsilon_x > 0$, so dass $\|x - x_n\| \geq \varepsilon$ für fast alle $n \in \N$.
 \end{beh}
 \begin{proof}[Beweis der Behauptung]
  Ansonsten gebe es ein $x \in K$, so dass es für jedes $\varepsilon > 0$ unendlich viele $n \in N$ mit $\|x - x_n\| < \varepsilon$ gibt. Insbesondere gibt es dann ein $n_1 \in \N$ mit $\|x - x_{n_1}\| < 1$. Dann gibt es auch ein $n_2 \in \N$ mit $n_2 > n_1$ und $\|x - x_n\| < 1/2$. Rekursiv ergibt sich, dass es für alle $j \geq 1$ ein $n_j \in \N$ gibt, so dass $n_j > n_{j-1}$ und $\|x - x_{n_j}\| < 1/j$. Es ist dann $n_j$ eine Teilfolge mit $x_{n_j} \to x$. Dies steht im Widerspruch zur Annahme, dass $(x_n)$ keine auf $K$ konvergente Teilfolge besitzt.
 \end{proof}
 
 Die $\varepsilon$-Bälle $\{B_{\varepsilon_x}(x) \mid x \in K\}$ bilden offenbar eine offene Überdeckung von $K$. Da $K$ überdeckungskompakt ist, besitzt diese offene Überdeckung eine endliche Teilüberdeckung. Es gibt also $x_1, \dotsc, x_s \in K$, so dass
 \begin{equation}\label{eqn: K endlich überdeckt}
  K \subseteq B_{\varepsilon_{x_1}}(x_1) \cup \dotsb \cup B_{\varepsilon_{x_s}}(x_s).
 \end{equation}
 Da jeder der $\varepsilon$-Bälle $B_{\varepsilon_x}(x)$ nur endlich viele Folgeglieder enthält, enthält auch $K$ wegen \eqref{eqn: K endlich überdeckt} nur endlich viele Folgeglieder, was offenbar nicht seien kann. Das zeigt, dass aus Überdeckungskompaktheit Folgenkompaktheit folgt.
\end{proof}


Wegen der Äquivalenz von Folgen- und Überdeckungskompaktheit in $\R^n$ spricht man auch nur von Kompaktheit; eine Teilmenge $K \subseteq \R^n$ heißt also \emph{kompakt}, falls sie folgenkompakt, bzw.\ überdeckungskompakt ist.






\section{Der Satz von Heine-Borel}
Der Satz von Heine-Borel beantwortet die Frage, wie kompakte Teilmengen von $\R^n$ aussehen.


\begin{thrm}[Satz von Heine-Borel]
 Eine Teilmenge $K \subseteq \R^n$ ist genau dann kompakt, wenn $K$ abgeschlossen und beschränkt ist.
\end{thrm}
\begin{proof}
 Wir haben bereits gesehen, dass jede (überdeckungs)kompakte Teilmenge sowohl abgeschlossen als auch beschränkt ist.
 
 Es sei andererseits $K$ abgeschlossen und beschränkt. Wir wollen zeigen, dass $K$ (folgen)kompakt ist. Es sei hierfür $(x_n)_{n \in \N}$ eine Folge auf $K$. Da $K$ beschränkt ist, besitzt $K$ nach dem Satz von Bolzano-Weierstraß eine konvergente Teilfolge $(x_{n_j})_{j \in \N}$. Da $(x_{n_j})$ eine Folge auf $K$ ist, und $K$ abgeschlossen ist, ist auch $\lim_{j \to \infty} x_{n_j} \in K$. Also ist $K$ (folgen)kompakt.
\end{proof}


Der Satz von Heine-Borel ist sehr nützlich, um Teilmengen von $\R^n$ auf Kompaktheit zu untersuchen.


\begin{bsp}
 \begin{itemize}
  \item
   Die $n$-dimensionale Sphäre $S^n = \{x \in \R^{n+1} \mid \|x\| = 1\}$ ist offenbar beschränkt, und wir wissen bereits, dass sie auch abgeschlossen ist. Also ist $S^n$ kompakt.
  \item
   Wir wissen bereits, dass der Einheitswürfel $[0,1]^n \subseteq \R^n$ abgeschlossen ist. Er ist auch beschränkt, denn für alle $x = (x_1, \dotsc, x_n) \in [0,1]^n$ ist $\|x\| \leq \sum_{i=1}^n |x_i| \leq n$. Also ist der Einheitswürfel kompakt.
  \item
   Das offen Intervall $(0,1)$ ist zwar beschränkt, aber nicht abgeschlossen, und somit auch nicht kompakt.
  \item
   Der unendliche Zylinder $S^1 \times \R \subseteq \R^3$ ist zwar abgeschlossen, aber nicht beschränkt, und somit ebenfalls nicht kompakt. (Für jedes $C > 0$ ist $x \coloneqq (1,0,C) \in S^1 \times \R$ mit $\|x\| > C$.)
  \item
   
  \end{itemize}
\end{bsp}











\end{document}
