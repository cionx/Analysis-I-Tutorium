\documentclass[a4paper,10pt]{article}
%\documentclass[a4paper,10pt]{scrartcl}

\usepackage{../mystyle}

\setromanfont[Mapping=tex-text]{Linux Libertine O}
% \setsansfont[Mapping=tex-text]{DejaVu Sans}
% \setmonofont[Mapping=tex-text]{DejaVu Sans Mono}

\title{Der Satz von Rolle}
\author{Jendrik Stelzner}
\date{\today}

\begin{document}
\maketitle

Wir wollen hier den Satz von Rolle angeben und beweisen. Als Vorbereitung hierdrauf wollen wir das Verhalten von differenzierbaren Funktionen an Extremstellen betrachten.


\section{Extremstellen}


\begin{defi}
 Es sei $X \subseteq \R^n$, $f \colon X \to \R$ und $x \in X$. $f$ heißt \emph{lokal maximal an $x$}, falls es eine Umgebung $V$ von $x$ gibt, so dass
 \[
  f(x) \geq f(y) \quad \text{für alle $y \in V \cap X$}.
 \]
 $f$ heißt \emph{lokal minimal an $x$}, falls es eine Umgebung $V$ von $x$ gibt, so dass
 \[
  f(x) \leq f(y) \quad \text{für alle $y \in V \cap X$}.
 \]
 Ist $f$ lokal maximal oder lokal minimal an $x$, so heißt $f$ \emph{lokal extremal an $x$}; dann heißt $x$ eine \emph{lokale Extremstelle von $f$}.
\end{defi}


Im Eindimensionalen differenzierbare Funktionen lässt sich mithilfe der Ableitung eine notwendige Bedingungen für das Vorhandensein einer lokalen Extremstelle angeben.


\begin{lem}
 Es sei $V \subseteq \R$ eine Umgebung von $x$ und $f \colon V \to \R$ lokal extremal an $x$. Dann ist $f'(x) = 0$.
\end{lem}
\begin{proof}
 Wir zeigen, dass $f$ nicht extremal an $x$ ist, wenn $f'(x) \neq 0$. Wir beschränken uns dabei auf den Fall $f'(x) > 0$, der Fall $f'(x) < 0$ verläuft analog.
 
 Da $f'(x) > 0$ ist
 \[
  \lim_{h \to \infty} \frac{f(x+h)-f(x)}{h} > 0.
 \]
 Es gibt daher ein $\varepsilon > 0$, so dass
 \[
  \frac{f(x+h)-f(x)}{h} > 0 \quad \text{für alle $h \neq 0$ mit $|h| < \varepsilon$}.
 \]
 Für alle $0 < h < \varepsilon$ ist daher
 \[
  \frac{f(x+h)-f(x)}{h} > 0
  \Leftrightarrow f(x+h)-f(x) > 0
  \Leftrightarrow f(x+h) > f(x),
 \]
 und für alle $-\varepsilon < h < 0$ ist
 \[
  \frac{f(x+h)-f(x)}{h} > 0
  \Leftrightarrow f(x+h)-f(x) < 0
  \Leftrightarrow f(x+h) < f(x).
 \]
 Es ist daher
 \begin{gather*}
  f(x) < f(y) \quad \text{für alle $y \in (x,x+\varepsilon)$}
 \shortintertext{und}
  f(x) > f(y) \quad \text{für alle $y \in (x-\varepsilon,x)$}.
 \end{gather*}
 Es kann also $f$ an $x$ nicht extremal sein.
\end{proof}


Wir können nun den Satz von Rolle beweisen. (Wir benötigen auch noch, dass abgeschlossene Intervalle $[a,b]$ kompakt sind --- siehe hierzu die entsprechende Übersicht zu kompakten Teilmengen von $\R^n$.)


\begin{thrm}[Rolle]
 Es seien $a,b \in \R$ mit $a < b$. Die Funktion $f \colon [a,b] \to \R$ sei stetig und auf $(a,b)$ differenzierbar. Es sei $f(a) = f(b)$. Dann gibt es ein $\xi \in (a,b)$ mit $f'(\xi) = 0$. 
\end{thrm}
\begin{proof}
 Wir können o.B.d.A. davon ausgehen, dass $f(a) = f(b) = 0$.  Ist $f$ konstant, also $f = 0$, so ist die Aussage klar. Ansonsten gibt es ein $c \in [a,b]$ mit $f(c) \neq 0$; daher ist
 \[
  \sup_{x \in [a,b]} f(x) > 0 \quad \text{oder} \quad \inf_{x \in [a,b]} f(x) < 0.
 \]
 Da $[a,b]$ kompakt ist werden beide Werte angenommen; es gibt also $\xi \in [a,b]$, so dass
 \[
  f(\xi) = \max_{x \in [a,b]} f(x) > 0 \quad \text{oder} \quad f(\xi) = \min_{x \in [a,b]} f(x) < 0.
 \]
 Wir erhalten in beiden Fällen, dass $\xi$ eine Extremstelle von $f$ ist und dass $\xi \notin \{a,b\}$. Da $\xi \in (a,b)$ eine Extremstelle von $f$ ist muss $f'(\xi) = 0$.
\end{proof}



\end{document}
