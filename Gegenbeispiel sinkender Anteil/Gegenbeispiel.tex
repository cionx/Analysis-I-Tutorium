\documentclass[a4paper,10pt]{article}
%\documentclass[a4paper,10pt]{scrartcl}

\usepackage{../mystyle}

\setromanfont[Mapping=tex-text]{Linux Libertine O}
% \setsansfont[Mapping=tex-text]{DejaVu Sans}
% \setmonofont[Mapping=tex-text]{DejaVu Sans Mono}


\begin{document}

Wir wollen eine streng monoton steigende Folge $(a_n)_{n \geq 1}$ mit $a_n \in \N$, $a_n \geq 1$ für alle $n \in \N$ konstruieren, so dass zwar
\[
 \sum_{n=1}^\infty \frac{1}{a_n} = \infty,
\]
aber die Häufigkeit der genutzen Summanden gegen $0$ geht, d.h.
\begin{equation}\label{eqn: Grenzwert}
 \lim_{n \to \infty} \frac{|\{k \in \N \mid a_k \in \{1, \dotsc, n\}\}|}{n} = 0.
\end{equation}

Die Idee der Konstruktion ist die folgende: Wir beginnen mit $a_1 = 1$. Nun nehmen wir jede zweite Zahl hinzu, bis die entsprechende Summe mindestens $2$ beträgt. Dann nehmen wir jede dritte Zahl hinzu, bis die entsprechende Summe mindestens $3$ beträgt. Dies führen wir dann fort, wobei im $n$-ten Schritt der Konstruktion solange jeder $n$-te Summand hinzugefügt wird, bis die bisherige Summe mindestens $n$ beträgt.

Für die so konstruierte Reihe gilt nach Konstruktion $\sum_{k=1}^\infty a_k = \infty$. Andererseits wird für jedes $m \geq 1$ nach dem $2m$-ten Schritt nur noch höchstens jeder $2m$-te Summand hinzugefügt, so dass die Anzahl der genutzten Summanden kleiner als $1/m$ wird. Wegen der Beliebigkeit von $m \geq 1$ geht die Häufigkeit der genutzen Summanden dann gegen $0$.

Wir wollen diese Konstruktion nun formalisieren: Im ersten Schritt setzen wir $a_1 = 1$, $b^{(1)}_1 \coloneqq 1$, und $\nu(1) = 1$.

Im zweiten Schritt beginnen setzen wir zunächst $b^{(2)}_1 \coloneqq a_1 + m = 3$. Da $1/3 < 1$ setzen wir weiter $b^{(2)}_2 \coloneqq a_1 + 2m = 5$. Da auch noch $1/3 + 1/5 = 8/15 < 1$ wählen wir weiter $b^{(2)}_3 \coloneqq a_1 + 3m = 7$. Wir führen diesen Vorgang bis zu dem minimalen $\nu(2) \in \N$ fort, so dass $b^{(2)}_1 + \dotsb + b^{(2)}_{\nu(2)} \geq 1$. Ein solches $\nu(2)$ existiert, da $\sum_{m=1}^\infty 1/(a_1 + 2m) = \infty$. (Es ist $\nu(2) = 7$, wir müssen also bis zum Summanden $1/15$ gehen.) Wir defineren dann $a_2$, \dots, $a_{\nu(2)+1}$ durch
\[
 a_2 \coloneqq b^{(2)}_1, a_3 \coloneqq b^{(2)}_2, \dotsc, a_{\nu(n)+1} \coloneqq b^{(n)}_{\nu(n)}.
\]

Haben wir bereits $n-1$ solcher Schritte durchlaufen, und die Werte $a_1$, \dots, $a_k$ definiert, so wählen wir im $n$-ten Schritt das minimale $\nu(n) \in \N$ mit
\[
 \frac{1}{a_k + n} + \frac{1}{a_k + 2n} + \dotsb + \frac{1}{a_k + \nu(n) n} \geq 1,
\]
und setzen
\[
 a_{k + i} \coloneqq b^{(n)}_i \coloneqq a_k + in \quad \text{für alle $1 \leq i \leq \nu(n)$}.
\]
Ein solches $\nu(n)$ existiert, da $\sum_{m=1}^\infty 1/(a_k + mn) = \infty$.

Da wir in jedem der Schritte mindestens einen neuen Wert konstruieren, ist $a_n$ für alle $n \geq 1$ definiert. Aus der Konstruktion ist sofort ersichtlch, dass für alle $m \in \N$
\begin{align*}
 \sum_{k=1}^\infty a_k
 &= \lim_{N \to \infty} \sum_{k=1}^N a_k
 \geq \sum_{k=1}^{\nu(1)+\dotsb+\nu(m)} a_k \\
 &= \sum_{n=1}^m \sum_{k=\nu(1)+\dotsb+\nu(n-1)+1}^{\nu(1)+\dotsb+\nu(n)} a_k
 = \sum_{n=1}^m \underbrace{\sum_{i=1}^{\nu(n)} b^{(n)}_i}_{\geq 1}
 \geq m,
\end{align*}
also $\sum_{k=1}^\infty a_k = \infty$.

Wir wollen nun noch zeigen, dass auch \eqref{eqn: Grenzwert} erfüllt ist. Sei hierfür $m \geq 1$ beliebig aber fest. Wir bemerken, dass ab $b^{(2m)}_1$ nur noch höchstens jeder $2m$-te Summand hinzugefügt wird. Für alle $n \in \N$ haben wir
\begin{align*}
     &\, |\{k \in \N \mid a_k \in \{1, \dotsc, n\}\}| \\
 \leq&\, \left|\left\{k \in \N \,\middle|\, a_k \in \left\{1, \dotsc, b^{(2m)}_1\right\}\right\}\right|
         + \left|\left\{k \in \N \,\middle|\, a_k \in \left\{b^{(2m)}_1, \dotsc, n\right\}\right\}\right| \\
  \leq&\, b^{(2m)}_1 + \left|\left\{k \in \N \,\middle|\, a_k \in \left\{b^{(2m)}_1, \dotsc, n\right\}\right\}\right|.
\end{align*}
Da ab $b^{(2m)}_1$ nur noch höchstens jeder $2m$-te Summand hinzugefügt wird, ist dabei für alle $n \geq b^{(2m)}_1$
\begin{align*}
     &\, \left|\left\{k \in \N \,\middle|\, a_k \in \left\{b^{(2m)}_1, \dotsc, n\right\}\right\}\right| \\
 \leq&\, \left|\left\{b^{(2m)}_1 + l(2m) \,\middle|\, l \in \N\right\} \cap \left\{b^{(2m)}_1, \dotsc, n\right\}\right| \\
    =&\, \left|\{l(2m) \mid l \in \N\} \cap \left\{0, \dotsc, n-b^{(2m)}_1\right\}\right| \\
 \leq&\, \frac{n-b^{(2m)}_1+1}{2m}.
\end{align*}
Für alle $n \geq b^{(2m)}_1$ ist daher
\[
 |\{k \in \N \mid a_k \in \{1, \dotsc, n\}\}|
 \leq b^{(2m)}_1 + \frac{n-b^{(2m)}_1 + 1}{2m},
\]
und somit
\[
 \frac{|\{k \in \N \mid a_k \in \{1, \dotsc, n\}|}{n}
 \leq \frac{b^{(2m)}_1}{n} + \frac{1}{2m} \frac{n-b^{(2m)}_1 + 1}{n}.
\]
Da
\[
 \lim_{n \to \infty} \underbrace{\frac{b^{(2m)}_1}{n}}_{\to 0} + \frac{1}{2m} \underbrace{\frac{n-b^{(2m)}_1 + 1}{n}}_{\to 1}
 = \frac{1}{2m}
\]
ist damit auch
\begin{align*}
     &\, \limsup_{n \to \infty} \frac{|\{k \in \N \mid a_k \in \{1, \dotsc, n\}\}|}{n} \\
 \leq&\, \limsup_{n \to \infty} \frac{b^{(2m)}_1}{n} + \frac{1}{2m} \frac{n-b^{(2m)}_1 + 1}{n} \\
    =&\, \lim_{n \to \infty} \frac{b^{(2m)}_1}{n} + \frac{1}{2m} \frac{n-b^{(2m)}_1 + 1}{n} \\
    =&\, \frac{1}{2m}.
\end{align*}

Aus der Beliebigkeit von $m \geq 1$ folgt damit, dass
\begin{align*}
 0 
 &\leq \liminf_{n \to \infty} \frac{|\{k \in \N \mid a_k \in \{1, \dotsc, n\}\}|}{n} \\
 &\leq \limsup_{n \to \infty} \frac{|\{k \in \N \mid a_k \in \{1, \dotsc, n\}\}|}{n}
 = 0,
\end{align*}
und somit bereits
\[
 \lim_{n \to \infty} \frac{|\{k \in \N \mid a_k \in \{1, \dotsc, n\}\}|}{n} = 0.
\]











\end{document}
