\documentclass[a4paper,10pt]{article}
%\documentclass[a4paper,10pt]{scrartcl}

\usepackage{../mystyle}

\setromanfont[Mapping=tex-text]{Linux Libertine O}
% \setsansfont[Mapping=tex-text]{DejaVu Sans}
% \setmonofont[Mapping=tex-text]{DejaVu Sans Mono}


\begin{document}

Wir wollen eine streng monoton steigende Folge $(a_n)_{n \geq 1}$ mit $a_n \in \N$, $a_n \geq 1$ für alle $n \geq 1$ konstruieren, so dass
\[
 \sum_{n=1}^\infty \frac{1}{a_n} = \infty,
\]
aber die Häufigkeit der genutzten Summanden gegen $0$ geht, d.h.
\begin{equation}\label{eqn: Grenzwert}
 \lim_{n \to \infty} \frac{|\{k \in \N \mid a_k \in \{1, \dotsc, n\}\}|}{n} = 0.
\end{equation}

Die Idee der Konstruktion ist die folgende: Wir beginnen mit $a_1 = 1$. Nun nehmen wir jeden zweiten Summanden hinzu, bis die Summe der bisherigen Summanden mindestens $2$ beträgt. Wir haben also also $a_2 = 3$, $a_3 = 5$, $a_4 = 7$ usw.\ bis zu $a_M$, so dass $\sum_{k=1}^M 1/a_k \geq 2$. Dann nehmen wir jeden dritten Summanden hinzu, bis die bisherige Summe mindstens $3$ beträgt. Dies führen wir dann fort, wobei im $n$-ten Schritt der Konstruktion solange jeder $n$-te Summand hinzugefügt wird, bis die bisherige Summe mindestens $n$ beträgt.

Für die so konstruierte Reihe gilt nach Konstruktion $\sum_{k=1}^\infty a_k = \infty$. Andererseits wird für jedes $m \geq 1$ nach dem $m$-ten Schritt nur noch höchstens jeder $m$-te Summand hinzugefügt, weshalb die  Häufigkeit der genutzen Summanden gegen $0$ geht.

Wir wollen diese Konstruktion nun formalisieren: Im ersten Schritt setzen wir $a_1 \coloneqq 1$, $b^{(1)}_1 \coloneqq 1$, und $\nu(1) \coloneqq 1$.

Im zweiten Schritt beginnen setzen wir zunächst $b^{(2)}_1 \coloneqq a_1 + 2 = 3$. Da $1/3 < 1$ setzen wir weiter $b^{(2)}_2 \coloneqq a_1 + 4 = 5$. Da auch noch $1/3 + 1/5 = 8/15 < 1$ wählen wir weiter $b^{(2)}_3 \coloneqq a_1 + 6 = 7$. Wir führen diesen Vorgang bis zu dem minimalen $\nu(2) \in \N$ fort, so dass
\[
 \frac{1}{b^{(2)}_1} + \frac{1}{b^{(2)}_2} + \dotsb + \frac{1}{b^{(2)}_{\nu(2)}}
 = \frac{1}{a_1 + 2} + \frac{1}{a_1 + 4} + \dotsb +  \frac{1}{a_1 + 2 \nu(2)}
 \geq 1.
\]
Ein solches $\nu(2)$ existiert, da $\sum_{m=1}^\infty 1/(a_1 + 2m) = \infty$. (Es ergibt sich, dass $\nu(2) = 7$, wir müssen also bis zum Summanden $1/15$ gehen.) Wir defineren dann $a_2$, \dots, $a_{1+\nu(2)}$ durch
\[
 a_2 \coloneqq b^{(2)}_1, a_3 \coloneqq b^{(2)}_2, \dotsc, a_{1+\nu(n)} \coloneqq b^{(n)}_{\nu(n)}.
\]
Die Zahl $\nu(2)$ gibt also an, wie viele Summanden wir im zweiten Schritt hinzugefügt haben.

Haben wir bereits $n-1$ solcher Schritte durchlaufen, so haben wir bereits die Folgenwerte $a_1$, \dots, $a_{\nu(1)+\dotsb+\nu(n-1)}$ definiert; zur besseren Lesbarkeit schreiben wir \mbox{$k \coloneqq \nu(1)+\dotsb+\nu(n-1)$}. Wir wählen im $n$-ten Schritt dann die minimale Anzahl neuer Summanden \mbox{$\nu(n) \in \N$} mit
\[
 \frac{1}{a_k + n} + \frac{1}{a_k + 2n} + \dotsb + \frac{1}{a_k + \nu(n)n} \geq 1,
\]
und setzen
\[
 a_{k + i} \coloneqq b^{(n)}_i \coloneqq a_k + in \quad \text{für alle $1 \leq i \leq \nu(n)$}.
\]
Ein solches $\nu(n)$ existiert, da $\sum_{i=1}^\infty 1/(a_k + in) = \infty$.  Die Zahl $\nu(n)$ gibt also an, wie viele Summanden im $n$-ten Schritt hinzugefügt wurden.

Da wir in jedem der Schritte mindestens einen Summanden hinzunehmen, ist $a_n$ für alle $n \geq 1$ definiert; es ist auch klar, dass die Folge $(a_n)$ streng monoton steigend ist. Aus der Konstruktion wird auch sofort ersichtlch, dass für alle $m \in \N$
\begin{align*}
 \sum_{k=1}^\infty a_k
 &= \lim_{N \to \infty} \sum_{k=1}^N a_k
 \geq \sum_{k=1}^{\nu(1)+\dotsb+\nu(m)} a_k \\
 &= \sum_{n=1}^m \sum_{k=\nu(1)+\dotsb+\nu(n-1)+1}^{\nu(1)+\dotsb+\nu(n)} a_k
 = \sum_{n=1}^m \underbrace{\sum_{i=1}^{\nu(n)} b^{(n)}_i}_{\geq 1}
 \geq m,
\end{align*}
also $\sum_{k=1}^\infty a_k = \infty$.

Wir wollen nun noch zeigen, dass auch \eqref{eqn: Grenzwert} erfüllt ist. Sei hierfür $m \geq 1$ beliebig aber fest. Wir bemerken, dass ab $c \coloneqq a_{\nu(1)+\dotsb+\nu(m-1)+1} = b^{(m)}_1$ nur noch höchstens jeder $m$-te Summand hinzugefügt wird. Für alle $n \in \N$ haben wir
\begin{align*}
     &\, |\{k \in \N \mid a_k \in \{1, \dotsc, n\}\}| \\
 \leq&\, \left|\left\{k \in \N \,\middle|\, a_k \in \left\{1, \dotsc, c\right\}\right\}\right|
         + \left|\left\{k \in \N \,\middle|\, a_k \in \left\{c, \dotsc, n\right\}\right\}\right| \\
  \leq&\, c + \left|\left\{k \in \N \,\middle|\, a_k \in \left\{c, \dotsc, n\right\}\right\}\right|.
\end{align*}
Da ab $c$ nur noch höchstens jeder $m$-te Summand hinzugefügt wird, ist dabei für alle $n \geq c$
\begin{align*}
     &\, \left|\left\{k \in \N \,\middle|\, a_k \in \left\{c, \dotsc, n\right\}\right\}\right| \\
 \leq&\, \left|\left\{c + \ell m \,\middle|\, \ell \in \N\right\} \cap \left\{c, \dotsc, n\right\}\right| \\
    =&\, \left|\{\ell m \mid \ell \in \N\} \cap \left\{0, \dotsc, n-c\right\}\right| \\
 \leq&\, \frac{n-c+1}{m}.
\end{align*}
Für alle $n \geq c$ ist daher
\[
 |\{k \in \N \mid a_k \in \{1, \dotsc, n\}\}|
 \leq c + \frac{n-c + 1}{m},
\]
und somit
\[
 \frac{|\{k \in \N \mid a_k \in \{1, \dotsc, n\}|}{n}
 \leq \frac{c}{n} + \frac{1}{m} \frac{n-c + 1}{n}.
\]
Da
\[
 \lim_{n \to \infty}\frac{c}{n} + \frac{1}{m} \frac{n-c + 1}{n}
 = \frac{1}{m}
\]
ist damit auch
\begin{align*}
     &\, \limsup_{n \to \infty} \frac{|\{k \in \N \mid a_k \in \{1, \dotsc, n\}\}|}{n} \\
 \leq&\, \limsup_{n \to \infty} \frac{c}{n} + \frac{1}{m} \frac{n-c + 1}{n} \\
    =&\, \lim_{n \to \infty} \frac{c}{n} + \frac{1}{m} \frac{n-c + 1}{n} \\
    =&\, \frac{1}{m}.
\end{align*}
Aus der Beliebigkeit von $m \geq 1$ folgt damit, dass
\begin{align*}
 0 
 &\leq \liminf_{n \to \infty} \frac{|\{k \in \N \mid a_k \in \{1, \dotsc, n\}\}|}{n} \\
 &\leq \limsup_{n \to \infty} \frac{|\{k \in \N \mid a_k \in \{1, \dotsc, n\}\}|}{n}
 \leq 0,
\end{align*}
und somit bereits
\[
 \lim_{n \to \infty} \frac{|\{k \in \N \mid a_k \in \{1, \dotsc, n\}\}|}{n} = 0.
\]











\end{document}
