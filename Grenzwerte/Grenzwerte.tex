\documentclass[a4paper,10pt]{article}
%\documentclass[a4paper,10pt]{scrartcl}

\usepackage{../mystyle}

\setromanfont[Mapping=tex-text]{Linux Libertine O}
% \setsansfont[Mapping=tex-text]{DejaVu Sans}
% \setmonofont[Mapping=tex-text]{DejaVu Sans Mono}

\title{Grenzwerte und Konvergenz}
\author{Jendrik Stelzner}
\date{\today}

\begin{document}
\maketitle

\tableofcontents





\section{Supremum und Infimum}



\subsection{Supremum und Infimum auf $\R$}


\begin{defi}
 Eine Teilmenge $A \subseteq \R$ heißt \emph{nach oben beschränkt}, falls es ein $s \in \R$ gibt, so dass $a \leq s$ für alle $a \in A$; $s$ heißt dann eine obere Schranke von $A$. $A$ heißt \emph{nach oben unbeschränkt}, falls $A$ nicht nach oben beschränkt ist.
 
 $A$ heißt \emph{nach unten beschränkt}, falls es ein $t \in \R$ gibt, so dass $t \leq a$ für alle $a \in A$; $t$ heißt dann untere Schranke von $A$. $A$ heißt \emph{nach unten unbeschränkt}, falls $A$ nicht nach unten beschränkt ist.
 
 $A$ heißt \emph{beschränkt}, falls $A$ sowohl nach oben als auch nach unten beschränkt ist.
\end{defi}


\begin{bsp}
 \begin{enumerate}
  \item
   $N \subseteq \R$ ist nach unten beschränkt und nach oben unbeschränkt.
  \item
   $\Z \subseteq \R$ ist nach unten und oben unbeschränkt.
  \item
   Das Einheitsintervall $[0,1] \subseteq \R$ ist beschränkt.
  \item
   Jede endliche Menge $X \subseteq \R$ ist beschränkt.
  \item
   Ist $B \subseteq \R$ nach unten, bzw.\ oben beschränkt, so ist $A \subseteq \R$ ebenfalls nach unten, bzw.\ oben beschränkt. Ist $B$ nach unten, bzw.\ oben unbeschränkt, so ist auch $A$ nach unten, bzw.\ oben unbeschränkt.
  \item
   Eine Teilmenge $A \subseteq \R$ ist genau dann nach unten, bzw.\ oben beschränkt, falls die Teilmenge
   \[
    -A = \{-a \mid a \in A\} \subseteq \R
   \]
   nach oben, bzw.\ unten beschränkt ist.
 \end{enumerate}
\end{bsp}


\begin{bem}
 Eine Teilmenge $A \subseteq \R$ ist genau dann beschränkt, falls es ein $R > 0$ gibt, so dass $|a| \leq R$ für alle $a \in A$.
\end{bem}
\begin{proof}
 Gibt es ein solches $R$, so ist $-R \leq a \leq R$ für alle $a \in A$ und somit ist $A$ sowohl nach oben als auch nach unten beschränkt, also beschränkt.
 
 Ist $A$ beschränkt, so gibt $s, t \in \R$, so dass $t \leq a \leq s$ für alle $a \in A$. Für $R \coloneqq \max\{|s|, |t|\}$ ist dann $|a| \leq R$ für alle $a \in A$.
\end{proof}


\begin{defi}
 Es sei $A \subseteq \R$ eine Teilmenge und $s \in \R$. $s$ heißt \emph{Supremum von $A$}, falls $s$ eine kleinste obere Schranke von $A$ ist, d.h. falls
 \begin{enumerate}
  \item
   $s$ ist eine obere Schranke von $A$, und
  \item
   für jede obere Schranke $t$ von $A$ ist $s \leq t$.
 \end{enumerate}
 $s$ heißt \emph{Infimum von $A$}, falls $s$ eine größte untere Schranke von $A$ ist, d.h. falls
 \begin{enumerate}
  \item
   $s$ ist eine untere Schranke von $A$, und
  \item
   für jede untere Schranke $t$ von $A$ ist $t \leq s$.
 \end{enumerate}
\end{defi}


\begin{bem}
 Existiert ein Supremum, bzw.\ Infimum einer Teilmenge $A \subseteq \R$, so ist dieses eindeutig, d.h. sind $s,t \in \R$ Suprema, bzw.\ Infima von $A$, so ist $s = t$. Falls das Supremum von $A$ existiert, so schreiben wir hierfür $\sup A$. Für das Infimum schreiben wir $\inf A$, sofern dieses existiert.
\end{bem}
\begin{proof}
 Es seien $s$ und $t$ Suprema von $A$. Da $s$ ein Supremum von $A$ und $t$ eine obere Schranke von $A$ ist, ist $s \leq t$. Durch Vertauschen der Rollen ergibt sich, dass auch $t \leq s$. Daher ist $s = t$.
 
 Sind $s$ und $t$ Infima von $A$, so ergibt sich $s = t$ analog.
\end{proof}


Ist $A \subseteq \R$ nicht nach oben beschränkt, so besitzt $A$ keine obere Schranke und somit auch kein Supremum. Wie sich herausstellt, ist diese Bedingung für nicht-leere Teilmengen von $\R$ schon hinreichend:


\begin{prop}
 Es sei $A \subseteq \R$ eine nicht-leere Teilmenge.
 \begin{enumerate}
  \item
   Ist $A$ nach oben beschränkt, so existiert $\sup A$.
  \item
   Ist $A$ nach unten beschränkt, so existiert $\inf A$.
 \end{enumerate}
\end{prop}


Wir wollen diese Proposition hier nicht beweisen, da der Beweis von der konkreten Konstruktion der reellen Zahlen abhängt.


\begin{lem}[Charakterisierung von $\sup$ und $\inf$]
 Es sei $A \subseteq \R$ eine nach oben beschränkte Teilmenge.
 \begin{enumerate}
  \item
   Für eine obere Schranke $s$ von $A$ ist $s = \sup A$ genau dann, wenn es für alle $\varepsilon > 0$ ein $a \in A$ mit $s-\varepsilon < a \leq s$ gibt.
  \item
   Für eine untere Schranke $s$ von $A$ ist $s = \inf A$ genau dann, wenn es für alle $\varepsilon > 0$ ein $a \in A$ mit $s \leq a < s + \varepsilon$ gibt.
 \end{enumerate}
\end{lem}
\begin{proof}
 \begin{enumerate}
  \item
   Angenommen es ist $s = \sup A$. Es sei $\varepsilon > 0$ beliebig aber fest. Da $s$ die kleinste obere Schranke von $A$ ist, ist $s-\varepsilon$ keine obere Schranke von $A$. Es gibt also $a \in A$ mit $s-\varepsilon < a$. Da $s$ eine obere Schranke von $A$ ist, ist auch $a \leq s$. Zusammen ist daher $s-\varepsilon < a \leq s$. Aus der Beliebigkeit von $\varepsilon > 0$ folgt die Implikation.
   
   Angenommen es ist $s \neq \sup A$. Dann gibt es eine kleinere obere Schranke $t < s$ von $A$. Für $\varepsilon \coloneqq (s-t)/2 > 0$ ist dann für alle $a \in A$
   \[
    a \leq t < t + \varepsilon = s - \varepsilon.
   \]
   Es gibt dann kein $a \in A$ mit $s-\varepsilon < a \leq s$.
  \item
   Der Beweis läuft analog zum vorherigen.
 \end{enumerate}
\end{proof}


Auch für Teilmengen von $\R$, die nicht beschränkt sind, lassen sich Supremum und Infimum definieren.


\begin{defi}\label{defi: Uneigentliche Konvergenz}
 Es sei $A \subseteq \R$ nach oben beschränkt. Ist $A$ nach oben unbeschränkt, so setzen wir $\sup A \coloneq \infty$. Ist $A$ nach unten unbeschränkt, so setzen wir $\inf A \coloneqq -\infty$.
 
 Für die leere Menge setzen wir $\sup \emptyset = -\infty$ und $\inf \emptyset = \infty$.
\end{defi}



\subsection{Verallgemeinerung}
Die Ideen des Supremums und Infimums lassen sich leicht auf geordnete Mengen verallgemeinern. Dabei verstehen wir unter einer geordneten Mengen eine Menge $X$ zusammen mit einer Ordnungsrelation $\leq$, die die folgenden Bedingungen erfüllt:
\begin{enumerate}
 \item
  Für alle $x \in X$ ist $x \leq x$. (Reflexivität)
 \item
  Für $x,y \in X$ mit $x \leq y$ und $y \leq x$ ist bereits $x = y$. (Anti-Symmetrie)
 \item
  Für $x,y,z \in X$ mit $x \leq y$ und $y \leq z$ ist auch $x \leq z$.
\end{enumerate}

Auf einer geordneten Menge lassen sich dann Beschränktheit von Teilmengen, Schranken für Teilmengen und damit auch die Begriffe des Supremums und Infimums verallgemeinern, indem man in den entsprechenden obigen Definitionen $\R$ durch $X$ ersetzt.

Mithilfe dieser Verallgemeinerung lässt sich auch Definition \ref{defi: Uneigentliche Konvergenz} anders verstehen: Die Menge $X \coloneqq \R \cup \{-\infty, \infty\}$ ist eine geordnete Menge, indem man die Ordnung von $\R$ um
\begin{itemize}
 \item
  $x \leq \infty$ für alle $x \in \R$,
 \item
  $-\infty \leq x$ für alle $x \in \R$,
 \item
  $-\infty \leq \infty$
\end{itemize}
ergänzt. Supremum und Infimum nach \ref{defi: Uneigentliche Konvergenz} entsprechen dann dem Supremum und Infimum auf $X$.





\section{Grenzwerte}
Im Folgenden handelt es sich, sofern nicht anders angegeben, bei allen Folgen um Folgen auf $\R$.


\begin{defi}
 Eine Folge $(x_n)_{n \in \N}$ heißt nach oben beschränkt, bzw.\ nach unten beschränkt, bzw.\ beschränkt, falls die Punktmenge $\{x_n \mid n \in \N\}$ nach oben beschränkt, bzw.\ nach unten beschränkt, bzw.\ beschränkt ist.
\end{defi}


\begin{defi}
 Es sei $(x_n)_{n \in \N}$ eine Folge auf $\R$ und $x \in \R$. \emph{$(x_n)$ konvergiert gegen $x$}, falls es für alle $\varepsilon > 0$ ein $N \in \N$ gibt, so dass $|x_n-x| < \varepsilon$ für alle $n \geq N$. Wir schreiben dann $\lim_{n \to \infty} x_n = x$, bzw.\ $x_n \to x$ für $n \to \infty$.
 
 Wir sagen, dass die Folge $(x_n)$ \emph{konvergent} ist, falls es ein $x \in \R$ gibt, so dass $x_n \to x$ für $n \to \infty$. $x$ heißt dann der \emph{Grenzwert} der Folge $(x_n)$.
\end{defi}


\begin{prop}
 Der Grenzwert einer konvergenten Folge ist eindeutig, d.h. ist $(x_n)_{n \in \N}$ eine konvergente Folge und sind $x, x' \in \R$ mit $x_n \to x$ und $x_n \to x'$ für $n \to \infty$, so ist $x = x'$.
\end{prop}
\begin{proof}
 Angenommen es ist $x \neq x'$. Dann ist $\varepsilon \coloneqq |x-x'| > 0$. Da $x_n \to x$ gibt es $n_1 \in \N$, so dass $|x_n - x| < \varepsilon/3$ für alle $n \geq n_1$. Da $x_n \to x'$ gibt es $n_2 \in \N$, so dass $|x_n - x'| < \varepsilon/3$ für alle $n \geq n_2$. Für $N \coloneqq \max \{n_1, n_2\}$ ist dann für alle $n \geq N$ sowohl $|x_n - x| < \varepsilon/3$ als auch $|x_n - x'| < \varepsilon/3$. Daher ist nach der Dreiecksungleichung
 \[
  |x - x'|
  = |(x_N - x') - (x_N - x)|
  \leq |x_N - x'| + |x_N - x|
  < \frac{\varepsilon}{3} + \frac{\varepsilon}{3}
  = \frac{2}{3} \varepsilon,
 \]
 im Widerspruch zu $\varepsilon = |x - x'| > 0$.
\end{proof}


\begin{bsp}
 \begin{enumerate}
  \item
   Wir haben $\lim_{n \to \infty} 1/n = 0$: Es sei $\varepsilon > 0$ beliebig aber fest. Nach Archimedes gibt es ein $N \in \N$, $N \geq 1$ mit $N > 1/\varepsilon$. Für alle $n \geq N$ ist dann $\varepsilon > 1/n$ und damit
   \[
    0 \leq \frac{1}{n} \leq \varepsilon \quad \text{für alle $n \geq N$}.,
   \]
   also $|0 - 1/n| < \varepsilon$. Dass $\lim_{n \to \infty} 1/n = 0$ folgt daher aus der Beliebigkeit von $\varepsilon > 0$.
  \item
   Die konstante Folge $(c)_{n \in \N}$ konvergiert gegen $c$.
  \item
   Es ist $\lim_{n \to \infty} n^{1/n} = 1$. Zum Beweis hiervon sei $\varepsilon > 0$ beliebig aber fest. Es sei $N \in \N$ mit $N \geq 2$ und $N \geq 1 + 2/\varepsilon^2$. Für alle $n \geq N$ ist dann $(n-1)\varepsilon^2/2 \geq 1$ und somit
   \begin{align*}
    (1+\varepsilon)^n
    &= \sum_{k=0}^n \binom{n}{k} \varepsilon^k
    = 1 + n \varepsilon + \frac{n(n-1)}{2}\varepsilon^2 + \dotso \\
    &> 1 + \frac{n(n-1)}{2}\varepsilon^2
    = 1 + n \frac{(n-1)\varepsilon^2}{2}
    \geq 1 + n
    > n.
   \end{align*}
  Daher ist für alle $n \geq N$
  \[
   1 \leq n^{1/n} < 1 + \varepsilon
  \]
  und somit $|1 - n^{1/n}| < \varepsilon$. Aus der Beliebigkeit von $\varepsilon > 0$ folgt, dass $n^{1/n} \to 1$ für $n \to \infty$.
 \item
  Die Folgen $((-1)^n)_{n \in \N}$ und $(n)_{n \in \N}$ konvergieren nicht.
 \end{enumerate}
\end{bsp}





\subsection{Konvergenz und Ordnung}


\begin{lem}
 Sei $(x_n)_{n \in \N}$ eine konvergente Folge. Dann ist $(x_n)$ beschränkt, d.h. es gibt $R > 0$, so dass $|x_n| < R$ für alle $n \in \N$.
\end{lem}
\begin{proof}
 Es sei $x \coloneqq \lim_{n \to \infty} x_n$. Da $x_n \to x$ für $n \to \infty$ gibt es ein $N \in \N$ mit $|x_n - x| < 1$ für alle $n \geq N$. Es ist dann nach der Dreiecksungleichung für alle $n \geq N$
 \[
  |x_n|
  = |x + x_n - x|
  \leq |x| + |x_n - x|
  < |x| + 1.
 \]
 Für
 \[
  R \coloneqq \max \{|x_0|, |x_1|, \dotsc, |x_{N-1}|, |x| + 1\}
 \]
 ist daher $|x_n| \leq R$ für alle $n \in \N$.
\end{proof}


\begin{lem}
 Es seien $(a_n)_{n \in \N}$ und $(b_n)_{n \in \N}$ konvergente Folgen mit $a_n \leq b_n$ für alle $n \in \N$. Dann ist auch $\lim_{n \to \infty} a_n \leq \lim_{n \to \infty} b_n$.
\end{lem}
\begin{proof}
 Es sei $a \coloneqq \lim_{n \to \infty} a_n$ und $b \coloneqq \lim_{n \to \infty} b_n$. Angenommen, es ist $b < a$. Dann ist $\varepsilon \coloneqq |a-b| = a-b > 0$. Da $a_n \to a$ für $n \to \infty$ gibt es $n_1 \in \N$ mit $|a_n - a| < \varepsilon/3$ für alle $n \geq n_1$. Da $b_n \to b$ für $n \to \infty$ gibt es $n_2 \in \N$ mit $|b_n - b| < \varepsilon/3$ für alle $n \geq n_2$. Für $N = \max \{n_1, n_2\}$ ist dann
 \[
  b_N < b + \frac{\varepsilon}{3} < b + \frac{2\varepsilon}{3} = a - \frac{\varepsilon}{3} < a_N,
 \]
 im Widerspruch zu $a_n \leq b_n$.
\end{proof}


\begin{lem}[Sandwich-Lemma]
 Es seien $(a_n)_{n \in \N}$, $(b_n)_{n \in \N}$ und $(c_n)_{n \in \N}$ Folgen mit $a_n \leq b_n \leq c_n$ für alle $n \in \N$. Konvergieren die beiden äußeren Folgen $(a_n)$ und $(c_n)$ und gilt $\lim_{n \to \infty} a_n = \lim_{n \to \infty} c_n$, so konvergiert auch $(b_n)$ und es gilt
 \[
  \lim_{n \to \infty} a_n = \lim_{n \to \infty} b_n = \lim_{n \to \infty} c_n.
 \]
\end{lem}
\begin{proof}
 Wir setzen $x \coloneqq \lim_{n \to \infty} a_n = \lim_{n \to \infty} c_n$. Es sei $\varepsilon > 0$ beliebig aber fest. Da $a_n \to x$ für $n \to \infty$ gibt es $n_1 \in \N$ mit $|a_n - x| < \varepsilon$ für alle $n \geq n_1$. Da $c_n \to x$ für $n \to \infty$ gibt es $n_2 \in \N$ mit $|c_n - x| < \varepsilon$ für alle $n \geq n_2$. Für $N \coloneqq \max \{n_1, n_2\}$ ist daher für alle $n \geq N$
 \[
  x-\varepsilon < a_n \leq b_n \leq c_n < x+\varepsilon,
 \]
 also $|x - b_n| < \varepsilon$ für alle $n \geq N$. Wegen der Beliebigkeit von $\varepsilon > 0$ zeigt dies, dass $\lim_{n \to \infty} b_n = x$.
\end{proof}


\begin{prop}[Bolzano-Weierstraß]
 Es sei $(x_n)_{n \in \N}$ eine beschränkte monotone Folge. Dann konvergiert $(x_n)$.
\end{prop}
\begin{proof}
 Wir betrachten den Fall, dass $(x_n)$ monoton steigend ist. Da $(x_n)$ nach oben beschränkt ist existiert $x \coloneqq \sup_{n \in \N} x_n$. Wir wollen zeigen, dass $x_n \to x$ für $n \to \infty$.
 
 Es sei $\varepsilon > 0$ beliebig aber fest. Aus der Charakterisierung des Supremums wissen wir, dass es $N \in \N$ mit $x - \varepsilon < x_N \leq x$. Wegen der Monotonie von $(x_n)$ erhalten wir, dass $x - \varepsilon < x_n$ für alle $n \in \N$, und dass $x_n \leq x$ für alle $n \geq N$ folgt direkt aus der Definition von $x$. Es ist also
 \[
  x - \varepsilon < x_n \leq x \quad \text{für alle $n \geq N$},
 \]
 und somit insbesondere $|x - x_n| < \varepsilon$ für alle $n \geq N$. Wegen der Beliebigkeit von $\varepsilon > 0$ folgt, dass $\lim_{n \to \infty} x_n = x$.
 
 Der Fall, dass $(x_n)$ monoton fallend ist, läuft analog.
\end{proof}





\subsection{Konvergenz und Rechenoperationen}


\begin{prop}
 Es seien $(a_n)_{n \in \N}$ und $(b_n)_{n \in \N}$ konvergente Folgen mit Grenzwerten $a \coloneqq \lim_{n \to \infty} a_n$ und $b \coloneqq \lim_{n \to \infty} b_n$.
 \begin{enumerate}
  \item
   Die Folge $(a_n + b_n)_{n \in \N}$ ist ebenfalls konvergent und
   \[
    \lim_{n \to \infty} (a_n + b_n)
    = a + b
    = \left(\lim_{n \to \infty} a_n\right) + \left(\lim_{n \to \infty} b_n\right).
   \]
  \item
   Für alle $c \in \R$ ist die Folge $(c a_n)_{n \in \N}$ konvergent und
   \[
    \lim_{n \to \infty} (c a_n)
    = c a
    = c \lim_{n \to \infty} a_n.
   \]
  \item
   Die Folge $(a_n \cdot b_n)_{n \in \N}$ ist ebenfalls konvergiert und
   \[
    \lim_{n \to \infty} (a_n \cdot b_n)
    = a \cdot b
    = \left(\lim_{n \to \infty} a_n\right) \cdot \left(\lim_{n \to \infty} b_n\right).
   \]
  \item
   Ist $a_n \neq 0$ für alle $n \in \N$ und $a \neq 0$, so konvergiert auch die Folge $(1/a_n)_{n \in \N}$ und es ist
   \[
    \lim_{n \to \infty} \frac{1}{a_n}
    = \frac{1}{a}
    = \frac{1}{\lim_{n \to \infty} a_n}
   \]
  \item
   Ist $b_n \neq 0$ für alle $n \in \N$ und $b \neq 0$, so ist auch $(a_n/b_n)_{n \in \N}$ konvergent und
   \[
    \lim_{n \to \infty} \frac{a_n}{b_n}
    = \frac{a}{b}
    = \frac{\lim_{n \to \infty} a_n}{\lim_{n \to \infty} b_n}.
   \]
 \end{enumerate}
\end{prop}
\begin{proof}
 \begin{enumerate}
  \item
   Es sei $\varepsilon > 0$ beliebig aber fest. Da $a_n \to a$ für $n \to \infty$ gibt es $n_1 \in \N$ mit $|a - a_n| < \varepsilon/2$ für alle $n \geq n_1$. Da $b_n \to b$ für $n \to \infty$ gibt es $n_2 \in \N$ mit $|b - b_n| < \varepsilon/2$ für alle $n \geq n_2$. Für $N \coloneqq \max \{n_1, n_2\}$ ist dann für alle $n \geq N$
   \begin{align*}
    |(a + b) - (a_n + b_n)|
    &= |(a - a_n) + (b - b_n)| \\
    &\leq |a - a_n| + |b - b_n|
    < \frac{\varepsilon}{2} + \frac{\varepsilon}{2}
    = \varepsilon.
   \end{align*}
   Aus der Beliebigkeit von $\varepsilon > 0$ folgt, dass $\lim_{n \to \infty} (a_n + b_n) = a + b$.
  \item
   Für $c = 0$ ist die Aussage klar. Ansonsten sei $\varepsilon > 0$ beliebig aber fest. Da $a_n \to a$ für $n \to \infty$ gibt es $N \in \N$ mit $|a - a_n| < \varepsilon/|c|$ für alle $n \geq N$. Es ist dann für alle $n \geq N$
   \[
    |ca - ca_n|
    = |c| |a - a_n|
    < |c| \frac{\varepsilon}{|c|}
    = \varepsilon.
   \]
   Aus der Beliebigkeit von $\varepsilon > 0$ folgt damit, dass $\lim_{n \to \infty} (ca_n) = ca$.
  \item
   Es sei $\varepsilon > 0$ beliebig aber fest. Da $(a_n)$ konvergiert, ist $(a_n)$ insbesondere beschränkt. Es gibt also ein $c > 0$ mit $|a_n| < c$ für alle $n \in \N$. Wir bemerken zunächst, dass für alle $n \in \N$
   \begin{equation}\label{eqn: Ungleichung Produkt}
    \begin{aligned}
     |a b - a_n b_n|
     &= |a b - a_n b + a_n b - a_n b_n| \\
     &\leq |a b - a_n b| + |a_n b - a_n b_n| \\
     &= |a - a_n| |b| + |a_n| |b - b_n| \\
     &\leq |b| |a - a_n| + c |b - b_n|.
    \end{aligned}
   \end{equation}
   Wir unterscheiden nun zwischen zwei Fällen: Ist $b = 0$, so ist nach \eqref{eqn: Ungleichung Produkt}
   \[
    |a b - a_n b_n| \leq c |b - b_n|
   \]
   Da $b_n \to b$ für $n \to \infty$ gibt es ein $N \in \N$ mit $|b - b_n| < \varepsilon/c$ für alle $n \geq N$. Es ist daher für alle $n \geq \N$
   \[
    |a b - a_n b_n| \leq c |b - b_n| < c \frac{\varepsilon}{c} = \varepsilon.
   \]
   Ist andererseits $b \neq 0$, und somit auch $|b| \neq 0$, so gibt es wegen $a_n \to a$ für $n \to \infty$ ein $n_1 \in \N$ mit $|a - a_n| < \varepsilon/(2|b|)$ für $n \geq n_1$, und wegen $b_n \to b$ für $n \to \infty$ ein $n_2 \in \N$ mit $|b - b_n| < \varepsilon/(2c)$ für alle $n \geq n_2$. Für $N \coloneqq \max \{n_1, n_2\}$ ist dann für alle $n \geq N$
   \[
    |a b - a_n b_n|
    \leq |b| |a - a_n| + c |b - b_n|
    < |b| \frac{\varepsilon}{2|b|} + c \frac{\varepsilon}{2c}
    = \varepsilon.
   \]
   In beiden Fällen folgt aus der Beliebigkeit von $\varepsilon > 0$, dass $(a_n b_n)$ konvergiert und $\lim_{n \to \infty} (a_n \cdot b_n) = a \cdot b$.
  \item
   Da $b_n \to b$ für $n \to \infty$ und $|b| > 0$ gibt es $n_1 \in \N$ mit $|b - b_n| < |b|/2$ für alle $n \geq n_1$. Da für alle $n \in \N$
   \[
    |b| = |b - b_n + b_n| \leq |b - b_n| + |b_n| \Rightarrow |b_n| \geq |b| - |b - b_n|
   \]
   ist dann für alle $n \geq n_1$
   \[
    |b_n|
    \geq |b| - |b - b_n|
    \geq |b| - \frac{|b|}{2}
    = \frac{|b|}{2},
   \]
   und damit auch
   \begin{equation}\label{eqn: Konvergenz Kehrwert}
    \left| \frac{1}{b} - \frac{1}{b_n} \right|
    = \left| \frac{b_n - b}{b b_n} \right|
    = \frac{|b - b_n|}{|b| |b_n|}
    \leq \frac{2 |b - b_n|}{|b|^2}.
   \end{equation}
   Es sei nun $\varepsilon > 0$ beliebig aber fest. Da $b_n \to b$ für $n \to \infty$ gibt es auch ein $n_2 \in \N$ mit $|b - b_n| < \varepsilon|b|^2/2$ für alle $n \geq n_2$. Zusammen mit \eqref{eqn: Konvergenz Kehrwert} ergibt sich für $N \coloneqq \max \{n_1, n_2\}$, dass für alle $n \geq N$
   \[
    \left|\frac{1}{b} - \frac{1}{b_n}\right|
    \leq \frac{2 |b - b_n|}{|b|^2}
    < \varepsilon.
   \]
   Aus der Beliebigkeit von $\varepsilon > 0$ folgt, dass $\lim_{n \to \infty} 1/b_n = 1/b$.
  \item
   Aus den vorherigen Aussagen folgt, dass $1/b_n \to 1/b$ für $n \to \infty$ und damit auch
   \[
    \frac{a_n}{b_n} = a_n \cdot \frac{1}{b_n} \to a \cdot \frac{1}{b} = \frac{a}{b}.
   \]
 \end{enumerate}
\end{proof}


\begin{bsp}
 \begin{enumerate}
  \item
   Wir wollen die Konvergenz der Folge $(a_n)_{n \in \N}$ mit
   \[
    a_n \coloneqq \frac{n^2 + 3n + 5}{6n^2 + n + 1}
   \]
   untersuchen. Erweitern des Bruches mit $1/n^2$ ergibt, dass für alle $n \geq 1$
   \[
    a_n
    = \frac{n^2 + 3n + 5}{6n^2 + n + 1}
    = \frac{1 + \frac{3}{n} + \frac{5}{n^2}}{6 + \frac{1}{n} + \frac{1}{n^2}}.
   \]
   Aus $\lim_{n \to \infty} 1/n = 0$ ergibt sich aus der obigen Proposition, dass auch $3/n \to 0$, $5/n^2 \to 0$ and $1/n^2 \to 0$ für $n \to \infty$. Also ist $1 + 3/n + 5/n^2 \to 1$ und $6 + 1/n + 1/n^2 \to 6$ für $n \to \infty$. Daher folgt nach der Proposition
   \[
    \lim_{n \to \infty} a_n = \frac{1}{6}.
   \]
  \item
   Für alle $0 \leq q < 1$ ist die Folge $(q^n)_{n \in \N}$ monoton fallend, da $q^{n+1} = q \cdot q^n \leq q^n$ und nach unten beschränkt (durch $0$), und somit konvergent. Für den Grenzwert $x = \lim_{n \to \infty} q^n$ folgt aus der Linearität des Grenzwertes
   \[
    x = \lim_{n \to \infty} q^n = \lim_{n \to \infty} q^{n+1} = q \lim_{n \to \infty} q^n = q x,
   \]
   also $(1-q)x = 0$, und wegen $q \neq 1$ somit $x = 0$. Allgemeiner gilt für alle $q \in \R$ mit $|q| < 1$, dass
   \[
    -|q|^n \leq q^n \leq |q|^n,
   \]
   und da wegen der Linearität des Grenzwertes auch $-|q|^n \to 0$ für $n \to \infty$ folgt aus dem Sandwich-Lemma, dass $q^n \to 0$ für $n \to \infty$.
 \end{enumerate}
\end{bsp}



\subsection{Uneigentliche Grenzwerte}
Nur weil eine Folge nicht konvergiert, bedeutet dies noch nicht, dass für $n \to \infty$ kein „vernünftiges“ Verhalten aufweist.


\begin{defi}
 Eine Folge $(x_n)_{n \in \N}$ konvergiert gegen $\infty$, falls es für jedes $R > 0$ ein $N \in \N$ gibt, so dass $a_n > R$ für alle $n \geq N$. Wir schreiben dann $\lim_{n \to \infty} x_n = \infty$ oder $x_n \to \infty$ für $n \to \infty$.
 
 $(x_n)$ konvergiert gegen $-\infty$, falls es für jedes $R > 0$ ein $N \in \N$ gibt mit $a_n < -R$ für alle $n \geq N$. Wir schreiben dann $\lim_{n \to \infty} x_n = -\infty$ oder $x_n \to -\infty$ für $n \to \infty$.
 
 Ist $x_n \to \infty$ oder $x_n \to -\infty$ für $n \to \infty$, so sagen wir, dass die Folge $(x_n)$ \emph{uneigentlich konvergent} ist.
\end{defi}


Unter Verwendung von uneigentlichen Grenzwerten ergibt sich, dass jede monotone Folge konvergiert:

Ist $(x_n)_{n \in \N}$ eine monoton steigende Folge, so gibt es zwei Fälle: Ist die Folge nach oben beschränkt, so konvergiert sie eigentlich nach Bolzano-Weierstraß. Ist sie nicht nach oben beschränkt, so gibt es für alle $R > 0$ ein $N \in \N$ mit $x_N > R$, und aus der Monotonie der Folge folgt, dass $x_n > R$ für alle $n \geq N$, weshalb in diesem Fall $x_n \to \infty$ für $n \to \infty$.

Analog ergibt sich, dass auch jede monoton fallende Folge entweder eigentlich konvergiert, oder uneigentlich gegen $-\infty$ konvergiert.





\section{Alternative Definitionen von Konvergenz}



\subsection{Limsup und Liminf}
Einer der Probleme mit dem Begriff des Grenzwertes ist, dass nicht jede Folge eigentlich oder uneigentlich konvergiert, etwa die Folge $((-1)^n)_{n \in \N}$.


\begin{defi}
 Sei $(x_n)_{n \in \N}$ eine Folge. Die Folge $(\sup_{k \geq n} x_k)_{n \in \N}$ ist monoton fallend und daher (eventuell uneigentlich) konvergent. Wir definieren
 \[
  \limsup_{n \to \infty} x_n \coloneqq \lim_{n \to \infty} \sup_{k \geq n} x_k.
 \]
 Analog definieren wir
 \[
  \liminf_{n \to \infty} x_n \coloneqq \lim_{n \to \infty} \inf_{k \geq n} x_k.
 \]
\end{defi}



\subsection{Cauchy-Folgen}


\begin{defi}
 Eine Folge $(x_n)_{n \in \N}$ heißt eine \emph{Cauchy-Folge}, falls es für alle $\varepsilon > 0$ ein $N \in \N$ gibt, so dass $|x_n - x_{n'}| < \varepsilon$ für alle $n, n' \geq N$.
\end{defi}


\begin{lem}
 Ist eine Folge $(x_n)_{n \in \N}$ konvergent, so ist $(x_n)$ eine Cauchy-Folge.
\end{lem}
\begin{proof}
 Es sei $x \coloneqq \lim_{n \to \infty} x_n$. Sei $\varepsilon > 0$. Da $x_n \to x$ für $n \to \infty$ gibt es $N \in \N$ mit $|x - x_n| < \varepsilon/2$ für alle $n \geq N$. Dann ist für alle $n, n' \geq N$ auch
 \[
  |x_n - x_{n'}|
  = |(x - x_{n'}) - (x - x_n)|
  \leq |x - x_{n'}| + |x - x_n|
  < \frac{\varepsilon}{2} + \frac{\varepsilon}{2}
  = \varepsilon.
 \]
 Wegen der Beliebigkeit von $\varepsilon > 0$ zeigt dies, dass $(x_n)$ eine Cauchy-Folge ist.
\end{proof}


Es stellt sich heraus, dass auf $\R$ der Begriff der Cauchy-Folge schon äquivalent zu dem einer konvergenten Folge ist.



\subsection{Äquivalente Definitionen}

\begin{prop}
 Für eine Folge $(x_n)_{n \in \N}$ sind äquivalent:
 \begin{enumerate}
  \item\label{enum: konvergent}
   $(x_n)$ konvergiert.
  \item\label{enum: Cauchy}
   $(x_n)$ ist eine Cauchy-Folge.
  \item\label{enum: limsup = liminf}
   $\liminf_{n \to \infty} x_n = \limsup_{n \to \infty} x_n$.
 \end{enumerate}
 Es gilt dann
 \[
  \liminf_{n \to \infty} x_n = \limsup_{n \to \infty} x_n = \lim_{n \to \infty} x_n.
 \]
\end{prop}
\begin{proof}
 (\ref{enum: konvergent} $\Rightarrow$ \ref{enum: Cauchy}) Dies wurde bereits gezeigt.
 
 (\ref{enum: Cauchy} $\Rightarrow$ \ref{enum: limsup = liminf}) Sei $\varepsilon > 0$ beliebig aber fest. Da $(x_n)$ eine Cauchy-Folge ist, gibt es ein $N \in \N$, so dass $|x_n - x_{n'}| < \varepsilon/2$ für alle $n, n' \geq N$. Für $a \coloneqq x_N$ ist daher
 \[
  |x_n - a| < \varepsilon/2 \quad \text{für alle $n \geq N$}.
 \]
 Da damit $x_n < a + \varepsilon/2$ für alle $n \geq N$ ist
 \[
  \sup_{k \geq n} x_k \leq a + \varepsilon/2 \quad \text{für alle $n \geq N$},
 \]
 und somit auch $\limsup_{n \to \infty} x_n = \lim_{n \to \infty} \sup_{k \geq n} \leq a + \varepsilon$. Analog ergibt sich, dass auch $a-\varepsilon/2 \leq \liminf_{n \to \infty} x_n$. Es ist also
 \[
  a - \frac{\varepsilon}{2}
  \leq \liminf_{n \to \infty} x_n
  \leq \limsup_{n \to \infty} x_n
  \leq a + \frac{\varepsilon}{2}
 \]
 und somit
 \[
  0
  \leq \limsup_{n \to \infty} x_n - \limsup_{n \to \infty} x_n
  \leq \left(a + \frac{\varepsilon}{2}\right) - \left(a - \frac{\varepsilon}{2}\right)
  = \varepsilon.
 \]
 Aus der Beliebigkeit von $\varepsilon > 0$ ergibt sich, dass $\limsup_{n \to \infty} x_n - \liminf_{n \to \infty} x_n = 0$, also $\limsup_{n \to \infty} x_n = \liminf_{n \to \infty} x_n$.
 
 (\ref{enum: limsup = liminf} $\Rightarrow$ \ref{enum: konvergent}) Wir definieren die Folgen $(g_n)_{n \in \N}$ und $(h_n)_{n \in \N}$ durch
 \[
  g_n \coloneqq \inf_{k \geq n} x_k \quad \text{und} \quad h_n \coloneqq \sup_{k \geq n} x_k.
 \]
 Per Definition von $(g_n)$ und $(h_n)$ ist $g_n \leq x_n \leq h_n$ für alle $n \in \N$, und nach der Definition des $\liminf$, bzw. $\limsup$ ist
 \[
 \lim_{n \to \infty} g_n = \liminf_{n \to \infty} x_n = \limsup_{n \to \infty} h_n = \lim_{n \to \infty} h_n.
 \]
 Nach dem Sandwich-Lemma ist auch $(x_n)$ konvergent und $\lim_{n \to \infty} x_n$ stimmt mit den obigen Ausdrücken überein.
\end{proof}





\section{Metrische Räume}
Wir wollen hier noch kurz eine Möglichkeit vorstellen, den Begriff der Konvergenz zu verallgemeinern. Hierfür wollen wir Räume betrachten, in denen man Abstände messen kann, sogenannte metrische Räume.


\begin{defi}
 Sei $X$ eine Menge. Eine \emph{Metrik auf $X$} ist eine Funktion $d \colon X \times X \to \R$, die die folgenden Bedingungen erfüllt:
 \begin{enumerate}
  \item
   Für alle $x,y \in X$ ist $d(x,y) \geq 0$.
  \item
   Für alle $x,y \in X$ ist $d(x,y) = 0$ genau dann wenn $x = y$.
  \item
   Für alle $x,y \in X$ ist $d(x,y) = d(y,x)$ (Symmetrie).
  \item
   Für alle $x,y,z \in X$ gilt die Dreiecksungleichung
   \[
    d(x,z) \leq d(x,y) + d(y,z).
   \]
 \end{enumerate}
 
 Ein \emph{metrischer Raum} ist ein Paar $(X,d)$ bestehend aus einer Menge $X$ und einer Metrik $d$ auf $X$.
\end{defi}


\begin{bem}
 Die erste der vier Bedingungen muss eigentlich nicht gefordert werden, da sie aus den anderen dreien folgt: Für alle $x,y \in X$ ist nämlich
 \[
  0 = d(x,x) \leq d(x,y) + d(y,x) = 2 d(x,y)
 \]
 und somit $0 \leq d(x,y)$.
\end{bem}


\begin{bsp}
 \begin{enumerate}
  \item
   Die Abbildung $d(x,y) = |x-y|$ definiert eine Metrik auf $\R$, die man als die \emph{Standardmetrik auf $\R$} bezeichnet.
  \item
   Allgemein definiert die Abbildung $d(x,y) = \|x-y\|$ eine Metrik auf $\R^n$, die man als die \emph{Standardmetrik auf $\R^n$} bezeichnet.
  \item
   Für eine beliebige Menge $X$ definiert die Abbildung
   \[
    d \colon X \times X \to \R,
    d(x,y) =
    \begin{cases}
     1 & \text{falls $x = y$}, \\
     0 & \text{falls $x \neq y$},
    \end{cases}
   \]
   eine Metrik auf $X$, die sogenannte \emph{diskrete Metrik}. (Die Dreiecksungleichung ergibt sich durch Fallunterscheidung.)
  \item
   Ist $(X,d)$ ein metrischer Raum und $Y \subseteq X$, so lässt sich die Metrik $d$ auf $Y$ einschränken zu
   \[
    d' \colon Y \times Y \to \R, d'(x,y) = d(x,y).
   \]
   Dann ist auch $(Y,d')$ ein metrischer Raum und ein \emph{Unterraum} von $(X,d)$.
  \item
   Sind $(X_1, d_1)$ und $(X_2, d_2)$ metrische Räume, so ist die Abbildung
   \begin{gather*}
    d \colon (X_1 \times X_2) \times (X_1 \times X_2) \to \R, \\
    d((x_1, y_1), (x_2, y_2)) = d_1(x_1, x_2) + d_2(y_1, y_2)
   \end{gather*}
   eine Metrik auf $X_1 \times X_2$.
 \end{enumerate}
\end{bsp}


Auf einem metrischen Raum kann man nun von der Konvergenz von Folgen sprechen:


\begin{defi}
 Sei $(X,d)$ ein metrischer Raum, $(x_n)_{n \in \N}$ eine Folge auf $X$ und $x \in X$. Wir sagen, dass $x_n$ gegen $x$ konvergiert, falls es für alle $\varepsilon > 0$ ein $N \in \N$ gibt, so dass $d(x, x_n) < \varepsilon$ für alle $n \geq N$. Wir schreiben dann $\lim_{n \to \infty} x_n = x$ oder $x_n \to x$ für $n \to \infty$.
\end{defi}


Die Konvergenz auf $\R$, wie wir sie bisher untersucht haben, entspricht unter dieser Definition genau der Konvergenz auf $(\R,d)$, wobei $d$ die Standardmetrik auf $\R$ ist.


Viele der Eigenschaften konvergenter Folgen, die wir bisher kennengelernt haben, lassen sich auch auf metrische Räume übertragen. So sind etwa Grenzwerte in metrischen Räumen eindeutig. Man mache sich jedoch bewusst, dass metrische Räume nicht mit zusätzlicher Struktur wie Ordnungsrelationen oder binären Operationen wie Addition oder Multiplikation versehen seien müssen. Daher ergeben  Konstruktionen wie „die Summe zweier Folgen” in allgemeinen metrischen Räumen keinen Sinn.


Zum Schluss wollen wir noch den Begriff der Cauchy-Folge in metrischen Räumen thematisieren. Der Begriff der Cauchy-Folge lässt sich ohne Probleme auf metrische Räume verallgemeinern.

\begin{defi}
 Sei $(X,d)$ ein metrischer Raum und $(x_n)_{n \in \N}$ eine Folge auf $X$. $(x_n)$ heißt \emph{Cauchy-Folge}, falls es für alle $\varepsilon > 0$ ein $N \in \N$ gibt, so dass $d(x_n, x_{n'}) < \varepsilon$ für alle $n, n' \geq N$.
\end{defi}

Wie bereits zuvor ergibt sich, dass jede konvergente Folge auch eine Cauchy-Folge ist. Es stellt sich jedoch heraus, dass in metrischen Räumen Cauchy-Folgen nicht unbedingt konvergieren.

Als einfaches Gegenbeispiele betrachte man $\Q \subseteq \R$. Es sei $d$ die Standardmetrik auf $\R$ und $d_\Q$ die Einschränkung von $d$ auf $\Q$, d.h. $d_\Q(x,y) = d(x,y) = |x-y|$ für alle $x,y \in \Q$.

Es existiert nun eine Folge $(x_n)_{n \in \N}$ rationaler Zahlen, so dass in $(\R, d)$ die Konvergenz $x_n \to \sqrt{2}$ für $n \to \infty$ gilt. In $(\R, d)$ handelt es sich deshalb bei $(x_n)$ um eine Cauchy-Folge. Da $d_\Q$ die Einschränkung von $d$ auf $\Q$ ist, ist deshalb $(x_n)$ auch in $(\Q,d_\Q)$ eine Cauchy-Folge. In $(\Q,d_\Q)$ konvergiert die Folge $(x_n)$ jedoch nicht, da $\sqrt{2} \notin \Q$. (Würde die Folge $(x_n)$ in $(\Q, d_\Q)$ gegen eine rationale Zahlen $q \in \Q$ konvergieren, so würde sie auch in $(\R, d)$ gegen $q$ konvergieren, was wegen $\sqrt{2} \neq q$ im Widerspruch zur Eindeutigkeit von Grenzwerten steht.)


Metrische Räume, in den alle Cauchy-Folgen konvergieren, sind also 


\begin{defi}
 Ein metrischer Raum $(X,d)$ heißt \emph{vollständig}, wenn jede Cauchy-Folge in $X$ auch konvergiert.
\end{defi}

\begin{bsp}
 Wie wir bereits wissen ist $\R$ (mit der Standardmetrik) vollständig. Daraus kann man folgern, dass auch $\R^n$ mit der Standardmetrik vollständig ist.
\end{bsp}














\end{document}
