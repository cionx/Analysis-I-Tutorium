\documentclass[a4paper,10pt]{article}
%\documentclass[a4paper,10pt]{scrartcl}

\usepackage{../mystyle}

\setromanfont[Mapping=tex-text]{Linux Libertine O}
% \setsansfont[Mapping=tex-text]{DejaVu Sans}
% \setmonofont[Mapping=tex-text]{DejaVu Sans Mono}

\title{Definitionen von Stetigkeit}
\author{Jendrik Stelzner}
\date{\today}

\begin{document}
\maketitle


Im Folgenden wollen wir die unterschiedlichen Definitionen der Stetigkeit einer Abbildung $f \colon \R \to \R$ angebeben und ihre Äquivalenz beweisen.


\tableofcontents





\section{Grundlegende Definitionen}


\subsection{$\varepsilon$-$\delta$-Stetigkeit}


\begin{defi}
 Eine Abbildung $f \colon \R \to \R$ heißt \emph{$\varepsilon$-$\delta$-stetig im Punkt $x \in \R$}, falls es für jedes $\varepsilon > 0$ ein $\delta > 0$ gibt, so dass
 \[
  |x-y| < \delta \Rightarrow |f(x)-f(y)| < \varepsilon \quad \text{für alle $y \in \R$}.
 \]
 Die Abbildung $f$ heißt \emph{$\varepsilon$-$\delta$-stetig}, falls $f$ $\varepsilon$-$\delta$-stetig an jeder Stelle $x \in \R$ ist.
\end{defi}


\begin{bsp}
 Wir betrachten die Abbildung $f \colon \R \to \R$, $x \mapsto x^2$ an einer Stelle $x \in \R$. Für alle $y \in \R$ haben wir
 \begin{equation}\label{eqn: Ungleichung für Quadrat}
  \begin{aligned}
   \left| x^2 - y^2 \right|
   &= |(x+y)(x-y)|
   = |x+y||x-y|
   \leq (|x|+|y|)|x-y| \\
   &\leq (|x|+|x|+|x-y|)|x-y|
   = 2|x||x-y| + |x-y|^2,
  \end{aligned}
 \end{equation}
 wobei wir die Dreiecksungleichung für \mbox{$|x+y| \leq |x| + |y|$} und \mbox{$|y| = |x| + |x-y|$} nutzen. Wir unterscheiden nun zwischen zwei Fällen:
 
 Ist $x = 0$, so ist $|x^2 - y^2| = |y|^2$ für alle $y \in \R$. Wählt man dann $\delta \coloneqq \sqrt{\varepsilon}$, so ist für alle $y \in \R$ mit $|y| = |x-y| < \delta$ auch $|x^2-y^2| = |y|^2 < \varepsilon$.
 
 Ist $x \neq 0$, so ergibt sich für $\delta \coloneqq \min\{ \varepsilon/(4|x|), \sqrt{\varepsilon/2} \}$ aus \eqref{eqn: Ungleichung für Quadrat}, dass für alle $y \in \R$ mit $|x-y| < \delta$
 \[
  \left| x^2 - y^2 \right| \leq 2|x||x-y| + |x-y|^2 < \frac{\varepsilon}{2} + \frac{\varepsilon}{2} = \varepsilon.
 \]
 
 Das zeigt, dass $f$ an jeder Stelle $x \in \R$ stetig ist
\end{bsp}


\begin{question}
 Es sei $f \colon \R \to \R$ $\varepsilon$-$\delta$-stetig an der Stelle $x \in \R$. Zeigen Sie: Ist $f(x) > 0$, so gibt es ein $\delta > 0$ mit $f(y) > 0$ für alle $y \in (x-\delta, x+\delta)$. Gilt die Aussage auch für $f(x) < 0$ oder $f(x) \neq 0$?
\end{question}
\begin{solution}
 Da $f$ an der Stelle $x$ $\varepsilon$-$\delta$-stetig ist, gibt es $\delta > 0$, so dass 
 \[
  |f(x) - f(y)| < \frac{f(x)}{2} \quad \text{für alle $y \in \R$ mit $|x-y| < \delta$}.
 \]
 Durch die Dreiecksungleichung ergibt sich, dass
 \[
  |f(y)| \geq |f(x)| - |f(x)-f(y)| \quad \text{für alle $y \in \R$}.
 \]
 Zusammen ergibt sich damit, dass für alle $y \in \R$ mit $|x-y| < \delta$
 \begin{align*}
  |f(y)|
  \geq |f(x)| - |f(x)-f(y)|
  > f(x) - \frac{f(x)}{2}
  = \frac{f(x)}{2}
  > 0.
 \end{align*}
 Dass $|x-y| < \delta$ bedeutet gerade, dass $y \in (x-\delta, x+\delta)$, wodurch sich die Aussage ergibt.
\end{solution}


\begin{question}
 Es seien $f, g \colon \R \to \R$. Zeigen Sie: Ist $f$ $\varepsilon$-$\delta$-stetig an der Stelle $x \in \R$ und $g$ $\varepsilon$-$\delta$-stetig an der Stelle $f(x)$, so ist die Komposition $g \circ f$ $\varepsilon$-$\delta$-stetig an der Stelle $x$. 
\end{question}
\begin{solution}
 Sei $\varepsilon > 0$ beliebig aber fest. Wegen der $\varepsilon$-$\delta$-Stetigkeit von $g$ an der Stelle $f(x)$ gibt es $\delta' > 0$, so dass
 \[
  |f(x)-y'| < \delta' \Rightarrow |g(f(x))-g(y')| < \varepsilon
  \quad \text{für alle $y' \in \R$}.
 \]
 Wegen der $\varepsilon$-$\delta$-Stetigkeit von $f$ an $x$ gibt es ein $\delta > 0$, so dass
 \[
  |x-y| < \delta \Rightarrow |f(x)-f(y)| < \delta'
  \quad \text{für alle $y \in \R$}.
 \]
 Für alle $y \in \R$ ist daher
 \[
  |x-y| < \delta
  \Rightarrow |f(x)-f(y)| < \delta'
  \Rightarrow |g(f(x))-g(f(y))| < \varepsilon.
 \]
 Wegen der Beliebigkeit von $\varepsilon > 0$ zeigt dies, dass $g \circ f$ $\varepsilon$-$\delta$-stetig an der Stelle $x$ ist.
\end{solution}


Wir erhalten damit, dass für zwei $\varepsilon$-$\delta$-stetige Abbildungen $f \colon \R \to \R$ und $g \colon \R \to \R$ auch die Verknüpfung $g \circ f$ $\varepsilon$-$\delta$-stetig ist.


\subsection{Folgenstetigkeit}


\begin{defi}
 Eine Abbildung $f \colon \R \to \R$ heißt \emph{folgenstetig an $x \in \R$}, falls für jedes Folge $(x_n)_{n \in \N}$ mit $x_n \to x$ für $n \to \infty$ auch die Folge $(f(x_n))_{n \in \N}$ konvergiert und
 \[
  \lim_{n \to \infty} f(x_n) = f(x) = f\left(\lim_{n \to \infty} x_n\right).
 \]
 $f$ heißt \emph{folgenstetig}, falls $f$ an jeder Stelle $x \in \R$ folgenstetig ist.
\end{defi}


\begin{bsp}
 Wir betrachten erneut die Abbildung $f \colon \R \to \R$, $x \mapsto x^2$ an einer Stelle $x \in \R$. Ist $(x_n)_{n \in \N}$ eine Folge mit $\lim_{n \to \infty} x_n = x$, so folgt aus den bekannten Eigenschaften konvergenter Folgen, dass auch die Folge $(x_n^2)_{n \in \N}$ konvergiert und
 \[
  \lim_{n \to \infty} x_n^2
  = \lim_{n \to \infty} (x_n \cdot x_n)
  = \left(\lim_{n \to \infty} x_n\right) \cdot \left(\lim_{n \to \infty} x_n\right)
  = x \cdot x
  = x^2.
 \]
 Das zeigt, dass $f$ an jeder Stelle $x \in \R$ folgenstetig ist.
\end{bsp}


\begin{question}
 Es seien $f, g \colon \R \to \R$ beide folgenstetig an der Stelle $x \in R$. Zeigen Sie, dass auch die Funktionen $f+g$ und $f \cdot g$ folgenstetig an der Stelle $x$ sind.
\end{question}
\begin{solution}
 Es sei $(x_n)_{n \in \N}$ eine Folge mit $\lim_{n \to \infty} x_n = x$. Da $f$ und $g$ folgenstetig an der Stelle $x$ sind, sind auch die Folgen $(f(x_n))_{n \in \N}$ und $(g(x_n))_{n \in \N}$ konvergent und es gilt
 \[
  \lim_{n \to \infty} f(x_n) = f(x)
  \quad
  \text{und}
  \quad
  \lim_{n \to \infty} g(x_n) = g(x).
 \]
 Nach den üblichen Rechenregeln für Folgen konvergieren daher auch die Folgen
 \begin{align*}
  ((f+g)(x_n))_{n \in \N}
  &= (f(x_n)+g(x_n))_{n \in \N}
 \shortintertext{und}
  ((f \cdot g)(x_n))_{n \in \N}
  &= (f(x_n) \cdot g(x_n))_{n \in \N},
 \end{align*}
 und es gilt
 \begin{gather*}
  \lim_{n \to \infty} (f+g)(x_n)
  = \lim_{n \to \infty} f(x_n) + g(x_n)
  = f(x) + g(x)
  = (f+g)(x)
 \shortintertext{sowie}
  \lim_{n \to \infty} (f \cdot g)(x_n)
  = \lim_{n \to \infty} f(x_n) \cdot g(x_n)
  = f(x) \cdot g(x)
  = (f \cdot g)(x).
 \end{gather*}
 Dies zeigt, dass auch $f + g$ und $f \cdot g$ folgenstetig an $x$ sind.
\end{solution}



\subsection{Stetigkeit über Grenzwerte}


\begin{defi}
 Es sei $f \colon \R \to \R$ eine Abbildung und $x_0 \in \R$. Für $y \in \R$ schreiben wir $\lim_{x \uparrow x_0} f(x) = y$, falls
 \[
  \text{für alle $\varepsilon > 0$ existiert $\delta > 0$, s.d.\ $|f(x_0)-f(x)| < \varepsilon$ für alle $x_0-\delta < x < x_0$},
 \]
 und bezeichnen $y$ dann als den \emph{linksseitigen Limes von $f$ an $x_0$}. Analog schreiben wir \mbox{$\lim_{x \downarrow x_0} f(x) = f(y)$}, falls
 \[
  \text{für alle $\varepsilon > 0$ existiert $\delta > 0$, s.d.\ $|f(x_0)-f(x)| < \varepsilon$ für alle $x_0 < x < x_0+\delta$}.
 \]
 Wir nennen $y$ dann denn \emph{rechtsseitigen Limes von $f$ an $x_0$}. Existieren links- und rechtsseitiger Limes von $f$ an $x_0$ und ist $\lim_{x \uparrow x_0} f(x) = \lim_{x \downarrow x_0} f(x)$, so nennen wir
 \[
  \lim_{x \to x_0} f(x) \coloneqq \lim_{x \uparrow x_0} f(x) = \lim_{x \downarrow x_0} f(x)
 \]
 den \emph{beidseitgen Limes von $f$ an $x_0$}.
\end{defi}


\begin{defi}
 Eine Abbildung $f \colon \R \to \R$ heißt \emph{linksstetig an der Stelle $x \in \R$}, falls $\lim_{y \uparrow x} f(y) = f(x)$. $f$ heißt \emph{rechtsstetig an der Stelle $x$}, falls $\lim_{y \downarrow x} f(y) = f(x)$. $f$ heißt \emph{beidseitig stetig an $x$}, falls $\lim_{y \to x} f(y) = f(x)$. (Insbesondere müssen die entsprechenden Grenzwerte existieren.)
 
 $f$ heißt \emph{linksstetig}, falls $f$ an jeder Stelle $x \in \R$ linksstetig ist, und \emph{rechtsstetig}, falls $f$ an jeder Stelle $x \in \R$ rechtsstetig ist. Ist $f$ an jeder Stelle $x \in \R$ beidseitig stetig, so heißt $f$ \emph{beidseitig stetig}.
\end{defi}


\begin{bem}
 Rechts-, Links- und beidseitige Limites sind eindeutig (sofern sie existieren).
\end{bem}


\begin{question}
 Zeigen Sie, dass $f$ genau dann beidseitig stetig ist, wenn $f$ links- und rechtsstetig ist.
\end{question}
\begin{solution}
 $f$ ist genau dann beidseitig stetig, wenn $f$ an jeder Stelle $x \in \R$ beidseitig stetig ist, wenn also $f$ an jeder Stelle $x \in \R$ sowohl links- als auch rechtsstetig ist. Dies ist äquivalent dazu, dass $f$ an jeder Stelle rechtstetig ist, und an jeder Stelle auch linksstetig ist. Dies bedeutet, dass $f$ links- und rechtsstetig ist.
\end{solution}


\begin{question}
 Zeigen Sie, dass für eine monoton steigende Funktion $f \colon \R \to \R$ an jeder Stelle $x \in \R$ sowohl der linksseitige als auch der rechtsseitige Limes exitieren, und dass
 \[
  \lim_{y \uparrow x} f(y) = \sup\{f(y) \mid y < x\}
  \quad
  \text{und}
  \quad
  \lim_{y \downarrow x} f(y) = \inf\{f(y) \mid y > x\}.
 \]
 Wie sieht es für eine monoton fallende Funktion aus?
\end{question}
\begin{solution}
 Es sei $f$ monoton steigend und $x \in \R$. Wir wollen zeigen, dass $a \coloneqq \sup_{y < x} f(y)$ die Eigenschaften des linksseitigen Limes erfüllt. Sei hierfür $\varepsilon > 0$ beliebig aber fest. Nach der $\varepsilon$-Charakterisierung des Supremums gibt es ein $y_0 < x$ mit $a-\varepsilon < f(y_0)$. Aus der Monotonie von $f$ folgt, dass
 \[
  a-\varepsilon < f(y_0) \leq f(y) \leq \sup_{y' < x} f(y') = a \quad \text{für alle $y_0 \leq y < x$}.
 \]
 Für $\delta \coloneqq x-y_0 > 0$ ist also $|f(y)-a| < \varepsilon$ für alle $y \in (x-\delta, x)$. Wegen der Beliebigkeit von $\varepsilon>0$ zeigt dies, dass $\lim_{y \uparrow x} f(y) = a$.
 
 Analog zeigt man, dass $\lim_{y \downarrow x} f(y) = \inf_{x < y} f(x)$. Für monoton fallende Funktionen zeigt man analog, dass obere und untere Grenzwerte an jeder Stelle existieren, und dass für alle $x \in \R$
 \[
  \lim_{y \uparrow x} f(y) = \inf_{y < x} f(y)
  \quad
  \text{und}
  \quad
  \lim_{y \downarrow x} f(y) = \sup_{y > x} f(y).
 \]
\end{solution}


\begin{question}\label{qst: Charakterisierung beidseitiger Grenzwert}
 Es sei $f \colon \R \to \R$ und $x \in \R$. Zeigen Sie, dass $\lim_{x \to x_0} f(x) = a$ genau dann, wenn
 \[
  \text{für alle $\varepsilon > 0$ gibt es $\delta > 0$ mit $|f(y)-a| < \varepsilon$ für $|x-y| < \delta$ und $y \neq x$}.
 \]
\end{question}
\begin{solution}
 Angenommen, es ist $a = \lim_{y \to x} f(y)$. Dann ist sowohl $\lim_{y \uparrow x} f(y) = a$ als auch $\lim_{y \downarrow x} f(y) = a$. Sei $\varepsilon > 0$ beliebig aber fest. Da $\lim_{y \uparrow x} f(y) = a$ gibt es $\delta_1 > 0$, so dass
 \[
  |f(y)-a| < \varepsilon \quad \text{für alle $y \in (x-\delta_1, x)$}.
 \]
 Da $\lim_{y \downarrow x} f(y) = a$ gibt es $\delta_2 > 0$, so dass
 \[
  |f(y)-a| < \varepsilon \quad \text{für alle $y \in (x, x+\delta_2)$}.
 \]
 Für $\delta \coloneqq \min\{\delta_1, \delta_2\}$ ist damit
 \[
  |f(y)-a| < \varepsilon \quad \text{für alle $y \in (x-\delta, x+\delta)$ mit $y \neq x$}.
 \]
 Wegen der Beliebigkeit von $\varepsilon > 0$ zeigt dies eine der Implikationen.
 
 Angenommen, es gibt für jedes $\varepsilon > 0$ ein $\delta > 0$, so dass $|f(y)-a| < \varepsilon$ für alle $y \in (x-\delta, x+\delta)$ mit $y \neq x$. Insbesondere gilt dann $|f(y)-a| < \varepsilon$ für alle $y \in (x-\delta,x)$ und $y \in (x,x+\delta)$, weshalb dann $\lim_{y \uparrow x} f(y) = a$ und $\lim_{y \downarrow x} f(y) = a$. Somit ist $\lim_{y \to x} f(y) = a$.
\end{solution}


\subsection{Äquivalenz der Stetigkeitsbegriffe}


Wir wollen nun zeigen, dass die verschiedenen Stetigkeitsbegriffe äquivalent zueinander sind.


\begin{prop}
 Es sei $f \colon \R \to \R$ und $x \in \R$. Dann sind äquivalent:
 \begin{enumerate}
  \item\label{enum: epsilon-delta-stetig}
   $f$ ist $\varepsilon$-$\delta$-stetig an der Stelle $x$.
  \item\label{enum: folgenstetig}
   $f$ ist folgenstetig an der Stelle $x$.
  \item\label{enum: beidseitig stetig}
   $f$ ist beidseitig stetig an der Stelle $x$.
 \end{enumerate}
\end{prop}
\begin{proof}
 (\ref{enum: epsilon-delta-stetig} $\Rightarrow$ \ref{enum: folgenstetig})
 Sei $(x_n)_{n \in \N}$ ein Folge mit $\lim_{n \to \infty} x_n = x$. Sei $\varepsilon > 0$ beliebig aber fest. Da $f$ $\varepsilon$-$\delta$-stetig an $x$ ist, gibt es $\delta > 0$ mit $|f(x)-f(y)| < \varepsilon$ falls $|x-y| < \delta$. Da $\lim_{n \to \infty} x_n = x$ gibt es $N \in \N$ mit $|x-x_n| < \delta$ für alle $n \geq N$. Für alle $n \geq N$ ist also $|f(x)-f(x_n)| < \varepsilon$. Wegen der Beliebigkeit von $\varepsilon > 0$ folgt, dass $\lim_{n \to \infty} f(x_n) = f(x)$. Das zeigt, dass $f$ folgenstetig an $x$ ist.

 (\ref{enum: folgenstetig} $\Rightarrow$ \ref{enum: epsilon-delta-stetig})
 Angenommen, $f$ ist nicht $\varepsilon$-$\delta$-stetig an $x$. Dann gibt es $\varepsilon > 0$, so dass es für jedes $\delta > 0$ ein $y \in \R$ mit $|x-y| < \delta$ und $|f(x)-f(y)| \geq \varepsilon$ gibt. Insbesondere gibt es für jedes $n \geq 1$ ein $x_n \in \R$ mit $|x - x_n| < 1/n$ und $|f(x) - f(x_n)| \geq \varepsilon$. Es ist dann $\lim_{n \to \infty} x_n = x$ aber nicht $\lim_{n \to \infty} f(x_n) = f(x)$. Dies steht im Widerspruch zur Folgenstetigkeit von $f$ an $x$.
 
 (\ref{enum: epsilon-delta-stetig} $\Rightarrow$ \ref{enum: beidseitig stetig})
 Sei $\varepsilon > 0$ beliebig aber fest. Da $f$ $\varepsilon$-$\delta$-stetig an $x$ ist, gibt es $\delta > 0$ mit $|f(x)-f(y)| < \varepsilon$ für alle $y \in \R$ mit $|x-y| < \delta$. Inbesondere ist $|f(x)-f(y)| < \varepsilon$ für alle $y \in (x-\delta,x)$ und für alle $y \in (x,x+\delta)$. Also ist $f$ sowohl rechts- als auch linksstetig an $x$, und somit beidseitig stetig an $x$.
 
 (\ref{enum: beidseitig stetig} $\Rightarrow$ \ref{enum: epsilon-delta-stetig})
 Es sei $\varepsilon > 0$ beliebig aber fest. Da $f$ beidseitig stetig an $x$ ist, existiert der beidseitige Limes $\lim_{y \to x} f(y)$ und es ist $f(x) = \lim_{y \to x} f(y)$. Nach Übung \ref{qst: Charakterisierung beidseitiger Grenzwert} gibt daher $\delta > 0$, so dass $|f(x)-f(y)| < \varepsilon$ für alle $y \in \R$ mit $|x-y| < \delta$ und $x \neq y$; für $x = y$ gilt dies offenbar ebenfalls. Also ist $f$ $\varepsilon$-$\delta$-stetig an $x$.
\end{proof}


Statt zwischen den verschiedenen Stetigkeitsbegriffen zu unterscheiden, sprechen wir von nun an nur noch von Stetigkeit.


\begin{question}
 Es sei
 \[
  \mc{O} \coloneqq \{f \colon \R \to \R \mid \text{$f$ ist stetig}\}.
 \]
 Zeigen Sie:
 \begin{enumerate}
  \item
   Alle konstanten Funktionen sind in $\mc{O}$ enthalten.
  \item
   $\mc{O}$ ist ein $\R$-Vektorraum unter punktweiser Addition und Skalarmultiplikation, d.h. für alle $f, g \in \mc{O}$ ist $(f+g)(x) = f(x)+g(x)$ und für alle $f \in \mc{O}$ und $\lambda \in \R$ ist $(\lambda f)(x) = \lambda f(x)$.
  \item
   Zeigen Sie, dass für zwei stetige Abbildungen $f, g \in \mc{O}$ das das punktweise Produkt $f \cdot g$ stetig ist, d.h. $(f \cdot g)(x) = f(x) \cdot g(x)$ für alle $x \in \R$.
 \end{enumerate}
 Insgesamt zeigt dies, dass $\mc{O}$ eine $\R$-Algebra bildet. (Um genau zu sein zeigt es, dass $\mc{O}$ eine $\R$-Unteralgebra von $\Abb(\R, \R)$ ist.
\end{question}


\begin{question}
 \begin{enumerate}
  \item
   Zeigen Sie, dass der Betrag
   \[
    |\cdot| \colon \R \to \R, x \mapsto |x|
   \]
   stetig ist.
  \item
   Folgern Sie, dass für zwei stetige Abbildungen $f, g \colon \R \to \R$ auch $\max(f,g)$ und $\min(f,g)$ stetig sind, wobei für alle $x \in \R$
   \[
    \max(f,g)(x) = \max\{f(x),g(x)\}
    \quad
    \text{und}
    \quad
    \min(f,g)(x) = \min\{f(x),g(x)\}.
   \]
 \end{enumerate}
\end{question}
\begin{solution}
 \begin{enumerate}
  \item
   Es sei $x \in \R$ und $(x_n)_{n \in \N}$ eine Folge mit $\lim_{n \to \infty} x_n = x$. Wie wir bereits wissen, ist dann auch die Folge $(|x_n|)_{n \in \N}$ konvergent und
   \[
    \lim_{n \to \infty} |x_n|
    = \left| \lim_{n \to \infty} x_n \right|
    = |x|.
   \]
   Wegen der Beliebigkeit der Folge $(x_n)$ folgt, dass der Betrag folgenstetig an $x$ ist. Aus der Beliebigkeit der Stelle $x \in \R$ folgt, dass der Betrag folgenstetig ist.
  \item
   Wir bemerke, dass für alle $x,y \in \R$
   \[
    \max(x,y) = \frac{x+y}{2} + \frac{|x-y|}{2} = \frac{x+y+|x-y|}{2},
   \]
   denn für $x \leq y$ ist $x-y \leq 0$ und somit
   \[
    \frac{x+y+|x-y|}{2}
    = \frac{x+y-(x-y)}{2}
    = y
    = \max(x,y),
   \]
   und für $y \leq x$ ist $x-y \geq 0$ und somit
   \[
    \frac{x+y+|x-y|}{2}
    = \frac{x+y+(x-y)}{2}
    = x
    = \max(x,y).
   \]
   Die Stetigkeit von
   \[
    \max(f,g) = \frac{f+g+|f-g|}{2}
   \]
   ergibt sich aus den schon bekannten Aussagen über Kombination stetiger Abbildungen. Analog zeigt man, dass
   \[
    \min(x,y) = \frac{x+y-|x-y|}{2} \quad \text{für alle $x,y \in \R$},
   \]
   und dass damit $\min(f,g)$ stetig ist.
 \end{enumerate}
\end{solution}


\begin{question}
 Es seien
 \[
  f \colon \R \setminus \{0\} \to \R, x \mapsto \sin\left(\frac{1}{x}\right)
 \]
 und
  \[
  g \colon \R \to \R,
  x \mapsto
  \begin{cases}
   x \sin\left(\frac{1}{x}\right) & \text{falls $x \neq 0$}, \\
                                0 & \text{falls $x = 0$}.
  \end{cases}
 \]
 \begin{enumerate}
  \item
   Zeigen Sie, dass sich $f$ nicht stetig auf $\R$ fortsetzen lässt, d.h. es gibt keine stetige Funktion $h \colon \R \to \R$ mit $h(x) = f(x)$ für alle $x \neq 0$.
  \item
   Zeigen Sie, dass die Abbildung $g$ stetig ist.
  \item
   Es sei $x_0 \in \R$ und $s \colon \R \setminus\{x_0\} \to \R$ eine Abbildung, die in einer Umgebung von $x_0$ beschränkt sei, d.h. es gebe ein $\varepsilon > 0$ und eine Konstante $C > 0$, so dass $|s(x)| \leq C$ für alle $x \in (x_0 - \varepsilon, x_0 + \varepsilon)$ mit $x \neq x_0$. Zeigen Sie: Für eine stetige Abbildung $h \colon \R \to \R$ mit $h(x_0) = 0$ ist die Abbildung $h \cdot s$ stetig an $x_0$.
 \end{enumerate}
\end{question}





\section{Topologische Stetigkeit}


\begin{defi}
 Für $x \in \R$ und $\varepsilon > 0$ definieren wir den \emph{offenen $\varepsilon$-Ball um $x$} als
 \[
  B_\varepsilon(x) \coloneqq \{y \in \R \mid |x-y| < \varepsilon\}.
 \]
\end{defi}


\begin{defi}
 Eine Teilmenge $U \subseteq \R$ heißt \emph{offen}, falls es für jedes $x \in U$ ein $\varepsilon > 0$ gibt, so dass $B_\varepsilon \subseteq U$.
\end{defi}


\begin{bsp}
 \begin{enumerate}
  \item
   Die leere Menge ist offen, da die Bedingung dort leer ist.
  \item
   Offene Intervalle sind offen: Ist $I = (a,b)$ eine offenes Intervall und $x \in I$, so ist für
   \[
    \varepsilon \coloneqq \min\{x-a, b-x\}
   \]
   auch $B_\varepsilon(x) \subseteq I$.
  \item
   Abgeschlossene nicht-leere Intervalle sind nicht offen: Ist $I = [a,b]$ eine abgeschlossenen Intervall, dass nicht leer ist (also $a \leq b$), so gibt es kein $\varepsilon > 0$, so dass $B_\varepsilon(a) \subseteq I$, dann es ist $a-\varepsilon/2 \in B_\varepsilon(x)$, aber $a-\varepsilon/2 \notin I$.
 \end{enumerate}
\end{bsp}


\begin{lem}
 \begin{enumerate}
  \item
   Die leere Menge $\emptyset$ sowie $\R$ selbst sind offen.
  \item
   Ist $\{U_i \mid i \in I\}$ eine beliebige Kollektion offener Mengen, so ist auch die Vereinigung $\bigcup_{i \in I} U_i$ offen.
  \item
   Sind $U_1, \dotsc, U_n \subseteq \R$ offen, so ist auch der Schnitt $U_1 \cap \dotsb \cap U_n$ offen.
 \end{enumerate}
\end{lem}


\begin{defi}
 Es sei $x \in \R$ ein Punkt. Eine Menge $V \subseteq \R$ heißt \emph{Umgebung von $x$}, falls es ein $\varepsilon > 0$ gibt, so dass $B_\varepsilon(x) \subseteq V$.
\end{defi}


\begin{question}
 Zeigen Sie, dass eine Menge $U \subseteq \R$ genau dann offen ist, wenn $U$ für jedes $x \in U$ eine Umgebung von $x$ ist.
\end{question}


\begin{question}
 Zeigen Sie, dass eine Menge $V \subseteq \R$ genau Umgebung eines Punktes $x \in \R$ ist, wenn es eine offene Menge $U \subseteq \R$ mit $x \in U \subseteq V$ gibt.
\end{question}


\begin{question}
 Zeigen Sie: 
 \begin{enumerate}
  \item
   Ist $V \subseteq \R$ Umgebung eines Punktes $x \in \R$, so auch jedes Teilmenge $W \subseteq \R$ mit $V \subseteq \R$ eine Umgebung von $V$
  \item
   Sind $V_1, \dotsc, V_n \subseteq \R$ Umgebungen von $x \in \R$, so ist auch $V_1 \cap \dotsb \cap V_n$ eine Umgebung von $x$.
 \end{enumerate}
\end{question}





\section{Lösungen der Übungsaufgaben}

\printsolutions





















\end{document}
