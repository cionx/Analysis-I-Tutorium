\documentclass[a4paper,10pt]{article}
%\documentclass[a4paper,10pt]{scrartcl}

\usepackage{../mystyle}
%\SetupExSheets{solution/print=true} %Zur Ausgabe der Lösungen direkt nach den Fragen

\title{Grundbegriffe der Analysis I in $\R^n$}
\author{Jendrik Stelzner}
\date{\today}

\begin{document}
\maketitle


\tableofcontents





\section{Vorbereitung}


Wir gehen davon aus, dass der Leser mit der Vektorraumstruktur des $\R^n$ vertraut ist. Die \emph{Norm} eines Vektors $x = (x_1, \dotsc, x_n) \in \R^n$ ist als
\[
 \|x\| = \sqrt{\sum_{i=1}^n x_i^2}
\]
definiert.

Auf $\R^1 = \R$ stimmt die Norm mit dem Betrag überein. Die Norm auf $\R^2$ entspricht unter der üblichen Identifikation von $\R^2$ mit $\C$ dem Betrag auf $\C$. Unsere Betrachtungen erhalten daher auch den Konvergenz- und den Stetigkeitsbegriff auf $\C$.


Die Norm hat die folgenden Eigenschaften:
\begin{itemize}
 \item
  Für alle $x \in \R^n$ ist $\|x\| \geq 0$, und es ist genau dann $\|x\| = 0$, wenn $x = 0$.
 \item
  Die Norm ist \emph{absolut homogen}, d.h. für alle $x \in \R^n$ und $\lambda \in \R$ ist
  \[
   \|\lambda x\| = |\lambda| \|x\|.
  \]
 \item
  Für alle $x,y \in \R^n$ gilt die \emph{Dreiecksungleichung}
  \[
   \|x+y\| \leq \|x\| + \|y\|.
  \]
\end{itemize}


Insbesondere ist daher $\|-x\| = \|x\|$ für alle $x \in \R^n$ und $\|x-y\| = \|y-x\|$ für alle $x, y \in \R^n$.


\begin{question}
 Zeigen Sie die \emph{umgekehrte Dreiecksungleichung}
 \[
  \|x-y\| \geq |\|x\|-\|y\|| \quad \text{für alle $x,y \in \R^n$}.
 \]
\end{question}
\begin{solution}
 Für alle $x,y \in \R^n$ ist
 \[
  \|x\| = \|y+x-y\| \leq \|y\| + \|x-y\|
 \]
 und damit $\|x\|-\|y\| \leq \|x-y\|$, sowie
 \[
  \|y\| = \|x+y-x\| \leq \|x\| + \|y-x\| = \|x\| + \|x-y\|
 \]
 und damit $\|y\|-\|x\| \leq \|x-y\|$. Es ist
 \[
  |\|x\|-\|y\|| = \|x\|-\|y\|
  \quad
  \text{oder}
  \quad
  |\|x\|-\|y\|| = \|y\|-\|x\|;
 \]
 in beiden Fällen erhalten wir, dass $|\|x\|-\|y\|| \leq \|x-y\|$.
\end{solution}


\begin{question}
 \begin{enumerate}
  \item
   Zeigen Sie, dass die Wurzelfunktion subadditiv ist, d.h.\ dass
   \[
    \sqrt{x+y} \leq \sqrt{x} + \sqrt{y} \quad \text{für alle $x,y \geq 0$}.
   \]
  \item
   Folgern Sie, dass für alle $x = (x_1, \dotsc, x_n) \in \R^n$ die beiden Ungleichungen
   \begin{gather*}
    |x_i| \leq \|x\| \quad \text{für alle $1 \leq i \leq n$}
   \shortintertext{und}
    \|x\| \leq \sum_{i=1}^n |x_i|
   \end{gather*}
   gelten. Folgern Sie außerdem, dass für alle $x, y \in \R^n$ mit Koordinaten $x = (x_1, \dotsc, x_n)$ und $y = (y_1, \dotsc, y_n)$ die beiden Ungleichungen
   \begin{gather*}
    |x_i - y_i| \leq \|x-y\| \quad \text{für alle $1 \leq i \leq n$}
   \shortintertext{und}
    \|x-y\| \leq \sum_{i=1}^n |x_i - y_i|
   \end{gather*}
   gelten.
 \end{enumerate}
\end{question}
\begin{solution}
 \begin{enumerate}
  \item
   Für alle $x, y \geq 0$ ist
   \begin{align*}a
                   &\, \sqrt{x+y} \leq \sqrt{x}+\sqrt{y} \\
    \Leftrightarrow&\, x + y \leq (\sqrt{x}+\sqrt{y})^2 = x + 2\sqrt{xy} + y \\
    \Leftrightarrow&\, 0 \leq \sqrt{xy},
   \end{align*}
   was offenbar gilt.
  \item
   Für alle $1 \leq i \leq n$ ist wegen der Monotonie der Wurzel
   \[
    \|x\|
    = \sqrt{\sum_{j=1}^n x_j^2}
    \geq \sqrt{x_i^2}
    = |x_i|,
   \]
   was die erste Ungleichung zeigt. Die zweite Ungleichung folgt mithilfe der Subadditivität der Wurzel durch
   \[
    \|x\|
    = \sqrt{\sum_{i=1}^n x_i^2}
    \leq \sum_{i=1}^n \sqrt{x_i^2}
    = \sum_{i=1}^n |x_i|.
   \]
   
   Die Ungleichungen für $\|x-y\|$ ergeben sich sofort aus den beiden bisher gezeigten.
 \end{enumerate}
\end{solution}


Diese Ungleichungen werden wir wiederholt verwenden, um das Verhalten auf $\R^n$ auf das Verhalten in den einzelnen Koordinaten zurückzuführen.


\begin{defi}
 Eine Teilmenge $X \subseteq \R^n$ heißt \emph{beschränkt}, falls es ein $C > 0$ gibt, so dass $\|x\| \leq C$ für alle $x \in X$.
\end{defi}


Als eine erste Anwendung unserer Ungleichungen wollen wir die Beschränktheit in $\R^n$ auf die Beschränktheit in $\R$ zurückführen:


\begin{question}\label{qst: Beschränktheit in Koordinaten}
 Für $1 \leq i \leq n$ sei $\pi_i \colon \R^n \to \R, (x_1, \dotsc, x_n) \mapsto x_i$ die Projektion in die $i$-te Koordinate. Zeigen Sie, dass $X \subseteq \R^n$ genau dann beschränkt ist, wenn $\pi_i(X)$ für alle $1 \leq i \leq n$ beschränkt ist.
\end{question}
\begin{solution}
 Angenommen $X$ ist beschränkt. Dann gibt es ein $C > 0$, so dass $\|x\| \leq C$ für alle $x \in X$. Für jedes $1 \leq i \leq n$ ist daher
 \[
  |\pi_i(x)| = |x_i| \leq \|x\| \leq C \quad \text{für alle $x = (x_1, \dotsc, x_n) \in X$},
 \]
 und somit
 \[
  |x| \leq C \quad \text{für alle $x \in \pi_i(X)$}.
 \]
 Also ist $\pi_i(X)$ für alle $1 \leq i \leq n$ beschränkt.
 
 Angenommen $\pi_i(X)$ ist für alle $1 \leq i \leq n$ beschränkt. Dann gibt es für jedes $1 \leq i \leq n$ ein $C_i > 0$ mit $\pi_i(x) \leq C_i$ für alle $x \in X$. Für $C \coloneqq \sum_{i=1}^n C_i > 0$ ist daher
 \[
  \|x\| \leq \sum_{i=1}^n |x_i| \leq \sum_{i=1}^n C_i = C
  \quad \text{für alle $x = (x_1, \dotsc, x_n) \in X$}.
 \]
 Also ist $X$ beschränkt.
\end{solution}


\begin{question}
 Es sei $X \subseteq \R^n$. Dann sind äquivalent:
 \begin{enumerate}
  \item\label{enum: beschränkt um 0}
   $X$ ist beschränkt.
  \item\label{enum: beschränkt um einen Punkt}
   Es gibt ein $y \in \R^n$ und $C > 0$ mit $\|y-x\| \leq C$ für alle $x \in X$.
  \item\label{enum: beschränkt um jeden Punkt}
   Für jedes $z \in \R^n$ gibt es ein $C_z > 0$ mit $\|z-x\| \leq C_z$ für alle $x \in X$.
 \end{enumerate}
\end{question}
\begin{solution}
 (\ref{enum: beschränkt um 0} $\Rightarrow$ \ref{enum: beschränkt um einen Punkt}) Da $X$ beschränkt ist gibt es ein $C > 0$ mit
 \[
  \|0 - x\| \leq \|x\| \leq C
  \quad \text{für alle $x \in X$}.
 \]
 Die Aussage gilt also für $y = 0$.
 
 (\ref{enum: beschränkt um einen Punkt} $\Rightarrow$ \ref{enum: beschränkt um jeden Punkt}) Für alle $z \in \R^n$ setzen wir $C_z \coloneqq \|z-y\| + C$. Es ist dann
 \[
  \|z-x\| \leq \|z-y\| + \|y-x\| \leq \|z-y\| + C = C_z
  \quad 
  \text{für alle $x \in X$}.
 \]
 
 (\ref{enum: beschränkt um jeden Punkt} $\Rightarrow$ \ref{enum: beschränkt um 0}) Nach Annahme gibt es $C_0 > 0$ mit
 \[
  \|x\| = \|0-x\| \leq C_0
  \quad \text{für alle $x \in X$}.
 \]
\end{solution}





\section{Folgen}


\begin{defi}
 Es sei $(x_n)_{n \in \N}$ eine Folge auf $\R^m$, also $x_n \in \R^m$ für alle $n \in \N$, und $x \in \R^m$. Wir sagen, dass $(x_n)$ gegen $x$ konvergiert, falls es für alle $\varepsilon > 0$ ein $N \in \N$ gibt, so dass
 \[
  \|x - x_n\| < \varepsilon \quad \text{für alle $n \geq N$}.
 \]
 Wir schreiben dann $\lim_{n \to \infty} x_n = x$ oder $x_n \to x$ für $n \to \infty$.
\end{defi}


Wie schon auf $\R$ ergibt sich auch auf $\R^n$, dass Grenzwerte eindeutig sind.


\begin{lem}
 Es sei $(x_n)$ eine Folge auf $\R^m$. Sind $x, x' \in \R^m$ mit $x_n \to x$ und $x_n \to x'$ für $n \to \infty$, so ist $x = x'$.
\end{lem}
\begin{proof}
 Es sei $\varepsilon > 0$ beliebig aber fest. Da $\lim_{n \to \infty} x_n = x$ gibt es ein $N_1 \in \N$ mit $\|x - x_n\| < \varepsilon/2$  für alle $n \geq N_1$. Da $\lim_{n \to \infty} x_n = x'$ gibt es ein $N_2 \in \N$ mit $\|x' - x_n\| < \varepsilon/2$  für alle $n \geq N_2$. Für $N \coloneqq \max\{N_1, N_2\}$ ist deshalb für alle $n \geq N$
 \begin{align*}
  \|x-x'\|
  &= \|x-x_n + x_n-x'\| \\
  &\leq \|x-x_n\| + \|x_n-x'\|
  < \frac{\varepsilon}{2} + \frac{\varepsilon}{2}
  = \varepsilon.
 \end{align*}
 Also ist $\|x-x'\| < \varepsilon$ für alle $\varepsilon > 0$. Es muss also $\|x-x'\| = 0$, also $x-x' = 0$ und somit $x = x'$.
\end{proof}


Da die Norm auf $\R^1 = \R$ mit dem Betrag übereinstimmt, handelt es sich um eine Erweiterung des Konvergenzbegriffes auf $\R$. Die Konvergenz auf $\R^m$ lässt sich auch auf die Konvergenz in $\R$ zurückführen. Eine Folge auf $\R^m$ konvergiert nämlich genau dann, wenn sie in jeder Koordinate konvergiert.


\begin{lem}\label{lem: convergence in coordinates}
 Es sei $(x_n)$ eine Folge auf $\R^m$ und $x \in \R^m$. In Koordinaten sei $x_n = (x^{(1)}_n, \dotsc, x^{(m)}_n)$ für alle $n \in \N$ und $x = (x^{(1)}, \dotsc, x^{(m)})$. Dann ist genau dann $\lim_{n \to \infty} x_n = x$, wenn $\lim_{n \to \infty} x^{(i)}_n = x^{(i)}$ für alle $1 \leq i \leq m$.
\end{lem}
\begin{proof}
 Ist $x_n \to x$, so ist $\|x_n - x\| \to 0$. Für alle $1 \leq i \leq m$ ist
 \[
  0 \leq \left|x^{(i)}_n - x^{(i)}\right| \leq \|x_n - x\| \quad \text{für alle $n \in \N$}
 \]
 und daher nach dem Sandwich-Lemma $|x^{(i)}_n - x^{(i)}| \to 0$. Also ist $x^{(i)}_n \to x^{(i)}$ für alle $1 \leq i \leq m$.
 
 Ist andererseits $x^{(i)}_n \to x^{(i)}$ für alle $1 \leq i \leq m$, so ist $|x^{(i)}_n - x^{(i)}| \to 0$ für alle $1 \leq i \leq m$. Damit ist auch $\sum_{i=1}^m |x^{(i)}_n - x^{(i)}| \to 0$. Da
 \[
  0 \leq \|x_n - x\| \leq \sum_{i=1}^m \left|x^{(i)}_n - x^{(i)}\right| \quad \text{für alle $n \in \N$}
 \]
 ist daher nach dem Sandwich-Lemma auch $\|x_n - x\| \to 0$, also $x_n \to x$.
\end{proof}


\begin{bem}
 Für den Fall $\C \cong \R^2$ erhalten wir, dass eine Folge komplexer Zahlen $(z_n)$ genau dann gegen $z \in \C$ konvergiert, wenn die Folge der Realteile $(\Re(z))$ gegen $\Re(z)$ konvergiert und die Folge der Imaginärteile $(\Im(z))$ gegen $\Im(z)$ konvergiert.
\end{bem}


Viele der Eigenschaften und Rechenregeln, die wir für konvergente Folgen auf $\R$ kenne, lassen sich auf Folgen auf $\R^n$ zu verallgemeinern. So sind Grenzwerte auch im Mehrdimensionalen verträglich mit Addition und Skalarmultiplikation. Im Komplexen ergibt sich auch eine Verträglichkeit mit der Multiplikation und Division, wie wir sie schon aus dem Reellen kennen.


\begin{question}
 Es seien $(\xi_n)$ und $(\zeta_n)$ zwei konvergente Folgen komplexer Zahlen mit Grenzwerten $\xi \coloneqq \lim_{n \to \infty} \xi_n$ und $\zeta \coloneqq \lim_{n \to \infty} \zeta_n$. Zeigen Sie, dass auch die Folge $(\xi_n \cdot \zeta_n)$ konvergiert und
 \[
  \lim_{n \to \infty} (\xi_n \cdot \zeta_n)
  = \xi \cdot \zeta
  = \left( \lim_{n \to \infty} \xi_n \right) \cdot \left( \lim_{n \to \infty} \zeta_n \right).
 \]
\end{question}
\begin{solution}
 Es seien $\xi_n = u_n + i v_n$ und $\zeta_n = x_n + i y_n$ für alle $n \in \N$ die Zerlegungen in Real- und Imaginärteile, sowie $\xi = u + i v$ und $\zeta = x + i y$ die Zerlegungen der Grenzwerte in Real- und Imaginärteile. Dass $\xi_n \to \xi$ ist nach Lemma \ref{lem: convergence in coordinates} äquivalent dazu, dass $u_n \to u$ und $v_n \to v$, und dass $\zeta_n \to \zeta$ ist äquivalent dazu, dass $x_n \to x$ und $y_n \to y$. Nach Lemma \ref{lem: convergence in coordinates} und den bekannten Rechenregeln für die Grenzwerte reelle Folgen folgt damit, dass
 \begin{align*}
  \xi_n \cdot \zeta_n
  &= (u_n + i v_n) \cdot (x_n + i y_n)
  = (u_n x_n - v_n y_n) + i (u_n y_n + v_n x_n) \\
  &\to (ux - vy) + i(uy + vx)
  = (u + iv) \cdot (x + iy)
  = \xi \cdot \zeta.
  \qedhere
 \end{align*}
\end{solution}


Für Folgen auf $\R$ kennen wir bereits den Begriff der Cauchy-Folge, und wissen, dass eine Folge genau dann konvergiert, wenn sie eine Cauchy-Folge ist. Dies lässt sich auf Folgen auf $\R^n$ verallgemeinern:


\begin{defi}
 Eine Folge $(x_n)$ auf $\R^m$ heißt \emph{Cauchy-Folge}, falls es für alle $\varepsilon > 0$ ein $N \in \N$ gibt, so dass $\|x_n - x_{n'}\| < \varepsilon$ für alle $n, n' \geq N$.
\end{defi}


Wir haben bereits gesehen, dass eine Folge im Mehrdimensionalen genau dann konvergiert, wenn sie in jeder Koordinate konvergiert. Es ergibt sich auch, dass eine Folge im Mehrdimensionalen genau dann eine Cauchy-Folge ist, wenn sie in jeder Komponente eine Cauchy-Folge ist.


\begin{question}
 Es sei $(x_n)$ eine Folge auf $\R^m$ mit Koordinaten $x_n = (x^{(1)}_n, \dotsc, x^{(m)}_n)$ für alle $n \in \N$. Zeigen Sie, dass $(x_n)$ genau dann eine Cauchy-Folge ist, wenn $(x^{(i)}_n)$ für alle $1 \leq i \leq m$ eine Cauchy-Folge ist.
\end{question}
\begin{solution}
 Wir wissen bereits, dass für alle $n, n' \in \N$
 \begin{gather*}
  \left| x^{(i)}_n - x^{(i)}_{n'} \right| \leq \|x_n - x_{n'}\| \quad \text{für alle $1 \leq i \leq m$},
 \shortintertext{und dass}
  \|x_n - x_{n'}\| \leq \sum_{i=1}^m \left| x^{(i)}_n - x^{(i)}_{n'} \right|.
 \end{gather*}
 
 Ist $(x_n)$ eine Cauchy-Folge, so gibt es für alle $\varepsilon > 0$ ein $N \in \N$, so dass $\|x_n - x_{n'}\| < \varepsilon$ für alle $n, n' \geq N$, und somit auch $|x^{(i)}_n - x^{(i)}_{n'}| < \varepsilon$ für alle $n \geq N$ für alle $1 \leq i \leq m$. Also ist $(x^{(i)}_n)$ für alle $1 \leq i \leq m$ eine Cauchy-Folge.
 
 Es sei andererseits $(x^{(i)}_n)$ für alle $1 \leq i \leq n$ eine Cauchy-Folge. Wir wollen zeigen, dass dann auch $(x_n)$ eine Cauchy-Folge ist. Sei hierfür $\varepsilon > 0$ beliebig aber fest. Für jedes $1 \leq i \leq m$ gibt es, da $(x^{(i)}_n)$ eine Cauchy-Folge ist, ein $N_i \in \N$ mit $|x^{(i)}_n - x^{(i)}_{n'}| < \varepsilon/m$ für alle $n, n' \geq N_i$. Wir setzen $N \coloneqq \max_{1 \leq i \leq m} N_i$. Für alle $n, n' \geq N$ ist dann
 \[
  \|x_n - x_{n'}\|
  \leq \sum_{i=1}^m \left| x^{(i)}_n - x^{(i)}_{n'} \right|
  < \sum_{i=1}^m \frac{\varepsilon}{m}
  = \varepsilon.
 \]
 Aus der Beliebigkeit von $\varepsilon > 0$ folgt, dass $(x_n)$ eine Cauchy-Folge ist.
\end{solution}


\begin{prop}[Vollständigkeit von $\R^m$]
 Es sei $(x_n)$ eine Folge auf $\R^m$. Dann ist $(x_n)$ genau dann eine Cauchy-Folge, wenn $(x_n)$ konvergiert.
\end{prop}
\begin{proof}
 $(x_n)$ ist genau dann eine Cauchy-Folge, wenn $(x_n)$ in jeder Koordinate eine Cauchy-Folge ist. Wegen der Vollständigkeit von $\R$ ist dies äquivalent dazu, dass $(x_n)$ in jeder Koordinate konvergiert. Dies ist wiederum äquivalent dazu, dass $(x_n)$ auf $\R^m$ konvergiert.
\end{proof}


Wir kennen den Satz von Bolzano-Weierstraß, der besagt, dass jede beschränkte Folge auf $\R$ eine konvergente Teilfolge besitzt. Die naheliegende Verallgemeinerung gilt auch für $\R^n$.


\begin{defi}
 Eine Folge $(x_n)$ auf $\R^m$ heißt beschränkt, falls die Menge der Folgeglieder $\{x_n \mid n \in \N\}$ beschränkt ist, falls es also ein $C > 0$ gibt, so dass $\|x_n\| \leq C$ für alle $n \in \N$.
\end{defi}


\begin{prop}[Satz von Bolzano-Weierstraß]
 Jede beschränkte Folge $(x_n)_{n \in \N}$ auf $\R^m$ besitzt eine konvergente Teilfolge.
\end{prop}
\begin{proof}
 In Koordinaten sei $x_n = (x^{(1)}_n, \dotsc, x^{(m)}_n)$ für alle $n \in \N$. Da die Folge $(x_n)$ beschränkt ist, ist die Folge $(x^{(i)}_n)_{n \in \N}$ für alle $1 \leq i \leq m$ beschränkt. (Siehe Übung \ref{qst: Beschränktheit in Koordinaten}.)
 
 Da die Folge $(x^{(1)}_n)_{n \in \N}$ beschränkt ist, besitzt sie nach dem Satz von Bolzano-Weierstraß auf $\R$ eine konvergente Teilfolge $n_{1,j}$; es sei
 \[
  x^{(1)} \coloneqq \lim_{j \to \infty} x^{(1)}_{n_{1,j}}.
 \]
 Da die Folge $(x^{(2)}_{n_{1,j}})_{j \in \N}$ beschränkt ist, besitzt sie nach dem Satz von Bolzano-Weierstraß auf $\R$ eine konvergente Teilfolge $n_{2,j}$; wir setzen
 \[
  x^{(2)} \coloneqq \lim_{j \to \infty} x^{(2)}_{n_{2,j}}.
 \]
 Da $x^{(1)}_{n_{1,j}} \to x^{(1)}$ für $j \to \infty$ ist auch $x^{(1)}_{n_{2,j}} \to x^{(1)}$ für $j \to \infty$, da es sich um eine Teilfolge handelt.
 
 Wir erhalten rekursiv Teilfolgen $n_{1,j}$, \dots, $n_{m,j}$ und $x^{(1)}, \dotsc, x^{(m)} \in \R$, so dass für jedes $1 \leq k \leq m$
 \[
  x^{(i)} = \lim_{j \to \infty} x^{(i)}_{n_{k,j}} \quad \text{für alle $1 \leq i \leq k$}.
 \]
 Inbesondere ist $n_{m,j}$ eine Teilfolge, so dass $x^{(i)}_{n_{m,j}} \to x^{(i)}$ für $j \to \infty$ für jedes $1 \leq i \leq m$. Daher ist $x_{n_{m,j}} \to (x^{(1)}, \dotsc, x^{(m)})$ für $j \to \infty$. Als besitzt $(x_n)$ eine konvergente Teilfolge.
\end{proof}


Die Vollständigkeit von $\R^n$ lässt sich auch durch den Satz von Bolzano-Weierstraß zeigen.


\begin{question}
 Es sei $(x_n)$ eine Cauchy-Folge auf $\R^m$.
 \begin{enumerate}
  \item
   Zeigen Sie, dass $(x_n)$ beschränkt ist.
  \item
   Zeigen Sie, dass $(x_n)$ genau dann konvergiert, wenn $(x_n)$ eine konvergente Teilfolge besitzt.
  \item
   Folgern Sie, dass $(x_n)$ konvergiert.
 \end{enumerate}
\end{question}
\begin{solution}
 \begin{enumerate}
  \item
   Da $(x_n)$ eine Cauchy-Folge ist, gibt es ein $N \in \N$ mit
   \[
    \|x_n - x_{n'}\| < 1 \quad \text{für alle $n, n' \geq N$}.
   \]
   Insbesondere ist daher $\|x_n - x_N\| < 1$ für alle $n \geq N$. Daher ist
   \[
    \|x_n\|
    \leq \|x_N\| + \|x_n - x_N\|
    < \|x_N\| + 1
    \quad \text{für alle $n \geq N$}.
   \]
   Für
   \[
    C \coloneqq \max \{\|x_0\|, \|x_1\|, \dotsc, \|x_{N-1}\|, \|x_N\| + 1\}
   \]
   ist deshalb $\|x_n\| \leq C$ für alle $n \in \N$.
  \item
   Konvergiert $(x_n)$, so ist auch jede Teilfolge von $(x_n)$ konvergent (mit gleichen Grenzwert).
   
   Sei andererseits $n_j$ eine Teilfolge, so dass $(x_{n_j})_{j \in \N}$ konvergiert. Wir schreiben $x \coloneqq \lim_{j \to \infty} x_{n_j}$. Wir wollen zeigen, dass bereits $x_n \to x$. Es sei hierfür $\varepsilon > 0$. Da $(x_n)$ eine Cauchy-Folge ist, gibt es ein $N \in \N$ mit
   \begin{equation}
    \|x_n - x_{n'}\| < \frac{\varepsilon}{2} \quad \text{für alle $n, n' \geq N$}.
   \end{equation}
   Da $x_{n_j} \to x$ für $j \to \infty$ gibt es ein $J \in \N$ mit
   \[
    \|x - x_{n_j}\| < \frac{\varepsilon}{2} \quad \text{für alle $j \geq J$},
   \]
   wobei wir o.B.d.A.\ davon ausgehen können, dass $J \geq N$. Insbesondere ist $\|x - x_{n_J}\| < \varepsilon/2$.
   
   Da $J \geq N$ ist
   \[
    \|x_J - x_n\| < \frac{\varepsilon}{2} \quad \text{für alle $n \geq N$}.
   \]
   Damit erhalten wir, dass
   \[
    \|x - x_n\|
    \leq \|x - x_J\| + \|x_J - x_n\|
    < \frac{\varepsilon}{2} + \frac{\varepsilon}{2}
    = \varepsilon
    \quad \text{für alle $n \geq N$}.
   \]
   Wegen der Beliebigkeit von $\varepsilon > 0$ zeigt dies, dass $x_n \to x$.
  \item
   Da die Folge $(x_n)$ beschränkt ist, besitzt sie nach dem Satz von Bolzano-Weierstraß eine konvergente Teilfolge, und ist somit konvergent.
 \end{enumerate}
\end{solution}







\section{Stetigkeit}


Nachdem wir nun den Begriff der Folgenkonvergenz ins Mehrdimensionale verallgemeinert haben, wollen wir uns jetzt der Stetigkeit zuwenden.


\begin{defi}
 Es sei $X \subseteq \R^n$ und $f \colon X \to \R^m$. $f$ heißt \emph{stetig an der Stelle $x \in X$} falls es für alle $\varepsilon > 0$ ein $\delta > 0$ gibt, so dass
 \[
  \|x-y\| < \delta \Rightarrow \|f(x)-f(y)\| < \varepsilon \quad \text{für alle $y \in X$}.
 \]
\end{defi}


% TODO: Komposition stetiger Abbildungen und die üblichen Rechenregeln


Da die Norm auf $\R^1 = \R$ mit dem Betrag übereinstimmt ist dieser Stetigkeitsbegriff eine Verallgemeinerung der $\varepsilon$-$\delta$-Stetigkeit auf $\R$. Für den Fall $\R^2 \cong \C$ erhalten wir so auch einen Stetigkeitsbegriff für Abbildungen $\R^n \to \C$ und Abbildungen $\C \to \R^n$.


\begin{question}
 Zeigen Sie, dass die Norm
 \[
  f \colon \R^n \to \R, x \mapsto \|x\|
 \]
 stetig ist.
\end{question}
\begin{solution}
 Es sei $x \in \R^n$ ein beliebiger Punkt. Wir wollen zeigen, dass $f$ stetig an $x$ ist. Sei hierfür $\varepsilon > 0$ beliebig aber fest. Für alle $y \in \R^n$ ist nach der umgekehrten Dreiecksungleichung
 \[
  \|x-y\| \geq |\|x\|-\|y\||.
 \]
 Für $\delta \coloneqq \varepsilon$ ist daher $\|x-y\| < \varepsilon$ für alle $y \in \R^n$ mit $\|x-y\| < \delta$. Aus der Beliebigkeit von $\varepsilon > 0$ folgt, dass $f$ an $x$ stetig ist, und aus der Beliebigkeit von $x \in \R^n$ folgt, dass $f$ stetig ist.
\end{solution}


\begin{question}
 Zeigen Sie, dass die Projektion $\pi_i \colon \R^n \to \R, (x_1, \dotsc, x_n) \mapsto x_i$ für alle $1 \leq i \leq n$ stetig ist.
\end{question}
\begin{solution}
 Für den Beweis fixieren wir ein beliebiges $1 \leq i \leq n$. Es sei $x \in \R^n$ beliebig aber fest. Wir wollen zeigen, dass $\pi_i$ stetig an $x$ ist. Sei hierfür $\varepsilon > 0$ beliebig aber fest. Wir wissen, dass für alle $y = (y_1, \dotsc, y_n) \in \R^n$ die Ungleichung $|x_i - y_i| \leq \|x-y\|$ gilt. Für $\delta \coloneqq \varepsilon$ haben wir daher, dass für alle $y \in \R^n$ mit $\|x-y\| < \delta$
 \[
  |\pi_i(x)-\pi_i(y)| = |x_i - y_i| \leq \|x-y\| < \delta = \varepsilon.
 \]
 Wegen der Beliebigkeit von $\varepsilon > 0$ zeigt dies, dass $\pi_i$ stetig an $x$ ist, und wegen der Beliebigkeit von $x \in \R^n$ erhalten wir weiter, dass $\pi_i$ stetig ist.
\end{solution}


Eine wichtige Eigenschaft stetiger Funktionen besteht darin, dass die Einschränkung stetiger Funktionen wieder stetig ist:


\begin{question}
 Es seien $A \subseteq B \subseteq \R^n$ und $f \colon B \to \R^m$. Zeigen Sie: Ist $f$ stetig, so ist auch die Einschränkung $f|_A$ stetig. (Die Einschränkung ist als
 \[
  f|_A \colon A \to \R^m, a \mapsto f(a)
 \]
 definiert.)
\end{question}
\begin{solution}
 Es sei $x \in A$ beliebig aber fest. Wir wollen zeigen, dass $f$ stetig an $x$ ist. Sie hierfür $\varepsilon > 0$ beliebig aber fest. Da $A \subseteq B$ ist auch $x \in B$. Da $f \colon B \to \R^m$ stetig ist, gibt es ein $\delta > 0$, so dass
 \[
  \|x-b\| < \delta \Rightarrow \|f(x)-f(b)\| < \varepsilon
  \quad \text{für alle $b \in B$}.
 \]
 Da $A \subseteq B$ ist deshalb auch
 \[
  \|x-a\| < \delta \Rightarrow \|f(x)-f(a)\| < \varepsilon
  \quad \text{für alle $a \in A$}.
 \]
 Da $f|_A(a) = f(a)$ für alle $a \in A$ zeigt dies wegen der Beliebigkeit von $\varepsilon > 0$, dass $f|_A$ stetig an $x$ ist. Wegen der Beliebigkeit von $a \in A$ folgt damit, dass $f|_A$ stetig ist.
\end{solution}


Wir haben bereits gesehen, dass sich Stetigkeit auf $\R$ auch durch Folgenstetigkeit verstehen lässt. Dies gilt auch für Stetigkeit im Mehrdimensionalen.


\begin{lem}
 Es sei $X \subseteq \R^m$, $f \colon X \to \R^k$ und $x \in X$. Dann sind äquivalent:
 \begin{enumerate}
  \item
   $f$ ist stetig an $x$.
  \item
   Für jede Folge $(x_n)$ auf $X$ mit $x_n \to x$ konvergiert auch die Folge $(f(x_n))$ und es gilt
   \[
    \lim_{n \to \infty} f(x_n) = f(x) = f\left( \lim_{n \to \infty} x_n \right).
   \]
 \end{enumerate}
\end{lem}
\begin{proof}
 Angenommen $f$ ist stetig an $x$. Es sei $(x_n)$ eine Folge auf $X$ mit $x_n \to x$. Wir wollen zeigen, dass $f(x_n) \to f(x)$. Hierfür sei $\varepsilon > 0$ beliebig aber fest. Da $f$ stetig an $x$ ist, gibt es ein $\delta > 0$, so dass
 \[
  \|x-y\| < \delta \Rightarrow \|f(x) - f(y)\| < \varepsilon \quad \text{für alle $y \in X$}.
 \]
 Da $x_n \to x$ gibt es ein $N \in \N$ mit $\|x - x_n\| < \delta$ für alle $n \geq N$. Wir erhalten damit, dass $\|f(x) - f(x_n)\| < \varepsilon$ für alle $n \geq N$. Aus der Beliebigkeit von $\varepsilon > 0$ folgt damit, dass $f(x_n) \to f(x)$.
 
 Angenommen $f$ ist nicht stetig an $x$. Dann gibt es ein $\varepsilon > 0$, so dass es für alle $\delta > 0$ ein $y \in X$ gibt, so dass zwar $\|x-y\| < \delta$, aber $\|f(x)-f(y)\| \geq \varepsilon$. Insbesondere gibt es daher für alle $n \geq 1$ ein $x_n \in X$ mit $\|x - x_n\| < 1/n$ und $\|f(x) - f(y)\| \geq \varepsilon$. Es ist dann $(x_n)$ eine Folge auf $X$ mit $x_n \to x$, aber $(f(x_n))$ konvergiert nicht gegen $f(x)$, da $\|f(x) - f(x_n)\| \geq \varepsilon$ für alle $n \in \N$.
\end{proof}


Auch im Mehrdimensionalen bezeichnet man diese beiden Stetigkeitsbegriffe als \emph{$\varepsilon$-$\delta$-Stetigkeit} und \emph{Folgenstetigkeit}.


\begin{question}
 Zeigen Sie erneut die Stetigkeit der Projektionen
 \[
  \pi_i \colon \R^m \to \R, (x_1, \dotsc, x_m) \to x_i
 \]
 indem Sie zeigen, dass sie folgenstetig sind.
\end{question}
\begin{solution}
 Für den Beweis fixieren ein beliebiges $1 \leq i \leq m$. Es sei $x \in \R^n$ beliebig aber fest. Wir wollen zeigen, dass $\pi_i$ an $x$ folgenstetig ist. Sei hierfür $(x_n)$ eine Folge auf $\R^m$ mit $x_n \to x$; in Koordinaten sei $x_n = (x^{(1)}_n, \dotsc, x^{(m)}_n)$ für alle $n \in \N$ und $x = (x_1, \dotsc, x_m)$. Dass $x_n \to x$ bedeutet insbesondere, dass $x^{(i)}_n \to x^{(i)}$. Es ist also
 \[
  \pi_i(x_n) = x^{(i)}_n \to x.
 \]
 Das zeigt, dass $\pi_i$ folgenstetig an $x$ ist, und wegen der Beliebigkeit von $x \in \R^n$, dass $\pi_i$ folgenstetig ist.
\end{solution}


\begin{question}
 Untersuchen Sie, ob sich die Abbildung
 \[
  f \colon \R^2 \setminus \{0\} \to \R, (x,y) \mapsto \frac{xy}{x^2 + y^2}
 \]
 stetig auf $\R^2$ fortsetzen lässt, d.h. ob es eine stetige Funktion $\hat{f} \colon \R^2 \to \R$ gibt, so dass $\hat{f}|_{\R^2 \setminus \{0\}} = f$.
\end{question}
\begin{solution}
 Angenommen, $f$ ließe sich zu einer stetigen Funktion $\hat{f} \colon \R^2 \to \R$ fortsetzen. Wir betrachten die beiden Folgen $(x_n)_{n \geq 1}$ und $(y_n)_{n \geq 1}$ auf $\R^2 \setminus \{0\}$ mit
 \[
  x_n = \left( \frac{1}{n}, 0 \right)
  \quad
  \text{und}
  \quad
  y_n = \left( \frac{1}{n}, \frac{1}{n} \right)
  \quad
  \text{für alle $n \geq 1$}.
 \]
 Da $1/n \to 0$ ist $x_n \to (0,0)$ und $y_n \to (0,0)$ nach Lemma \ref{lem: convergence in coordinates}. Da $\hat{f}$ stetig ist konvergieren auch die Folgen $(\hat{f}(x_n))$ und $(\hat{f}(y_n))$, und es gilt
 \[
  \lim_{n \to \infty} \hat{f}(x_n) = \hat{f}(0,0) = \lim_{n \to \infty} \hat{f}(y_n).
 \]
 Da $\hat{f}|_{\R^2 \setminus \{0\}} = f$ ist $\hat{f}(x_n) = f(x_n)$ und $\hat{f}(y_n) = f(y_n)$ für alle $n \geq 1$. Also konvergieren die Folgen $(f(x_n))$ und $(f(y_n))$ und es ist
 \[
  \lim_{n \to \infty} f(x_n) = \hat{f}(0,0) = \lim_{n \to \infty} f(y_n).
 \]
 Wir haben jedoch
 \begin{gather*}
  \lim_{n \to \infty} f(x_n)
  = \lim_{n \to \infty} 0
  = 0
 \shortintertext{und}
  \lim_{n \to \infty} f(y_n)
  = \lim_{n \to \infty} \frac{\frac{1}{n^2}}{\frac{1}{n^2} + \frac{1}{n^2}}
  = \lim_{n \to \infty} \frac{1}{2}
  = \frac{1}{2}.
 \end{gather*}
 
 $f$ lässt sich also nicht stetig auf $\R^2 \setminus \{0\}$ fortsetzen.
\end{solution}


Wir haben bereits gesehen, dass sich Folgenkonvergenz auf $\R^n$ durch Konvergenz in den einzelnen Koordinaten beschreiben lässt. Eine analoge Aussage gilt auch für die Stetigkeit: Eine Funktion ist genau dann stetig, wenn sie in jeder Koordinate stetig ist.


\begin{lem}
 Es sei $X \subseteq \R^m$ und $f \colon X \to \R^k$ mit Koordinaten $f = (f_1, \dotsc, f_k)$, d.h. $f_1, \dotsc, f_k \colon X \to \R$ mit
 \[
  f(x) = (f_1(x), \dotsc, f_k(x)) \quad \text{für alle $x \in X$}.
 \]
 Dann ist $f$ genau dann stetig an der Stelle $x \in X$, falls $f_i$ für alle $1 \leq i \leq k$ stetig an der Stelle $x$ ist.
\end{lem}
\begin{proof}
 Wir wollen zeigen, dass $f$ genau dann folgenstetig an $x$ ist, wenn alle $f_i$ folgenstetig an $x$ sind. Hierfür sei $(x_n)$ eine Folge auf $X$ mit $x_n \to x$. Dass $f(x_n) \to f(x)$ bedeutet in Koordinaten, dass
 \[
  (f_1(x_n), \dotsc, f_k(x_n)) \to (f_1(x), \dots, f_k(x)).
 \]
 Dies ist nach Lemma \ref{lem: convergence in coordinates} äquivalent dazu, dass $f_i(x_n) \to f_i(x)$ für alle $1 \leq i \leq k$. Dass $f$ mit der Folge $(x_n)$ verträglich ist, ist also äquivalent dazu, dass die $f_i$ alle veträglich mit $(x_n)$ sind. Deshalb ist $f$ genau dann folgenstetig an $x$, wenn die $f_i$ alle folgenstetig an $x$ sind.
\end{proof}





\section{Umgebungen und offene Mengen}
Wir haben bereits kurz die Begriffe Umgebungen und offene Mengen auf $\R$ angesprochen. Auch diese Begriffe wollen wir auf $\R^n$ verallgemeinern. Wir wollen auch zeigen, wie sich Folgenkonvergenz und Stetigkeit durch Umgebungen und offene Mengen charakterisieren lassen.


\subsection{Definition und grundlegende Eigenschaften}


\begin{defi}
 Für $x \in \R^n$ und $\varepsilon > 0$ ist
 \[
  B_\varepsilon(x) \coloneqq \{y \in \R^n \mid \|x-y\| < \varepsilon\}
 \]
 der \emph{offene $\varepsilon$-Ball um $x$}.
\end{defi}


\begin{bem}
 Im Eindimensionalen ist $B_\varepsilon(x) = (x-\varepsilon,x+\varepsilon)$.
\end{bem}


\begin{bem}
 Für $x \in \R^n$ und $0 < \varepsilon \leq \varepsilon'$ ist $B_\varepsilon(x) \subseteq B_{\varepsilon'}(x)$ und für $0 < \varepsilon < \varepsilon'$ ist $B_\varepsilon(x) \subsetneq B_{\varepsilon'}(x)$.
\end{bem}


Mithilfe von $\varepsilon$-Bällen lassen sich die Begriffe einer Umgebung und einer offenen Menge auf das Mehrdimensionale verallgemeinern.


\begin{defi}
 Es sei $x \in \R^n$. Eine Teilmenge $V \subseteq \R^n$ heißt \emph{Umgebung von $x$}, falls es es ein $\varepsilon > 0$ gibt, so dass $B_\varepsilon(x) \subseteq V$.
\end{defi}


\begin{bsp}
 \begin{enumerate}
  \item
   Das Intervall $(-1,1)$ ist eine Umgebung von $0$, denn für jedes $\varepsilon > 0$ mit $\varepsilon < 1$ ist
   \[
    B_\varepsilon(0) = (-\varepsilon, \varepsilon) \subseteq (-1,1).
   \]
   (Wie wir in Kürze sehen können, ist $(-1,1)$ für jedes $x \in (-1,1)$ eine Umgebung.)
  \item
   Das Intervall $[-1,1]$ ist ebenfalls eine Umgebung von $0$. Es ist jedoch keine Umgebung von $-1$ oder $1$.
  \item
   Die nicht-negativen reellen Zahlen $[0,\infty)$ sind für jedes $x > 0$ eine Umgebung, nicht jedoch von $0$ selbst.
 \end{enumerate}
\end{bsp}


Die Umgebungen eines Punktes $x$ erfüllen gewissen Verträglichkeiten bzgl.\ passender Mengenoperationen und -relationen.


\begin{lem}\label{lem: Eigenschaften von Umgebungen}
 Es sei $x \in \R^n$.
 \begin{enumerate}
  \item
   $\R^n$ ist eine Umgebung von $x$.
  \item
   Ist $V \subseteq \R^n$ eine Umgebung von $x$ und $W \subseteq \R^n$ eine beliebige Teilmenge mit $V \subseteq W$, so ist auch $W$ eine Umgebung von $x$.
  \item
   Sind $V_1, \dotsc, V_s \subseteq \R^n$ Umgebungen von $x$, so ist auch $V_1 \cap \dotsb \cap V_s$ eine Umgebung von $x$.
 \end{enumerate}
\end{lem}
\begin{proof}
 \begin{enumerate}
  \item
   Für beliebige $\varepsilon > 0$ ist $B_\varepsilon(x) \subseteq \R^n$, also ist $\R^n$ eine Umgebung von $x$.
  \item
   Da $V$ eine Umgebung von $x$ ist, gibt es ein $\varepsilon > 0$ mit $B_\varepsilon(x) \subseteq V$. Da $V \subseteq W$ ist damit auch $B_\varepsilon(x) \subseteq W$. Also ist $W$ eine Umgebung von $x$.
  \item
   Da die $V_i$ Umgebungen von $x$ sind, gibt es für jedes $1 \leq i \leq s$ ein $\varepsilon_i  > 0$ mit $B_{\varepsilon_i}(x) \subseteq V_i$. Für $\varepsilon \coloneqq \min_{1 \leq i \leq s} \varepsilon_i > 0$ ist daher
   \[
    B_\varepsilon(x) \subseteq B_{\varepsilon_i}(x) \subseteq V_i
    \quad \text{für alle $1 \leq i \leq s$}.
   \]
   Also ist auch $B_\varepsilon(x) \subseteq V_1 \cap \dotsb \cap V_s$. Daher ist $V_1 \cap \dotsb \cap V_s$ eine Umgebung von $x$.
  \qedhere
 \end{enumerate}
\end{proof}


\begin{bem}
 Ist $X$ eine Menge, so ist ein \emph{Filter auf $X$} eine Kollektion von Teilmengen $\mc{F} \subseteq \mc{P}(X)$, so dass
 \begin{enumerate}
  \item
   $\mc{F}$ ist nicht leer,
  \item
   $\emptyset \notin \mc{F}$,
  \item
   ist $S \in \mc{F}$, so ist für alle $T \subseteq X$ mit $S \subseteq T$ auch $T \in \mc{F}$, und
  \item
   für alle $S_1, \dotsc, S_n \in \mc{F}$ ist auch $S_1 \cap \dotsb \cap S_n \in \mc{F}$.
 \end{enumerate}
 Lemma \ref{lem: Eigenschaften von Umgebungen} zeigt, dass die Umgebungen eines Punktes $x \in \R^n$ einen Filter bilden.
\end{bem}


\begin{defi}
 Eine Teilmenge $U \subseteq \R^n$ heißt \emph{offen}, falls es für alle $x \in U$ ein $\varepsilon > 0$ gibt, so dass $B_\varepsilon(x) \subseteq U$.
\end{defi}


\begin{bsp}
 Wir wollen zeigen, dass offene $\varepsilon$-Bälle offen sind, d.h. dass $B_\varepsilon(x)$ für alle $x \in \R^n$ und $\varepsilon > 0$ offen ist.
 
 Es sei hierfür $x \in X$ und $\varepsilon > 0$. Um zu zeigen, dass $B_\varepsilon(x)$ offen ist, müssen wir zeigen, dass es für jedes $y \in B_\varepsilon(x)$ ein $\delta > 0$ gibt, so dass $B_\delta(y) \subseteq B_\varepsilon(x)$. Sei hierfür $y \in B_\varepsilon(x)$ beliebig aber fest. Wir setzen
 \[
  \delta \coloneqq \varepsilon - |x-y|.
 \]
 Da $|x-y| < \varepsilon$ ist $\delta > 0$. Für alle $z \in B_\delta(y)$ ist
 \[
  |x-z|
  \leq |x-y| + |y-z|
  < |x-y| + \delta
  = |x-y| + \varepsilon - |x-y|
  = \varepsilon,
 \]
 und somit $z \in B_\varepsilon(x)$. Also ist $B_\delta(y) \subseteq B_\varepsilon(x)$. 
 \begin{center}
  \tikzsetnextfilename{epsilon_balls_are_open}
  \begin{tikzpicture}[scale=3]
   % big circle
   \draw[fill=black] (0,0) circle (0.02) node[below] {$x$};
   \draw[thick, dashed] (0,0) circle (1);
   \draw[<->] (0,0) to node[above] {$\varepsilon$} (1,0);
   % small circle
   \draw[fill=black] (0.5,0.5) circle (0.02) node[below] {$y$};
   \draw[thick, dashed] (0.5,0.5) circle ({1 - sqrt(0.5});
   \draw[<->] (0.5, 0.5) to node[above] {$\delta$} ({sqrt(0.5)}, {sqrt(0.5)});
  \end{tikzpicture}
 \end{center}
 Aus der Beliebigkeit von $y \in B_\varepsilon(x)$ folgt, dass $B_\varepsilon(x)$ offen ist.
\end{bsp}


\begin{bsp}
 \begin{enumerate}
  \item
   Offene Intervalle sind offen.
  \item
   Die Teilmenge $\Q \subseteq \R^n$ ist nicht offen: Für $x \in \Q$ und $\varepsilon > 0$ gibt es immer eine irrationale Zahl $y \in (x-\varepsilon, x+\varepsilon)$, weshalb $B_\varepsilon(x) \subsetneq \Q$. Also enthält $\Q$ um keine Zahl $x \in \Q$ einen $\varepsilon$-Ball. (In gewisser Weise ist $\Q$ maximal nicht-offen, da es um keinen seiner Punkte einen $\varepsilon$-Ball enthält.)
  \item
   Wir wollen zeigen, dass die Teilmenge
   \[
    U \coloneqq \{x \in \R^n \mid \|x\| > 1\}
   \]
   offen ist. Hierfür sei $x \in U$, also $\|x\| > 1$. Wir setzen
   \[
    \varepsilon \coloneqq \|x\|-1.
   \]
   Da $\|x\| > 1$ ist $\varepsilon > 0$. Wir behaupten, dass $B_\varepsilon(x) \subseteq U$. Dies ergibt sich daraus, dass für alle $y \in B_\varepsilon(x)$
   \begin{align*}
    \|y\|
    &= \|x - (x - y)\|
    \geq |\|x\| - \|x-y\||
    \geq \|x\| - \|y-x\| \\
    &> \|x\| - \varepsilon 
    = \|x\| - (\|x\| - 1)
    = 1,
   \end{align*}
   und somit $y \in U$. Also ist $B_\varepsilon(x) \subseteq U$. Wegen der Beliebigkeit von $x \in U$ folgt, dass $U$ offen ist.
  \item
   Die Teilmenge
   \[
    X \coloneqq \R \setminus \left\{ \frac{1}{n} \,\middle|\, n \in \N, n \geq 1 \right\}
   \]
   ist nicht offen, da sie keinen $\varepsilon$-Ball um $0$ enthält: Für jedes $\varepsilon > 0$ gibt es ein $n \in \N$, $n \geq 1$ mit $0 < 1/n < \varepsilon$, und somit $B_\varepsilon(x) \subsetneq X$.
  \item
   Das offene Quadrat $(0,1)^2 \subseteq \R^2$ ist offen.
  \item
   Die obere Halbebene
   \[
    \mathbb{H} \coloneqq \{(x,y) \in \R^2 \mid y > 0\}
   \]
   ist offen.
 \end{enumerate}
\end{bsp}


Vorstellungsmäßig ist eine Menge $U \subseteq \R^n$ offen, wenn sie keine „Randpunkte“ enthält. Diese Vorstellung lässt sich auch mathematisch präzise formulieren (und erweist sich dann als richtig), darauf werden wir allerdings nicht eigehen.


\begin{question}
 Es seien $a,b \in \R$ mit $a < b$. Zeigen Sie, dass das offene Intervall $(a,b)$ offen ist. Zeigen sie ferner, dass das abgeschlossene Intervall $[a,b]$ nicht offen ist.
\end{question}
\begin{solution}
 Da
 \[
  (a,b) = B_{\frac{b-a}{2}}\left(\frac{a+b}{2}\right)
 \]
 ist das offene Intervall $(a,b)$ offen.
 
 Um zu zeigen, dass das abgeschlossen Intervall $[a,b]$ nicht offen ist, betrachten wir den Punkt $a \in [a,b]$. Für jedes $\varepsilon > 0$ ist
 \[
  a - \frac{\varepsilon}{2} \in B_\varepsilon(a)
  \quad
  \text{aber}
  \quad
  a - \frac{\varepsilon}{2} \notin [a,b],
 \]
 und deshalb $B_\varepsilon(x) \nsubseteq [a,b]$. Also enthält $[a,b]$ keinen $\varepsilon$-Ball um $a$ und ist daher nicht offen. (Analog ergibt sich, dass $[a,b]$ auch keinen $\varepsilon$-Ball um $b$ enthält. Für jedes Punkt $x \in (a,b)$ hingegen enthält $[a,b]$ einen $\varepsilon$-Ball um $x$.)
\end{solution}


Analog zu den Eigenschaften von Umgebungen bzgl.\ Mengenrelationen und -operationen (Lemma \ref{lem: Eigenschaften von Umgebungen}) gibt es auch entsprechende Eigenschaften für offene Mengen.


\begin{lem}\label{lem: Eigenschaften von offenen Mengen}
 Offene Mengen erfüllen die folgenden Eigenschaften:
 \begin{enumerate}
  \item
   Die Teilmengen $\emptyset$ und $\R^n$ sind offen.
  \item
   Sind $U_1, \dotsc, U_s \subseteq \R^n$ offen, so ist auch der Schnitt $U_1 \cap \dotsb \cap U_s$ offen. Endliche Schnitte offener Mengen sind also offen.
  \item
   Für eine Kollektion $\{U_i \mid i \in I\}$ offener Mengen $U_i \subseteq \R^n$ ist auch die Vereinigung $\bigcup_{i \in I} U_i$ offen.
 \end{enumerate}
\end{lem}
\begin{proof}
 \begin{enumerate}
  \item
   Da die leere Menge keine Element enthält ist die Bedingung, dass es für alle $x \in \emptyset$ ein $\varepsilon > 0$ mit $B_\varepsilon(x) \subseteq \emptyset$ gibt, trivialerweise erfüllt. Dass $\R^n$ offen ist, ist ebenfalls klar, da $B_\varepsilon(x) \subseteq \R^n$ für alle $\varepsilon > 0$.
  \item
   Ist $U_1 \cap \dotsb \cap U_s = \emptyset$ so ist die Aussage klar. Ansonsten sei $x \in U_1 \cap \dotsb \cap U_s$. Dann ist $x \in U_i$ für alle $1 \leq i \leq s$. Da die $U_i$ offen sind, gibt es für jedes $1 \leq i \leq s$ ein $\varepsilon_i > 0$ mit $B_{\varepsilon_i}(x) \subseteq U_i$. Es sei $\varepsilon \coloneqq \min_{1 \leq i \leq s} \varepsilon_i$. Wir haben
   \[
    B_\varepsilon(x) \subseteq B_{\varepsilon_i}(x) \subseteq U_i
    \quad
    \text{für alle $1 \leq i \leq s$},
   \]
   und somit $B_\varepsilon(x) \subseteq U_1 \cap \dotsb \cap U_s$. Aus der Beliebigkeit von $x \in U_1 \cap \dotsb \cap U_s$ folgt, dass $U_1 \cap \dotsb \cap U_s$ offen ist.
  \item
   Es sei $x \in \bigcup_{i \in I} U_i$. Dann gibt es ein $j \in I$ mit $x \in U_j$. Da $U_j$ offen ist, gibt es ein $\varepsilon > 0$, so dass $B_\varepsilon(x) \subseteq U_j$. Es ist dann auch
   \[
    B_\varepsilon(x) \subseteq U_j \subseteq \bigcup_{i \in I} U_i.
   \]
   Aus der Beliebigkeit von $x \in \bigcup_{i \in I} U_i$ folgt, dass $\bigcup_{i \in I} U_i$ offen ist.
  \qedhere
 \end{enumerate}
\end{proof}


An dieser Stelle eine Warnung: Unendliche Schnitte offener Mengen sind nicht notwendigerweise offen. Man betrachte etwa die offenen Intervalle
\[
 U_n \coloneqq (-1-1/n, 1+1/n)  \quad \text{für alle $n \in \N$, $n \geq 1$}.
\]
Wir wissen bereits, dass die $U_n$ alle offen sind. Wir wissen aber auch, dass
\[
 \bigcap_{n=1}^\infty U_n = [-1,1]
\]
nicht offen ist.


Für die Definition von Umgebungen und offenen Mengen haben wir die Existenz von $\varepsilon$-Bällen verwendet. Es erscheint daher naheliegend, Umgebungen durch offene Mengen und offene Mengen durch Umgebungen auszudrücken.


\begin{question}
 Zeigen Sie, dass $V \subseteq \R^n$ genau dann eine Umgebung von $x \in \R^n$ ist, falls es eine offene Menge $U \subseteq \R^n$ gibt, so dass $x \in U$ und $U \subseteq V$.
\end{question}
\begin{solution}
 Ist $V$ ist eine Umgebung von $x$, so gibt es ein $\varepsilon > 0$ mit $B_\varepsilon(x) \subseteq V$. Dann ist $B_\varepsilon(x)$ eine offene Menge mit $x \in B_\varepsilon(x)$ und $B_\varepsilon(x) \subseteq V$.
 
 Angenommen, es gibt eine offene Menge $U \subseteq \R^n$ mit $x \in U$ und $U \subseteq V$. Da $U$ offen ist, gibt es ein $\varepsilon > 0$ mit $B_\varepsilon(x) \subseteq U$. Da $U \subseteq V$ ist damit auch $B_\varepsilon(x) \subseteq V$, also ist $V$ eine Umgebung von $x$.
\end{solution}


So wie sich Umgebungen durch offene Mengen charakterisieren lassen, kann man offene Mengen auch durch Umgebungen charakterisieren.


\begin{lem}
 Eine Teilmenge $U \subseteq \R^n$ ist genau dann offen, wenn $U$ für jedes $x \in U$ eine Umgebung ist.
\end{lem}
\begin{proof}
 $U$ ist genau dann eine Umgebung von $x \in U$, falls es ein $\varepsilon > 0$ gibt, so dass $B_\varepsilon(x) \subseteq U$. Dass $U$ für jedes $x \in U$ eine Umgebung ist, ist also äquivalent dazu, dass es für jedes $x \in U$ ein $\varepsilon > 0$ mit $B_\varepsilon(x) \subseteq U$ gibt. Dies bedeutet gerade, dass $U$ offen ist.
\end{proof}


\begin{question}
 Zeigen Sie, dass eine Teilmenge $U \subseteq \R^n$ genau dann offen ist, wenn sie die Vereinigung von $\varepsilon$-Bällen ist.
\end{question}
\begin{solution}
 Da $\varepsilon$-Bälle offen sind, und die Vereinigung offener Mengen ebenfalls wieder offen ist, sind Vereinigungen von $\varepsilon$-Bällen offen.
 
 Sei andererseits $U$ offen. Dann gibt es für jedes $x \in U$ ein $\varepsilon_x > 0$ mit $B_{\varepsilon_x}(x) \subseteq U$. Da
 \[
  U
  = \bigcup_{x \in U} \{x\}
  \subseteq \bigcup_{x \in U} B_{\varepsilon_x}(x)
  \subseteq U
 \]
 ist bereits $U = \bigcup_{x \in U} B_{\varepsilon_x}(x)$. Also ist $U$ die Vereinigung von $\varepsilon$-Bällen.
\end{solution}


\subsection{Charakterisierung von Folgenkonvergenz}


Per Definition ist für alle $x, y \in \R^n$ und $\varepsilon > 0$
\[
 \|x-y\| < \varepsilon \Leftrightarrow y \in B_\varepsilon(x).
\]
Wir können daher die Definition der Folgenkonvergenz durch $\varepsilon$-Bälle umschreiben: Ist $(x_n)$ eine Folge auf $\R^m$ und $x \in \R^m$, so ist
\begin{align*}
                 &\, \text{$(x_n)$ konvergiert mit $x_n \to x$ für $n \to \infty$   } \\
 \Leftrightarrow &\, \text{für alle $\varepsilon > 0$ gibt es $N \in \N$ mit $\|x - x_n\| < \varepsilon$ für alle $n \geq N$} \\
 \Leftrightarrow &\, \text{für alle $\varepsilon > 0$ gibt es $N \in \N$ mit $x_n \in B_\varepsilon(x)$ für alle $n \geq N$} \\
 \Leftrightarrow &\, \text{für alle $\varepsilon > 0$ ist $x_n \in B_\varepsilon(x)$ für fast alle $n \in \N$}.
\end{align*}


\begin{bem}
 Wir sagen, dass eine Eigenschaft \emph{für fast alle $n \in \N$} gilt, wenn sie nur für endlich viele $n$ nicht gilt. Eine Eigenschaft gilt nicht für fast alle $n \in \N$, genau dann wenn sie für unendlich viele $n \in \N$ nicht gilt.
\end{bem}


Aus der obigen Formulierung kann man durch Verwendung des Umgebungsbegriffes auch noch das $\varepsilon$ entfernen.


\begin{lem}
 Es sei $(x_n)$ eine Folge auf $\R^m$ und $x \in \R^m$. Dann sind äquivalent:
 \begin{enumerate}
  \item
   $(x_n)$ konvergiert mit $\lim_{n \to \infty} x_n = x$.
  \item
   Für jede Umgebung $V \subseteq \R^m$ von $x$ ist $x_n \in V$ für fast alle $n \in \N$.
 \end{enumerate}
\end{lem}
\begin{proof}
 Angenommen $(x_n)$ konvergiert gegen $x$. Ist $V \subseteq \R^m$ eine Umgebung von $x$, so gibt es ein $\varepsilon > 0$ mit $B_\varepsilon(x) \subseteq V$. Da $x_n \to x$ ist $x_n \in B_\varepsilon(x) \subseteq V$ für fast alle $n \in \N$.
 
 Angenommen $(x_n)$ konvergiert nicht gegen $x$. Dann gibt es ein $\varepsilon > 0$, so dass $x_n \notin B_\varepsilon(x)$ für unendlich viele $n \in \N$. $B_\varepsilon(x)$ ist aber eine Umgebung von $x$.
\end{proof}

\begin{bem}
 Diese Charakterisierung der Folgenkonvergenz ist erlaubt es häufig, Aussagen über konvergente Folgen auf abstraktere Weise zu zeigen als mit $\varepsilon$-Gehacke. Außerdem ermöglichst sie es, die Idee von Folgenkonvergenz auf abstraktere Settings zu verallgemeinern, in denen man zwar keinen Begriff von $\varepsilon$-Bällen mehr hat, aber noch von Umgebungen.
\end{bem}


\subsection{Charakterisierung von Stetigkeit}


Auch die Definition der Stetigkeit lässt sich durch $\varepsilon$-Bälle umschreiben: Für eine Abbildung $f \colon X \to \R^m$ mit Definitionsbereich $X \subseteq \R^n$ und $x \in X$ ist
\begin{align*}
                &\, \text{$f$ ist stetig an $x$} \\
 \Leftrightarrow&\, \forall \varepsilon > 0 \exists \delta > 0 \forall y \in X : \|x-y\| < \delta \Rightarrow \|f(x)-f(y)\| < \varepsilon \\
 \Leftrightarrow&\, \text{für jedes $\varepsilon > 0$ gibt es ein $\delta > 0$ mit $f(B_\delta(x) \cap X) \subseteq B_\varepsilon(f(x))$} \\
 \Leftrightarrow&\, \text{für jedes $\varepsilon > 0$ gibt es ein $\delta > 0$ mit $B_\delta(x) \cap X \subseteq f^{-1}(B_\varepsilon(f(x)))$}.
\end{align*}


\begin{question}
 Es sei $X \subseteq \R^n$ und $f \colon X \to \R^m$. Ein Punkt $x \in X$ heißt \emph{isolierter Punkt}, falls es ein $\varepsilon > 0$ gibt, so dass $B_\varepsilon(x) \cap X = \{x\}$. Zeigen Sie, dass $f$ an allen isolierten Punkten stetig ist.
\end{question}
\begin{solution}
 Es sei $x \in X$ ein isolierter Punkt. Dann gibt es ein $\delta > 0$ mit $B_\delta(x) \cap X = \{x\}$. Für beliebige $\varepsilon > 0$ ist deshalb
 \[
  f(B_\delta(x)) = f(\{x\}) = \{f(x)\} \subseteq B_\varepsilon(f(x)),
 \]
 und somit $f$ stetig an $x$.
\end{solution}


Auch hier lassen sich $\varepsilon$ und $\delta$ durch Verwendung von Umgebungen und offenen Mengen umgehen: Lokal lässt sich Stetigkeit durch Umgebungen charakterisieren und global durch offene Mengen.


\begin{lem}
 Es sei $X \subseteq \R^n$ und $f \colon X \to \R^m$.
 \begin{enumerate}
  \item
   Es sei $x \in X$. Dann sind äquivalent:
   \begin{enumerate}
    \item\label{enum: stetig an x epsilon delta}
     $f$ ist stetig an $x$.
    \item\label{enum: stetig an x Umgebungen}
     Für jede Umgebung $W \subseteq \R^m$ von $f(x)$ gibt es eine Umgebung $V \subseteq \R^n$ von $x$ mit
     \[
      f^{-1}(W) = V \cap X.
     \]
   \end{enumerate}
  \item
   Es sind äquivalent:
   \begin{enumerate}
    \item\label{enum: stetig epsilon delta}
     $f$ ist stetig.
    \item\label{enum: stetig offene Mengen}
     Für jede offene Menge $V \subseteq \R^m$ gibt es eine offene Menge $U \subseteq \R^n$ mit
     \[
      f^{-1}(V) = U \cap X.
     \]
   \end{enumerate}
 \end{enumerate}
\end{lem}


Vorstellungsmäßig bedeutet Stetigkeit an einem einzelnen Punkt, dass Urbilder von Umgebungen wieder Umgebungen sind; Stetigkeit auf dem ganzen Definitionsbereich bedeutet, dass Urbilder offener Mengen offen sind. Das Schneiden mit $X$ ergibt sich nur als technisches Detail (das leider den Beweis verhässlicht.)


\begin{proof}
 \begin{enumerate}
  \item
   (\ref{enum: stetig an x epsilon delta} $\Rightarrow$ \ref{enum: stetig an x Umgebungen}) Es sei $W \subseteq \R^m$ eine Umgebung von $f(x)$. Dann gibt es ein $\varepsilon > 0$ mit $B_\varepsilon(f(x)) \subseteq W$. Da $f$ stetig an $x$ ist, gibt es ein $\delta > 0$ mit
   \[
    B_\delta(x) \cap X \subseteq f^{-1}(B_\varepsilon(f(x))) \subseteq f^{-1}(W).
   \]
   Die Menge
   \[
    V \coloneqq B_\delta(x) \cup f^{-1}(W)
   \]
   ist daher eine Umgebung von $x$ mit
   \begin{align*}
    f^{-1}(W)
    &= f^{-1}(W) \cap X \\
    &\subseteq \left( B_\delta(x) \cup f^{-1}(W) \right) \cap X \\
    &= (B_\delta(x) \cap X) \cup (f^{-1}(W) \cap X) \\
    &\subseteq f^{-1}(W) \cup f^{-1}(W) \\
    &= f^{-1}(W),
   \end{align*}
   und somit
   \[
    f^{-1}(W) = (B_\delta(x) \cup f^{-1}(W)) \cap X = V \cap X.
   \]
   
   (\ref{enum: stetig an x Umgebungen} $\Rightarrow$ \ref{enum: stetig an x epsilon delta}) Es sei $\varepsilon > 0$ beliebig aber fest. Da $B_\varepsilon(f(x))$ eine Umgebung von $f(x)$ ist gibt es eine Umgebung $V \subseteq \R^n$ von $x$ mit $f^{-1}(B_\varepsilon(f(x))) = V \cap X$, also $f(V \cap X) \subseteq B_\varepsilon(x)$. Da $V$ eine Umgebung von $x$ ist, gibt es ein $\delta > 0$ mit $B_\delta(x) \subseteq V$. Da
   \[
    f(B_\delta(x) \cap X) \subseteq f(V \cap X) \subseteq B_\varepsilon(f(x))
   \]
   ist damit
   \[
    \|x-y\| < \delta \Rightarrow \|f(x)-f(y)\| < \varepsilon \quad \text{für alle $y \in X$}.
   \]
  \item
   Zum Beweis dieser Äquivalenz werden wir die gerade gezeigte Äquivalenz verwenden.
   
   (\ref{enum: stetig epsilon delta} $\Rightarrow$ \ref{enum: stetig offene Mengen}) Es sei $W \subseteq \R^m$ eine offene Menge. Da $f$ stetig ist, ist $f$ an jeder Stelle $x \in X$ stetig. Da $W$ offen ist, ist $W$ für jedes $y \in W$ eine Umgebung von $y$; also ist $W$ für jedes $x \in f^{-1}(W) \subseteq X$ eine Umgebung von $f(x)$.
   
   Nach dem ersten Teil des Lemmas gibt es für jedes $x \in f^{-1}(W)$ eine Umgebung $V_x$ von $x$ mit $f^{-1}(W) = V_x \cap X$. Da $V_x$ eine Umgebung von $x$ ist, gibt es für alle $x \in f^{-1}(W)$ ein $\varepsilon_x > 0$ mit $B_{\varepsilon_x}(x) \subseteq V_x$. Wir betrachten die offene Menge
   \[
    U \coloneqq \bigcup_{x \in f^{-1}(W)} B_{\varepsilon_x}(x).
   \]
   Es ist
   \begin{align*}
    f^{-1}(W)
    &= \bigcup_{x \in f^{-1}(W)} \{x\} \\
    &\subseteq \bigcup_{x \in f^{-1}(W)} B_{\varepsilon_x}(x) \cap X \\
    &\subseteq \bigcup_{x \in f^{-1}(W)} V_x \cap X \\
    &= \bigcup_{x \in f^{-1}(W)} f^{-1}(W)
     = f^{-1}(W),
   \end{align*}
   und damit bereits
   \[
    f^{-1}(W)
    = \bigcup_{x \in f^{-1}(W)} B_\varepsilon(x) \cap X
    =  U \cap X.
   \]
   
   (\ref{enum: stetig offene Mengen} $\Rightarrow$ \ref{enum: stetig epsilon delta}) Es sei $x \in X$. Es sei $W$ eine Umgebung von $f(x)$. Wir wollen zeigen, dass es eine Umgebung $V$ von $x$ gibt, so dass $f^{-1}(W) = V \cap X$.
   
   Da $W$ eine Umgebung von $f(x)$ ist, gibt es ein $\varepsilon > 0$ mit $B_\varepsilon(f(x)) \subseteq W$. Da $B_\varepsilon(f(x))$ offen ist gibt es nach Annahme eine offene Menge $U \subseteq \R^n$ mit $f^{-1}(B_\varepsilon(f(x))) = U \cap X$. Da $x \in f^{-1}(B_\varepsilon(f(x)))$ muss $x \in U$. Da $U$ offen ist, ist $U$ insbesondere eine Umgebung von $x$. Wir setzen
   \[
    V \coloneqq f^{-1}(W) \cup U.
   \]
   Da $U$ eine Umgebung von $x$ ist, und $U \subseteq V$, ist auch $V$ eine Umgebung von $x$. Für diese Umgebung $V$ von $x$ gilt
   \begin{align*}
    V \cap X
    &= (f^{-1}(W) \cup U) \cap X \\
    &= (f^{-1}(W) \cap X) \cup (U \cap X) \\
    &= f^{-1}(W) \cup f^{-1}(B_\varepsilon(f(x))) \\
    &\subseteq f^{-1}(W) \cup f^{-1}(W) \\
    &= f^{-1}(W).
   \end{align*}
   Das zeigt nach dem ersten Teil des Lemmas, dass $f$ stetig an $x$ ist. Wegen der Beliebigkeit von $x \in X$ zeigt dies, dass $f$ stetig ist.
  \qedhere
 \end{enumerate}
\end{proof}


\begin{bem}
 Wir erhalten insbesondere, dass eine Abbildung $f \colon \R^n \to \R^m$ genau dann stetig in $x \in \R^n$ ist, wenn für jede Umgebung $W \subseteq \R^m$ von $f(x)$ ihr Urbild $f^{-1}(W) \subseteq \R^n$ eine Umgebung von $x$ ist; wir erhalten ferner, dass $f$ genau dann stetig auf ganz $\R^n$ ist, wenn für jede offene Menge $U \subseteq \R^m$ auch ihr Urbild $f^{-1}(U) \subseteq \R^n$ offen ist.
\end{bem}


Die Charakterisierung von Stetigkeit durch Umgebungen und offene Mengen mag zunächst sehr kompliziert und unpraktisch erscheinen; ihn der Praxis erspart sie jedoch häufig Rechenarbeit: Statt mit einer Vielzahl von $\varepsilon$ und $\delta$ zu arbeiten und sich mit Indizes herumzuschlagen übersetzt man das Problem in die Sprache der Umgebungen und offenen Mengen und argumentiert dann damit, wie sich stetige Funktionen auf diese auswirken.


Eine häufige Anwendung dieser Charakterisierung besteht etwa darin, dass man die Offenheit einer Teilmenge $U \subseteq \R^n$ zeigt, indem man eine stetige Funktion $f \colon \R^n \to \R^m$ und eine offene Menge $V \subseteq \R^m$ konstuiert, so dass $U = f^{-1}(V)$; dies kann zu kurzen und eleganten Beweisen führen.


\begin{bsp}
 \begin{itemize}
  \item
   Wir wollen erneut zeigen, dass die Menge
   \[
    U \coloneqq \{x \in \R^n \mid \|x\| > 1\}
   \]
   offen ist. Hierfür betrachten wir die Norm
   \[
    f \colon \R^n \to \R, x \mapsto \|x\|.
   \]
   Wir wissen bereits, dass $f$ stetig ist. Da die Menge $(1,\infty) \subseteq \R$ offen ist, ist auch $U = f^{-1}((1,\infty))$ offen.
  \item
   Die Offenheit von $\varepsilon$-Bällen ergibt sich dadurch, dass für jedes $x \in \R^n$ die Abbildung
   \[
    f \colon \R^n \to \R, y \mapsto \|x-y\|
   \]
   stetig ist, die Menge $(0,\varepsilon) \subseteq \R$ offen ist, und $B_\varepsilon(x) = f^{-1}((0,\varepsilon))$.
 \end{itemize}
\end{bsp}


\begin{question}
 Zeigen Sie, dass die Menge
 \[
  U \coloneqq \{(x,y) \in \R^2 \mid x + 3y + 2 > y^2 - 6x^3 \}
 \]
 offen ist.
\end{question}
\begin{proof}
 Für alle $(x,y) \in \R^2$ ist
 \[
  x + 3y + 2 > y^2 - 6x^3
  \Leftrightarrow 6x^3 + x - y^2 + 3y > -2.
 \]
 Da die Projektionen $\R^2 \to \R, (x_1, x_2) \mapsto x_i$ für $i=1,2$ stetig sind, ist nach den üblichen Rechenregel für stetige Funktionen auch
 \[
  f \colon \R^2 \to \R, (x,y) \mapsto 6x^3 + x - y^2 + 3y
 \]
 stetig, da
 \[
  f = 6 \pi_1^3 + \pi_1 - \pi_2^2 + 3 \pi_2.
 \]
 Da $(-2,\infty) \subseteq \R$ offen ist, ergibt sich, dass auch $U = f^{-1}((-2,\infty))$ offen ist.
\end{proof}


\begin{question}
 Zeigen Sie, dass für alle $x \in \R$ und $\varepsilon > 0$ die Menge
 \[
  U \coloneqq \{y \in \R^n \mid \|x-y\| > \varepsilon\}
 \]
 offen ist.
\end{question}
\begin{solution}
 Die Abbildung $f \colon \R^n \to \R, y \mapsto \|x-y\|$ ist stetig, und die Teilmenge $(\varepsilon,\infty) \subseteq \R$ ist offen, also ist $U = f^{-1}((\varepsilon,\infty))$ offen.
\end{solution}





\section{Abgeschlossene Mengen}


Zuletzt wollen wir noch auf den Begriff einer abgeschlossenen Menge eingehen; wie der Name bereits vermuten lässt gibt es einen engen Zusammehang zwischen offenen und abgeschlossenen Mengen.


\begin{defi}
 Eine Teilmenge $A \subseteq \R^n$ heißt \emph{abgeschlossen}, falls ihr Komplement $A^c \subseteq \R^n$ offen ist.
\end{defi}


\begin{bsp}
 \begin{itemize}
  \item
   Für alle $a,b \in \R$ mit $a < b$ ist das abgeschlossene Intervall $[a,b]$ abgeschlossen, denn $[a,b]^c = (-\infty,a) \cup (b, \infty)$ ist offen.
  \item
   Für alle $x \in \R^n$ und $\varepsilon > 0$ ist der \emph{abgeschlossene $\varepsilon$-Ball}
   \[
    \overline{B}_\varepsilon(x) \coloneqq \{y \in \R^n \mid \|x-y\| \leq \varepsilon\}
   \]
   abgeschlossen, denn
   \[
    \overline{B}_\varepsilon(x)^c = \{y \in \R^n \mid \|x-y\| > \varepsilon\}
   \]
   ist offen.
  \item
   Punkte sind abgeschlossen, d.h.\ für alle $x \in \R^n$ ist die einpunktige Menge $\{x\} \subseteq \R^n$ abgeschlossen: Es ist $\{x\}^c = \R^n \setminus \{x\}$. Für $y \in \R^n \setminus \{x\}$ ist $y \neq x$ und somit $\varepsilon \coloneqq \|x-y\| > 0$. Da $x \notin B_\varepsilon(y)$ ist $B_\varepsilon(y) \subseteq \R^n \setminus \{x\}$. Das zeigt, dass $\R^n \setminus \{x\}$ offen ist, also $\{x\}$ abgeschlossen.
  \item
   Das offene Intervall $(0,1) \subseteq \R$ ist nicht abgeschlossen, denn
   \[
    (0,1)^c = (-\infty,0] \cup [1,\infty)
   \]
   ist nicht offen (die Menge enthält keine $\varepsilon$-Bälle um $0$ oder $1$).
  \item
   Da $\emptyset$ und $\R^n$ offen sind, die beide Mengen auch abgeschlossen. Es lässt sich zeigen, dass dies die einzigen Teilmengen von $\R^n$ sind, die sowohl offen als auch abgeschlossen sind.
 \end{itemize}
\end{bsp}


Offene Mengen kann man sich als die Mengen vorstellen, die keine Randpunkte enthalten. Dual dazu kann man sich abgeschlossene Mengen als jene Mengen vorstellen, die alle ihre Randpunkte enthalten. Diese Aussage werden wir durch Lemma \ref{lem: Folgenabgeschlossenheit} präzisieren.


Man beachte, dass nicht jede Teilmenge $X \subseteq \R^n$ offen oder abgeschlossen ist. $[0,1) \subseteq \R$ etwa ist nicht offen (da die Menge keinen $\varepsilon$-Ball um $0$ enthält) aber auch nicht abgeschlossen, da
\[
 [0,1)^c = (-\infty,0) \cup [1,\infty)
\]
nicht offen ist (da die Menge keinen $\varepsilon$-Ball um $1$ enthält).


So wie sich abgeschlossen Mengen in gewisser Weise dual zu offenen Mengen verhalten, ergeben sich dual zu Lemma \ref{lem: Eigenschaften von offenen Mengen} auch entsprechende Aussagen über abgeschlossene Mengen:


\begin{question}
 Zeigen Sie:
 \begin{enumerate}
  \item
   $\emptyset, \R^n \subseteq \R^n$ sind abgeschlossen.
  \item
   Für abgeschlossene Mengen $A_1, \dotsc, A_s \subseteq \R^n$ ist auch $A_1 \cup \dotsb \cup A_s$ abgeschlossen. Endliche Vereinigungen abgeschlossener Mengen sind also abgeschlossen.
  \item
   Ist $\{A_i\}_{i \in I}$ eine Kollektion von abgeschlossenen Teilmengen $A_i \subseteq \R^n$, so ist auch der Schnitt $\bigcap_{i \in I} A_i \subseteq \R^n$ abgeschlossen. Beliebige Schnitter abgeschlossener Mengen sind also abgeschlossen.
 \end{enumerate}
\end{question}
\begin{solution}
 \begin{enumerate}
  \item
   Da $\emptyset^c = \R^n$ offen ist, ist $\emptyset$ abgeschlossen, und da $(\R^n)^c = \emptyset$ offen ist, ist auch $\R^n$ abgeschlossen.
  \item
   Da $A_1$, \dots, $A_s$ abgeschlossen sind, sind $A_1^c$, \dots, $A_s^c$ offen. Also ist auch $A_1^c \cap \dotsb \cap A_s^c$ offen. Da
   \[
    A_1^c \cap \dotsb \cap A_s^c
    = (A_1 \cup \dotsb \cup A_s)^c
   \]
   ist deshalb $A_1 \cup \dotsb \cup A_s$ abgeschlossen.
  \item
   Da die $A_i$ abgeschlossen sind, ist $A_i^c$ für alle $i \in I$ offen. Also ist auch
   \[
    \left( \bigcap_{i \in I} A_i \right)^c
    = \bigcup_{i \in I} A_i^c
   \]
   offen, und somit $\bigcap_{i \in I} A_i$ abgeschlossen.
 \end{enumerate}
\end{solution}


\begin{bem}
 Da unendliche Schnitte offener Mengen nicht mehr offen seien müssen, sind auch unendliche Vereinigung abgeschlossener Mengen nicht notwendigerweise abgeschlossen. So sind etwa die abgeschlossen Intervalle $A_n$ mit
 \[
  A_n \coloneqq \left[ \frac{1}{n}, 3-\frac{1}{n} \right]
  \quad \text{für alle $n \in \N$, $n \geq 1$}
 \]
 abgeschlossen, aber
 \[
  \bigcup_{n=1}^\infty A_n
  = \bigcup_{n=1}^\infty \left[ \frac{1}{n}, 3-\frac{1}{n} \right]
  = (0, 3)
 \]
 ist nicht abgeschlossen.
\end{bem}


\begin{question}
 Es seien $M$ und $N$ Mengen und $f \colon M \to N$ eine beliebige Funktion. Zeigen Sie, dass für alle $Y \subseteq N$
 \[
  f^{-1}(Y^c) = (f^{-1}(Y))^c.
 \]
 (Also ist $f^{-1}(N \setminus Y) = M \setminus f^{-1}(Y)$.)
\end{question}
\begin{solution}
 Für alle $x \in M$ ist
 \[
  x \in f^{-1}(Y^c)
  \Leftrightarrow f(x) \in Y^c
  \Leftrightarrow f(x) \notin Y
  \Leftrightarrow x \notin f^{-1}(Y)
  \Leftrightarrow x \in (f^{-1}(Y))^c.
 \]
\end{solution}


Wie wir bereits gesehen haben, ist eine Abbildung $f \colon \R^n \to \R^m$ genau dann offen, wenn für jede offene Menge $U \subseteq \R^m$ auch das Urbild $f^{-1}(U) \subseteq \R^n$ offen ist. Analog können wir Stetigkeit auch durch abgeschlossene Mengen charakterisieren.


\begin{lem}
 Es sei $X \subseteq \R^n$ und $f \colon X \to \R^m$. Dann ist $f$ genau dann stetig, falls es für jede abgeschlossene Teilmenge $C \subseteq \R^m$ eine abgeschlossen Teilmenge $A \subseteq \R^n$ gibt, so dass $f^{-1}(C) = A \cap X$.
 
 Insbesondere ist eine Abbildung $f \colon \R^n \to \R^m$ genau dann stetig, wenn die Urbilder abgeschlossener Mengen ebenfalls abgeschlossen sind.
\end{lem}
\begin{proof}
 Angenommen $f$ ist stetig. Es sei $C \subseteq \R^m$ abgeschlossen. Da $C$ abgeschlossen ist, ist $C^c$ offen. Da $f$ stetig ist, gibt es eine offene Menge $U \subseteq \R^n$ mit $f^{-1}(C^c) = U \cap X$. Daher ist
 \[
  f^{-1}(C)
  = X \setminus (X \setminus f^{-1}(C))
  = X \setminus f^{-1}(C^c)
  = X \setminus (U \cap X)
  = U^c \cap X.
 \]
 Da $U$ offen ist, ist $U^c$ abgeschlossen.
 
 Es sei andererseits $U \subseteq \R^m$ offen. Dann ist $U^c$ abgeschlossen, es gibt also nach Annahme eine abgeschlossene Teilmenge $A \subseteq \R^n$ mit $f^{-1}(U^c) = A \cap X$. Analog zur vorherigen Rechnung ergibt sich, dass
 \[
  f^{-1}(U) = A^c \cap X;
 \]
 da $A$ abgeschlossen ist, ist $A^c$ offen, also ist $f$ stetig.
\end{proof}


Diese Charakterisierung stetiger Abbildungen erlaubt es uns mit wenig Arbeit die Abgeschlossenheit einer Menge zu überprüfen.


\begin{bsp}
 \begin{itemize}
  \item
   Die $n$-dimensionale Sphäre
   \[
    S^n \coloneqq \{x \in \R^{n+1} \mid \|x\| = 1\}
   \]
   ist abgeschlossen, denn die Norm
   \[
    f \colon \R^{n+1} \to \R, x \mapsto \|x\|
   \]
   ist stetig, $\{1\} \in \R$ ist abgeschlossen und $S^n = f^{-1}(\{1\})$.
  \item
   Der $n$-dimensionale Einheitswürfel $[0,1]^n \subseteq \R^n$ ist abgeschlossen: Für alle $1 \leq i \leq n$ definieren wir
   \[
    A_i \coloneqq \{(x_1, \dotsc, x_n) \in \R^n \mid 0 \leq x_i \leq 1\}.
   \]
   Es ist klar, dass $[0,1]^n = A_1 \cap \dotsb \cap A_n$. Da (beliebige) Schnitte abgeschlossener Mengen wieder abgeschlossen sind, genügt es zu zeigen, dass die $A_i$ abgeschlossen sind. Dies ergibt sich daraus, dass die kanonischen Projektionen $\pi_i \colon \R^n \to \R$ stetig sind, das Einheitsintervall $[0,1] \subseteq \R$ abgeschlossen ist und $A_i = \pi_i^{-1}([0,1])$ für alle $1 \leq i \leq n$.
 \end{itemize}
\end{bsp}


\begin{question}
 Zeigen Sie, dass der „unendliche Zylinder“
 \[
  Z \coloneqq S^1 \times \R = \{(x,y,z) \in \R^3 \mid \|(x,y)\| = 1\}
 \]
 abgeschlossen ist.
\end{question}
\begin{solution}
 Die Abbildung
 \[
  f \colon \R^3 \to \R, (x,y,z) \mapsto \sqrt{x^2 + y^2}
 \]
 ist stetig, da $f = \sqrt{\pi_1^2 + \pi_2^2}$. Da $\{1\} \subseteq \R$ abgeschlossen ist, ist also auch der Zylinder $Z = f^{-1}(\{1\})$ abgeschlossen.
\end{solution}


Eine wichtige Eigenschaft abgeschlossener Mengen besteht darin, dass sie unter Grenzwerten von Folgen abgeschlossen sind.


\begin{lem}\label{lem: Folgenabgeschlossenheit}
 Es sei $A \subseteq \R^m$. Dann sind äquivalent:
 \begin{enumerate}
  \item\label{enum: topologisch abgeschlossen}
   $A$ ist abgeschlossen.
  \item\label{enum: folgenabgeschlossen}
   Für jede Folge $(a_n)$ auf $A$, die auf $\R^m$ konvergiert, ist auch $\lim_{n \to \infty} a_n \in A$.
 \end{enumerate}
\end{lem}
\begin{proof}
 (\ref{enum: topologisch abgeschlossen} $\Rightarrow$ \ref{enum: folgenabgeschlossen}) Angenommen $A$ ist abgeschlossen. Es sei $(x_n)$ eine konvergente Folge auf $\R^m$ mit $x_n \to x$ aber $x \notin A$. Da $x \notin A$ ist $x \in A^c$, und da $A$ abgeschlossen ist, ist $A^c$ offen. Es gibt also ein $\varepsilon > 0$ mit $B_\varepsilon(x) \subseteq A^c$. Da $x_n \to x$ ist $x_n \in B_\varepsilon(x)$ für fast alle $n \in \N$. Also ist $x_n \in A^c$ für fast alle $n \in \N$, bzw.\ äquivalent $x_n \in A$ für nur endlich viele $n \in \N$. Insbesondere kann $(x_n)$ keine Folge auf $A$ sein.
 
 (\ref{enum: folgenabgeschlossen} $\Rightarrow$ \ref{enum: topologisch abgeschlossen}) Angenommen $A$ ist nicht abgeschlossen. Dann ist $A^c$ nicht offen. Es gibt also ein $x \in A^c$, so dass $B_\varepsilon(x) \nsubseteq A^c$ für alle $\varepsilon > 0$, bzw. äquivalent $B_\varepsilon(x) \cap A \neq \emptyset$ für alle $\varepsilon > 0$. Insbesondere gibt es daher für alle $n \geq 1$ ein $a_n \in A$ mit $\|x-a_n\| < 1/n$. Dann ist per Konstruktion $a_n \to x$, aber da $x \in A^c$ ist $x \notin A$.
\end{proof}


\begin{bsp}
 \begin{itemize}
  \item
   Wir betrachten noch einmal die $m$-dimensionale Sphäre
   \[
    S^m = \{x \in \R^{m+1} \mid \|x\| = 1\}.
   \]
   Ist $(x_n)$ eine Folge auf $S^m$, die gegen ein $x \in \R^{m+1}$ konvergiert, so ist $\|x_n\| = 1$ für alle $n \in \N$. Wegen der Stetigkeit der Norm $\|\cdot\|$ ist dann auch
   \[
    \|x\|
    = \|\lim_{n \to \infty} x_n\|
    = \lim_{n \to \infty} \|x_n\|
    = \lim_{n \to \infty} 1
    = 1,
   \]
   also $x \in S^m$. Also ergibt sich auch durch diese Charakterisierung der Abgeschlossenheit, dass $S^m$ abgeschlossen ist.
  \item
   Allgmeiner ergibt sich so auch die Aussage, dass Urbilder abgeschlossener Mengen unter stetigen Funktionen wieder abgeschlossen sind: Es sei $f \colon \R^m \to \R^k$ stetig und $A \subseteq \R^k$ abgeschlossen. Für jede Folge $(x_n)$ auf $f^{-1}(A)$ ist dann $f(x_n) \in A$ für alle $n \in \N$. Konvergiert $(x_n)$ gegen ein $x \in \R^m$, so konvergiert $f(x_n)$ wegen der Stetigkeit von $f$ gegen $f(x)$. Da $A$ abgeschlossen ist muss auch $f(x) \in A$. Damit ist dann auch $x \in f^{-1}(A)$.
  \item
   Auch die Abgeschlossenheit des Einheitswürfels $[0,1]^m \subseteq \R^m$ ergibt sich durch die Abgeschlossenheit unter Grenzwerten von Folgen: Es sei $(x_n)$ eine Folge auf $[0,1]^m$, die gegen ein $x \in \R^m$ konvergiert. In Koordinaten sei $x_n = (x^{(1)}_n, \dotsc, x^{(m)}_n)$ für alle $n \in \N$ und $x = (x^{(1)}, \dotsc, x^{(m)})$. Da $x_n \to x$ ist $x^{(i)}_n \to x^{(i)}$ für alle $1 \leq i \leq m$. Da $0 \leq x^{(i)}_n \leq 1$ für alle $n \in \N$ ist auch $0 \leq x^{(i)} \leq 1$ für alle $0 \leq i \leq m$. Also ist auch $x \in [0,1]^m$.
 \end{itemize}
\end{bsp}


Das letzte Beispiel lässt sich Verallgemeinern:


\begin{question}
 Es seien $A_1, \dotsc, A_m \subseteq \R$ abgeschlossen. Zeigen Sie, dass $A_1 \times \dotsb \times A_m \subseteq \R^m$ abgeschlossen ist.
\end{question}
\begin{solution}
 Es sei $(x_n)$ eine Folge auf $A_1 \times \dotsb \times A_m$ die gegen $x \in \R^m$ konvergiert. In Koordinaten gelte $x_n = (x^{(1)}_n, \dotsc, x^{(m)}_n)$ für alle $n \in \N$ sowie $x = (x^{(1)}, \dotsc, x^{(m)})$. Da $x_n \to x$ ist $x^{(i)}_n \to x^{(i)}$ für alle $1 \leq i \leq m$. Für alle $1 \leq i \leq m$ ist $x^{(i)}_n \in A_i$ für alle $n \in \N$, und da $A_i$ abgeschlossen ist damit auch $x^{(i)} \in A_i$. Es ist also $x \in A_1 \times \dotsb \times A_m$.
\end{solution}


\begin{question}
 Es sei $f \colon \R \to \R$ eine stetige Funktion. Zeigen Sie, dass der Graph
 \[
  G \coloneqq \{(x,f(x)) \mid x \in \R\}
 \]
 abgeschlossen ist.
\end{question}
\begin{solution}
 Wir geben zwei Beweise an: Zuerst einen durch Folgenstetigkeit und dann einen durch Urbilder abgeschlossener Mengen.
 
 Es sei $(y_n)$ eine Folge auf $G$ mit $y_n \to y$ für $y \in \R^2$. Wir wollen zeigen, dass $y \in G$. In Koordinaten sei $y_n = (y^{(1)}_n, y^{(2)}_n)$ für alle $n \in \N$ und $y = (y^{(1)}, y^{(2)})$. Da $(y_n)$ eine Folge auf $G$ ist, ist $y^{(2)}_n = f(y^{(1)}_n)$ für alle $n \in \N$. Schreiben wir $x_n \coloneqq y^{(1)}_n$, so ist also $y_n = (x_n, f(x_n))$ für alle $n \in \N$.
 
 Da $y_n \to y$ ist $x_n \to y^{(1)}$ und $f(x_n) \to y^{(2)}$. Da $f$ stetig ist, folgt aus $x_n \to y^{(1)}$, dass $f(x_n) \to f(y^{(1)})$. Aus der Eindeutigkeit von Grenzwerten folgt damit, dass
 \[
  y^{(2)} = \lim_{n \to \infty} f(x_n) = f\left(y^{(1)}\right).
 \]
 Also ist $y = (y^{(1)}, y^{(2)}) = (y^{(1)}, f(y^{(1)}))$ und somit auch $y \in G$. Das zeigt, dass $G$ abgeschlossen ist.
 
 Alternativ bemerken wir, dass die Abbildung
 \[
  g \colon \R^2 \to \R, (x,y) \mapsto f(x)-y
 \]
 stetig ist, da $g = (f \circ \pi_1) - \pi_2$, und $G = g^{-1}(\{0\})$ für die abgeschlossene Menge $\{0\} \subseteq \R$.
\end{solution}





\newpage

\section{Lösungen}

\printsolutions



\end{document}
