\documentclass[a4paper,10pt]{article}
%\documentclass[a4paper,10pt]{scrartcl}

\usepackage{../mystyle}

\setromanfont[Mapping=tex-text]{Linux Libertine O}
% \setsansfont[Mapping=tex-text]{DejaVu Sans}
% \setmonofont[Mapping=tex-text]{DejaVu Sans Mono}

\title{Konvexität der Exponentialfunktion}
\author{Jendrik Stelzner}
\date{\today}

\begin{document}
\maketitle


\begin{defi}
 Eine Abbildung $f \colon \R \to \R$ heißt konvex, falls für alle $x,y \in \R$
 \[
  f(\lambda x + (1-\lambda)y) \leq \lambda f(x) + (1-\lambda) f(y)
  \quad \text{für alle $\lambda \in [0,1]$}.
 \]
\end{defi}


\begin{question}
 Es sei $f, g \colon \R \to \R$ konvex. Zeigen Sie: Ist $g$ monoton steigend, so ist auch $g \circ f$ konvex.
\end{question}
\begin{solution}
 Es seien $x,y \in \R$ und $\lambda \in [0,1]$. Da $f$ konvex ist, ist
 \[
  f(\lambda x + (1-\lambda)y) \leq \lambda f(x) + (1-\lambda) f(y).
 \]
 Wegen der Monotonie von $g$ ergibt sich damit, dass
 \[
  g(f(\lambda x + (1-\lambda)y)) \leq g(\lambda f(x) + (1-\lambda) f(y)).
 \]
 Durch die Konvexität von $g$ erhalten wir auch, dass
 \[
  g(\lambda f(x) + (1-\lambda) f(y))
  \leq \lambda g(f(x)) + (1-\lambda) g(f(y)).
 \]
 Insgesamt erhalten wir damit, ass
 \begin{align*}
      &\, (g \circ f)(\lambda x + (1-\lambda) y) \\
     =&\, g(f(\lambda x + (1-\lambda) y)) \\
  \leq&\, g(\lambda f(x) + (1-\lambda) f(y)) \\
  \leq&\, \lambda g(f(x)) + (1-\lambda) g(f(y)) \\
     =&\, \lambda (g \circ f)(x) + (1-\lambda) (g \circ f)(y).
 \end{align*}
\end{solution}


Wir wollen nun zeigen, dass die Exponentialfunktion $\exp \colon \R \to \R$ konvex ist. Wir erinnern daran, dass
\[
 \exp(x) = \sum_{k=0}^\infty \frac{x^k}{k!} \quad \text{für alle $x \in \R$},
\]
und dass
\begin{equation}\label{eqn: exp Gruppenhomo}
 \exp(x+y) = \exp(x) \exp(y) \quad \text{für alle $x,y \in \R$}.
\end{equation}
Außerdem ist $\exp(x) > 0$ für alle $x \in \R$.


\begin{question}
 Zeigen Sie, dass $\exp$ auf $[0,\infty)$ konvex ist, d.h. dass für alle $x,y \in [0,\infty)$
 \[
  \exp(\lambda x + (1-\lambda) y)
  \leq \lambda \exp(x) + (1-\lambda) \exp(y)
  \quad
  \text{für alle $\lambda \in [0,1]$}.
 \]
 (\emph{Hinweis}: Nutzen Sie, dass die Abbildungen $x \mapsto x^k$ auf $[0,\infty)$ konvex sind.)
\end{question}


\begin{question}
 Folgern Sie, dass $\exp$ konvex ist. (\emph{Hinweis:} Nutzen Sie \eqref{eqn: exp Gruppenhomo} um Translationen von $\exp$ in Skalierungen zu transformieren.)
\end{question}






\newpage


\printsolutions





\end{document}
