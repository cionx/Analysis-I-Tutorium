\documentclass[a4paper,10pt]{article}
%\documentclass[a4paper,10pt]{scrartcl}

\usepackage{../mystyle}

\setromanfont[Mapping=tex-text]{Linux Libertine O}
% \setsansfont[Mapping=tex-text]{DejaVu Sans}
% \setmonofont[Mapping=tex-text]{DejaVu Sans Mono}

\title{Konvexität der Exponentialfunktion}
\author{Jendrik Stelzner}
\date{\today}

\begin{document}
\maketitle


\begin{defi}
 Eine Abbildung $f \colon \R \to \R$ heißt konvex, falls für alle $x,y \in \R$
 \[
  f(\lambda x + (1-\lambda)y) \leq \lambda f(x) + (1-\lambda) f(y)
  \quad \text{für alle $\lambda \in [0,1]$}.
 \]
\end{defi}


\begin{question}
 Es sei $f, g \colon \R \to \R$ konvex. Zeigen Sie: Ist $g$ monoton steigend, so ist auch $g \circ f$ konvex.
\end{question}
\begin{solution}
 Es seien $x,y \in \R$ und $\lambda \in [0,1]$. Da $f$ konvex ist, ist
 \[
  f(\lambda x + (1-\lambda)y) \leq \lambda f(x) + (1-\lambda) f(y).
 \]
 Wegen der Monotonie von $g$ ergibt sich damit, dass
 \[
  g(f(\lambda x + (1-\lambda)y)) \leq g(\lambda f(x) + (1-\lambda) f(y)).
 \]
 Durch die Konvexität von $g$ erhalten wir auch, dass
 \[
  g(\lambda f(x) + (1-\lambda) f(y))
  \leq \lambda g(f(x)) + (1-\lambda) g(f(y)).
 \]
 Insgesamt erhalten wir damit, ass
 \begin{align*}
      &\, (g \circ f)(\lambda x + (1-\lambda) y) \\
     =&\, g(f(\lambda x + (1-\lambda) y)) \\
  \leq&\, g(\lambda f(x) + (1-\lambda) f(y)) \\
  \leq&\, \lambda g(f(x)) + (1-\lambda) g(f(y)) \\
     =&\, \lambda (g \circ f)(x) + (1-\lambda) (g \circ f)(y).
 \end{align*}
\end{solution}


Wir wollen nun zeigen, dass die Exponentialfunktion $\exp \colon \R \to \R$ konvex ist. Wir erinnern daran, dass
\[
 \exp(x) = \sum_{k=0}^\infty \frac{x^k}{k!} \quad \text{für alle $x \in \R$},
\]
und dass
\begin{equation}\label{eqn: exp Gruppenhomo}
 \exp(x+y) = \exp(x) \exp(y) \quad \text{für alle $x,y \in \R$}.
\end{equation}
Außerdem ist $\exp(x) > 0$ für alle $x \in \R$.


\begin{question}
 Zeigen Sie, dass $\exp$ auf $[0,\infty)$ konvex ist, d.h. dass für alle $x,y \in [0,\infty)$
 \[
  \exp(\lambda x + (1-\lambda) y)
  \leq \lambda \exp(x) + (1-\lambda) \exp(y)
  \quad
  \text{für alle $\lambda \in [0,1]$}.
 \]
 (\emph{Hinweis}: Nutzen Sie, dass für alle $k \in \N$ die Abbildung $x \mapsto x^k$ auf $[0,\infty)$ konvex ist.)
\end{question}
\begin{solution}
 Für alle $k \in \N$ ist die Abbildung
 \[
  [0, \infty) \to \R, x \mapsto x^k
 \]
 konvex, d.h. für alle $x,y \in [0,\infty)$ ist
 \[
  (\lambda x + (1-\lambda) y)^k
  \leq \lambda x^k + (1-\lambda) y^k
  \quad
  \text{für alle $\lambda \in [0,1]$}.
 \]
 Damit ergibt sich, dass für alle $x,y \in \R$ und $\lambda \in [0,1]$
 \begin{align*}
  \exp(\lambda x + (1-\lambda) y)
  &= \sum_{k=0}^\infty \frac{(\lambda x + (1-\lambda)^y)^k}{k!} \\
  &\leq \sum_{k=0}^\infty \frac{\lambda x^k + (1-\lambda) y^k}{k!} \\
  &= \lambda \sum_{k=0}^\infty \frac{x^k}{k!} + (1-\lambda) \sum_{k=0}^\infty \frac{y^k}{k!} \\
  &= \lambda \exp(x) + (1-\lambda) \exp(y).
 \end{align*}
\end{solution}



\begin{question}
 Folgern Sie, dass $\exp$ konvex ist. (\emph{Hinweis:} Nutzen Sie \eqref{eqn: exp Gruppenhomo} um das Verschieben der Exponentialfunktion in eine Skalierung umzuwindeln.)
\end{question}
\begin{solution}
 Es seien $x,y \in \R$ beliebig aber fest. Es sei $z \in \R$, so dass $x+z, y+z \geq 0$, etwa $z \coloneqq \max\{|x|,|y|\}$. Für alle $\lambda \in [0,1]$ ist dann
 \begin{align*}
      &\, \exp(\lambda x + (1-\lambda) y) \\
     =&\,  \exp(-z) \exp(z) \exp(\lambda x + (1-\lambda) y) \\
     =&\,  \exp(-z) \exp(\lambda (x+z) + (1-\lambda) (y+z)) \\
  \leq&\,  \exp(-z) \left( \lambda \exp(x+z) + (1-\lambda) \exp(y+z) \right) \\
     =&\,  \lambda \exp(-z) \exp(x+z) + (1-\lambda) \exp(-z) \exp(y+z) \\
     =&\,  \lambda \exp(x) + (1-\lambda) \exp(y).
 \end{align*}
 Dabei haben wir genutzt, dass $\exp(-z) > 0$. und $\exp(-z) = 1/\exp(z)$.
\end{solution}







\newpage


\printsolutions





\end{document}
