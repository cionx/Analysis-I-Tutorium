\documentclass[a4paper,10pt]{article}
%\documentclass[a4paper,10pt]{scrartcl}

\usepackage{../mystyle}
\SetupExSheets{solution/print=true} %Zur Ausgabe der Lösungen direkt nach den Fragen

\setromanfont[Mapping=tex-text]{Linux Libertine O}
% \setsansfont[Mapping=tex-text]{DejaVu Sans}
% \setmonofont[Mapping=tex-text]{DejaVu Sans Mono}

\title{Grenzwerte von Funktionen}
\author{Jendrik Stelzner}
\date{\today}

\begin{document}
\maketitle

\tableofcontents





\section{Häufungspunkte}


Es sei $A \subseteq \R^n$ und $f \colon A \to \R^m$ (nicht notwendigerweise stetig). Wir wollen untersuchen, wie sich $f$ an einer Stelle $x \in \R^n$ verhält, bzw.\ verhalten sollte. Um das Verhalten von $f$ an $x$ zu untersuchen, brauchen wir, dass $f$ „in der Nähe“ von $x$ definiert ist. Hierfür brauchen wir, dass $x$ „nahe“ an $A$ ist. Dies motiviert die folgende Definition:


\begin{defi}
 Es sei $A \subseteq \R^n$. Ein Punkt $x \in \R^n$ heißt \emph{Häufungspunkt von $A$}, falls es für alle $\varepsilon > 0$ ein $a \in A$ mit $\|x-a\| < \varepsilon$ und $x \neq a$ gibt (also $0 < \|x-a\| < \varepsilon$).
 
 Wir bezeichnen die Menge aller Häufungspunkte von $A$ mit $A'$.
\end{defi}


\begin{question}
 Es sei $x \in \R^n$. Der \emph{punktierte offene $\varepsilon$-Ball um $x$} ist die Menge
 \[
  \dot{B}_\varepsilon(x)
  \coloneqq B_\varepsilon(x) \setminus \{x\}
  = \{y \in \R^n \mid 0 < \|x-y\| < \varepsilon\}.
 \]
 Für eine Umgebung $V$ von $x$ ist $\dot{V} \coloneqq V \setminus \{x\}$ die entsprechende \emph{punktierte Umgebung von $x$}
 
 Es sei $A \subseteq \R^n$ und $x \in \R^n$. Zeigen Sie, dass die folgenden Bedingungen äquivalent sind:
 \begin{enumerate}
  \item\label{enum: Häufungspunkt}
   $x$ ist ein Häufungspunkt von $A$.
  \item\label{enum: punktierte Bälle}
   Für jedes $\varepsilon > 0$ gibt es ein $a \in A$ mit $a \in \dot{B}_\varepsilon(x)$.
  \item\label{enum: punktierte Umgebungen}
   Für jede punktierte Umgebung $\dot{V}$ von $x$ gibt es ein $a \in A$ mit $a \in \dot{V}$.
 \end{enumerate}
\end{question}
\begin{solution}
 (\ref{enum: Häufungspunkt} $\Leftrightarrow$ \ref{enum: punktierte Bälle}) \ref{enum: punktierte Bälle} ist eine direkte Umformulierung der Definition von \ref{enum: Häufungspunkt}.
 
 (\ref{enum: punktierte Bälle} $\Leftrightarrow$ \ref{enum: punktierte Umgebungen}) Jeder punktierte $\varepsilon$-Ball um $x$ ist auch eine punktierte Umgebung von $x$. Andererseits enthält jede punktierte Umgebung von $x$ einen punktierten $\varepsilon$-Ball um $x$.
\end{solution}


Vorstellungsmäßig ist $x \in \R^n$ ein Häufungspunkt von $A \subseteq \R^n$, falls sich $x$ \emph{von außen} durch Punkte aus $A$ annähern lässt.


\begin{bsp}
 \begin{itemize}
  \item
   $x = 0$ ist ein Häufungspunkt von $A \coloneqq \{1/n \mid n \geq 1\} \subseteq \R$: Da $\lim_{n \to \infty} 1/n = 0$ gibt es für jedes $\varepsilon > 0$  ein $n \geq 1$ mit $|x-1/n| < \varepsilon$, wobei klar ist, dass $1/n \neq 0$.
  \item
   Der Punkt $x = 2$ ist \emph{kein} Häufungspunkt der Menge $A \coloneqq [0,1] \cup \{2\} \subseteq \R$, denn das einzige $a \in A$ mit $|x-a| < 1/2$ ist $a = 2$.
  \item
   Ist $U \subseteq \R^n$ offen, so ist jeder Punkt $x \in U$ ein Häufungspunkt von $U$: Da $U$ offen ist gibt es ein $\delta > 0$ mit $B_\delta(x) \subseteq U$. Für jedes $\varepsilon > 0$ gibt es für $\omega \coloneqq \min\{\varepsilon,\delta\}$ daher ein
   \[
    y \in B_\omega(x) \subseteq B_\delta(x) \subseteq U \quad \text{mit $y \neq x$},
   \]
   und es gilt $\|x-y\| < \omega \leq \varepsilon$.
  \item
   Allgemeiner ergibt sich mit dieser Argumentation, dass $x \in \R^n$ ein Häufungspunkt von $V \subseteq \R^n$ ist, falls $V$ eine Umgebung von $x$ ist. (Es genügt bereits eine punktierte Umgebung.)
 \end{itemize}
\end{bsp}


\begin{question}
 Es sei $A \subseteq \R^m$ und $x \in \R^m$. Zeigen Sie, dass $x$ genau dann ein Häufungspunkt von $A$ ist, falls es eine Folge $(a_n)$ auf $A \setminus \{x\}$ gibt, so dass $a_n \to x$.
\end{question}
\begin{solution}
 Angenommen, $x$ ist ein Häufungspunkt von $A$. Dann gibt es für jedes $n \geq 1$ ein $a_n \in A \setminus \{x\}$ mit $|x-a_n| < 1/n$. Die Folge $(x_n)_{n \geq 1}$ konvergiert per Konstruktion gegen $x$.
 
 Angenommen, eine solche Folge $(a_n)_{n \in \N}$ existiert. Dann gibt es für jedes $\varepsilon > 0$ ein $N \in \N$, so dass $|x-a_n| < \varepsilon$ für alle $n \geq N$. Inbesondere ist $a_N \in A$ mit $|x-a_N| < \varepsilon$ und $a_N \neq x$.
\end{solution}


\begin{question}
 Es sei $M \subseteq \R^n$ endlich. Zeigen Sie, dass $M' = \emptyset$.
\end{question}
\begin{solution}
 Es sei $x \in \R^n$. Ist $x \notin M$, so ergibt sich für
 \[
  \varepsilon \coloneqq \min_{m \in M} \|x-m\| > 0,
 \]
 dass es kein $m \in M$ mit $\|x-m\| < \varepsilon$ gibt. Also ist $x$ dann kein Häufungspunkt von $M$. Ist $x \in M$, so ergibt sich für
 \[
  \varepsilon
  \coloneqq
  \begin{cases}
   \min_{m \in M, m \neq x} \|x-m\| & \text{falls $|M| \geq 2$}, \\
                                  1 & \text{falls $M = \{x\}$},
  \end{cases}
 \]
 dass $x$ das einzige $m \in M$ mit $\|x-m\| < \varepsilon$ ist. Also ist $x$ auch dann kein Häufungspunkt von $M$.
\end{solution}


\begin{question}
 Bestimmen Sie $\Z'$.
\end{question}
\begin{solution}
 Es sei $x \in \R$. Ist $x \notin \Z$, so gibt es für
 \[
  \varepsilon \coloneqq \min\{ \lceil x \rceil - x, x - \lfloor x \rfloor \}
 \]
 kein $n \in \Z$ mit $\|x-n\| < \varepsilon$. Also ist $x$ dann kein Häufungspunkt von $\Z$. Ist andererseits $x \in \Z$, so gibt es außer $x$ kein $n \in \Z$ mit $\|x-n\| < 1/2$, weshalb $x$ auch dann kein Häufungspunkt von $\Z$ ist.
 
 Also ist kein $x \in \R$ ein Häufungspunkt von $\Z$, und somit $\Z' = \emptyset$.
\end{solution}


\begin{question}
 Es seien $A, B \subseteq \R$. Zeigen Sie:
 \begin{enumerate}
  \item
   Ist $A \subseteq B$, so ist $A' \subseteq B'$.
  \item
   Es ist $(A \cup B)' = A' \cup B'$.
 \end{enumerate}
\end{question}
\begin{solution}
 \begin{enumerate}
  \item
   Es sei $x \in A'$. Für jedes $\varepsilon > 0$ gibt es dann ein $a \in A$ mit $\|x-a\| < \varepsilon$ und $a \neq x$. Da $a \in A \subseteq B$ folgt, dass es für jedes $\varepsilon > 0$ ein $b \in B$ mit $\|x-b\| < \varepsilon$ und $b \neq x$ gibt. Also ist $x$ ein Häufungspunkt von $B$, also $x \in B'$. Aus der Beliebigkeit von $x \in A'$ folgt, dass $A' \subseteq B'$.
  \item
   Da $A \subseteq A \cup B$ ist $A' \subseteq (A \cup B)'$, und da $B \subseteq A \cup B$ ist $B' \subseteq (A \cup B)'$. Also ist auch $A' \cup B' \subseteq (A \cup B)'$.
   
   Angenommen, es ist $x \notin A' \cup B'$. Dann gibt es $\varepsilon_A, \varepsilon_B > 0$, so dass es kein $a \in A$ mit $\|x-a\| < \varepsilon_A$ und $a \neq x$ gibt, und auch kein $b \in B$ mit $\|x-b\| < \varepsilon_B$ und $b \neq x$. Für $\varepsilon \coloneqq \min\{\varepsilon_A, \varepsilon_B\}$ gibt es daher kein $c \in A \cup B$ mit $\|x-c\| < \varepsilon$ und $c \neq x$. Also ist dann $x \notin (A \cup B)'$. Das zeigt, dass auch $(A \cup B)' \subseteq A' \cup B'$.
  \qedhere
 \end{enumerate}
\end{solution}


\begin{question}
 Bestimmen Sie $A'$ für $A \coloneqq [0,1] \cup [2,3]$.
\end{question}
\begin{solution}
 \begin{beh}
  Für alle $a, b \in \R$ mit $a < b$ ist
  \[
   [a,b]' = [a,b].
  \]
 \end{beh}
 \begin{proof}[Beweis der Behauptung]
  Für $x < a$  ist $a-x > 0$. Für alle $y \in [a,b]$ ist wegen $y \geq a$ aber
  \[
   \|x-y\| = y-x \geq a-x,
  \]
  es gibt also kein $y \in [a,b]$ mit $\|x-y\| < a-x$. Daher ist $x \notin [a,b]'$.  Analog ergibt sich, dass auch $x \notin [a,b]'$ für $x > b$. Also ist $[a,b]' \subseteq [a,b]$.
  
  Dass $a,b \in [a,b]'$ ergibt sich durch die Folgen $(x_n)$ auf $(a,b]$ und $(y_n)$ auf $[a,b)$ mit
  \[
   x_n \coloneqq a + \frac{b-a}{n+1}
   \quad
   \text{und}
   \quad
   y_n \coloneqq b - \frac{b-a}{n+1}
   \quad
   \text{für alle $n \in \N$}.
  \]
  Dass $x \in [a,b]'$ für $a < x < b$ ergibt sich daraus, dass $[a,b]$ eine Umgebung für diese $x$ ist. Damit ergibt sich, dass $[a,b] \subseteq [a,b]'$.
 \end{proof}
 
 Aus der Behauptung ergibt sich direkt, dass
 \[
  ([0,1] \cup [2,3])'
  = [0,1]' \cup [2,3]'
  = [0,1] \cup [2,3].
 \]
\end{solution}





\section{Grenzwerte von Funktionen}


\begin{defi}
 Es sei $A \subseteq \R^n$, $f \colon A \to \R^m$ und $x \in \R^n$ ein Häufungspunkt von $A$. Für $y \in \R^m$ schreiben wir
 \[
  \lim_{\substack{a \to x \\ a \in A}} f(a) = y,
 \]
 falls es für jedes $\varepsilon > 0$ ein $\delta > 0$ gibt, so dass
 \[
  \|x-a\| < \delta \Rightarrow \|y-f(a)\| < \varepsilon
  \quad
  \text{für alle $a \in A$ mit $a \neq x$}.
 \]
 Wir nennen $y$ dann den \emph{Grenzwert von $f$ an $x$ über $A$}. Wir schreiben auch $f(a) \to y$ für $a \to x$ über $a \in A$
\end{defi}


Die Bedingung $a \neq x$ sorgt dafür, dass nur das Verhalten von $f$ in der Nähe von $x$ von Bedeutung ist, unabhängig von $f(x)$ selbst. So können wir untersuchen, wie sich $f$ an $x$ verhalten \emph{sollte}.


Grenzwerte von Funktionen sind eindeutig:


\begin{lem}
 Es sei $A \subseteq \R^n$, $f \colon A \to \R^m$ und $x \in \R^n$ ein Häufungspunkt von $A$. Sind $y, y' \in \R^m$, so dass $f(a) \to y$ und $f(a) \to y'$ für $a \to x$ über $a \in A$, so ist $y = y'$.
\end{lem}
\begin{proof}
 Es sei $\varepsilon > 0$ beliebig aber fest. Da $f(a) \to y$ für $a \to x$ über $a \in A$ gibt es ein $\delta_1 > 0$ mit
 \[
  \|x-a\| < \delta_1 \Rightarrow \|y-f(a)\| < \frac{\varepsilon}{2}
  \quad \text{für alle $a \in A$ mit $a \neq x$}.
 \]
 Da $f(a) \to y'$ für $a \to x$ über $a \in A$ gibt es auch ein $\delta_2 > 0$, so dass
 \[
  \|x-a\| < \delta_2 \Rightarrow \|y'-f(a)\| < \frac{\varepsilon}{2}
  \quad \text{für alle $a \in A$ mit $a \neq x$}.
 \]
 Da $x$ eine Häufungspunkt von $A$ ist, gibt es für $\delta \coloneqq \min \{\delta_1, \delta_2\} > 0$ ein $a \in A$ mit $\|x-a\| < \delta$ und $a \neq x$. Deshalb ist
 \[
  \|y-y'\|
  \leq \|y-f(a)\| + \|y'-f(a)\|
  < \frac{\varepsilon}{2} + \frac{\varepsilon}{2}
  = \varepsilon.
 \]
 Da $\|y-y'\| < \varepsilon$ für alle $\varepsilon > 0$ ist bereits $\|y-y'\| = 0$, also $y = y'$
\end{proof}


Für diese Eindeutigkeit ist es entscheidet, dass $x$ ein Häufungspunkt von $A$ ist; daher definieren wir Funktionsgrenzwerte nur für diesen Fall.


\begin{bsp}
 \begin{itemize}
  \item
   Wir betrachten die \emph{Signumabbildung}
   \[
    \sgn \colon \R \to \R,
    x \mapsto
    \begin{cases}
     -1 & \text{falls $x < 0$}, \\
      0 & \text{falls $x = 0$}, \\
      1 & \text{falls $x > 0$}.
    \end{cases}
   \]
   $0$ ist ein gemeinsamer Häufungspunkt von $(-\infty,0)$ und $(0,\infty)$, und es ist
   \[
    \lim_{\substack{x \to 0 \\ x \in (-\infty, 0)}} f(x) = -1,
    \quad
    \text{und}
    \quad
    \lim_{\substack{x \to 0 \\ x \in (0, \infty)}} f(x) = 1.
   \]
  \item
   Wir betrachten die Abbildung
   \[
    f \colon (0,\infty) \to \R, x \mapsto \sin \frac{1}{x}.
   \]
   $0$ ist ein Häufungspunkt der beiden Mengen
   \[
    A \coloneqq \left\{ \frac{1}{\frac{\pi}{2} + n \cdot 2\pi} \,\middle|\, n \in \N \right\}
    \quad
    \text{und}
    \quad
    B \coloneqq \left\{ \frac{1}{\frac{3\pi}{2} + n \cdot 2\pi} \,\middle|\, n \in \N \right\},
   \]
   und es ist
   \[
    \lim_{\substack{x \to 0 \\ x \in A}} f(x) = 1
    \quad
    \text{und}
    \quad
    \lim_{\substack{x \to 0 \\ x \in B}} f(x) = -1.
   \]
   $0$ ist auch ein Häufungspunkt der Menge $(0,\infty)$, der Grenzwert
   \[
    \lim_{\substack{x \to 0 \\ x \in (0,\infty)}} f(x)
   \]
   existiert jedoch nicht.
 \end{itemize}
\end{bsp}


\begin{lem}\label{lem: Funktionsgrenzwerte durch Folgen}
 Es sei $A \subseteq \R^m$, $f \colon A \to \R^k$ und $x \in \R^m$ ein Häufungspunkt von $A$. Für $y \in \R^k$ sind äquivalent:
 \begin{enumerate}
  \item\label{enum: Grenzwert mit epsilon delta}
   $\lim_{a \to x, a \in A} f(a) = y$.
  \item\label{enum: Grenzwert mit Folgen}
   Für jede Folge $(a_n)$ auf $A \setminus \{x\}$ mit $a_n \to x$ ist $f(a_n) \to y$.
 \end{enumerate}
 (Da $x$ ein Häufungspunkt von $A$ ist, existiert eine entsprechende Folge.)
\end{lem}
\begin{proof}
 (\ref{enum: Grenzwert mit epsilon delta} $\Rightarrow$ \ref{enum: Grenzwert mit Folgen}) Es sei $(a_n)$ eine Folge auf $A \setminus \{x\}$ mit $a_n \to x$. Wir wollen zeigen, dass $f(a_n) \to y$. Sei hierfür $\varepsilon > 0$ beliebig aber fest. Da $\lim_{a \to x, a \in A} f(a) = y$ gibt es ein $\delta > 0$, so dass
 \[
  \|x-a\| < \delta \Rightarrow \|y-f(a)\| < \varepsilon
  \quad \text{für alle $a \in A$ mit $a \neq x$}.
 \]
 Da $a_n \to x$ gibt es ein $N \in \N$ mit $\|x - a_n\| < \delta$ für alle $n \geq N$. Da $a_n \neq x$ ist deshalb $\|y-f(a_n)\| < \varepsilon$ für alle $n \geq N$.
 
 (\ref{enum: Grenzwert mit Folgen} $\Rightarrow$ \ref{enum: Grenzwert mit epsilon delta}) Angenommen, es ist nicht $\lim_{a \to x, a \in A} f(a) = y$. Dann gibt es ein $\varepsilon > 0$, so dass es für alle $\delta > 0$ ein $a \in A$ gibt, so dass zwar $a \neq x$ und $\|x-a\| < \delta$, aber $\|y-f(a)\| \geq \varepsilon$. Insbesondere gibt es deshalb für alle $n \geq 1$ ein $a_n \in A \setminus \{x\}$ mit $\|x-a_n\| < 1/n$ aber $\|y-f(a_n)\| \geq \varepsilon$. Dann ist $(a_n)$ eine Folge auf $A \setminus \{x\}$ mit $a_n \to x$, aber es gilt nicht $f(a_n) \rightarrow y$.
\end{proof}


Aus dieser Beschreibung von Funktionsgrenzwerten durch Folgen ergeben sich direkt zwei einfache Konsequenzen: Zum einen sehen wir, dass sich Funktionsgrenzwerte auch koordinatenweise beschreiben lassen.


\begin{lem}
 Es sei $f \colon A \to \R^m$ mit Definitionsbereich $A \subseteq \R^n$ und $x \in \R^n$ ein Häufungspunkt von $A$. In Koordinaten sei $f = (f_1, \dotsc, f_m)$. Für $y = (y_1, \dotsc, y_m) \in \R^m$ ist genau dann $\lim_{a \to x, a \in A} f(a) = y$, falls $\lim_{a \to x, a \in A} f_i(a) = y_i$ für alle $1 \leq i \leq m$.
\end{lem}
\begin{proof}
 Dass $\lim_{a \to x, a \in A} f(a) = y$ ist äquivalent dazu, dass für jede Folge $(a_n)$ auf $A \setminus \{x\}$ mit $a_n \to x$ auch $f(a_n) \to y$. Dies ist äquivalent dazu, dass für jede Folge $(a_n)$ auf $A \setminus \{x\}$ auch $f_i(a_n) \to y_i$ für alle $1 \leq i \leq m$. Dies bedeutet wiederum, dass $\lim_{a \to x, a \in A} f_i(a) = y_i$ für alle $1 \leq i \leq n$.
\end{proof}


Ein weiteres Ergebnis ist, dass Funktionsgrenzwerte mit den üblichen Rechenregeln im $\R^n$ verträglich sind.


\begin{prop}
 Es seien $A \subseteq \R^n$, $f, f_1, f_2 \colon A \to \R^m$, $\lambda \in \R$ und $x \in \R^n$ ein Häufungspunkt von $A$.
 \begin{enumerate}
  \item
   Existieren die Grenzwerte $\lim_{a \to x, a \in A} f_1(a)$ und $\lim_{a \to x, a \in A} f_2(a)$, so existiert auch der Grenzwert $\lim_{x \to a, a \in A} (f_1 + f_2)(a)$ und es gilt
   \[
    \lim_{\substack{a \to x \\ a \in A}} (f_1 + f_2)(a)
    =
    \left( \lim_{\substack{a \to x \\ a \in A}} f_1(a) \right)
    + \left( \lim_{\substack{a \to x \\ a \in A}} f_2(a) \right).
   \]
  \item
   Existiert der Grenzwert $\lim_{a \to x, a \in A} f(a)$, so existiert auch $\lim_{a \to x, a \in A} (\lambda f)(a)$, und es gilt
   \[
    \lim_{\substack{a \to x \\ a \in A}} (\lambda f)(a)
    = \lambda \lim_{\substack{a \to x \\ a \in A}} f(a).
   \]
 \end{enumerate}
 Im Fall $n = 1$, also für $\R^1 = \R$, gilt auch eine Verträglichkeit mit Multiplikation und Division.
 \begin{enumerate}[resume]
  \item
   Existieren die Grenzwerte $\lim_{a \to x, a \in A} f_1(a)$ und $\lim_{a \to x, a \in A} f_2(a)$, so existiert auch der Grenzwert $\lim_{x \to a, a \in A} (f_1 \cdot f_2)(a)$ und es gilt
   \[
    \lim_{\substack{a \to x \\ a \in A}} (f_1 \cdot f_2)(a)
    =
    \left( \lim_{\substack{a \to x \\ a \in A}} f_1(a) \right)
    \cdot \left( \lim_{\substack{a \to x \\ a \in A}} f_2(a) \right).
   \]
  \item
   Existieren die beiden Grenzwerte $\lim_{a \to x, a \in A} f_1(a)$ und $\lim_{a \to x, a \in A} f_2(a)$, und ist \mbox{$f_2(a) \neq 0$} für alle $a \in A \setminus \{x\}$ sowie $\lim_{a \to x, a \in A} f_2(a) \neq 0$, so existiert auch der Grenzwert $\lim_{a \to x, a \in A} f_1(a)/f_2(a)$ und es gilt
   \[
    \lim_{\substack{a \to x \\ a \in A}} \frac{f_1(a)}{f_2(a)}
    = \frac{\lim_{a \to x, a \in A} f_1(a)}{\lim_{a \to x, a \in A} f_2(a)}
   \]
 \end{enumerate}
 (Sehen wir $\R^2 \cong \C$, so ergibt sich auch eine Verträglichkeit mit der Multiplikation und Division im Komplexen; hierdrauf gehen wir hier aber nicht weiter ein.)
\end{prop}


Wie wir bereits gesehen haben, können für eine Funktion $f \colon X \to \R^m$ mit Definitionsbereich $X \subseteq \R^n$ und Teilmengen $A, B \subseteq X$ mit gemeinsamen Häufungspunkt $x \in \R^n$ die beiden Grenzwerte $\lim_{a \to x, a \in A} f(a)$ und $\lim_{b \to x, b \in B} f(b)$ existieren, aber dennoch
\[
 \lim_{\substack{a \to x \\ a \in A}} f(a)
 \neq
 \lim_{\substack{b \to x \\ b \in B}} f(b).
\]
Es kann auch passieren, dass einer der beiden Grenzwerte existiert, der andere jedoch nicht. Es gibt also im Allgemeinen keinen Zusammehang zwischen dem Grenzwert von $f$ über $A$ und dem Grenzwert über $B$. Unter bestimmten Umständen lassen sich die beiden Grenzwerte aber vergleichen:


\begin{lem}\label{lem: Grenzwerte auf Teilmengen}
 Es seien $A \subseteq B \subseteq \R^n$ und $f \colon B \to \R^m$. Ist $x \in \R^n$ ein gemeinsamer Häufungspunkt von $A$ und $B$, sodass der Grenzwert $\lim_{b \to x, b \in B} f(b)$ existiert, so existiert auch $\lim_{a \to x, a \in A} f(a)$, und es gilt
 \[
  \lim_{\substack{a \to x \\ a \in A}} f(a)
  = \lim_{\substack{b \to x \\ b \in B}} f(b).
 \]
\end{lem}
\begin{proof}
 Zur besseren Lesbarkeit setzen wir $y \coloneqq \lim_{b \to x, b \in B} f(b)$. Wir wollen zeigen, dass auch $\lim_{a \to x, x \in A} f(a) = y$. Es sei hierfür $\varepsilon > 0$ beliebig aber fest. Da \mbox{$y = \lim_{b \to x, b \in B} f(b)$} gibt es ein $\delta > 0$, so dass
 \[
  \|x-b\| < \delta \Rightarrow \|y-f(b)\| < \varepsilon \quad \text{für alle $b \in B$ mit $b \neq x$}.
 \]
 Da $A \subseteq B$ ist daher insbesondere
 \[
  \|x-a\| < \delta \Rightarrow \|y-f(a)\| < \varepsilon \quad \text{für alle $a \in A$ mit $a \neq x$}.
 \]
 Wegen der Beliebigkeit von $\varepsilon > 0$ folgt, dass $\lim_{a \to x, a \in A} f(a) = y$.
\end{proof}


\begin{kor}\label{kor: gemeinsamer Häufungspunkt mit gemeinsamen Grenzwert}
 Es sei $f \colon X \to \R^m$ mit Definitionsbereich $X \subseteq \R^n$. Es seien $A, B \subseteq X$ Teilmengen, so dass $x \in \R^n$ ein gemeinsamer Häufungspunkt von $A$ und $B$ ist, und die beiden Grenzwerte $\lim_{a \to x, a \in A} f(a)$ und $\lim_{b \to x, b \in B} f(b)$ existieren. Ist $x$ auch ein Häufungspunkt von $A \cap B$, so ist
 \[
  \lim_{\substack{a \to x \\ a \in A}} f(a)
  = \lim_{\substack{b \to x \\ b \in B}} f(b).
 \]
\end{kor}
\begin{proof}
 Da die beiden Grenzwerte $\lim_{a \to x, a \in A} f(a)$ und $\lim_{b \to x, b \in B} f(b)$ existieren, und $A \cap B \subseteq A$ und $A \cap B \subseteq B$, erhalten wir aus Lemma \ref{lem: Grenzwerte auf Teilmengen}, dass auch der Grenzwert $\lim_{c \to x, c \in A \cap B} f(c)$ existiert und
 \[
  \lim_{\substack{a \to x \\ a \in A}} f(a)
  = \lim_{\substack{c \to x \\ c \in A \cap B}} f(c)
  = \lim_{\substack{b \to x \\ b \in B}} f(b).
  \qedhere
 \]
\end{proof}





\section{Grenzwerte und Stetigkeit}
Die Definition von Funktionsgrenzwerten erinnert stark an das $\varepsilon$-$\delta$-Kriterium für die Stetigkeit einer Funktion. Diese Ähnlichkeit legt die Vermutung nahe, dass sich die Stetigkeit einer Funktion durch die Betrachtung von passenden Funktionsgrenzwerten untersuchen lässt.


\begin{lem}\label{lem: Unabhängigkeit von Umgebung}
 Es sei $f \colon A \to \R^m$ mit Definitionsbereich $A \subseteq \R^n$. $f$ sei auf einer Umgebung von $x \in \R^n$ definiert, d.h.\ es gebe eine Umgebung $V$ von $x$ mit $V \subseteq A$. Dann sind die folgenden beiden Bedingungen äquivalent:
 \begin{enumerate}
  \item\label{enum: eine Umgebung}
   Der Grenzwert $\lim_{a \to x, a \in V} f(a)$ existiert für eine Umgebung $V \subseteq \R^n$ von $x$ mit $V \subseteq A$.
  \item\label{enum: alle Umgebungen}
   Der Grenzwert $\lim_{a \to x, a \in U} f(a)$ existiert für jede Umgebung $U \subseteq \R^n$ von $x$ mit $U \subseteq A$.
 \end{enumerate}
 Der Grenzwert $\lim_{a \to x, a \in U} f(a)$ ist dabei unabhängig von der Wahl der Umgebung $U$ von $x$ mit $U \subseteq A$.
\end{lem}
\begin{proof} 
 (\ref{enum: alle Umgebungen} $\Rightarrow$ \ref{enum: eine Umgebung}) Nach Annahme existiert eine Umgebung $V \subseteq \R^n$ von $x$, so dass $f$ auf $V$ definiert ist, also mit $V \subseteq A$; diese Implikation ist daher klar.
 
 (\ref{enum: eine Umgebung} $\Rightarrow$ \ref{enum: alle Umgebungen}) Es sei $V$ eine entsprechende Umgebung von $x$ und $y \coloneqq \lim_{a \to x, a \in V} f(a)$. Es sei $U \subseteq \R^n$ eine beliebige Umgebung von $x$ mit $U \subseteq A$. Wir wollen zeigen, dass $\lim_{a \to x, a \in U} f(a)$ existiert und $\lim_{a \to x, a \in U} f(a) = y$.
 
 Es sei hierfür $\varepsilon > 0$ beliebig aber fest. Da $y = \lim_{a \to x, a \in V} f(a)$ gibt es $\delta_1 > 0$, so dass
 \[
  \|x-a\| < \delta_1 \Rightarrow \|y-f(a)\| < \varepsilon
  \quad \text{für alle $a \in A$ mit $a \neq x$}.
 \]
 Da $V$ eine Umgebung von $x$ ist, gibt es außerdem ein $\delta_2 > 0$ mit $B_{\delta_2}(x) \subseteq V$. Für $\delta' \coloneqq \min\{\delta_1, \delta_2\}$ ist deshalb $B_{\delta'}(x) \subseteq V$ und
 \[
  \|y-f(a)\| < \varepsilon
  \quad \text{für alle $a \in B_{\delta'}(x)$ mit $a \neq x$}.
 \]
 
 Da auch $U$ eine Umgebung von $x$ ist, gibt es ein $\delta'' > 0$ mit $B_{\delta''}(x) \subseteq U$. Für $\delta \coloneqq \min\{\delta', \delta''\}$ ist also $B_\delta(x) \subseteq U$ mit
 \[
  \|y-f(a)\| < \varepsilon
  \quad \text{für alle $a \in B_\delta(x)$ mit $a \neq x$}.
 \]
 Damit erhalten wir, dass
 \[
  \|x-a\| < \delta \Rightarrow \|y-f(a)\| < \varepsilon \quad \text{für alle $a \in U$ mit $a \neq x$}.
 \]
 Wegen der Beliebigkeit von $\varepsilon > 0$ zeigt dies, dass $\lim_{a \to x, a \in U} f(a) = y$.
\end{proof}


\begin{defi}
 Es sei $f \colon A \to \R^m$ mit Definitionsbereich $A \subseteq \R^n$. Ist $f$ auf einer Umgebung $V$ von $x \in \R^n$ definiert, also $V \subseteq A$, so schreiben wir
 \[
  \lim_{a \to x} f(a)
  \quad
  \text{für}
  \quad
  \lim_{\substack{a \to x \\ a \in V}} f(a)
 \]
 und nennen dies den \emph{Grenzwert von $f$ an $x$}.
\end{defi}


Die Wohldefiniertheit, also Unabhängigkeit von $V$, folgt aus Lemma \ref{lem: Unabhängigkeit von Umgebung}.


\begin{prop}\label{prop: Stetigkeit durch Funktionsgrenzwerte}
 Es sei $f \colon A \to \R^m$ mit Definitionsbereich $A \subseteq \R^n$ auf einer Umgebung von $x$ definiert. Dann ist $f$ genau dann stetig an $x$, wenn \mbox{$\lim_{a \to x} f(a) = f(x)$}.
\end{prop}
\begin{proof}
 Angenommen $f$ ist stetig an $x$. Es sei $V \subseteq \R^n$ eine Umgebung von $x$ mit $V \subseteq A$. Wir wollen zeigen, dass $\lim_{a \to x, a \in V} f(a) = f(x)$. Hierfür sei $\varepsilon > 0$ beliebig aber fest. Da $f$ stetig an $x$ ist gibt es $\delta > 0$, so dass
 \[
  \|x-a\| < \delta \Rightarrow \|f(x)-f(a)\| < \varepsilon
  \quad \text{für alle $a \in A$}.
 \]
 Daher ist insbesondere
 \[
  \|x-a\| < \delta \Rightarrow \|f(x)-f(a)\| < \varepsilon
  \quad \text{für alle $a \in V$ mit $a \neq x$}.
 \]
 Wegen der Beliebigkeit von $\varepsilon > 0$ folgt, dass $\lim_{a \to x, a \in V} f(a) = f(x)$.
 
 Angenommen, es ist $\lim_{x \to a} f(a) = f(x)$. Wir wollen zeigen, dass $f$ stetig an $x$ ist. Hierfür sei $\varepsilon > 0$ beliebig aber fest. Es sei $V \subseteq \R^n$ eine Umgebung von $x$ mit $V \subseteq A$. Da $V$ eine Umgebung von $x$ ist, gibt es ein $\delta_1 > 0$ mit $B_{\delta_1}(x) \subseteq V$. Da $\lim_{a \to x} f(a) = f(x)$ ist $\lim_{a \to x, a \in V} f(a) = f(x)$, es gibt daher ein $\delta_2 > 0$ mit
 \[
  \|x-a\| < \delta_2 \Rightarrow \|f(x)-f(a)\| < \varepsilon
  \quad \text{für alle $a \in V$ mit $a \neq x$},
 \]
 und wir können offenbar auch $a = x$ zulassen. Für $\delta \coloneqq \min\{\delta_1, \delta_2\}$ haben wir nun $B_\delta(x) \subseteq B_{\delta_1}(x) \subseteq V$ und $\|x-a\| < \delta \leq \delta_2$ für alle $a \in B_\delta(x)$, und somit
 \[
  \|x-a\| < \delta \Rightarrow \|f(x)-f(a)\| < \varepsilon
  \quad \text{für alle $a \in A$.}
 \]
 Wegen der Beliebigkeit von $\varepsilon > 0$ zeigt dies die Stetigkeit von $f$ an $x$.
\end{proof}





\section{Links-, rechts- und beidseitige Grenzwerte}
Wir wollen uns nun einem Sonderfall von Funktionsgrenzwerten zuwenden.


\begin{defi}
 Es sei $f \colon A \to \R^m$ mit Definitionsbereich $A \subseteq \R$ und $x \in \R$.
 
 Gibt es ein $r > 0$, so dass $(x-r, x) \subseteq A$, so schreiben wir
 \[
  \lim_{a \uparrow x} f(a)
  \quad
  \text{für}
  \quad
  \lim_{\substack{a \to x \\ a \in (x-r,x)}} f(a),
 \]
 und nennen dies den \emph{linksseitigen Grenzwert von $f$ an $x$}.
 
 Gibt es ein $r > 0$, so dass $(x,x+r) \subseteq A$, so schreiben wir
 \[
  \lim_{a \downarrow x} f(a)
  \quad
  \text{für}
  \quad
  \lim_{\substack{a \to x \\ a \in (x,x+r)}} f(a),
 \]
 und nennen dies den \emph{rechtsseitigen Grenzwert von $f$ an $x$}.
 
 Existiert ein $r > 0$, so dass $(x-r,x) \cup (x,x+r) \subseteq A$, so schreiben wir
 \[
  \lim_{a \to x} f(a)
  \quad
  \text{für}
  \quad
  \lim_{\substack{a \to x \\ a \in (x-r,x) \cup (x,x+r)}} f(a),
 \]
 und nennen dies den \emph{beidseitigen Grenzwert von $f$ an $x$}.
\end{defi}


Die Wohldefiniertheit der jeweiligen Ausdrücke, also die Unabhängigkeit von $r$, ergibt sich aus Korollar \ref{kor: gemeinsamer Häufungspunkt mit gemeinsamen Grenzwert}.


\begin{bem}
 Explizit bedeutet die Definition folgendes: Ist $f \colon A \to \R^n$ mit Definitionsbereits $A \subseteq \R$, so ist für $x \in \R$ und $y \in \R^n$ genau dann $\lim_{a \uparrow x} f(a) = y$, falls
 \begin{enumerate}
  \item
   es gibt ein $r > 0$, so dass $f$ auf $(x-r,x)$ definiert ist, und
  \item
   für alle $\varepsilon > 0$ gibt es ein $\delta > 0$, so dass
   \[
   \|y - f(a)\| < \varepsilon
   \quad
   \text{für alle $a \in A$ mit $a \in (x-\delta,x)$}.
   \]
 \end{enumerate}
 Analoges gilt auch für $\lim_{y \downarrow x} f(y)$ und $\lim_{y \to x} f(y)$.
\end{bem}



\begin{bem}
 Ist $V \subseteq \R$ eine Umgebung von $x \in \R$, und $f \colon V \to \R^m$, so ist die Notation $\lim_{a \to x} f(a)$ doppelt belegt: Zum einen steht die Notation für $\lim_{a \to x, a \in V} f(a)$. Zum anderen gibt es, da $V$ eine Umgebung von $x$ ist, ein $r > 0$ mit $(x-r,x+r) \subseteq V$; dann steht die Notation auch für $\lim_{a \to x, a \in (x-r,x) \cup (x,x+r)} f(a)$.
 
 Die beiden Definitionen sind in diesem Fall allerdings gleichbedeutend: Per Definition ist der Grenzwertes $\lim_{a \to x, a \in (x-r,x) \cup (x,x+r)} f(a)$ gleichbedeutend zum Grenzwert $\lim_{a \to x, a \in (x-r,x+r)} f(a)$. Dieser Grenzwert ist nach Lemma \ref{lem: Unabhängigkeit von Umgebung} der gleiche wie $\lim_{a \to x, a \in V} f(a)$.
\end{bem}


Der beidseitige Grenzwert lässt sich auch als Kombination des links- und rechtsseitigen Grenzwertes definieren:


\begin{lem}
 Es sei $A \subseteq \R$ und $f \colon A \to \R^m$. Für $x \in \R$ und $y \in \R^m$ sind äquivalent:
 \begin{enumerate}
  \item\label{enum: beidseitiger Grenzwert}
   Der beidseitige Grenzwert $\lim_{a \to x} f(a)$ existiert und $\lim_{a \to x} f(a) = y$.
  \item\label{enum: beide Grenzwerte}
   Die beiden Grenzwerte $\lim_{a \uparrow x} f(a)$ und $\lim_{a \downarrow x} f(a)$ existieren und es ist
   \[
    \lim_{a \uparrow x} f(a) = y = \lim_{a \downarrow x} f(a).
   \]
 \end{enumerate}
\end{lem}
\begin{proof}
 (\ref{enum: beidseitiger Grenzwert} $\Rightarrow$ \ref{enum: beide Grenzwerte}) Da $\lim_{a \to x} f(a)$ existiert, gibt es ein $r > 0$, so dass $f$ auf $(x-r,x) \cup (x,x+r)$ definiert ist und $\lim_{a \to x, a \in (x-r,x) \cup (x,x+r)} f(a)$ existiert. Dann ist $f$ auf $(x-r,x)$ und auf $(x,x+r)$ definiert, und nach Lemma \ref{lem: Grenzwerte auf Teilmengen} gilt
 \[
  \lim_{a \uparrow x} f(a)
  = \lim_{\substack{a \to x \\ a \in (x-r,x)}} f(a)
  = \lim_{\substack{a \to x \\ a \in (x-r,x) \cup (x,x+r)}} f(a)
  = \lim_{a \to x} f(a).
 \]
 Analog ergibt sich, dass auch $\lim_{a \downarrow} f(a)$ existiert und
 \[
  \lim_{a \downarrow x} f(a) = \lim_{a \to x} f(a)
 \]
 
 (\ref{enum: beide Grenzwerte} $\Rightarrow$ \ref{enum: beidseitiger Grenzwert}) Da die beiden Grenzwerte $\lim_{a \uparrow x} f(a)$ und $\lim_{a \downarrow x} f(a)$ existieren gibt es $r_-, r_+ > 0$, so dass $f$ auf $(x-r_-,x)$ und auf $(x,x+r_+)$ definiert ist. Also ist $f$ für \mbox{$r \coloneqq \min\{r_-, r_+\}$} auf $(x-r,x) \cup (x,x+r)$ definiert. Wir wollen zeigen, dass der Grenzwert $\lim_{a \to x} f(a)$ existiert und $\lim_{a \to x} f(a) = y$. Hierfür sei $\varepsilon > 0$ beliebig aber fest. Da $\lim_{a \uparrow x} f(a) = y$ gibt es ein $\delta_- > 0$, so dass
 \[
  \|x-a\| < \delta_- \Rightarrow \|y-f(a)\| < \varepsilon
  \quad \text{für alle $a \in (x-r_-,x)$},
 \]
 und da $\lim_{a \downarrow x} f(a) = y$ gibt es ein $\delta_+ > 0$, so dass
 \[
  \|x-a\| < \delta_+ \Rightarrow \|y-f(a)\| < \varepsilon
  \quad \text{für alle $a \in (x,x+r_+)$}.
 \]
 Für $\delta \coloneqq \min\{\delta_-, \delta_+\}$ ist daher
 \[
  \|x-a\| < \delta \Rightarrow \|y-f(a)\| < \varepsilon
  \quad \text{für alle $a \in (x-r,x) \cup (x,x+r)$}.
 \]
 Aus der Beliebigkeit von $\varepsilon > 0$ folgt, dass $\lim_{a \to x} f(a) = y$.
\end{proof}


Aus unserer bisherigen, allgemeinen Untersuchung von Funktionsgrenzwerten ergibt sich für die Sonderfälle von links-, rechts- und beidseitigen Grenzwerten ohne weitere Arbeit die Verträglichkeit mit den üblichen Rechenoperationen. Aus Lemma \ref{lem: Funktionsgrenzwerte durch Folgen} ergibt sich außerdem, dass sich links-, rechts- und beidseitige Grenzwerte durch Folgen beschreiben lassen.


\begin{bsp}
 \begin{itemize}
  \item
    Der Grenzwert $\lim_{x \to 0} \sin(1/x)$ existiert nicht: Die beiden Folgen $(a_n)$ und $(b_n)$ mit
   \[
    a_n \coloneqq \frac{1}{\frac{\pi}{2} + n \cdot 2\pi}
    \quad
    \text{und}
    \quad
    b_n \coloneqq \frac{1}{\frac{3\pi}{2} + n \cdot 2\pi}
    \quad
    \text{für alle $n \in \N$}
   \]
   konvergieren gegen $0$, aber die beiden Grenzwerte
   \begin{gather*}
    \lim_{n \to \infty} \sin \frac{1}{a_n}
    = \lim_{n \to \infty} 1
    = 1
   \shortintertext{und}
    \lim_{n \to \infty} \sin \frac{1}{b_n}
    = \lim_{n \to \infty} -1
    = -1
   \end{gather*}
   unterscheiden sich.
  \item
   Es ist $\lim_{x \to 0} x \sin(1/x) = 0$: Wir wissen bereits, dass die Abbildung
   \[
    f \colon \R \to \R
    \quad
    \text{mit}
    \quad
    f(x) \coloneqq
    \begin{cases}
     x \sin \frac{1}{x} & \text{für $x \neq 0$}, \\
                      0 & \text{für $x = 0$},
    \end{cases}
   \]
   stetig ist. (Stetigkeit an $x \neq 0$ ist klar, und an $x = 0$ ergibt die Stetigkeit aus dem $\varepsilon$-$\delta$-Kriterium.) Daher ist nach Proposition \ref{prop: Stetigkeit durch Funktionsgrenzwerte}
   \[
    \lim_{x \to 0} x \sin \frac{1}{x}
    = \lim_{x \to 0} f(x)
    = f(0)
    = 0.
   \]
  \item
   Für $p, q \in \N$ mit $p, q \geq 1$ ist
   \[
    \lim_{x \to 1} \frac{x^p-1}{x^q-1} = \frac{p}{q}.
   \]
   Das Problem besteht darin, dass $1^q-1 = 1^p-1 = 0$. Dieses Problem beseitigen wir dadurch, dass wir aus den Polynomen $x^p-1$ und $x^q-1$ den Linearfaktor $x-1$ ausklammern (dies ist möglich, da $1$ ein Nullstelle der beiden Polynome ist). Wir erhalten so, dass für alle $x \notin \{-1,1\}$
   \[
    \frac{x^p-1}{x^q-1}
    = \frac{(x-1)\sum_{k=0}^{p-1} x^k}{(x-1)\sum_{k=0}^{q-1} x^k}
    = \frac{\sum_{k=0}^{p-1} x^k}{\sum_{k=0}^{q-1} x^k}.
   \]
   Daher ist
   \[
    \lim_{x \to 1} \frac{x^p-1}{x^q-1}
    = \lim_{x \to 1} \frac{\sum_{k=0}^{p-1} x^k}{\sum_{k=0}^{q-1} x_k}
    = \frac{\sum_{k=0}^{p-1} 1}{\sum_{k=0}^{q-1} 1}
    = \frac{p}{q}.
   \]
  \item
   Wir wollen untersuchen, wie sich die Grenzwert von $x \sqrt{1 + 4/x^2}$ an $x = 0$ verhalten --- von links, rechts und beidseitig. Hierfür bemerken wir, dass für alle $x \neq 0$
   \[
    x = \sgn(x) |x| = \sgn(x) \sqrt{x^2}
   \]
   und somit
   \begin{align*}
    x \sqrt{1 + \frac{4}{x^2}}
    &= \sgn(x) \sqrt{x^2} \sqrt{1 + \frac{4}{x^2}} \\
    &= \sgn(x) \sqrt{x^2 \left(1 +\frac{4}{x^2}\right)}
    = \sgn(x) \sqrt{x^2 + 4}.
   \end{align*}
   Daher ist nach den üblichen Stetigkeitsargumenten (Proposition \ref{prop: Stetigkeit durch Funktionsgrenzwerte})
   \begin{align*}
    \lim_{x \uparrow 0} x \sqrt{1 + \frac{4}{x^2}}
    &= \lim_{x \uparrow 0} \sgn(x) \sqrt{x^2 + 4}
    = \lim_{x \uparrow 0} -1 \cdot \sqrt{x^2 + 4} \\
    &= -\lim_{x \uparrow 0} \sqrt{x^2 + 4}
    = - \sqrt{0^2 + 4}
    = -2.
   \end{align*}
   Analog ergibt sich, dass
   \[
    \lim_{x \downarrow 0} x \sqrt{1 + \frac{4}{x^2}}
    = 2.
   \]
   Damit kennen wir das Verhalten von links- und rechtsseitigen Grenzwert. Der beidseitige Grenzwert existiert nicht, da links- und rechtsseitiger Grenzwert verschieden sind.
  \item
   Wir wollen den Grenzwert
   \[
    \lim_{x \uparrow 2} \frac{x^2 - 14x + 24}{|x-2| + |x^2-4|}
   \]
   untersuchen. Hierfür bemerken wir, dass für alle $x \neq 2$
   \begin{align*}
    \frac{x^2 - 14x + 24}{|x-2| + |x^2-4|}
    &= \frac{(x-2)(x-12)}{|x-2| + |x-2| |x+2|} \\
    &= \frac{(x-2)(x-12)}{\sgn(x-2)(x-2) + \sgn(x-2)(x-2)|x+2|} \\
    &= \frac{x-12}{\sgn(x-2) + \sgn(x-2)|x+2|} \\
    &= \sgn(x-2) \frac{x-12}{1 + |x+2|}.
   \end{align*}
   Daher ist
   \begin{align*}
    \lim_{x \uparrow 2} \frac{x^2 - 14x + 24}{|x-2| + |x^2-4|}
    &= \lim_{x \uparrow 2} \sgn(x-2) \frac{x-12}{1+|x+2|} \\
    = \lim_{x \uparrow 2} -1 \cdot \frac{x-12}{1+|x+2|}
    &= -\lim_{x \uparrow 2} \frac{x-12}{1+|x+2|}
    = - \frac{2-12}{1+|2+2|}
    = 2.
   \end{align*}
   Analog ergibt sich, dass auch
   \[
    \lim_{x \downarrow 2} \frac{x^2 - 14x + 24}{|x-2| + |x^2-4|} = -2.
   \]
 \end{itemize}
\end{bsp}


\begin{bsp}
 Ist $f \colon \R \to \R$ monoton steigend, so existiert an jeder Stelle $x \in \R$ sowohl der links- als auch der rechtsseitige Genzwert, und es ist
 \[
  \lim_{a \uparrow x} f(a) = \sup_{a < x} f(a)
  \quad
  \text{und}
  \quad
  \lim_{a \downarrow x} f(a) = \inf_{a > x} f(a)
 \]

 Zum Beweis zeigen wir, dass
 \[
  y \coloneqq \sup_{a < x} f(a)
 \]
 die Eigenschaften des linksseitigen Limes erfüllt. Hierfür sei $\varepsilon > 0$ beliebig aber fest. Nach der $\varepsilon$-Charakterisierung des Supremums gibt es ein $a_0 < x$ mit $y-\varepsilon < f(a_0)$. Aus der Monotonie von $f$ folgt, dass
 \[
  y-\varepsilon < f(a_0) \leq f(a) \leq \sup_{a' < x} f(a') = y \quad \text{für alle $a_0 \leq a < x$}.
 \]
 Für $\delta \coloneqq x-a_0 > 0$ ist also
 \[
  \|y - f(a)\| < \varepsilon \quad \text{für alle $a \in (x-\delta, x)$}.
 \]
 Wegen der Beliebigkeit von $\varepsilon > 0$ zeigt dies, dass $\lim_{a \uparrow x} f(a) = y$. Analog zeigt man, dass $\lim_{a \downarrow x} f(a) = \inf_{a > x} f(a)$.
 
 Ist $f$ monoton fallend, so ergibt sich analog, dass an jeder Stelle $x \in \R$
  \[
  \lim_{a \uparrow x} f(a) = \inf_{a < x} f(a)
  \quad
  \text{und}
  \quad
  \lim_{a \downarrow x} f(a) = \sup_{a > x} f(a).
 \]
\end{bsp}


Zusammen mit Proposition \ref{prop: Stetigkeit durch Funktionsgrenzwerte} ergibt sich damit die aus der Vorlesung bekannte Charakterisierung der Stetigkeit einer monotonen Funktion:


\begin{kor}
 Ist $f \colon \R \to \R$ monoton steigend, so ist $f$ genau dann stetig an der Stelle $x \in \R$, wenn
 \[
  \sup_{a < x} f(a) = f(x) = \inf_{a > x} f(a).
 \]
 Ist $f$ monoton fallend, so ist $f$ genau dann stetig an $x$, wenn
 \[
  \inf_{a < x} f(a) = f(x) = \sup_{a > x} f(a).
 \]
\end{kor}
\begin{proof}
 $f$ ist genau dann stetig an $x$, wenn $\lim_{a \to x} f(a) = f(x)$. Dies ist äquivalent dazu, dass $\lim_{a \uparrow x} f(a)$ und $\lim_{a \downarrow x} f(a)$ existieren und
 \begin{equation}\label{eqn: beide Grenzwerte}
  \lim_{a \uparrow x} f(a) = f(x) = \lim_{a \downarrow x} f(a).
 \end{equation}
 Ist $f$ monoton steigend, so existieren die Grenzwerte $\lim_{a \uparrow x} f(a)$ und $\lim_{a \downarrow x} f(a)$, und es ist
 \[
  \lim_{a \uparrow x} f(a) = \sup_{a < x} f(a)
  \quad
  \text{und}
  \quad
  \lim_{a \downarrow x} f(a) = \inf_{a > x} f(a).
 \]
 Also übersetzt sich Bedingung \eqref{eqn: beide Grenzwerte} dann in
 \[
  \sup_{a < x} f(a) = f(x) = \inf_{a > x} f(a).
 \]
 Die zweite Aussage ergibt sich analog.
\end{proof}





\section{Uneigentliche Grenzwerte}
Wie bereits für Folgen lassen sich auch für Funktionen uneigentliche Grenzwerte definieren.


\begin{defi}
 Es sei $A \subseteq \R^n$, $f \colon A \to \R$ und $x \in \R^n$ ein Häufungspunkt von $A$. Wir schreiben $\lim_{a \to x, a \in A} f(a) = \infty$, falls es für alle $R > 0$ ein $\delta > 0$ gibt, so dass
 \[
  \|x-a\| < \delta \Rightarrow f(a) \geq R \quad \text{für alle $a \in A$ mit $a \neq x$}.
 \]
 Analog definieren wir $\lim_{a \to x, a \in A} f(a) = -\infty.$
\end{defi}


\begin{bsp}
 \begin{itemize}
  \item
   Es ist
   \[
    \lim_{x \uparrow 0} \frac{1}{x} = -\infty
    \quad
    \text{und}
    \quad
    \lim_{x \downarrow 0} \frac{1}{x} = \infty.
   \]
   Der Grenzwert $\lim_{x \to 0} 1/x$ exstiert nicht (auch nicht uneigentlich).
  \item
   $0$ ist ein Häufungspunkt von $\R^n \setminus \{0\}$ und wir haben
   \[
    \lim_{x \to 0, x \neq 0} \frac{1}{\|x\|} = \infty.
   \]
  \item
   Es ist
   \[
    \lim_{x \downarrow 0} \frac{1}{\exp(x) - 1} = \infty
    \quad
    \text{und}
    \quad
    \lim_{x \uparrow 0} \frac{1}{\exp(x) - 1} = -\infty.
   \]
 \end{itemize}
\end{bsp}


\begin{defi}
 Es sei $f \colon X \to \R^n$  mit Definitionsbereich $X \subseteq \R$. Für $y \in \R^n$ sagen wir, dass $\lim_{x \to \infty} f(x) = y$, falls
 \begin{enumerate}
  \item
   es gibt $r_0 \in \R$, so dass $f(x)$ für alle $x \geq r_0$ definiert ist, und
  \item
   für alle $\varepsilon > 0$ gibt es $r \geq r_0$, so dass $\|y - f(x)\| < \varepsilon$ für alle $x \geq r$.
 \end{enumerate}
 Analog definiert man die Schreibweise $\lim_{x \to -\infty} f(x) = y$.
\end{defi}


\begin{bsp}
 \begin{itemize}
  \item
   Es ist
   \[
    \lim_{x \to -\infty} \exp(x) = 0.
   \]
  \item
   Es ist
   \[
    \lim_{x \to \infty} \frac{1}{x} = 0
    \quad
    \text{und}
    \quad
    \lim_{x \to -\infty} \frac{1}{x} = 0.
   \]
 \end{itemize}
\end{bsp}


\begin{defi}
 Es sei $X \subseteq \R$ und $f \colon X \to \R$. Wir schreiben, dass $\lim_{x \to \infty} f(x) = \infty$, falls
 \begin{enumerate}
  \item
   es gibt $r_0 \in \R$, so dass $f(x)$ für alle $x \geq r_0$ definiert ist, und
  \item
   für alle $R > 0$ gibt es $r > r_0$, so dass $f(x) \geq R$ für alle $x \geq r$.
 \end{enumerate}
 Analog definiert man die Ausdrücke $\lim_{x \to \infty} f(x) = -\infty$, $\lim_{x \to -\infty} f(x) = \infty$ und $\lim_{x \to -\infty} f(x) = -\infty$. 
\end{defi}


\begin{bsp}
 \begin{itemize}
  \item
   Es ist
   \[
    \lim_{x \to \infty} x^2 = \infty
    \quad
    \text{und}
    \quad
    \lim_{x \to -\infty} x^2 = \infty,
   \]
   und es ist
   \[
    \lim_{x \to \infty} x^3 = \infty
    \quad
    \text{und}
    \quad
    \lim_{x \to -\infty} x^3 = -\infty.
   \]
  \item
   Es ist
   \[
    \lim_{x \to \infty} \exp(x) = \infty.
   \]
 \end{itemize}
\end{bsp}

\begin{bsp}
 Es seien $a,b \in \R$. Wir wollen den Grenzwert 
 \[
  \lim_{x \to \infty} \sqrt{(x-a)(x-b)}-x
 \]
 untersuchen. Hierfür bemerken wir zunächst, dass $(x-a)(x-b) \geq 0$ für alle \mbox{$x \geq \max\{a,b\}$}; der Ausdruck $\sqrt{(x-a)(x-b)}$ ist also für alle $x \geq \max\{a,b\}$ definiert. Zur Bestimmung des Grenzwerts wollen wir den Ausdruck $\sqrt{(x-a)(x-b)}-x$ zunächst umschreiben; für alle $x > \max\{a,b,0\}$ haben wir
 \begin{align*}
  \sqrt{(x-a)(x-b)}-x
  &= \frac{(\sqrt{(x-a)(x-b)}-x)(\sqrt{(x-a)(x-b)}+x)}{\sqrt{(x-a)(x-b)}+x} \\
  &= \frac{(x-a)(x-b)-x^2}{\sqrt{(x-a)(x-b)}+x}
  = \frac{-(a+b)x + ab}{\sqrt{x^2-(a+b)x+ab}+x} \\
  &= \frac{-(a+b)+\frac{ab}{x}}{\sqrt{1-\frac{a+b}{x}+\frac{ab}{x^2}}+1}
 \end{align*}
 Daher ist
 \[
  \lim_{x \to \infty} \sqrt{(x-a)(x-b)}-x
  = \lim_{x \to \infty} \frac{-(a+b)+\frac{ab}{x}}{\sqrt{1-\frac{a+b}{x}+\frac{ab}{x^2}}+1}
  = -\frac{a+b}{2}.
 \]
\end{bsp}


Auch für uneigentliche Grenzwerte gelten (intuitive) Rechenregeln, von denen wir hier einige angeben wollen:
\begin{enumerate}
 \item
  Für $y \in \R^m$ ist genau dann $\lim_{x \to \infty} f(x) = y$, wenn $\lim_{x \downarrow 0} f(1/x) = y$. Die Aussage gilt für eine reellwertige Funktion auch für $y \in \{-\infty, \infty\}$.
 \item
  Wenn $\lim_{x \to a, a \in A} f(x) = \infty$ oder $\lim_{x \to a, a \in A} f(x) = -\infty$, dann ist
  \[
   \lim_{x \to a, a \in A} 1/f(x) = 0.
  \]
 \item
  Ist $\lim_{x \to a, a \in A} f(x) = \infty$, so ist $\lim_{x \to a, a \in A} -f(x) = -\infty$. Analoges gilt für $\lim_{x \to \infty} f(x)$ und $\lim_{x \to -\infty} f(x)$.
\end{enumerate}


Auch uneigentliche Grenzwerte lassen sich durch Folgen charakterisieren.


\begin{enumerate}[resume]
 \item
  Ist $A \subseteq \R^m$, $f \colon A \to \R$ und $x \in A$ ein Häufungspunkt von $A$, so ist genau dann $\lim_{a \to x, a \in A} f(a) = \infty$, falls für jede Folge $(a_n)$ auf $A \setminus \{x\}$ mit $a_n \to x$ auch $f(a_n) \to \infty$. Eine analoge Aussage gilt für $\lim_{a \to x, a \in A} f(a) = -\infty$.
 \item
  Für $y \in \R^m$ ist genau dann $\lim_{x \to \infty} f(x) = y$, wenn für jede Folge $(x_n)$ auf $\R$ mit $x_n \to \infty$ auch $f(x_n) \to y$. Eine analoge Aussage gilt für $\lim_{x \to -\infty} f(x)$. Ist $f$ eine reellwertige Funktion, so gilt die Aussage auch für $y \in \{-\infty, \infty\}$.
\end{enumerate}









\newpage


\section{Lösungen}


\printsolutions



\end{document}
