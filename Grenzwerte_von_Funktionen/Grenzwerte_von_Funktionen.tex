\documentclass[a4paper,10pt]{article}
%\documentclass[a4paper,10pt]{scrartcl}

\usepackage{../mystyle}

\setromanfont[Mapping=tex-text]{Linux Libertine O}
% \setsansfont[Mapping=tex-text]{DejaVu Sans}
% \setmonofont[Mapping=tex-text]{DejaVu Sans Mono}

\title{Grenzwerte von Funktionen}
\author{Jendrik Stelzner}
\date{\today}

\begin{document}
\maketitle

\tableofcontents





\section{Häufungspunkte}


Es sei $A \subseteq \R^n$ und $f \colon A \to \R$. Wir wollen untersuchen, wie sich $f$ an einer Stelle $x \in \R^n$ verhält, bzw.\ verhalten sollte. Um das Verhalten von $f$ an $x$ zu untersuchen, brauchen wir, dass $f$ „in der Nähe“ von $x$ definiert ist. Hierfür brauchen wir, dass $x$ „nahe“ an $A$ ist. Dies motiviert die folgende Definition:


\begin{defi}
 Es sei $A \subseteq \R^n$. Ein Punkt $x \in \R^n$ heißt \emph{Häufungspunkt von $A$}, falls es für alle $\varepsilon > 0$ ein $a \in A$ mit $\|x-a\| < \varepsilon$ und $x \neq a$ gibt.
 
 Wir bezeichnen die Menge aller Häufungspunkte von $A$ mit $A'$.
\end{defi}


Vorstellungsmäßig ist $x \in \R^n$ ein Häufungspunkt von $A \subseteq \R^n$, falls sich $x$ \emph{von außen} durch Punkte aus $A$ annähern lässt.


\begin{bsp}
 \begin{itemize}
  \item
   $x = 0$ ist ein Häufungspunkt von $A \coloneqq \{1/n \mid n \geq 1\} \subseteq \R$: Da $\lim_{n \to \infty} 1/n = 0$ gibt es für jedes $\varepsilon > 0$  ein $n \geq 1$ mit $|x-1/n| < \varepsilon$, wobei klar ist, dass $1/n \neq 0$.
  \item
   Der Punkt $x = 2$ ist \emph{kein} Häufungspunkt der Menge $A \coloneqq [0,1] \cup \{2\} \subseteq \R$, denn das einzige $a \in A$ mit $|x-a| < 1/2$ ist $a = 2$.
  \item
   Ist $U \subseteq \R^n$ offen, so ist jeder Punkt $x \in U$ ein Häufungspunkt von $U$: Da $U$ offen ist, gibt es ein $\delta > 0$ mit $B_\delta(x) \subseteq U$. Für jedes $\varepsilon > 0$ gibt es für $\omega \coloneqq \min\{\varepsilon,\delta\}$ daher ein
   \[
    y \in B_\omega(x) \subseteq B_\delta(x) \subseteq U \quad \text{mit $y \neq x$},
   \]
   und es gilt $\|x-y\| < \omega \leq \varepsilon$.
  \item
   Allgemeiner ergibt sich mit dieser Argumentation, dass $x \in \R^n$ ein Häufungspunkt von $V \subseteq \R^n$ ist, falls $V$ eine Umgebung von $x$ ist.
 \end{itemize}
\end{bsp}


\begin{question}
 Es sei $A \subseteq \R^m$ und $x \in \R^m$. Zeigen Sie, dass $x$ genau dann ein Häufungspunkt von $A$ ist, falls es eine Folge $(a_n)$ auf $A \setminus \{x\}$ gibt, so dass $a_n \to x$.
\end{question}
\begin{solution}
 Angenommen, $x$ ist ein Häufungspunkt von $A$. Dann gibt es für jedes $n \geq 1$ ein $a_n \in A \setminus \{x\}$ mit $|x-a_n| < 1/n$. Die Folge $(x_n)_{n \geq 1}$ konvergiert per Konstruktion gegen $x$.
 
 Angenommen, eine solche Folge $(a_n)_{n \in \N}$ existiert. Dann gibt es für jedes $\varepsilon > 0$ ein $N \in \N$, so dass $|x-a_n| < \varepsilon$ für alle $n \geq N$. Inbesondere ist $a_N \in A$ mit $|x-a_N| < \varepsilon$ und $a_N \neq x$.
\end{solution}
 

\begin{question}
 Es sei $M \subseteq \R^n$ endlich. Zeigen Sie, dass $M' = \emptyset$.
\end{question}
\begin{solution}
 Es sei $x \in \R^n$. Ist $x \notin M$, so ergibt sich für
 \[
  \varepsilon \coloneqq \frac{1}{2} \min_{m \in M} \|x-m\| > 0,
 \]
 dass es kein $m \in M$ mit $\|x-m\| < \varepsilon$ gibt. Also ist $x$ dann kein Häufungspunkt von $M$. Ist $x \in M$, so ergibt sich für
 \[
  \varepsilon
  \coloneqq
  \begin{cases}
   \frac{1}{2} \min_{m \in M, m \neq x} \|x-m\| & \text{falls $|M| \geq 2$} \\
                                               1 & \text{falls $M = \{x\}$},
  \end{cases}
 \]
 dass $x$ das einzige $m \in M$ mit $\|x-m\| < \varepsilon$ ist. Also ist $x$ auch dann kein Häufungspunkt von $M$.
\end{solution}


\begin{question}
 Bestimmen Sie $\Z'$.
\end{question}
\begin{solution}
 Es sei $x \in \R$. Ist $x \notin \Z$, so gibt es für
 \[
  \varepsilon \coloneq \frac{1}{2} \min\{ \lceil x \rceil - x, x - \lfloor x \rfloor \}
 \]
 kein $n \in \Z$ mit $\|x-n\| < \varepsilon$. Also ist $x$ dann kein Häufungspunkt von $\Z$. Ist andererseits $x \in \Z$, so gibt es außer $x$ kein $n \in \Z$ mit $\|x-n\| < 1/2$, weshalb $x$ auch dann kein Häufungspunkt von $\Z$ ist.
 
 Also ist kein $x \in \R$ ein Häufungspunkt von $\Z$, und somit $\Z' = \emptyset$.
\end{solution}


\begin{question}
 Es seien $A, B \subseteq \R$. Zeigen Sie:
 \begin{enumerate}
  \item
   Ist $A \subseteq B$, so ist $A' \subseteq B'$.
  \item
   Es ist $(A \cup B)' = A' \cup B'$.
 \end{enumerate}
\end{question}
\begin{solution}
 \begin{enumerate}
  \item
   Es sei $x \in A'$. Für jedes $\varepsilon > 0$ gibt es daher ein $a \in A$ mit $\|x-a\| < \varepsilon$ und $a \neq x$. Da $a \in A \subseteq B$ folgt, dass es für jedes $\varepsilon > 0$ ein $b \in B$ mit $\|x-b\| < \varepsilon$ und $b \neq x$ gibt. Also ist $x$ ein Häufungspunkt von $B$, also $b \in B'$. Aus der Beliebigkeit von $a \in A'$ folgt, dass $A' \subseteq B'$.
  \item
   Da $A \subseteq A \cup B$ ist $A' \subseteq (A \cup B)'$, und da $B \subseteq A \cup B$ ist $B' \subseteq (A \cup B)'$. Also ist auch $A' \cup B' \subseteq (A \cup B)'$.
   
   Angenommen, es ist $x \notin A' \cup B'$. Dann gibt es $\varepsilon_A, \varepsilon_B > 0$, so dass es kein $a \in A$ mit $\|x-a\| < \varepsilon$ und $x \neq a$ gibt, und auch kein $b \in B$ mit $\|x-b\| < \varepsilon$ und $x \neq b$. Für $\varepsilon \coloneqq \min\{\varepsilon_A, \varepsilon_B\}$ gibt es daher kein $y \in A \cup B$ mit $\|x-y\| < \varepsilon$ und $y \neq x$. Also ist dann $x \notin (A \cup B)'$. Das zeigt, dass auch $(A \cup B)' \subseteq A' \cup B'$.
  \qedhere
 \end{enumerate}
\end{solution}


\begin{question}
 Bestimmen Sie $A'$ für $A \coloneqq [0,1] \cup [2,3]$.
\end{question}
\begin{solution}
 \begin{beh}
  Für alle $a, b \in \R$ mit $a < b$ ist
  \[
   [a,b]' = [a,b].
  \]
 \end{beh}
 \begin{proof}[Beweis der Behauptung]
  Für $x < a$  ist $a - x > 0$, es gibt aber kein $y \in [a,b]$ mit $\|x-y\| < a-x$; also ist dann $x \notin [a,b]'$ Analog ergibt sich, dass auch $x \notin [a,b]'$ für $x > b$. Also ist $[a,b]' \subseteq [a,b]$.
  
  Dass $a,b \in [a,b]'$ ergibt sich durch die Folgen $(x_n)$ auf $(a,b]$ und $(y_n)$ auf $[a,b)$ mit
  \[
   x_n \coloneqq a + \frac{b-a}{n+1}
   \quad
   \text{und}
   \quad
   y_n \coloneqq b - \frac{b-a}{n+1}
   \quad
   \text{für alle $n \in \N$}.
  \]
  Dass $x \in [a,b]'$ für $a < x < b$ ergibt sich daraus, dass $[a,b]$ eine Umgebung für diese $x$ ist. Damit ergibt sich, dass $[a,b] \subseteq [a,b]'$.
 \end{proof}
 
 Aus der Behauptung ergibt sich direkt, dass
 \[
  ([0,1] \cup [2,3])'
  = [0,1]' \cup [2,3]'
  = [0,1] \cup [2,3].
 \]
\end{solution}





\section{Grenzwerte von Funktionen}


\begin{defi}
 Es sei $A \subseteq \R^n$, $f \colon A \to \R^m$ und $x \in \R^n$ ein Häufungspunkt von $A$. Für $y \in \R^m$ schreiben wir
 \[
  \lim_{\substack{a \to x \\ a \in A}} f(a) = y,
 \]
 falls es für jedes $\varepsilon > 0$ ein $\delta > 0$ gibt, so dass
 \[
  \|x-a\| < \delta \Rightarrow \|f(x)-f(a)\| < \varepsilon
  \quad
  \text{für alle $a \in A$ mit $a \neq x$}.
 \]
 Wir nennen $y$ dann den \emph{Grenzwert von $f$ an $x$ über $A$}.
\end{defi}


\begin{bsp}
 \begin{itemize}
  \item
   Wir betrachten die \emph{Signumabbildung}
   \[
    \sgn \colon \R \to \R,
    x \mapsto
    \begin{cases}
     -1 & \text{falls $x < 0$}, \\
      0 & \text{falls $x = 0$}, \\
      1 & \text{falls $x > 0$}.
    \end{cases}
   \]
   $0$ ist ein gemeinsamer Häufungspunkt von $(-\infty,0)$ und $(0,\infty)$, und es ist
   \[
    \lim_{\substack{x \to 0 \\ x \in (-\infty, 0)}} f(x) = -1,
    \quad
    \text{und}
    \quad
    \lim_{\substack{x \to 0 \\ x \in (0, \infty)}} f(x) = 1.
   \]
  \item
   Wir betrachten die Abbildung
   \[
    f \colon (0,\infty) \to \R, x \mapsto \sin \frac{1}{x}.
   \]
   $0$ ist ein Häufungspunkt der beiden Mengen
   \[
    A \coloneqq \left\{ \frac{1}{\frac{\pi}{2} + n \cdot 2\pi} \,\middle|\, n \in \N \right\}
    \quad
    \text{und}
    \quad
    B \coloneqq \left\{ \frac{1}{\frac{3\pi}{2} + n \cdot 2\pi} \,\middle|\, n \in \N \right\},
   \]
   und es ist
   \[
    \lim_{\substack{x \to 0 \\ x \in A}} f(x) = 1
    \quad
    \text{und}
    \quad
    \lim_{\substack{x \to 0 \\ x \in B}} f(x) = -1.
   \]
   $0$ ist auch ein Häufungspunkt der Menge $(0,\infty)$, der Grenzwert
   \[
    \lim_{\substack{x \to 0 \\ x \in (0,\infty)}} f(x)
   \]
   existiert jedoch nicht.
 \end{itemize}
\end{bsp}


\begin{lem}
 Es sei $A \subseteq \R^m$, $f \colon A \to \R^k$ und $x \in \R^m$ ein Häufungspunkt von $A$. Für $y \in \R^k$ sind äquivalent:
 \begin{enumerate}
  \item\label{enum: Grenzwert mit epsilon delta}
   $\lim_{a \to x, a \in A} f(a) = y$.
  \item\label{enum: Grenzwert mit Folgen}
   Für jede Folge $(a_n)$ auf $A \setminus \{x\}$ mit $a_n \to x$ gilt, dass auch $f(a_n) \to y$.
 \end{enumerate}
 (Da $x$ ein Häufungspunkt von $A$ ist, existiert auch eine entsprechende Folge.)
\end{lem}
\begin{proof}
 (\ref{enum: Grenzwert mit epsilon delta} $\Rightarrow$ \ref{enum: Grenzwert mit Folgen}) Es sei $(a_n)$ eine Folge auf $A \setminus \{x\}$ mit $a_n \to x$. Wir wollen zeigen, dass $f(a_n) \to y$. Sei hierfür $\varepsilon > 0$ beliebig aber fest. Da $\lim_{a \to x, a \in A} f(a) = y$ gibt es ein $\delta > 0$, so dass
 \[
  \|x-a\| < \delta \Rightarrow \|f(x)-f(a)\| < \varepsilon
  \quad \text{für alle $a \in A$ mit $a \neq x$}.
 \]
 Da $a_n \to x$ gibt es ein $N \in \N$ mit $\|x - a_n\| < \delta$ für alle $n \geq N$. Da $a_n \neq x$ ist deshalb $\|f(x)-f(a_n)\| < \varepsilon$ für alle $n \geq N$.
 
 (\ref{enum: Grenzwert mit Folgen} $\Rightarrow$ \ref{enum: Grenzwert mit epsilon delta}) Angenommen es ist nicht $\lim_{a \to x, a \in A} f(a) = y$. Dann gibt es ein $\varepsilon > 0$, so dass es für alle $\delta > 0$ ein $a \in A$ gibt, so dass zwar $x \neq a$ und $\|x-a\| < \delta$, aber $\|y-f(a)\| \geq \varepsilon$. Insbesondere gibt es deshalb für alle $n \geq 1$ ein $a_n \in A \setminus \{x\}$ mit $\|x-a_n\| < 1/n$ aber $\|y-f(a_n)\| \geq \varepsilon$. Dann ist $(a_n)$ eine Folge auf $A \setminus \{x\}$ mit $a_n \to x$, aber nicht $f(a_n) \rightarrow y$.
\end{proof}


Mithilfe dieses Lemmas können wir viele Aussagen für die Grenzwerte von Folgen auf Grenzwerte von Funktionen übertragen.


\begin{prop}
 Es seien $A \subseteq \R^n$, $f, f_1, f_2 \colon A \to \R^m$, $\lambda \in \R$ und $x \in \R^n$ ein Häufungspunkt von $A$.
 \begin{enumerate}
  \item
   Der Grenzwert $\lim_{a \to x, a \in A} f(a)$ ist eindeutig (wenn er existiert).
  \item
   Existieren die Grenzwerte $\lim_{a \to x, a \in A} f_1(a)$ und $\lim_{a \to x, a \in B} f_2(a)$, so existiert auch der Grenzwert $\lim_{x \to a, a \in A} (f_1 + f_2)(a)$ und es gilt
   \[
    \lim_{\substack{a \to x \\ a \in A}} (f_1 + f_2)(a)
    =
    \left( \lim_{\substack{a \to x \\ a \in A}} f_1(a) \right)
    + \left( \lim_{\substack{a \to x \\ a \in A}} f_2(a) \right).
   \]
  \item
   Existiert der Grenzwert $\lim_{a \to x, a \in A} f(a)$, so existiert auch $\lim_{a \to x, a \in A} (\lambda f)(a)$, und es gilt
   \[
    \lim_{\substack{a \to x \\ a \in A}} (\lambda f)(a)
    = \lambda \lim_{\substack{a \to x \\ a \in A}} f(a).
   \]
 \end{enumerate}
 Im Fall $n = 1$, also für $\R^1 = \R$, gilt auch eine Verträglichkeit mit Multiplikation und Division.
 \begin{enumerate}[resume]
  \item
   Existieren die Grenzwerte $\lim_{a \to x, a \in A} f_1(a)$ und $\lim_{a \to x, a \in B} f_2(a)$, so existiert auch der Grenzwert $\lim_{x \to a, a \in A} (f_1 \cdot f_2)(a)$ und es gilt
   \[
    \lim_{\substack{a \to x \\ a \in A}} (f_1 \cdot f_2)(a)
    =
    \left( \lim_{\substack{a \to x \\ a \in A}} f_1(a) \right)
    \cdot \left( \lim_{\substack{a \to x \\ a \in A}} f_2(a) \right).
   \]
  \item
   Existieren die beiden Grenzwerte $\lim_{a \to x, a \in A} f_1(a)$ und $\lim_{a \to x, a \in A} f_2(a)$, und ist \mbox{$f_2(a) \neq 0$} für alle $a \in A \setminus \{x\}$ sowie $\lim_{a \to x, a \in A} f_2(a) \neq 0$, so existiert auch der Grenzwert $\lim_{a \to x, a \in A} f_1(a)/f_2(a)$ und es gilt
   \[
    \lim_{\substack{a \to x \\ a \in A}} \frac{f_1(a)}{f_2(a)}
    = \frac{\lim_{a \to x, a \in A} f_1(a)}{\lim_{a \to x, a \in A} f_2(a)}
   \]
 \end{enumerate}
 (Sehen wir $\R^2 \cong \C$, so ergibt sich auch eine Verträglichkeit mit der Multiplikation und Division im Komplexen; hierdrauf wollen wir jetzt aber nicht weiter eingehen.)
\end{prop}


Wie wir bereits gesehen haben, können für eine Funktion $f \colon X \to \R^m$ mit Definitionsbereich $X \subseteq \R^n$ und Teilmengen $A, B \subseteq X$ mit gemeinsamen Häufungspunkt $x \in \R^n$ die beiden Grenzwerte $\lim_{a \to x, a \in A} f(a)$ und $\lim_{b \to x, b \in B} f(b)$ existieren, aber dennoch
\[
 \lim_{\substack{a \to x \\ a \in A}} f(a)
 \neq
 \lim_{\substack{b \to x \\ b \in B}} f(b).
\]
Es kann auch passieren, dass einer der beiden Grenzwerte existiert, der andere jedoch nicht. Es gibt also im Allgemeinen keinen Zusammehang zwischen dem Grenzwert von $f$ über $A$ und dem Grenzwert über $B$. Unter bestimmten Umständen lassen sich die entsprechenden Grenzwerte dennoch vergleichen:


\begin{lem}\label{lem: Grenzwerte auf Teilmengen}
 Es seien $A \subseteq B \subseteq \R^n$ und $f \colon B \to \R^m$. Ist $x \in \R^n$ ein gemeinsamer Häufungspunkt von $A$ und $B$, sodass $\lim_{b \to x, b \in B} f(b)$ existiert, so existiert auch der Grenzwert $\lim_{a \to x, a \in A} f(a)$, und es gilt
 \[
  \lim_{\substack{a \to x \\ a \in A}} f(a)
  = \lim_{\substack{b \to x \\ b \in B}} f(b).
 \]
\end{lem}
\begin{proof}
 Zur besseren Lesbarkeit setzen wir $y \coloneqq \lim_{b \to x, b \in B} f(b)$. Wir wollen zeigen, dass auh $\lim_{a \to x, x \in A} f(a) = y$. Sei hierfür $\varepsilon > 0$ beliebig aber fest. Da $y = \lim_{b \to x, b \in B} f(b)$ gibt es ein $\delta > 0$, so dass
 \[
  \|x-b\| < \delta \Rightarrow \|y-f(b)\| < \varepsilon \quad \text{für alle $b \in B$ mit $b \neq x$}.
 \]
 Da $A \subseteq B$ ist daher insbesondere
 \[
  \|x-a\| < \delta \Rightarrow \|y-f(a)\| < \varepsilon \quad \text{für alle $a \in A$ mit $a \neq x$}.
 \]
 Wegen der Beliebigkeit von $\varepsilon > 0$ folgt, dass $\lim_{a \to x, a \in A} f(a) = y$.
\end{proof}


\begin{kor}\label{kor: gemeinsamer Häufungspunkt mit gemeinsamen Grenzwert}
 Es sei $f \colon X \to \R$ mit Definitionsbereich $X \subseteq \R$. Es seien $A, B \subseteq X$ Teilmengen, so dass $x \in \R$ ein gemeinsamer Häufungspunkt von $A$, $B$ ist, und die beiden Grenzwerte $\lim_{a \to x, a \in A} f(a)$ und $\lim_{b \to x, b \in B} f(b)$ existieren. Ist $x$ auch ein Häufungspunkt von $A \cap B$, so ist
 \[
  \lim_{\substack{a \to x \\ a \in A}} f(a)
  = \lim_{\substack{b \to x \\ b \in B}} f(b).
 \]
\end{kor}
\begin{proof}
 Da die beiden Grenzwerte $\lim_{a \to x, a \in A} f(a)$ und $\lim_{b \to x, b \in B} f(b)$ existieren, und $A \cap B \subseteq A$ und $A \cap B \subseteq B$, erhalten wir aus Lemma \ref{lem: Grenzwerte auf Teilmengen}, dass auch der Grenzwert $\lim_{c \to x, c \in A \cap B} f(c)$ existiert und
 \[
  \lim_{\substack{a \to x \\ a \in A}} f(a)
  = \lim_{\substack{c \to x \\ c \in A \cap B}} f(c)
  = \lim_{\substack{b \to x \\ b \in B}} f(b).
  \qedhere
 \]
\end{proof}





\section{Links-, rechts- und beidseitige Grenzwerte}
Wir wollen uns nun einem Sonderfall von Funktionsgrenzwerten zuwenden.


\begin{defi}
 Es sei $f \colon A \to \R^m$ mit Definitionsbereich $A \subseteq \R$ und $x \in \R$.
 
 Gibt es ein $r > 0$, so dass $(x-r, x) \subseteq A$, so schreiben wir
 \[
  \lim_{a \uparrow x} f(a)
  \quad
  \text{für}
  \quad
  \lim_{\substack{a \to x \\ a \in (x-r,x)}} f(a),
 \]
 und nennen dies den \emph{linksseitigen Grenzwert von $f$ an $x$}.
 
 Gibt es ein $r > 0$, so dass $(x,x+r) \subseteq A$, so schreiben wir
 \[
  \lim_{a \downarrow x} f(a)
  \quad
  \text{für}
  \quad
  \lim_{\substack{a \to x \\ a \in (x,x+r)}} f(a),
 \]
 und nennen dies den \emph{rechtsseitigen Grenzwert von $f$ an $x$}.
 
 Existiert ein $r > 0$, so dass $(x-r,x) \cup (x,x+r) \subseteq A$, so schreiben wir
 \[
  \lim_{a \to x} f(a)
  \quad
  \text{für}
  \quad
  \lim_{\substack{a \to x \\ a \in (x-r,x) \cup (x,x+r)}} f(a),
 \]
 und nennen dies den \emph{beidseitigen Grenzwert von $f$ an $x$}.
\end{defi}


Die Wohldefiniertheit der jeweiligen Ausdrücke, also die Unabhängigkeit von $r$, ergibt sich aus Korollar \ref{kor: gemeinsamer Häufungspunkt mit gemeinsamen Grenzwert}.


Der beidseitige Grenzwert lässt sich auch als Kombination des links- und rechtsseitigen Grenzwertes definieren:


\begin{lem}
 Es sei $A \subseteq \R$ und $f \colon A \to \R$. Für $x \in \R$ und $y \in \R$ sind äquivalent:
 \begin{enumerate}
  \item
   Der beidseitige Grenzwert $\lim_{a \to x} f(a)$ existiert und $\lim_{a \to x} f(a) = y$.
  \item
   Die Grenzwerte $\lim_{a \uparrow x} f(a)$ und $\lim_{a \downarrow x} f(a)$ existieren und es ist
   \[
    \lim_{a \uparrow x} f(a) = y = \lim_{a \downarrow x} f(a).
   \]
 \end{enumerate}
\end{lem}





\section{Uneigentliche Grenzwerte}


\begin{defi}
 Es sei $A \subseteq \R^n$, $f \colon A \to \R^m$ und $x \in \R^n$ ein Häufungspunkt von $A$. Wir schreiben dass $\lim_{a \to x, a \in A} f(x) = \infty$, falls es für alle $R > 0$ ein $\delta > 0$ gibt, so dass
 \[
  \|x-a\| < \delta \Rightarrow \|f(a)\| \geq R \quad \text{für alle $a \in A$ mit $a \neq x$}.
 \]
 Wir schreiben $\lim_{a \to x, a \in A} f(x) = -\infty$, falls es für alle $R > 0$ ein $\delta > 0$ gibt, so dass
 \[
  \|x-a\| < \delta \Rightarrow \|f(a)\| \leq -R \quad \text{für alle $a \in A$ mit $a \neq x$}.
 \]
\end{defi}


\begin{defi}
 Es sei $f \colon X \to \R$ eine Abbildung mit Definitionsbereits $X \subseteq \R$. Für $y \in \R$ sagen wir, dass $\lim_{x \to \infty} f(x) = y$, falls
 \begin{enumerate}
  \item
   es gibt $r_0 \in \R$, so dass $f(x)$ für alle $x \geq r_0$ definiert ist, und
  \item
   für alle $\varepsilon > 0$ gibt es $r \geq r_0$, so dass $|f(x) - y| < \varepsilon$ für alle $x \geq r$.
 \end{enumerate}
 Wir sagen, dass $\lim_{x \to -\infty} f(x) = y$, falls
 \begin{enumerate}
  \item
   es gibt $r_0 \in \R$, so dass $f(x)$ für alle $x \leq r_0$ definiert ist, und
  \item
   für alle $\varepsilon > 0$ gibt es $r \leq r_0$, so dass $|f(x) - y| < \varepsilon$ für alle $x \leq r$.
 \end{enumerate}
\end{defi}









\newpage


\section{Lösungen der Übungen}


\printsolutions



\end{document}
