\documentclass[a4paper,10pt]{article}
%\documentclass[a4paper,10pt]{scrartcl}

\usepackage{../mystyle}

\setromanfont[Mapping=tex-text]{Linux Libertine O}
% \setsansfont[Mapping=tex-text]{DejaVu Sans}
% \setmonofont[Mapping=tex-text]{DejaVu Sans Mono}

\title{Grenzwerte von Funktionen}
\author{Jendrik Stelzner}
\date{\today}

\begin{document}
\maketitle

\tableofcontents





\section{Häufungspunkte}


Es sei $A \subseteq \R$ und $f \colon A \to \R$. Wir wollen untersuchen, wie sich $f$ an einer Stelle $x \in \R$ verhält, bzw.\ verhalten sollte. Hierfür betrachten wir die folgenden beiden Ansätze:


\begin{enumerate}
 \item
  Falls $x \in A$, was ist $f(x)$?
 \item
  Was verhält sich $f$ in der Nähe von $x$?
\end{enumerate}


Der zweite Ansatz hat den Vorteil, dass wir ihn auch für $x \notin A$ nutzen können, sofern $f$ in der Nähe von $x$ definiert ist.

Falls $x \in A$ können wir auch untersuchen, inwiefern sich die Ergebnisse der beiden Ansätze unterscheidenn, d.h.\ inwiefern das Verhalten von $f$ an der Stelle $x$ mit dem Verhalten in der Nähe von $x$ verträglich ist.


\begin{defi}
 Es sei $A \subseteq \R$. Ein Punkt $x \in \R$ heißt \emph{Häufungspunkt von $A$}, falls es für alle $\varepsilon > 0$ ein $a \in A$ mit $|x-a| < \varepsilon$ und $x \neq a$ gibt.
 
 Wir bezeichnen die Menge aller Häufungspunkte von $A$ mit $A'$.
\end{defi}


Vorstellungsmäßig ist $x \in \R$ ein Häufungspunkt von $A \subseteq \R$, falls sich $x$ \emph{von außen} durch Punkte aus $A$ annähern lässt.


\begin{bsp}
 \begin{itemize}
  \item
   $x = 0$ ist ein Häufungspunkte der Menge $A \coloneqq \{1/n \mid n \geq 1\}$. Da $\lim_{n \to \infty} 1/n = 0$ gibt es für jedes $\varepsilon > 0$  ein $n \geq 1$ mit $|x-1/n| < \varepsilon$, und es ist $1/n \neq 0$.
  \item
   Der Punkt $x = 2$ ist \emph{kein} Häufungspunkt der Menge $A \coloneqq [0,1] \cup \{2\}$, denn das einzige $a \in A$ mit $|x-a| < 1/2$ ist $a = 2$.
 \end{itemize}
\end{bsp}


\begin{question}
 Es sei $A \subseteq \R$ und $x \in \R$. Zeigen Sie, dass $x$ genau dann ein Häufungspunkt von $A$ ist, falls es eine Folge $(a_n)_{n \in \N}$ gibt, so dass $a_n \in A \setminus \{x\}$ für alle $n \in \N$ und $\lim_{n \to \infty} a_n = x$.
\end{question}
\begin{solution}
 Angenommen, $x$ ist ein Häufungspunkt von $A$. Dann gibt es für jedes $n \geq 1$ ein $a_n \in A \setminus \{x\}$ mit $|x-a_n| < 1/n$. Die Folge $(x_n)_{n \geq 1}$ konvergiert per Konstruktion gegen $x$.
 
 Angenommen, eine solche Folge $(a_n)_{n \in \N}$ existiert. Dann gibt es für jedes $\varepsilon > 0$ ein $N \in \N$, so dass $|x-a_n| < \varepsilon$ für alle $n \geq N$. Inbesondere ist $a_N \in A$ mit $|x-a_N| < \varepsilon$ und $a_N \neq x$.
\end{solution}


\begin{question}
 Es sei $M$ eine endliche Menge. Zeigen Sie, dass $M' = \emptyset$.
\end{question}


\begin{question}
 Bestimmen Sie $\Z'$.
\end{question}


\begin{question}
 Es seien $A, B \subseteq \R$. Zeigen Sie:
 \begin{enumerate}
  \item
   Ist $A \subseteq B$, so ist $A' \subseteq B'$.
  \item
   Es ist $(A \cup B)' = A' \cup B'$.
 \end{enumerate}
\end{question}


\begin{question}
 Bestimmen Sie $A'$ für $A \coloneqq [0,1] \cup [2,3]$.
\end{question}





\section{Grenzwerte von Funktionen}


\begin{defi}
 Es sei $A \subseteq \R$ und $x \in \R$ ein Häufungspunkt von $A$. Für $y \in \R$ schreiben wir
 \[
  \lim_{\substack{a \to x \\ a \in A}} f(a) = y,
 \]
 falls es für jedes $\varepsilon > 0$ ein $\delta > 0$ gibt, so dass
 \[
  |x-a| < \delta \Rightarrow |f(x)-f(a)| < \varepsilon
  \quad
  \text{für alle $a \in A$ mit $a \neq x$}.
 \]
 Wir nennen $y$ dann den \emph{Grenzwert von $f$ an $x$ über $A$}.
\end{defi}


\begin{bsp}
 \begin{itemize}
  \item
   Wir betrachten die \emph{Signumabbildung}
   \[
    \sgn \colon \R \to \R,
    x \mapsto
    \begin{cases}
     -1 & \text{falls $x < 0$}, \\
      0 & \text{falls $x = 0$}, \\
      1 & \text{falls $x > 0$}.
    \end{cases}
   \]
   Der Punkt $0$ ist ein Häufungspunkt von $(-\infty,0)$ und $(0,\infty)$, und es ist
   \[
    \lim_{\substack{x \to 0 \\ x \in (-\infty, 0)}} f(x) = -1,
    \quad
    \text{und}
    \quad
    \lim_{\substack{x \to 0 \\ x \in (0, \infty)}} f(x) = 1.
   \]
  \item
   Wir betrachten die Abbildung
   \[
    f \colon (0,\infty) \to \R, x \mapsto \sin \frac{1}{x}.
   \]
   $0$ ist ein Häufungspunkt der beiden Mengen
   \[
    A \coloneqq \left\{ \frac{1}{\frac{\pi}{2} + n \cdot 2\pi} \mid n \in \N \right\}
    \quad
    \text{und}
    \quad
    B \coloneqq \left\{ \frac{1}{\frac{3\pi}{2} + n \cdot 2\pi} \mid n \in \N \right\},
   \]
   und es ist
   \[
    \lim_{\substack{x \to 0 \\ x \in A}} f(x) = 1
    \quad
    \text{und}
    \quad
    \lim_{\substack{x \to 0 \\ x \in B}} f(x) = -1.
   \]
   $0$ ist auch ein Häufungspunkt der Menge $(0,\infty)$, der Grenzwert
   \[
    \lim_{\substack{x \to 0 \\ x \in (0,\infty)}} f(x)
   \]
   existiert jedoch nicht.
 \end{itemize}
\end{bsp}


\begin{lem}
 Es sei $A \subseteq \R$, $f \colon A \to \R$ und $x \in \R$ ein Häufungspunkt von $A$. Für $y \in \R$ sind äquivalent:
 \begin{enumerate}
  \item
   $\lim_{a \to x, a \in A} f(a) = x$.
  \item
   Für jede Folge $(a_n)_{n \in \N}$ mit $a_n \in A \setminus \{x\}$ für alle $n \in \N$ und $\lim_{n \to \infty} a_n = x$ gilt, dass $\lim_{n \to \infty} f(a_n) = f(x)$.
 \end{enumerate}
\end{lem}


Mithilfe dieses Lemmas können wir viele Aussagen für die Grenzwerte von Folgen auf Grenzwerte von Funktionen übertragen.


\begin{prop}
 Es seien $A \subseteq \R$, $x$ ein Häufungspunkt von $A$, $f, f_1, f_2 \colon A \to \R$ und $\lambda \in \R$.
 \begin{enumerate}
  \item
   Der Grenzwert $\lim_{a \to x, a \in A} f(a)$ ist eindeutig (wenn er existiert).
  \item
   Existieren die Grenzwerte $\lim_{a \to x, a \in A} f_1(a)$ und $\lim_{a \to x, a \in B} f_2(a)$, so existiert auch der Grenzwert $\lim_{x \to a, a \in A} (f_1 + f_2)(a)$ und es gilt
   \[
    \lim_{\substack{a \to x \\ a \in A}} (f_1 + f_2)(a)
    =
    \left( \lim_{\substack{a \to x \\ a \in A}} f_1(a) \right)
    + \left( \lim_{\substack{a \to x \\ a \in A}} f_2(a) \right).
   \]
  \item
   Existiert der Grenzwert $\lim_{a \to x, a \in A} f(a)$, so existiert auch $\lim_{a \to x, a \in A} (\lambda f)(a)$, und es gilt
   \[
    \lim_{\substack{a \to x \\ a \in A}} (\lambda f)(a)
    = \lambda \lim_{\substack{a \to x \\ a \in A}} f(a).
   \]
  \item
   Existieren die Grenzwerte $\lim_{a \to x, a \in A} f_1(a)$ und $\lim_{a \to x, a \in B} f_2(a)$, so existiert auch der Grenzwert $\lim_{x \to a, a \in A} (f_1 \cdot f_2)(a)$ und es gilt
   \[
    \lim_{\substack{a \to x \\ a \in A}} (f_1 \cdot f_2)(a)
    =
    \left( \lim_{\substack{a \to x \\ a \in A}} f_1(a) \right)
    \cdot \left( \lim_{\substack{a \to x \\ a \in A}} f_2(a) \right).
   \]
  \item
   Existieren die beiden Grenzwerte $\lim_{a \to x, a \in A} f_1(a)$ und $\lim_{a \to x, a \in A} f_2(a)$, und ist \mbox{$f_2(a) \neq 0$} für alle $a \in A \setminus \{x\}$ sowie $\lim_{a \to x, a \in A} f_2(a) \neq 0$, so existiert auch der Grenzwert $\lim_{a \to x, a \in A} f_1(a)/f_2(a)$ und es gilt
   \[
    \lim_{\substack{a \to x \\ a \in A}} \frac{f_1(a)}{f_2(a)}
    = \frac{\lim_{a \to x, a \in A} f_1(a)}{\lim_{a \to x, a \in A} f_2(a)}
   \]
 \end{enumerate}
\end{prop}


Wie wir bereits gesehen haben, können für eine Funktion $f \colon X \to \R$ mit Definitionsbereich $X \subseteq \R$ und Teilmengen $A, B \subseteq X$ mit gemeinsamen Häufungspunkt $x \in \R$ die beiden Grenzwerte $\lim_{a \to x, a \in A} f(a)$ und $\lim_{b \to x, b \in B} f(b)$ existieren, aber dennoch
\[
 \lim_{\substack{a \to x \\ a \in A}} f(a)
 \neq
 \lim_{\substack{b \to x \\ b \in B}} f(b).
\]
Es kann auch passieren, dass einer der beiden Grenzwerte existiert, der andere jedoch nicht. Es gibt also im Allgemeinen keinen Zusammehang zwischen dem Grenzwert von $f$ über $A$ und dem Grenzwert über $B$.


\begin{lem}\label{lem: Grenzwerte auf Teilmengen}
 Es seien $A, B \subseteq \R$ und $f \colon B \to \R$. Ist $x$ ein gemeinsamer Häufungspunkt von $A$ und $B$, sodass $\lim_{b \to x, b \in B} f(b)$ existiert, so existiert auch der Grenzwert $\lim_{a \to x, a \in A} f(a)$, und es gilt
 \[
  \lim_{\substack{a \to x \\ a \in A}} f(a)
  = \lim_{\substack{b \to x \\ b \in B}} f(b).
 \]
\end{lem}


\begin{kor}
 Es sei $f \colon X \to \R$ mit Definitionsbereich $X \subseteq \R$. Es seien $A, B \subseteq X$ Teilmengen, so dass $x \in \R$ ein gemeinsamer Häufungspunkt von $A$, $B$ ist, und die beiden Grenzwerte $\lim_{a \to x, a \in A} f(a)$ und $\lim_{b \to x, b \in B} f(b)$ existieren. Ist $x$ auch ein Häufungspunkt von $A \cap B$, so ist
 \[
  \lim_{\substack{a \to x \\ a \in A}} f(a)
  = \lim_{\substack{b \to x \\ b \in B}} f(b).
 \]
\end{kor}





\section{Links- rechts- und beidseitige Grenzwerte}
Wir wollen uns nun einem Sonderfall von Funktionsgrenzwerten zuwenden.


\begin{defi}
 Es sei $A \subseteq \R$, $f \colon A \to \R$ und $x \in \R$.
 
 Gibt es ein $r > 0$, so dass $(x-r, x) \subseteq A$, so schreiben wir
 \[
  \lim_{a \uparrow x} f(a)
  \quad
  \text{für}
  \quad
  \lim_{\substack{a \to x \\ a \in (x-r,x)}} f(a),
 \]
 und nennen dies den \emph{linksseitigen Grenzwert von $f$ an $x$}.
 
 Gibt es ein $r > 0$, so dass $(x,x+r) \subseteq A$, so schreiben wir
 \[
  \lim_{a \downarrow x} f(a)
  \quad
  \text{für}
  \quad
  \lim_{\substack{a \to x \\ a \in (x,x+r)}} f(a),
 \]
 und nennen dies den \emph{rechtsseitigen Grenzwert von $f$ an $x$}.
 
 Existiert ein $r > 0$, so dass $(x-r,x) \cup (x,x+r) \subseteq A$, so schreiben wir
 \[
  \lim_{a \to x} f(a)
  \quad
  \text{für}
  \quad
  \lim_{\substack{a \to x \\ a \in (x-r,x) \cup (x,x+r)}} f(a),
 \]
 und nennen dies den \emph{beidseitigen Grenzwert von $f$ an $x$}.
\end{defi}


Die Wohldefiniertheit der jeweiligen Ausdrücke, also die Unabhängigkeit von $r$, ergibt sich aus Lemma \ref{lem: Grenzwerte auf Teilmengen}.


Der beidseitige Grenzwert lässt sich auch als Kombination des links- und rechtsseitigen Grenzwertes definieren:


\begin{lem}
 Es sei $A \subseteq \R$ und $f \colon A \to \R$. Für $x \in \R$ und $y \in \R$ sind äquivalent:
 \begin{enumerate}
  \item
   Der beidseitige Grenzwert $\lim_{a \to x} f(a)$ existiert und $\lim_{a \to x} f(a) = y$.
  \item
   Die Grenzwerte $\lim_{a \uparrow x} f(a)$ und $\lim_{a \downarrow x} f(a)$ existieren und es ist
   \[
    \lim_{a \uparrow x} f(a) = y = \lim_{a \downarrow x} f(a).
   \]
 \end{enumerate}
\end{lem}





\section{Uneigentliche Grenzwerte}


\begin{defi}
 Es sei $A \subseteq \R$, $f \colon A \to \R$ und $x \in \R$ ein Häufungspunkt von $A$. Wir schreiben dass $\lim_{a \to x, a \in A} f(x) = \infty$, falls es für alle $R > 0$ ein $\delta > 0$ gibt, so dass
 \[
  |f(a)| \geq R \quad \text{für alle $a \in A$ mit $|x-a| < \delta$ und $a \neq x$}.
 \]
 Wir schreiben $\lim_{a \to x, a \in A} f(x) = -\infty$, falls es für alle $R > 0$ eine $\delta > 0$ gibt, so dass
 \[
  |f(a)| \leq -R \quad \text{für alle $a \in A$ mit $|x-a| < \delta$ und $a \neq x$}.
 \]
\end{defi}




\begin{defi}
 Es sei $f \colon X \to \R$ eine Abbildung mit Definitionsbereits $X \subseteq \R$. Für $y \in \R$ sagen wir, dass $\lim_{x \to \infty} f(x) = y$, falls
 \begin{enumerate}
  \item
   es gibt $r_0 \in \R$, so dass $f(x)$ für alle $x \geq r_0$ definiert ist, und
  \item
   für alle $\varepsilon > 0$ gibt es $r \geq r_0$, so dass $|f(x) - y| < \varepsilon$ für alle $x \geq r$.
 \end{enumerate}
 Wir sagen, dass $\lim_{x \to -\infty} f(x) = y$, falls
 \begin{enumerate}
  \item
   es gibt $r_0 \in \R$, so dass $f(x)$ für alle $x \leq r_0$ definiert ist, und
  \item
   für alle $\varepsilon > 0$ gibt es $r \leq r_0$, so dass $|f(x) - y| < \varepsilon$ für alle $x \leq r$.
 \end{enumerate}
\end{defi}









\newpage


\section{Lösungen der Übungen}


\printsolutions



\end{document}
