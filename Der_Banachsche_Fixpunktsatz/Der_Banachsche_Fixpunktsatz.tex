\documentclass[a4paper,10pt]{article}
%\documentclass[a4paper,10pt]{scrartcl}

\usepackage{../mystyle}

\setromanfont[Mapping=tex-text]{Linux Libertine O}
% \setsansfont[Mapping=tex-text]{DejaVu Sans}
% \setmonofont[Mapping=tex-text]{DejaVu Sans Mono}

\title{Der Banachsche Fixpunktsatz}
\author{Jendrik Stelzner}
\date{\today}

\begin{document}
\maketitle


Wir wollen hier den Banachschen Fixpunktsatz für $\R^n$ formulieren und beweisen.


\begin{defi}
 Eine Abbildung $f \colon \R^n \to \R^n$ heißt \emph{Kontraktion}, falls es eine Konstante $0 < L < 1$ gibt, so dass
 \[
  \|f(x)-f(y)\| \leq L\|x-y\| \quad \text{für alle $x,y \in \R^n$}.
 \]
\end{defi}


\begin{question}
 Bestimmen Sie, welche der folgenden Abbildungen eine Kontraktion ist:
 \begin{enumerate}
  \item
   $f_1 \colon \R \to \R, x \mapsto \frac{x}{4} - \frac{2}{3}$
  \item
   $f_2 \colon \R \to \R, x \mapsto x^2$
  \item
   $f_3 \colon \R \to \R: x \mapsto |x|^{1/2}$
  \item
   $f_4 \colon \R^2 \to \R^2, \vect{x\\y} \mapsto \vect{1/2 & -1/3 \\ 1/3 & 1/2} \vect{x\\y}$
 \end{enumerate}
\end{question}
\begin{solution}
 \begin{enumerate}
  \item
   $f_1$ ist eine Kontraktion, da für alle $x,y \in \R$
   \[
    |f_1(x)-f_1(y) = \left| \frac{x}{4} - \frac{y}{4} \right| = \frac{1}{4}|x-y|
   \]
   mit $1/4 < 1$.
  \item
   $f_2$ ist keine Kontraktion, denn
   \[
    |f_2(2)-f_2(1)| = |2^2 - 1^2| = 3 > 1 = |2-1|.
   \]
  \item
   $f_3$ ist keine Kontrakiton. Ansonsten gebe es eine Konstante $0 < L < 1$ mit $|f_3(x)-f_3(y)| \leq L|x-y|$ für alle $x, y \in \R$. Insbesondere wäre dann für alle $n \geq 1$
   \[
    \frac{1}{n^{1/2}}
    = \left|f_3\left(\frac{1}{n}\right) - f_3(0)\right|
    \leq L\left| \frac{1}{n} - 0\right|
    = \frac{L}{n}
   \]
   und somit $n^{1/2} \leq L$, also $n \leq L^2$. Dies gilt offenbar nicht.
  \item
   Wir sehen $\R^2 \cong \mathbb{C}$. Die Abbildung $f_4$ entspricht dann der Multiplikation mit der komplexen Zahl
   \[
    \xi = \frac{1}{2} + \frac{1}{3} i,
   \]
   d.h.\ $f_4(z) = \xi z$ für alle $z \in \mathbb{C}$. Für alle $z \in \mathbb{C}$ ist daher insbesondere
   \[
    |f_4(z)| = |\xi z| = |\xi| |z|.
   \]
   Da $|\xi| = \sqrt{1/4 + 1/9} < 1$ ergibt sich, dass $f_4$ ein Kontraktion ist (mit Konstante $|\xi|$).
 \end{enumerate}
\end{solution}


\begin{question}
 Für eine Abbildung $f \colon \R^n \to \R^m$ gebe es eine Konstante $L > 0$, so dass
 \[
  \|f(x)-f(y)\| \leq L\|x-y\| \quad \text{für alle $x,y \in \R^n$}.
 \]
 Zeigen Sie, dass $f$ stetig ist. (Man bezeichnet eine solche Abbildung als Lipschitz-stetig (mit Konstante $L$).) Folgern Sie, dass Kontraktionen stetig sind.
\end{question}
\begin{solution}
 $f$ erfüllt an jeder Stelle $x \in \R^n$ das $\varepsilon$-$\delta$-Kriterium, denn für beliebiges $\varepsilon > 0$ ergibt sich für $\delta \coloneqq \varepsilon/L$, dass für alle $y \in \R^n$ mit $\|x-y\| < \delta$
 \[
  \|f(x)-f(y)\| \leq L\|x-y\| < L\delta = L \frac{\varepsilon}{L} = \varepsilon.
 \]
 Also ist $f$ stetig. Kontraktionen sind per Definition Lipschitz-stetig mit Konstante $L < 1$ und somit stetig.
\end{solution}


\begin{thrm}[Banachscher Fixpunktsatz]
 Es sei $T \colon \R^m \to \R^m$ eine Kontraktion. Dann besitzt $T$ einen eindeutigen Fixpunkt, d.h.\ es existiert genau ein $\xi \in \R^m$ mit $T(\xi) = \xi$. Außerdem gilt für jedes $x \in \R^m$, dass $\lim_{n \to \infty} T^n(x) = \xi$. (Hier bezeichnet $T^n$ die $n$-fache Komposition von $T$ mit sich selbst.)
\end{thrm}


Der Beweis des Satzes lässt sich in kleinere Zwischenschritte aufteilen:


\begin{question}
 Es sei $T \colon \R^m \to \R^m$ eine Kontraktion.
 \begin{enumerate}
  \item
   Zeigen Sie die Eindeutigkeit des Fixpunktes von $T$. (Wir setzen hier noch keine Existenz voraus.)
  \item
   Zeigen Sie, dass für jeden Startwert $x \in \R^m$ die Folge $(x_n)_{n \in \N}$ mit $x_n \coloneqq T^n(x)$ konvergiert. (\emph{Hinweis}: Zeigen Sie, dass es sich um eine Cauchy-Folge handelt.)
  \item
   Zeigen Sie, dass für jedes $x \in \R^m$ der Grenzwert $\xi \coloneqq \lim_{n \to \infty} T^n(x)$ ein Fixpunkt von $T$ ist.
 \end{enumerate}
\end{question}
\begin{solution}
 Es sei $0 < L < 1$, so dass $\|T(x)-T(y)\| \leq L\|x-y\|$ für alle $x,y \in \R^m$.
 \begin{enumerate}
  \item
   Es seien $\xi, \zeta \in \R^m$ zwei Fixpunkte von $T$. Dann ist
   \[
    \|\xi-\zeta\| = \|T(\xi)-T(\zeta)\| \leq L\|\xi-\zeta\|.
   \]
   Da $0 < L < 1$ folgt, dass $\|\xi-\zeta\| = 0$ und somit $\xi = \zeta$.
  \item
   Für alle $n \geq 1$ ist
   \[
    \|x_{n+1} - x_n\|
    = \|T(x_n) - T(x_{n-1})\|
    \leq L\|x_n - x_{n-1}\|.
   \]
   Für $M \coloneqq \|x_1 - x_0\|$ ergibt sich damit induktiv, dass
   \[
    \|x_{n+1} - x_n\| \leq L^n \|x_1 - x_0\| = M L^n
    \quad \text{für alle $n \in \N$}.
   \]
   Für alle $N' \in \N$ und $m, m' \geq N$, wobei o.B.d.A. $m \geq m'$, ist daher
   \begin{align*}
    \|x_m - x_{m'}\|
    &\leq \|x_m - x_{m-1}\| + \|x_{m-1} - x_{m-2}\| + \dotsb + \|x_{m'+1} - x_{m'}\| \\
    &\leq M L^{m-1} + \dotsb + M L^{m'}
    = M L^{m'} \sum_{k=0}^{m-1-m'} L^k \\
    &\leq M L^{N'} \sum_{k=0}^\infty L^k
    = \frac{M}{1-L} L^{N'}.
   \end{align*}
   Da $\lim_{N' \to \infty} ML^{N'}/(1-L) = 0$ gibt es für alle $\varepsilon > 0$ ein $N \in \N$, so dass $ML^N/(1-L)< \varepsilon$, und damit $\|x_m - x_{m'}\| < \varepsilon$ für alle $m, m' \geq N$. Dies zeigt, dass $(x_n)$ eine Cauchy-Folge ist.
  \item
   $T$ ist eine Kontraktion, und damit insbesondere stetig. Für beliebiges $x \in \R^m$ erhalten wir unter Verwendung der Folgenstetigkeit für $\xi \coloneqq \lim_{n \to \infty} T^n(x)$, dass
   \[
    T(\xi)
    = T\left(\lim_{n \to \infty} T^n(x)\right)
    = \lim_{n \to \infty} T(T^n(x))
    = \lim_{n \to \infty} T^{n+1}(x)
    = \xi.
   \]
 \end{enumerate}
\end{solution}


\begin{bem}
 Der Banachsche Fixpunktsatz gilt allgemeiner für alle vollständige metrische Räume. Der Beweis hierfür läuft analog.
\end{bem}


\newpage


\printsolutions




\end{document}
