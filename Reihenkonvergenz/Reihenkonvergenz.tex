\documentclass[a4paper,10pt]{article}
%\documentclass[a4paper,10pt]{scrartcl}

\usepackage{../mystyle}

\setromanfont[Mapping=tex-text]{Linux Libertine O}
% \setsansfont[Mapping=tex-text]{DejaVu Sans}
% \setmonofont[Mapping=tex-text]{DejaVu Sans Mono}

\title{Grundlegendes zur Konvergenz von Reihen}
\author{Jendrik Stelzner}
\date{\today}

\begin{document}
\maketitle

\tableofcontents



\section{Vorbereitung}


Wir werden einige grundlegende Eigenschaften über die Konvergenz von Folgen nutzen, die bisher nicht gezeigt wurden. Wir überlassen die entsprechenden Beweise den geneigten Lesern als Übung.


\begin{ExerciseList}
 \Exercise
  Konvergiert die Folge $(a_n)_{n \in \N}$, so konvergiert auch die Folge der Beträge $(|a_n|)_{n \in \N}$ und es gilt
  \[
   \lim_{n \to \infty} |a_n| = \left| \lim_{n \to \infty} a_n \right|.
  \]
 \Answer
  Sei $a \coloneqq \lim_{n \to \infty} a_n$ und $\varepsilon > 0$ beliebig aber fest. Dann gibt es ein $N \in \N$ mit $|a - a_n| < \varepsilon$ für alle $n \geq N$. Für alle $n \geq N$ gilt nach der umgekehrten Dreiecksungleichung
  \[
   ||a| - |a_n|| \leq |a - a_n| < \varepsilon.
  \]
  Wegen der Beliebigkeit von $\varepsilon > 0$ zeigt dies, dass $|a_n| \to |a|$ für $n \to \infty$.
  
 \Exercise
  Für eine Folge $(a_n)_{n \in \N}$ und $C \in \R$ gilt: Es gibt genau dann $y < C$ und $N \in \N$ mit $a_n \leq y$ für alle $n \geq N$, falls $\limsup_{n \to \infty} a_n < C$.
 \Answer
  Gibt es solche $y$ und $N$, so ist $\sup_{k \geq N} a_k \leq y$ und somit
  \[
   \limsup_{n \to \infty} a_n
   = \inf_{n \in \N} \sup_{k \geq n} a_k
   \leq \sup_{k \geq N} a_k
   \leq y
   < 1.
  \]
  
  Sei andererseits $x \coloneqq \limsup_{n \to \infty} a_n < 1$. Da $\limsup_{n \to \infty} a_n = \inf_{n \in \N} \sup_{k \geq N} a_k$ gibt es wegen der $\varepsilon$-Charakterisierung des Infimums für alle $\varepsilon > 0$ ein $N' \in \N$ mit $\sup_{k \geq N'} a_k < x + \varepsilon$. 
\end{ExerciseList}


\section{Definition}


\begin{defi}
 Für eine Folge $(a_n)_{n \in \N}$ ist die Folge der \emph{Partialsummen} $(s_n)_{n \in \N}$ als
 \[
  s_n \coloneqq \sum_{k=0}^n a_k
 \]
 definiert; $s_n$ heißt die \emph{$n$-te Partialsumme} (der Folge $(a_n)$). Diese Folge der Partialsummen bezeichnet man als \emph{Reihe} und schreibt man als $\sum_{k=0}^\infty a_k$. Konvergiert die Reihe $\sum_{k=0}^\infty a_k$, d.h. konvergiert die Folge der Partialsummen $(s_n)$, so bezeichnet man den Grenzwert $\lim_{n \to \infty} s_n = \lim_{n \to \infty} \sum_{k=0}^n a_k$ ebenfalls als $\sum_{k=0}^\infty a_k$ und nennt dies den \emph{Wert der Reihe}. Die Folge $(a_n)$ heißt dann summierbar.
 
 Für $N \in \N$ ist die Reihe $\sum_{k=N}^\infty a_k$ als die Reihe $\sum_{k=0}^\infty a_{k+N}$ definiert.  
\end{defi}


\begin{bem}
 Die Reihe $\sum_{k=N}^\infty a_k$ wird nach dieser Definition als die Folge der Partialsummen $(s_n)_{n \in \N}$ mit $s_n = \sum_{k=0}^n a_{k+N}$ verstanden. Alternativ kann man die Reihe auch als die Folge der Partialsummen $(s'_n)_{n \in \N}$ mit
 \[
  s'_n \coloneqq \sum_{k=N}^\infty a_k
 \]
 definieren. Dies macht praktisch keinen Unterschied, da dann
 \[
  s'_n =
  \begin{cases}
   0       & \text{falls } n < N, \\
   s_{n-N} & \text{falls } n \geq N.
  \end{cases}
 \]
 Die Folge $(s'_n)$ ist also die Folge $(s_n)$ mit Nullen aufgefüllt.
\end{bem}



Man bemerke, dass man mit der Notation $\sum_{k=0}^\infty a_k$ sowohl die Folge der Partialsummen $(\sum_{k=0}^n a_k)_{n \in \N}$ als auch der Grenzwert dieser Folge bezeichnet. Soll also gezeigt werden, dass die Reihe $\sum_{k=0}^\infty a_k$ konvergiert, so ist damit gemeint, dass die Folge der Partialsummen $(\sum_{k=0}^n a_k)_{n \in \N}$ konvergieren soll. Soll der Wert der Reihe $\sum_{k=0}^\infty a_k$ bestimmt werden, so soll der Grenzwert $\lim_{n \to \infty} \sum_{k=0}^n a_k$ ermittelt werden.


\begin{defi}
 Eine Reihe $\sum_{k=0}^\infty a_k$ heißt \emph{absolut konvergent}, wenn die Reihe der Beträge $\sum_{k=0}^\infty |a_k|$ konvergiert. Die Folge $(a_n)$ heißt dann absolut summierbar.
\end{defi}


\begin{bem}
 Ist $\sum_{k=0}^\infty a_k$ eine Reihe und $(s_n)_{n \in \N}$ die Folge der Partialsummen, also $s_n = \sum_{k=0}^n a_k$, so schreibt man für $\lim_{n \to \infty} s_n = \infty$ und $\lim_{n \to \infty} s_n = -\infty$ ebenfalls $\sum_{k=0}^\infty a_k = \infty$, bzw. $\sum_{k=0}^\infty a_k = -\infty$. Die Reihe $\sum_{k=0}^\infty a_k$ bezeichnet man aber in diesen Fällen nicht als konvergent.
\end{bem}





\section{Grundlegende Eigenschaften}

\begin{lem}\label{lem: Reihe bedeutet Nullfolge}
 Konvergiert für eine Folge $(a_n)_{n \in \N}$ die Reihe $\sum_{k=0}^n a_k$, so ist $(a_n)$ eine Nullfolge, d.h. $\lim_{n \to \infty} a_n = 0$.
\end{lem}
\begin{proof}
 Wir betrachten die Folge der Partialsummen $(s_n)_{n \in \N}$, also $s_n \coloneqq \sum_{k=0}^n a_k$. Dass die Reihe $\sum_{k=0}^\infty a_k$ konvergiert bedeutet gerade, dass die Folge $(s_n)$ konvergiert. Es sei
 \[
  s \coloneqq \lim_{n \to \infty} s_n.
 \]
 Wir bemerken nun, dass für alle $n \geq 1$
 \[
  s_n - s_{n-1}
  = \left( \sum_{k=0}^n a_k \right) - \left( \sum_{k=0}^{n-1} a_k \right)
  = a_n.
 \]
 Durch die üblichen Rechenregeln konvergenter Folgen ergibt sich daher, dass
 \[
  a_n = s_n - s_{n-1} \to s-s = 0
 \]
 für $n \to \infty$. Also ist $(a_n)$ konvergent und $\lim_{n \to \infty} a_n = 0$.
\end{proof}


\begin{prop}
 \begin{enumerate}
  \item
   Konvergieren die Reihen $\sum_{k=0}^\infty a_k$ und $\sum_{k=0}^\infty b_k$, so konvergiert auch die Reihe $\sum_{k=0}^\infty (a_k + b_k)$ und es gilt
   \[
    \sum_{k=0}^\infty (a_k + b_k) = \left( \sum_{k=0}^\infty a_k \right) + \left( \sum_{k=0}^\infty b_k \right).
   \]
  \item
   Konvergiert die Reihe $\sum_{k=0}^\infty a_k$ so konvergiert für jedes $\lambda \in \R$ auch die Reihe $\sum_{k=0}^\infty (\lambda a_k)$ und es gilt
   \[
    \sum_{k=0}^\infty (\lambda a_k) = \lambda \sum_{k=0}^\infty a_k.
   \]
  \item
   Für eine Reihe $\sum_{k=0}^\infty a_k$ gilt für jedes $N \in \N$: Die Reihe $\sum_{k=0}^\infty a_k$ konvergiert genau dann, wenn die Reihe $\sum_{k=N}^\infty a_k$ konvergiert. Es ist dann
   \[
    \sum_{k=0}^\infty a_k = \sum_{k=0}^{N-1} a_k + \sum_{k=N}^\infty a_k.
   \]
 \end{enumerate}
\end{prop}
\begin{proof}
 Die Aussagen ergeben sich direkt aus den bekannten Rechenregeln für endliche Summen und konvergente Folgen. Ein genaues Formulieren bleibt den Lesern als Übung überlassen.
\end{proof}


\begin{kor}
 Die Menge der summierbaren Folgen
 \[
  \Sigma \coloneqq \{ (a_n)_{n \in \N} \mid \text{$(a_n)$ ist summierbar}\}
 \]
 bildet unter punktweiser Addition und Skalarmultiplikation einen $\R$-Vektorraum, und die Abbildung
 \[
  \Sigma \to \R, (a_n)_{n \in \N} \mapsto \sum_{k=0}^\infty a_k
 \]
 ist $\R$-linear.
\end{kor}


Wir erhalten aus dem Lemma auch, dass $\lim_{n \to \infty} \sum_{k=n}^\infty a_k = 0$, denn wir haben
\[
 \sum_{k=n}^\infty a_k
 = \sum_{k=0}^\infty a_k - \sum_{k=0}^{n-1} a_k
 \xrightarrow{n \to \infty} \sum_{k=0}^\infty a_k - \sum_{k=0}^\infty a_k
 = 0.
\]


\begin{lem}
 Ist die Reihe $\sum_{k=0}^\infty a_k$ konvergent, so ist
 \[
  \left| \sum_{k=0}^\infty a_k \right| \leq \sum_{k=0}^\infty |a_k|.
 \]
\end{lem}


\begin{proof}[Beweis des Lemmas]
 Ist die Reihe $\sum_{k=0}^\infty a_k$ nicht absolut konvergent, so haben wir $\sum_{k=0}^\infty |a_k| = \infty$ und die Aussage ist klar. Ansonsten gilt für alle $n \in \N$ nach der Dreiecksungleichung für endliche Summen
 \[
  \left| \sum_{k=0}^n a_k \right| \leq \sum_{k=0}^n |a_k|,
 \]
 so dass wir im Grenzwert
 \[
  \left| \sum_{k=0}^\infty a_k \right|
  = \left| \lim_{n \to \infty} \sum_{k=0}^n a_k \right|
  = \lim_{n \to \infty} \left| \sum_{k=0}^n a_k \right|
  \leq \lim_{n \to \infty} \sum_{k=0}^n |a_k|
  = \sum_{k=0}^\infty |a_k|
 \]
 haben.
\end{proof}


Wir wollen nun auf den Zusammenhang zwischen konvergenten und absolut konvergenten Reihen zurückkommen.


\begin{prop}
 Ist eine Reihe $\sum_{k=0}^\infty a_k$ absolut konvergent, so ist sie auch konvergent.
\end{prop}
\begin{proof}
 Es sei $(s_n)_{n \in \N}$ die Folge der Partialsummen, also $s_n = \sum_{k=0}^n a_k$. Wir wollen zeigen, dass $(s_n)$ eine Cauchy-Folge ist. Sei hierfür $\varepsilon > 0$ beliebig aber fest. Für alle $n \in \N$ haben wir für alle $m, m' \geq n$
 \[
  |s_m - s_{m'}|
  = \left| \sum_{k=\min\{m,m'\}}^{\max\{m,m'\}} a_k \right|
  \leq \sum_{k=\min\{m,m'\}}^{\max\{m,m'\}} |a_k|
  \leq \sum_{k=n}^\infty |a_k|.
 \]
 Da $\lim_{n \to \infty} \sum_{k=n}^\infty |a_k| = 0$ gibt es ein $N \in \N$ mit $\sum_{k=N}^\infty |a_k| < \varepsilon$. Aus der obigen Ungleichung ergibt sich damit, dass $m, m' \geq N$ ist dann $|s_m - s_{m'}| < \varepsilon$.
\end{proof}


\begin{lem}
 Es seien $\sum_{k=0}^\infty a_k$ und $\sum_{k=0}^\infty b_k$ Reihen mit $0 \leq a_n \leq b_n$ für alle $n \in \N$. Dann ist $\sum_{k=0}^\infty a_k \leq \sum_{k=0}^\infty b_k$.
\end{lem}
\begin{proof}
 Für die Partialsummen gilt für jedes $n \in \N$
 \[
  \sum_{k=0}^n a_k \leq \sum_{k=0}^n b_k.
 \]
 Die Aussage ergibt sich damit direkt daraus, dass Monotonie von Folgen unter Grenzwerten erhalten bleibt.
\end{proof}





\section{Konvergenzkriterien}


\subsection{Majoranten- und Minorantenkriterium}

\begin{lem}[Majoranten- und Minorantenkriterium]
 Es sei $\sum_{k=0}^\infty a_k$ und eine Reihe reeller Zahlen und $(b_n)_{n \in \N}$ eine Folge mit $b_n \geq 0$ für alle $n \in \N$.
 \begin{enumerate}
  \item
   Ist $|a_n| \leq b_n$ für alle $n \in \N$ und $\sum_{k=0}^\infty b_k < \infty$, so ist $\sum_{k=0}^\infty a_k$ absolut konvergent. (Dies ist das \emph{Majorantenkriterium}.)
  \item
   Ist $b_n \leq |a_n|$ für alle $n \in \N$ und $\sum_{k=0}^\infty a_k = \infty$, so ist $\sum_{k=0}^\infty a_k$ nicht absolut konvergent. (Dies ist das \emph{Minorantenkriterium}.)
 \end{enumerate}
\end{lem}
\begin{proof}
 \begin{enumerate}
  \item
   Für alle $n \in \N$ ist
   \[
    \sum_{k=0}^n |a_k| \leq \sum_{k=0}^n b_k \leq \sum_{k=0}^\infty b_k,
   \]
   also die Folge von Partialsummen $(\sum_{k=0}^n |a_k|)_{n \in \N}$ monoton steigend und nach oben beschränkt, und somit konvergent.
  \item
   Wäre $\sum_{k=0}^\infty a_k$ absolut konvergent, so würde die Reihe der Beträge $\sum_{k=0}^\infty |a_k|$ konvergieren, und nach dem Majorantenkriterium würde auch $\sum_{k=0}^\infty b_k$ konvergieren, im Widerspruch zu Annahme.
 \end{enumerate}
\end{proof}


\subsection{Quotientenkriterium}


\begin{prop}[Quotientenkriterium]
 Es sei $(a_n)_{n \in \N}$ eine Folge mit $a_n \neq 0$ für alle $n \in \N$ und es gebe $0 \leq y < 1$ und $N \in \N$ mit $|a_{n+1}/a_n| < y$ für alle $n \geq N$. Dann ist die Reihe $\sum_{k=0}^\infty a_k$ absolut konvergent.
\end{prop}
\begin{proof}
 Wir haben
 \begin{align*}
  \sum_{k=N}^\infty |a_k|
  &= \sum_{k=N}^\infty |a_N| \prod_{j=N}^{k-1} \frac{|a_{j+1}|}{|a_j|}
  = |a_N| \sum_{k=N}^\infty \prod_{j=N}^{k-1} \frac{|a_{j+1}|}{|a_j|} \\
  &\leq |a_N| \sum_{k=N}^\infty y^{k-N}
  = |a_N| \sum_{k=0}^\infty y^k
  = \frac{|a_N|}{1-y},
 \end{align*}
 und somit
 \[
  \sum_{k=0}^\infty |a_k|
  = \sum_{k=0}^{N-1} |a_k| + \sum_{k=N}^\infty |a_k|
  \leq \sum_{k=0}^{N-1} |a_k| + \frac{|a_N|}{1-y}
  < \infty.
  \qedhere
 \]
\end{proof}



\subsection{Wurzelkriterium}



\subsection{Cauchysches Verdichtungskriterium}


\subsection{Leibniz-Kriterium}




\section{Beispiele}



\subsection{Endliche Reihen}


\subsection{Die geometrische Reihe}


\subsection{Die allgemeine harmonische Reihe}


\subsection{Die (alternierende) harmonische Reihe}


\subsection{Weitere Beispiele}





\section{Potenzreihen}


\subsection{Definition}


\subsection{Konvergenzradius}


\subsection{Beispiele}




\section{Lösungen der Aufgaben}

\shipoutAnswer

























\end{document}
