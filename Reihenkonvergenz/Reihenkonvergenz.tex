\documentclass[a4paper,10pt]{article}
%\documentclass[a4paper,10pt]{scrartcl}

\usepackage{../mystyle}

\setromanfont[Mapping=tex-text]{Linux Libertine O}
% \setsansfont[Mapping=tex-text]{DejaVu Sans}
% \setmonofont[Mapping=tex-text]{DejaVu Sans Mono}

\title{Grundlegendes zur Konvergenz von Reihen}
\author{Jendrik Stelzner}
\date{\today}

\begin{document}
\maketitle

\tableofcontents





\section{Definition}


\begin{defi}
 Für eine Folge $(a_n)_{n \in \N}$ ist die Folge der \emph{Partialsummen} $(s_n)_{n \in \N}$ als
 \[
  s_n \coloneqq \sum_{k=0}^n a_k
 \]
 definiert; $s_n$ heißt die \emph{$n$-te Partialsumme} (der Folge $(a_n)$). Diese Folge der Partialsummen bezeichnet man als \emph{Reihe} und schreibt man als $\sum_{k=0}^\infty a_k$. Konvergiert die Reihe $\sum_{k=0}^\infty a_k$, d.h. konvergiert die Folge der Partialsummen $(s_n)$, so bezeichnet man den Grenzwert $\lim_{n \to \infty} s_n = \lim_{n \to \infty} \sum_{k=0}^n a_k$ ebenfalls als $\sum_{k=0}^\infty a_k$ und nennt dies den \emph{Wert der Reihe}.
 
 Für $N \in \N$ ist die Reihe $\sum_{k=N}^\infty a_k$ als die Reihe $\sum_{k=0}^\infty a_{k+N}$ definiert.  
\end{defi}


\begin{bem}
 Die Reihe $\sum_{k=N}^\infty a_k$ wird nach dieser Definition als die Folge der Partialsummen $(s_n)_{n \in \N}$ mit $s_n = \sum_{k=0}^n a_{k+N}$ verstanden. Alternativ kann man die Reihe auch als die Folge der Partialsummen $(s'_n)_{n \in \N}$ mit
 \[
  s'_n \coloneqq \sum_{k=N}^\infty a_k
 \]
 definieren. Dies macht praktisch keinen Unterschied, da dann
 \[
  s'_n =
  \begin{cases}
   0       & \text{falls } n < N, \\
   s_{n-N} & \text{falls } n \geq N.
  \end{cases}
 \]
 Die Folge $(s'_n)$ ist also die Folge $(s_n)$ mit Nullen aufgefüllt.
\end{bem}


Man bemerke, dass man mit der Notation $\sum_{k=0}^\infty a_k$ sowohl die Folge der Partialsummen $(\sum_{k=0}^n a_k)_{n \in \N}$ als auch der Grenzwert dieser Folge bezeichnet. Soll also gezeigt werden, dass die Reihe $\sum_{k=0}^\infty a_k$ konvergiert, so ist damit gemeint, dass die Folge der Partialsummen $(\sum_{k=0}^n a_k)_{n \in \N}$ konvergieren soll. Soll der Wert der Reihe $\sum_{k=0}^\infty a_k$ bestimmt werden, so soll der Grenzwert $\lim_{n \to \infty} \sum_{k=0}^n a_k$ ermittelt werden.


\begin{defi}
 Eine Reihe $\sum_{k=0}^\infty a_k$ heißt \emph{absolut konvergent}, wenn die Reihe der Beträge $\sum_{k=0}^\infty |a_k|$ konvergiert.
\end{defi}


\begin{bem}
 Ist $\sum_{k=0}^\infty a_k$ eine Reihe und $(s_n)_{n \in \N}$ die Folge der Partialsummen, also $s_n = \sum_{k=0}^n a_k$, so schreibt man für $\lim_{n \to \infty} s_n = \infty$ und $\lim_{n \to \infty} s_n = -\infty$ ebenfalls $\sum_{k=0}^\infty a_k = \infty$, bzw. $\sum_{k=0}^\infty a_k = -\infty$. Die Reihe $\sum_{k=0}^\infty a_k$ bezeichnet man aber in diesen Fällen nicht als konvergent.
\end{bem}





\section{Grundlegende Eigenschaften}

\begin{lem}
 Konvergiert für eine Folge $(a_n)_{n \in \N}$ die Reihe $\sum_{k=0}^n a_k$, so ist $(a_n)$ eine Nullfolge, d.h. $\lim_{n \to \infty} a_n = 0$.
\end{lem}
\begin{proof}
 Wir betrachten die Folge der Partialsummen $(s_n)_{n \in \N}$, also $s_n \coloneqq \sum_{k=0}^n a_k$. Dass die Reihe $\sum_{k=0}^\infty a_k$ konvergiert bedeutet gerade, dass die Folge $(s_n)$ konvergiert. Es sei
 \[
  s \coloneqq \lim_{n \to \infty} s_n.
 \]
 Wir bemerken nun, dass für alle $n \geq 1$
 \[
  s_n - s_{n-1}
  = \left( \sum_{k=0}^n a_k \right) - \left( \sum_{k=0}^{n-1} a_k \right)
  = a_n.
 \]
 Durch die üblichen Rechenregeln konvergenter Folgen ergibt sich daher, dass
 \[
  a_n = s_n - s_{n-1} \to s-s = 0
 \]
 für $n \to \infty$. Also ist $(a_n)$ konvergent und $\lim_{n \to \infty} a_n = 0$.
\end{proof}


\begin{prop}
 \begin{enumerate}
  \item
   Konvergieren die Reihen $\sum_{k=0}^\infty a_k$ und $\sum_{k=0}^\infty b_k$, so konvergiert auch die Reihe $\sum_{k=0}^\infty (a_k + b_k)$ und es gilt
   \[
    \sum_{k=0}^\infty (a_k + b_k) = \left( \sum_{k=0}^\infty a_k \right) + \left( \sum_{k=0}^\infty b_k \right).
   \]
  \item
   Konvergiert die Reihe $\sum_{k=0}^\infty a_k$ so konvergiert für jedes $\lambda \in \R$ auch die Reihe $\sum_{k=0}^\infty (\lambda a_k)$ und es gilt
   \[
    \sum_{k=0}^\infty (\lambda a_k) = \lambda \sum_{k=0}^\infty a_k.
   \]
  \item
   Für eine Reihe $\sum_{k=0}^\infty a_k$ gilt für jedes $N \in \N$: Die Reihe $\sum_{k=0}^\infty a_k$ konvergiert genau dann, wenn die Reihe $\sum_{k=N}^\infty a_k$ konvergiert. Es ist dann
   \[
    \sum_{k=0}^\infty a_k = \sum_{k=0}^{N-1} a_k + \sum_{k=N}^\infty a_k.
   \]
 \end{enumerate}
\end{prop}
\begin{proof}
 Die Aussagen ergeben sich direkt aus den bekannten Rechenregeln für endliche Summen und konvergente Folgen. Ein genaues Formulieren bleibt den Lesern als Übung überlassen.
\end{proof}


\begin{lem}
 Ist die Reihe $\sum_{k=0}^\infty a_k$ konvergent, so ist
 \[
  \left| \sum_{k=0}^\infty a_k \right| \leq \sum_{k=0}^\infty |a_k|.
 \]
\end{lem}

\begin{bem}
 Im Beweis des Lemmas werden wir die folgende kleine Aussage über konvergente Folgen nutzen, die man normalerweise im Laufe der Vorlesung behandelt: Ist eine Folge $(a_n)_{n \in \N}$ reeller Zahlen konvergent, so ist auch die Folge der Beträge $(|a_n|)_{n \in \N}$ konvergent und es gilt
 \[
  \lim_{n \to \infty} |a_n| = \left| \lim_{n \to \infty} a_n \right|.
 \]
 Dies folgt leicht mithilfe der umgekehrten Dreiecksungleichung: Sei $a \coloneqq \lim_{n \to \infty} a_n$ und $\varepsilon > 0$ beliebig aber fest. Dann gibt es ein $N \in \N$ mit $|a - a_n| < \varepsilon$ für alle $n \geq N$. Für alle $n \geq N$ ist dann auch
 \[
  ||a| - |a_n|| \leq |a - a_n| < \varepsilon.
 \]
 Wegen der Beliebigkeit von $\varepsilon > 0$ zeigt dies, dass $|a_n| \to |a|$ für $n \to \infty$.
\end{bem}

\begin{proof}[Beweis des Lemmas]
 Ist die Reihe $\sum_{k=0}^\infty a_k$ nicht absolut konvergent, so haben wir $\sum_{k=0}^\infty |a_k| = \infty$ und die Aussage ist klar. Ansonsten gilt für alle $n \in \N$ nach der Dreiecksungleichung für endliche Summen
 \[
  \left| \sum_{k=0}^n a_k \right| \leq \sum_{k=0}^n |a_k|,
 \]
 so dass wir im Grenzwert
 \[
  \left| \sum_{k=0}^\infty a_k \right|
  = \left| \lim_{n \to \infty} \sum_{k=0}^n a_k \right|
  = \lim_{n \to \infty} \left| \sum_{k=0}^n a_k \right|
  \leq \lim_{n \to \infty} \sum_{k=0}^n |a_k|
  = \sum_{k=0}^\infty |a_k|
 \]
 haben.
\end{proof}


\begin{prop}
 Ist eine Reihe $\sum_{k=0}^\infty a_k$ absolut konvergent, so ist sie auch konvergent.
\end{prop}
\begin{proof}
 Es sei $(s_n)_{n \in \N}$ die Folge der Partialsummen, also $s_n = \sum_{k=0}^n a_k$. Wir wollen zeigen, dass $(s_n)$ eine Cauchy-Folge ist. Sei hierfür $\varepsilon > 0$ beliebig aber fest. Für alle $m, m' \in \N$ haben wir
 \[
  |s_m - s_{m'}|
  = \left| \sum_{k=\min\{m,m'\}}^{\max\{m,m'\}} a_k \right|
  \leq \sum_{k=\min\{m,m'\}}^{\max\{m,m'\}} |a_k|
  \leq \sum_{k=\min\{m,m'\}}^\infty |a_k|.
 \]
 
 \begin{beh}
  Es gibt ein $N \in \N$ mit
  \[
   \sum_{k = N}^\infty |a_k| < \varepsilon.
  \]
 \end{beh}
 \begin{proof}[Beweis der Behauptung]
  Es sei $a \coloneqq \sum_{k=0}^\infty |a_k|$. Für alle $n \in \N$ haben wir
 \end{proof}
 
\end{proof}






















\end{document}
