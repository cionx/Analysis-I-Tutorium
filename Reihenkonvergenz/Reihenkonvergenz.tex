\documentclass[a4paper,10pt]{article}
%\documentclass[a4paper,10pt]{scrartcl}

\usepackage{../mystyle}

\setromanfont[Mapping=tex-text]{Linux Libertine O}
% \setsansfont[Mapping=tex-text]{DejaVu Sans}
% \setmonofont[Mapping=tex-text]{DejaVu Sans Mono}

\title{Grundlegendes zur Konvergenz von Reihen}
\author{Jendrik Stelzner}
\date{\today}

\begin{document}
\maketitle

\tableofcontents





\section{Vorbereitung}


Wir werden im Folgenden einige grundlegende Eigenschaften über die Konvergenz von Folgen nutzen, die bisher nicht gezeigt wurden.


\begin{question}
 Konvergiert die Folge $(a_n)_{n \in \N}$, so konvergiert auch die Folge der Beträge $(|a_n|)_{n \in \N}$ und es gilt
 \[
  \lim_{n \to \infty} |a_n| = \left| \lim_{n \to \infty} a_n \right|.
 \]
\end{question}
\begin{solution}
 Sei $a \coloneqq \lim_{n \to \infty} a_n$ und $\varepsilon > 0$ beliebig aber fest. Dann gibt es ein $N \in \N$ mit $|a - a_n| < \varepsilon$ für alle $n \geq N$. Für alle $n \geq N$ gilt nach der umgekehrten Dreiecksungleichung
 \[
  ||a| - |a_n|| \leq |a - a_n| < \varepsilon.
 \]
 Wegen der Beliebigkeit von $\varepsilon > 0$ zeigt dies, dass $|a_n| \to |a|$ für $n \to \infty$.
\end{solution}


\begin{question}
 Für eine Folge $(a_n)_{n \in \N}$ und $C \in \R$ gilt: Es gibt genau dann $y < C$ und $N \in \N$ mit $a_n \leq y$ für alle $n \geq N$, falls $\limsup_{n \to \infty} a_n < C$.
\end{question}
\begin{solution}
 Gibt es solche $y$ und $N$, so ist $\sup_{k \geq N} a_k \leq y$ und somit
 \[
  \limsup_{n \to \infty} a_n
  = \inf_{n \in \N} \sup_{k \geq n} a_k
  \leq \sup_{k \geq N} a_k
  \leq y
  < 1.
 \]
 Sei andererseits $x \coloneqq \limsup_{n \to \infty} a_n < C$. Da $\limsup_{n \to \infty} a_n = \inf_{n \in \N} \sup_{k \geq N} a_k$ gibt es wegen der $\varepsilon$-Charakterisierung des Infimums für alle $\varepsilon > 0$ ein $N' \in \N$ mit $\sup_{k \geq N'} a_k < x + \varepsilon$. Für $\varepsilon \coloneqq (C-x)/2$ gibt es daher ein $N \in \N$ mit
 \[
  \sup_{k \geq N} a_k < x + \varepsilon = \frac{C + x}{2} < C.
 \]
 Wählen wir $y \coloneqq x + \varepsilon = (C+x)/2$ so ist also $y < C$ und $a_k \leq y$ für alle $k \geq N$.
\end{solution}


\begin{question}
 Konvergieren für eine Folge $(a_n)_{n \in \N}$ die beiden Teilfolgen $(a_{2n})$ und $(a_{2n+1})$ und gilt $\lim_{n \to \infty} a_{2n} = \lim_{n \to \infty} a_{2n+1}$, so konvergiert bereits $(a_n)$ selbst.
\end{question}
\begin{solution}
 Es sei $a \coloneqq \lim_{n \to \infty} a_{2n} = \lim_{n \to \infty} a_{2n+1}$. Es sei $\varepsilon > 0$ beliebig aber fest. Da $\lim_{n \to \infty} a_{2n} = a$ gibt es $N_1 \in \N$ mit
 \[
  |a_{2n} - a| < \varepsilon \quad \text{für alle $n \geq N_1$}.
 \]
 Da $\lim_{n \to \infty} a_{2n+1} = a$ gibt es $N_2 \in \N$ mit
 \[
  |a_{2n+1} - a| < \varepsilon \quad \text{für alle $n \geq N_2$}.
 \]
 Für $N \coloneqq \max \{2N_1, 2N_2+1\}$ ist dann $|a_n - a| < \varepsilon$ für alle $n \geq N$. Aus der Beliebigkeit von $\varepsilon > 0$ folgt, dass $\lim_{n \to \infty} a_n = a$.
\end{solution}





\section{Definition}


\begin{defi}
 Für eine Folge $(a_n)_{n \in \N}$ ist die Folge der \emph{Partialsummen} $(s_n)_{n \in \N}$ als
 \[
  s_n \coloneqq \sum_{k=0}^n a_k
 \]
 definiert; $s_n$ heißt die \emph{$n$-te Partialsumme} (der Folge $(a_n)$). Diese Folge der Partialsummen bezeichnet man als \emph{Reihe} und schreibt man als $\sum_{k=0}^\infty a_k$. Konvergiert die Reihe $\sum_{k=0}^\infty a_k$, d.h. konvergiert die Folge der Partialsummen $(s_n)$, so bezeichnet man den Grenzwert $\lim_{n \to \infty} s_n = \lim_{n \to \infty} \sum_{k=0}^n a_k$ ebenfalls als $\sum_{k=0}^\infty a_k$ und nennt dies den \emph{Wert der Reihe}. Die Folge $(a_n)$ heißt dann summierbar.
 
 Für $N \in \N$ ist die Reihe $\sum_{k=N}^\infty a_k$ als die Reihe $\sum_{k=0}^\infty a_{k+N}$ definiert.  
\end{defi}


\begin{bem}
 Die Reihe $\sum_{k=N}^\infty a_k$ wird nach dieser Definition als die Folge der Partialsummen $(s_n)_{n \in \N}$ mit $s_n = \sum_{k=0}^n a_{k+N}$ verstanden. Alternativ kann man die Reihe auch als die Folge der Partialsummen $(s'_n)_{n \in \N}$ mit
 \[
  s'_n \coloneqq \sum_{k=N}^\infty a_k
 \]
 definieren. Dies macht praktisch keinen Unterschied, da dann
 \[
  s'_n =
  \begin{cases}
   0       & \text{falls } n < N, \\
   s_{n-N} & \text{falls } n \geq N.
  \end{cases}
 \]
 Die Folge $(s'_n)$ ist also die Folge $(s_n)$ mit Nullen aufgefüllt.
\end{bem}


Man bemerke, dass man mit der Notation $\sum_{k=0}^\infty a_k$ sowohl die Folge der Partialsummen $(\sum_{k=0}^n a_k)_{n \in \N}$ als auch der Grenzwert dieser Folge bezeichnet. Soll also gezeigt werden, dass die Reihe $\sum_{k=0}^\infty a_k$ konvergiert, so ist damit gemeint, dass die Folge der Partialsummen $(\sum_{k=0}^n a_k)_{n \in \N}$ konvergieren soll. Soll der Wert der Reihe $\sum_{k=0}^\infty a_k$ bestimmt werden, so soll der Grenzwert $\lim_{n \to \infty} \sum_{k=0}^n a_k$ ermittelt werden.


\begin{defi}
 Eine Reihe $\sum_{k=0}^\infty a_k$ heißt \emph{absolut konvergent}, wenn die Reihe der Beträge $\sum_{k=0}^\infty |a_k|$ konvergiert. Die Folge $(a_n)$ heißt dann absolut summierbar.
\end{defi}


\begin{bem}
 Ist $\sum_{k=0}^\infty a_k$ eine Reihe und $(s_n)_{n \in \N}$ die Folge der Partialsummen, also $s_n = \sum_{k=0}^n a_k$, so schreibt man für $\lim_{n \to \infty} s_n = \infty$ und $\lim_{n \to \infty} s_n = -\infty$ ebenfalls $\sum_{k=0}^\infty a_k = \infty$, bzw. $\sum_{k=0}^\infty a_k = -\infty$. Die Reihe $\sum_{k=0}^\infty a_k$ bezeichnet man aber in diesen Fällen nicht als konvergent.
\end{bem}


\begin{question}
 Zeigen Sie, dass für alle $-1 < q < 1$ die Reihe $\sum_{k=0}^\infty q^k$ konvergiert bestimmen Sie ihren Wert. Man bezeichnet Reihen dieser Form als \emph{geometrische Reihe}.
\end{question}
\begin{solution}
 Für alle $N \in \N$ gilt bekanntermaßen
 \[
  \sum_{k=0}^N q^k = \frac{1-q^{N+1}}{1-q}.
 \]
 Da $\lim_{N \to \infty} q^N = 0$ folgt, dass die Folge der Partialsummen $(\sum_{k=0}^N q^k)_{k \in \N}$ konvergiert und
 \[
  \sum_{k=0}^\infty q^k
  = \lim_{N \to \infty} \sum_{k=0}^N q^k
  = \lim_{N \to \infty} \frac{1-q^{N+1}}{1-q}
  = \frac{1}{1-q}.
 \]
\end{solution}


\begin{question}
 Zeigen Sie, dass die Reihe $\sum_{k=1}^\infty \frac{1}{k}$ gegen $\infty$ divergiert. Man bezeichnet diese Reihe als die \emph{harmonische Reihe}.
\end{question}
\begin{solution}
 Da die Folge der Partialsummen monoton steigend ist, genügt es zu zeigen, dass es für alle $R \in \N$ ein $N \in \N$ gibt, so dass $\sum_{k=1}^N 1/k \geq R$. Dies ergibt sich daraus, dass für alle $n \in \N$
 \[
  \sum_{k=1}^{2^n} \frac{1}{k}
  \geq \sum_{k=2}^{2^n} \frac{1}{k}
  = \sum_{\ell=0}^{n-1} \sum_{k=2^\ell+1}^{2^{\ell+1}} \frac{1}{k}
  \geq \sum_{\ell=0}^{n-1} 2^\ell \frac{1}{2^{\ell+1}}
  = \frac{n}{2}.
 \]
\end{solution}





\section{Grundlegende Eigenschaften}


Wir wollen nun einige grundlegende Eigenschaften von (konvergenten) Reihen angeben und beweisen.


\begin{lem}\label{lem: Reihe bedeutet Nullfolge}
 Konvergiert für eine Folge $(a_n)_{n \in \N}$ die Reihe $\sum_{k=0}^n a_k$, so ist $(a_n)$ eine Nullfolge, d.h. $\lim_{n \to \infty} a_n = 0$.
\end{lem}
\begin{proof}
 Wir betrachten die Folge der Partialsummen $(s_n)_{n \in \N}$, also $s_n \coloneqq \sum_{k=0}^n a_k$. Dass die Reihe $\sum_{k=0}^\infty a_k$ konvergiert bedeutet gerade, dass die Folge $(s_n)$ konvergiert. Es sei
 \[
  s \coloneqq \lim_{n \to \infty} s_n.
 \]
 Wir bemerken nun, dass für alle $n \geq 1$
 \[
  s_n - s_{n-1}
  = \left( \sum_{k=0}^n a_k \right) - \left( \sum_{k=0}^{n-1} a_k \right)
  = a_n.
 \]
 Durch die üblichen Rechenregeln konvergenter Folgen ergibt sich daher, dass
 \[
  a_n = s_n - s_{n-1} \to s-s = 0
 \]
 für $n \to \infty$. Also ist $(a_n)$ konvergent und $\lim_{n \to \infty} a_n = 0$.
\end{proof}

\begin{question}
 Für welche $q \in \R$ konvergiert die Reihe $\sum_{k=0}^\infty q^k$?
\end{question}
\begin{solution}
 Wir wissen bereits, dass die Reihe für $|q| < 1$ absolut konvergiert. Für $|q| \geq 1$ ist $(q^n)_{n \in \N}$ keine Nullfolge, da $|q^n| = |q|^n \geq 1$ für alle $n \in \N$, und die Reihe $\sum_{k=0}^\infty q^k$ somit nicht konvergent.
\end{solution}


Man beachte, dass die Umkehrung des Lemmas nicht gilt, d.h. nicht jede Nullfolge ist auch summierbar. Ein einfaches Gegenbeispiel hierfür ist die harmonische Reihe $\sum_{k=1}^\infty \frac{1}{k}$.


\begin{prop}
 \begin{enumerate}
  \item
   Konvergieren die Reihen $\sum_{k=0}^\infty a_k$ und $\sum_{k=0}^\infty b_k$, so konvergiert auch die Reihe $\sum_{k=0}^\infty (a_k + b_k)$ und es gilt
   \[
    \sum_{k=0}^\infty (a_k + b_k) = \left( \sum_{k=0}^\infty a_k \right) + \left( \sum_{k=0}^\infty b_k \right).
   \]
  \item
   Konvergiert die Reihe $\sum_{k=0}^\infty a_k$ so konvergiert für jedes $\lambda \in \R$ auch die Reihe $\sum_{k=0}^\infty (\lambda a_k)$ und es gilt
   \[
    \sum_{k=0}^\infty (\lambda a_k) = \lambda \sum_{k=0}^\infty a_k.
   \]
  \item
   Für eine Reihe $\sum_{k=0}^\infty a_k$ gilt für jedes $N \in \N$: Die Reihe $\sum_{k=0}^\infty a_k$ konvergiert genau dann, wenn die Reihe $\sum_{k=N}^\infty a_k$ konvergiert. Es ist dann
   \[
    \sum_{k=0}^\infty a_k = \sum_{k=0}^{N-1} a_k + \sum_{k=N}^\infty a_k.
   \]
 \end{enumerate}
\end{prop}
\begin{proof}
 Die Aussagen ergeben sich direkt aus den bekannten Rechenregeln für endliche Summen und konvergente Folgen. Ein genaues Formulieren bleibt den Lesern als Übung überlassen.
\end{proof}


\begin{kor}
 Die Menge der summierbaren Folgen
 \[
  \Sigma \coloneqq \{ (a_n)_{n \in \N} \mid \text{$(a_n)$ ist summierbar}\}
 \]
 bildet unter punktweiser Addition und Skalarmultiplikation einen $\R$-Vektorraum, und die Abbildung
 \[
  \Sigma \to \R, (a_n)_{n \in \N} \mapsto \sum_{k=0}^\infty a_k
 \]
 ist $\R$-linear.
\end{kor}


Wir erhalten aus dem Lemma auch, dass $\lim_{n \to \infty} \sum_{k=n}^\infty a_k = 0$, denn wir haben
\[
 \sum_{k=n}^\infty a_k
 = \sum_{k=0}^\infty a_k - \sum_{k=0}^{n-1} a_k
 \xrightarrow{n \to \infty} \sum_{k=0}^\infty a_k - \sum_{k=0}^\infty a_k
 = 0.
\]


Wie bereits für endliche Summen gilt auch für Reihen eine Dreiecksungleichung.


\begin{lem}
 Ist die Reihe $\sum_{k=0}^\infty a_k$ konvergent, so ist
 \[
  \left| \sum_{k=0}^\infty a_k \right| \leq \sum_{k=0}^\infty |a_k|.
 \]
\end{lem}
\begin{proof}[Beweis des Lemmas]
 Ist die Reihe $\sum_{k=0}^\infty a_k$ nicht absolut konvergent, so haben wir $\sum_{k=0}^\infty |a_k| = \infty$ und die Aussage ist klar. Ansonsten gilt für alle $n \in \N$ nach der Dreiecksungleichung für endliche Summen
 \[
  \left| \sum_{k=0}^n a_k \right| \leq \sum_{k=0}^n |a_k|,
 \]
 so dass wir im Grenzwert
 \[
  \left| \sum_{k=0}^\infty a_k \right|
  = \left| \lim_{n \to \infty} \sum_{k=0}^n a_k \right|
  = \lim_{n \to \infty} \left| \sum_{k=0}^n a_k \right|
  \leq \lim_{n \to \infty} \sum_{k=0}^n |a_k|
  = \sum_{k=0}^\infty |a_k|
 \]
 haben.
\end{proof}


Wir wollen nun auf den Zusammenhang zwischen konvergenten und absolut konvergenten Reihen zurückkommen.


\begin{lem}
 Ist eine Reihe $\sum_{k=0}^\infty a_k$ absolut konvergent, so ist sie auch konvergent.
\end{lem}
\begin{proof}
 Es sei $(s_n)_{n \in \N}$ die Folge der Partialsummen, also $s_n = \sum_{k=0}^n a_k$. Wir wollen zeigen, dass $(s_n)$ eine Cauchy-Folge ist. Sei hierfür $\varepsilon > 0$ beliebig aber fest. Für alle $n \in \N$ haben wir für alle $m, m' \geq n$
 \[
  |s_m - s_{m'}|
  = \left| \sum_{k=\min\{m,m'\}}^{\max\{m,m'\}} a_k \right|
  \leq \sum_{k=\min\{m,m'\}}^{\max\{m,m'\}} |a_k|
  \leq \sum_{k=n}^\infty |a_k|.
 \]
 Da $\lim_{n \to \infty} \sum_{k=n}^\infty |a_k| = 0$ gibt es ein $N \in \N$ mit $\sum_{k=N}^\infty |a_k| < \varepsilon$. Aus der obigen Ungleichung ergibt sich damit, dass $m, m' \geq N$ ist dann $|s_m - s_{m'}| < \varepsilon$.
\end{proof}


Wie wir noch sehen werden gilt die Umkehrung des Lemmas nicht, d.h. es gibt konvergente Reihen, die nicht absolut konvergent sind.


\begin{lem}
 Es seien $\sum_{k=0}^\infty a_k$ und $\sum_{k=0}^\infty b_k$ Reihen mit $0 \leq a_n \leq b_n$ für alle $n \in \N$. Dann ist $\sum_{k=0}^\infty a_k \leq \sum_{k=0}^\infty b_k$.
\end{lem}
\begin{proof}
 Für die Partialsummen gilt für jedes $n \in \N$
 \[
  0 \leq \sum_{k=0}^n a_k \leq \sum_{k=0}^n b_k,
 \]
 und die Folge der Partialsummen ist monoton steigend. Die Aussage ergibt sich damit direkt daraus, dass Monotonie von Folgen unter Grenzwerten erhalten bleibt.
\end{proof}





\section{Konvergenzkriterien}


Wir wollen nun Kriterien entwickeln, mit denen sich entscheiden lässt, ob eine Folge (absolut) konvergiert. Lemma \ref{lem: Reihe bedeutet Nullfolge} liefert uns bereits die notwendige Bedingung, dass $(a_n)_{n \in \N}$ eine Nullfolge seien muss, damit $\sum_{k=0}^\infty a_k$ konvergiert. Wie wir bereits wissen, ist diese Bedingung aber nicht hinreichend. Solche hinreichenden Bedingungen wollen wir nun ermitteln. Dabei spielt die Absolute Konvergenz eine besondere Rolle, da sich diese unter gewissen Umständen durch einen Vergleich mit der geometrischen Reihe untersuchen lässt.


\subsection{Majoranten- und Minorantenkriterium}

Eine der einfachsten Ideen um die absolutive Konvergenz einer Reihe $\sum_{k=0}^\infty a_n$ zu untersuchen besteht darin, die Reihe der Beträge $\sum_{k=0}^\infty |a_k|$ passend gegen eine konvergente, bzw. divergente Reihe abzuschätzen.


\begin{lem}
 Es sei $\sum_{k=0}^\infty a_k$ und eine Reihe reeller Zahlen und $(b_n)_{n \in \N}$ eine Folge mit $b_n \geq 0$ für alle $n \in \N$.
 \begin{enumerate}
  \item
   Ist $|a_n| \leq b_n$ für alle $n \in \N$ und $\sum_{k=0}^\infty b_k < \infty$, so ist $\sum_{k=0}^\infty a_k$ absolut konvergent. (Majorantenkriterium)
  \item
   Ist $b_n \leq |a_n|$ für alle $n \in \N$ und $\sum_{k=0}^\infty a_k = \infty$, so ist $\sum_{k=0}^\infty a_k$ nicht absolut konvergent. (Minorantenkriterium)
 \end{enumerate}
\end{lem}
\begin{proof}
 \begin{enumerate}
  \item
   Für alle $n \in \N$ ist
   \[
    \sum_{k=0}^n |a_k| \leq \sum_{k=0}^n b_k \leq \sum_{k=0}^\infty b_k,
   \]
   also die Folge von Partialsummen $(\sum_{k=0}^n |a_k|)_{n \in \N}$ monoton steigend und nach oben beschränkt, und somit konvergent.
  \item
   Wäre $\sum_{k=0}^\infty a_k$ absolut konvergent, so würde die Reihe der Beträge $\sum_{k=0}^\infty |a_k|$ konvergieren, und nach dem Majorantenkriterium würde auch $\sum_{k=0}^\infty b_k$ konvergieren, im Widerspruch zu Annahme.
  \qedhere
 \end{enumerate}
\end{proof}


\begin{question}
 Untersuchen Sie die Konvergenz der Reihen $\sum_{k=1}^\infty \frac{1}{\sqrt{k}}$, $\sum_{k=1}^\infty k^{-k}$ und $\sum_{k=1}^\infty \frac{1}{k!}$.
\end{question}
\begin{solution}
 Für alle $k \geq 1$ ist $\sqrt{k} \leq k$ und somit $1/\sqrt{k} \geq 1/k$. Da $\sum_{k=1}^\infty 1/k = \infty$, folgt aus dem Minorantenkriterium, dass auch $\sum_{k=1}^\infty 1/\sqrt{k} = \infty$.
 
 Für alle $k \geq 1$ ist $k! \geq 2^{k-1}$, und somit $1/k! \geq 1/2^{k-1}$. Da die geometrische Reihe $\sum_{k=1}^\infty 1/2^{k-1} = \sum_{k=0}^\infty (1/2)^k$ konvergiert, folgt aus dem Majorantenkriterium, dass auch die Reihe $\sum_{k=1}^\infty 1/k!$ konvergiert.
 
 Für alle $k \geq 1$ ist $k! \leq k^k$ $k^{-k} = 1/k^k \leq 1/k!$. Da $\sum_{k=1}^\infty 1/k!$ konvergiert, folgt aus dem Majorantenkriterium, dass auch die Reihe $\sum_{k=1}^\infty k^{-k}$ konvergiert.
\end{solution}

\begin{question}\label{qst: absolut konvergent und beschränkt}
 Es seien $\sum_{k=0}^\infty a_k$ eine absolut konvergente Reihe und $(b_k)_{k \in \N}$ eine beschränkte Folge. Zeigen  Sie, dass auch die Reihe $\sum_{k=0}^\infty a_k b_k$ absolut konvergiert.
\end{question}
\begin{solution}
 Da die Folge $(b_k)_{k \in \N}$ beschränkt ist, gibt es eine Konstante $C > 0$ mit $|b_k| \leq C$ für alle $k \in \N$. Für alle $k \in \N$ haben wir
 \[
  |a_k b_k|
  = |a_k| |b_k|
  \leq C |a_k|,
 \]
 und die Konvergenz der Reihe $\sum_{k=0}^\infty C |a_k|$ folgt direkt aus der absoluten Konvergenz von $\sum_{k=0}^\infty |a_k|$. Also ist die Reihe $\sum_{k=0}^\infty a_k b_k$ nach dem Majorantenkriterium konvergent.
\end{solution}


Das Majorantenkriterium führt zusammen mit der geometrischen Reihe zu zwei wichtigen Konvergenzkriterien, dem \emph{Quotientenkriterium} und dem \emph{Wurzelkriterium}.





\subsection{Quotientenkriterium}
Ein erstes wichtes Konvergenzkriterium, dass durch den Vergleich mit der harmonischen Reihe entsteht, 


\begin{prop}
 Es sei $(a_n)_{n \in \N}$ eine Folge mit $a_n \neq 0$ für alle $n \in \N$ und es gebe $0 \leq y < 1$ und $N \in \N$ mit $|a_{n+1}/a_n| \leq y$ für alle $n \geq N$. Dann ist die Reihe $\sum_{k=0}^\infty a_k$ absolut konvergent.
\end{prop}
\begin{proof}
 Wir haben
 \begin{align*}
  \sum_{k=N}^\infty |a_k|
  &= \sum_{k=N}^\infty |a_N| \prod_{j=N}^{k-1} \frac{|a_{j+1}|}{|a_j|}
  = |a_N| \sum_{k=N}^\infty \prod_{j=N}^{k-1} \frac{|a_{j+1}|}{|a_j|} \\
  &\leq |a_N| \sum_{k=N}^\infty y^{k-N}
  = |a_N| \sum_{k=0}^\infty y^k
  = \frac{|a_N|}{1-y},
 \end{align*}
 und somit
 \[
  \sum_{k=0}^\infty |a_k|
  = \sum_{k=0}^{N-1} |a_k| + \sum_{k=N}^\infty |a_k|
  \leq \sum_{k=0}^{N-1} |a_k| + \frac{|a_N|}{1-y}
  < \infty.
  \qedhere
 \]
\end{proof}


Wie wir bereits wissen, sind die Voraussetzungen des Quotientenkriteriums äquivalent dazu, dass $\limsup_{n \to \infty} |a_{n+1}/a_n| < 1$. Man bemerke jedoch, dass sich für den Fall $\limsup_{n \to \infty} |a_{n+1}/a_n| \geq 1$ keine Aussage über die Konvergenz von $\sum_{k=0}^\infty a_k$ treffen lässt.


\begin{question}
 Geben Sie divergente, bzw. absolut konvergente Reihen $\sum_{k=0}^\infty a_k$ und $\sum_{k=0}^\infty b_k$ an, so dass 
 \[
  \limsup_{n \to \infty} \left|\frac{a_{n+1}}{a_n}\right| = 1
  \quad
  \text{und}
  \quad
  \limsup_{n \to \infty} \left|\frac{b_{n+1}}{b_n}\right| = \infty.
 \]
\end{question}
\begin{solution}
 Die harmonische Reihe $\sum_{k=1}^\infty 1/k$ divergiert und es gilt
 \[
  \limsup_{n \to \infty} \frac{1/n}{1/(n+1)}
  = \limsup_{n \to \infty} \frac{n+1}{n}
  = \limsup_{n \to \infty} 1+\frac{1}{n+1}
  = 1.
 \]
 Die Reihe $\sum_{k=0}^\infty 2^k$ divergiert ebenfalls, da $(2^n)_{n \in \N}$ keine Nullfolge ist, und es gilt
 \[
  \limsup_{n \to \infty} \frac{2^{n+1}}{2^n}
  = \limsup_{n \to \infty} 2
  = 2
  > 1.
 \]

 Für die Folge $(a_n)_{n \in \N}$ mit $a_{2n} \coloneqq a_{2n+1} \coloneqq 1/2^n$ für alle $n \in \N$ gilt für alle $n \in \N$
 \[
  \frac{a_{n+1}}{a_n} =
  \begin{cases}
            1  & \text{falls $n$ gerade} \\
   \frac{1}{2} & \text{falls $n$ ungerade}.
  \end{cases}
 \]
 Es ist also $\limsup_{n \to \infty} |a_{n+1}/a_n| = 1$. Die Reihe konvergiert aber offenbar mit
 \[
  \sum_{k=0}^\infty a_k
  = 2 \sum_{k=0}^\infty \frac{1}{2^k}
  = 2 \cdot 2
  = 4.
 \]
 Für die Folge $(b_n)_{n \in \N}$ mit
 \[
  b_n \coloneqq
  \begin{cases}
     \frac{1}{2^n} & \text{falls $n$ gerade}, \\
   \frac{1}{n 2^n} & \text{falls $n$ ungerade},
  \end{cases}
 \]
 ist die Reihe $\sum_{k=0}^\infty b_k$ nach dem Majorantenkriterium absolut konvergent (mit Majorante $(1/2^n)_{n \in \N}$). Da für alle $n \in \N$
 \[
  \frac{b_{n+1}}{b_n} =
  \begin{cases}
   \frac{1}{2(n+1)} & \text{falls $n$ gerade}, \\
                n/2 & \text{falls $n$ ungerade},
  \end{cases}
 \]
 ist jedoch $\limsup_{n \to \infty} |b_{n+1}/b_n| = \infty$.
\end{solution}

\begin{question}
 Untersuchen Sie die Reihen $\sum_{k=0}^\infty 2^k/k!$, $\sum_{k=0}^\infty k^2/(k^3 + 5)$ und $\sum_{k=0}^\infty k^2/2^k$ auf Konvergenz.
\end{question}
\begin{solution}
 Es ist
 \[
  \limsup_{n \to \infty} \frac{2^{k+1} / (k+1)!}{2^k / k!}
  = \limsup_{n \to \infty}\frac{2}{k+1} = 0,
 \]
 also konvergiert die Reihe $\sum_{k=0}^\infty 2^k/k!$ nach dem Quotientenkriterium absolut.
 
 Für alle $n \geq 2$ ist $n^3 \geq 5$ und somit
 \[
  \frac{n^2}{n^3 + 5} \geq \frac{n^3}{n^3 + n^3} = \frac{1}{2n}.
 \]
 Da $\sum_{k=1}^\infty \frac{1}{2n} = \infty$ folgt aus dem Minorantenkriterium, dass $\sum_{k=0}^\infty k^2/(k^3 + 5) = \infty$.
 
 Da
 \begin{align*}
  \limsup_{n \to \infty} \frac{(n+1)^2/2^{n+1}}{n^2/2^n}
  &= \limsup_{n \to \infty} \frac{1}{2} \frac{n^2 + 2n + 1}{n^2} \\
  &= \limsup_{n \to \infty} \frac{1}{2} \left( 1 + \frac{2}{n} + \frac{1}{n^2} \right)
  = \frac{1}{2}
 \end{align*}
 konvergiert die Reihe $\sum_{k=0}^\infty k^2 / 2^k$ nach dem Quotientenkriterium absolut.
\end{solution}


\begin{question}
 Es sei $-1 < q < 1$. Zeigen Sie, dass die Reihe $\sum_{k=1}^\infty k q^k$ konvergiert und bestimmen sie den Grenzwert.
\end{question}
\begin{solution}
 Da
 \[
  \limsup_{n \to \infty} \left| \frac{(n+1)q^{n+1}}{nq^n} \right|
  = \limsup_{n \to \infty} \left(1 + \frac{1}{n}\right) |q|
  = |q|
  < 1
 \]
 konvergiert die Reihe nach dem Quotientenkriterium. Zur Bestimmung des Grenzwertes $\xi \coloneqq \sum_{k=1}^\infty k q^k$ bemerken wir, dass
 \begin{align*}
  \xi
  &= \sum_{k=1}^\infty k q^k
  = \sum_{k=0}^\infty (k+1) q^{k+1} \\
  &= \sum_{k=0}^\infty k q^{k+1} + \sum_{k=0}^\infty q^{k+1} \\
  &= q \sum_{k=0}^\infty k q^k + q \sum_{k=0}^\infty q^k \\
  &= q \xi + \frac{q}{1-q}.
 \end{align*}
 Durch Umstellen ergibt sich, dass $\xi = q/(1-q)^2$.
\end{solution}


\begin{question}
 Bestimmen sie alle $x \in \R$, für welche die Reihe $\sum_{k=0}^\infty x^k/k!$ konvergiert und für welche $x$ sie absolut konvergiert.
\end{question}
\begin{solution}
 Für $x = 0$ konvergiert die Reihe offenbar absolut. Für $x \neq 0$ haben wir
 \[
  \limsup_{n \to \infty} \left| \frac{x^{k+1}/(k+1)!}{x^k/k!} \right|
  = \limsup_{n \to \infty} \left| \frac{x^{k+1} k!}{x^k (k+1)!} \right|
  = \limsup_{n \to \infty} \frac{|x|}{k+1}
  = 0,
 \]
 und die Reihe konvergiert nach dem Quotientenkriterium absolut.
\end{solution}





\subsection{Wurzelkriterium}

Ein weiteres wichtiges Konvergenzkriterium, dass sich mithilfe der geometrischen Reihe ergibt, ist das \emph{Wurzelkriterium}.

\begin{prop}
 Es sei $(a_n)_{n \in \N}$ eine Folge und es gebe $y < 1$ und $N \in \N$, so dass $|a_n|^{1/n} < y$ für alle $n \geq N$. Dann konvergiert die Reihe $\sum_{k=0}^\infty a_k$ absolut.
\end{prop}
\begin{proof}
 Für alle $n \geq N$ gilt $|a_n|^{1/n} < y$ und damit $|a_n| < y^n$. Daher ist
 \[
  \sum_{k=N}^\infty |a_k|
  \leq \sum_{k=N}^\infty y^k
  = y^N \sum_{k=N}^\infty y^{k-N}
  = y^N \sum_{k=0}^\infty y^k
  = \frac{y^N}{1-y},
 \]
 und somit
 \[
  \sum_{k=0}^\infty |a_k|
  = \sum_{k=0}^{N-1} |a_k| + \sum_{k=N}^\infty |a_k|
  \leq \sum_{k=0}^{N-1} |a_k| + \frac{y^N}{1-y}
  < \infty.
  \qedhere
 \]
\end{proof}

Wie wir bereits wissen, sind die Voraussetzungen des Wurzelkriteriums äquivalent dazu, dass $\limsup_{n \to \infty} |a_n|^{1/n} < 1$. Wie bereits beim Quotientenkriterium können wir im Fall $\limsup_{n \to \infty} |a_n|^{1/n} = 1$ keine Aussage über das Konvergenzverhalten der Reihe treffen. Im Gegensatz zum Quotientenkriterium wissen wir jedoch im Fall $\limsup_{n \to \infty} |a_n|^{1/n} > 1$, dass die Reihe divergiert!


\begin{bem}
 Es lässt sich zeigen, dass das Wurzelkriterium stärker ist als das Quotientenkriterium, d.h. wenn sich absolute Konvergenz durch das Quotientenkriterium zeigen lässt, dann auch durch das Wurzelkriterium. Insbesondere hilft in den Fällen, in denen das Wurzelkriterium keine Aussage trifft (also wenn $\limsup_{n \to \infty} |a_n|^{1/n} = 1$), auch das Quotientenkriterium nicht weiter.
\end{bem}


\begin{question}
 Gebe eine divergente Reihe $\sum_{k=0}^\infty a_k$ mit $\limsup_{n \to \infty} |a_n|^{1/n} = 1$ an.
\end{question}
\begin{solution}
 Die harmonische Reihe $\sum_{k=1}^\infty 1/k$ divergiert mit
 \[
  \limsup_{n \to \infty} \frac{1}{n^{1/n}} = 1.
 \]
\end{solution}


\begin{question}
 Zeigen Sie, dass die Reihe $\sum_{k=0}^\infty a_k$ divergiert, falls $\limsup_{n \to \infty} |a_n|^{1/n} > 1$.
\end{question}
\begin{solution}
 Da $\limsup_{n \to \infty} |a_n|^{1/n}| = \inf_{n \in 1} \sup_{k \geq n} |a_k|^{1/k}$ ist $1 < \sup_{k \geq n} |a_k|^{1/k}$ für alle $n \geq 1$. Es gibt daher für jedes $n \geq 1$ ein $k \geq 1$ mit $1 < |a_k|^{1/k}$ und somit $|a_k| > 1$. Also ist $(a_n)_{n \in \N}$ keine Nullfolge, und die Reihe $\sum_{k=0}^\infty a_k$ somit nicht konvergent.
\end{solution}


\begin{question}
 Für welche $x \in \R$ konvergiert die Reihe $\sum_{k=0}^\infty \frac{1}{2^k} x^k$?
\end{question}
\begin{solution}
 Für alle $x \in \R$ ist
 \[
  \sum_{k=0}^\infty \frac{1}{2^k} x^k
  = \sum_{k=0}^\infty \left(\frac{x}{2}\right)^k.
 \]
 Diese Reihe konvergiert bekanntermaßen genau dann, wenn $|x/2| < 1$, also wenn $-2 < x < 2$.
\end{solution}





\subsection{Cauchysches Verdichtungskriterium}


\begin{prop}
 Es sei $(a_n)_{n \geq 1}$ eine monoton fallende Folge mit $a_n \geq 0$ für alle $n \in \N$. Dann konvergiert die Reihe $\sum_{k=0}^\infty a_k$ genau dann, wenn die Reihe $\sum_{\ell=0}^\infty 2^\ell a_{2^\ell}$ konvergiert.
\end{prop}
\begin{proof}
 Wir haben
 \[
  \sum_{k=1}^\infty a_k
  = \sum_{\ell=0}^\infty \sum_{k=2^\ell}^{2^{\ell+1}-1} a_k.
 \]
 Da $(a_n)$ monoton fallend ist, ist für alle $\ell \in \N$
 \begin{gather*}
  \sum_{k=2^\ell}^{2^{\ell+1}-1} a_k
  \leq 2^\ell a_{2^\ell}
 \shortintertext{sowie}
  \sum_{k=2^\ell}^{2^{\ell+1}-1} a_k
  \geq 2^\ell a_{2^{\ell+1}}.
 \end{gather*}
 Konvergiert die Reihe $\sum_{\ell=0}^\infty 2^\ell a_{2^\ell}$, so konvergiert wegen
 \[
  \sum_{k=1}^\infty a_k \leq \sum_{\ell=0}^\infty 2^\ell a_{2^\ell}
 \]
 dann auch die Reihe $\sum_{k=1}^\infty a_k$. Konvergiert die Reihe $\sum_{k=1}^\infty a_k$, so konvergiert wegen
 \[
  \sum_{k=1}^\infty a_k \geq \sum_{\ell=0}^\infty 2^\ell a_{2^{\ell+1}}
 \]
 dann auch die Reihe $\sum_{\ell=0}^\infty 2^\ell a_{2^{\ell+1}}$, und wegen
 \[
  \sum_{\ell=0}^\infty 2^\ell a_{2^{\ell+1}} = \frac{1}{2} \sum_{\ell=1}^\infty 2^\ell a_{2^\ell}
 \]
 damit auch die Reihe $\sum_{\ell=0}^\infty 2^\ell a_{2^\ell}$.
\end{proof}


\begin{question}
 Zeigen Sie, dass die Reihe $\sum_{k=1}^\infty 1/k^\alpha$ genau dann konvergiert, wenn $\alpha>1$. (Diese Reihen sind die \emph{allgemeinen harmonischen Reihen}.)
\end{question}
\begin{solution}
 Für $\alpha \leq 0$ ist $n^\alpha \leq 1$ und damit $1/n^\alpha \geq 1$ für alle $n \geq 1$. Die Reihe divergiert dann gegen $\infty$. 
 
 Für $\alpha > 0$ ist die Folge $(1/n^\alpha)_{n \geq 1}$ monoton fallend mit $1/n^\alpha \geq 0$ für alle $n \geq 1$. Nach dem Cauchy-Kriterium konvergiert die Reihe $\sum_{k=1}^\infty 1/k^\alpha$ daher genau dann, wenn die Reihe
 \[
  \sum_{\ell=0}^\infty 2^\ell \frac{1}{\left(2^\ell\right)^\alpha}
  = \sum_{\ell=0}^\infty 2^{\ell(1-\alpha)}
  = \sum_{\ell=0}^\infty \left(2^{1-\alpha}\right)^\ell
 \]
 konvergiert. Dies ist bekanntermaßen genau dann der Fall, wenn $2^{1-\alpha} < 1$, wenn also $\alpha > 1$.
 
 Die Reihe $\sum_{k=1}^\infty 1/k^\alpha$ konvergiert also genau dann, wenn $\alpha > 1$.
\end{solution}





\subsection{Leibniz-Kriterium}


\begin{prop}
 Es sei $(a_n)_{n \in \N}$ eine monoton fallende Folge mit $a_n \geq 0$ für alle $n \in \N$ und $\lim_{n \to \infty} a_n = 0$. Dann konvergiert die alternierende Reihe $\sum_{k=0}^\infty (-1)^n a_n$.
\end{prop}
\begin{proof}
 Wir betrachten die Folge der Partialsummen $(s_n)_{n \in \N}$, also
 \[
  s_n = \sum_{k=0}^\infty (-1)^k a_k.
 \]
 
 Da die Folge $(a_n)$ monoton fallend ist, ist für alle $n \in \N$
 \[
  s_{2(n+1)} - s_{2n} = a_{2n+2} - a_{2n+1} \leq 0.
 \]
 Die Teilfolge $(s_{2n})$ ist also monoton fallend. Analog ergibt sich, dass die Teilfolge $(s_{2n+1})$ monoton steigend ist.
 
 Für alle $n \in \N$ haben wir außerdem
 \[
  s_{2n} \geq s_{2n} - a_{2n+1} =  s_{2n+1}.
 \]
 Es ist also für alle $n \in \N$
 \begin{gather*}
  s_0 \geq s_2 \geq s_4 \geq \dotsb \geq s_{2n} \geq s_{2n+1},
 \shortintertext{sowie}
  s_{2n} \geq s_{2n+1} \geq \dotsb \geq s_5 \geq s_3 \geq s_1.
 \end{gather*}
 Die Teilfolge $(s_{2n})$ ist also durch $s_1$ nach unten beschränkt, und die Teilfolge $(s_{2n+1})$ durch $s_0$ nach oben beschränkt. Die Teilfolgen $(s_{2n})$ und $(s_{2n+1})$ konvergieren also.
 
 Es sei $s \coloneqq \lim_{n \to \infty} s_{2n}$ und $s' \coloneqq \lim_{n \to \infty} s_{2n+1}$. Da
 \[
  s - s' = \lim_{n \to \infty} (s_{2n} - s_{2n+1}) = \lim_{n \to \infty} a_{2n+1} = 0
 \]
 ist $s = s'$. Also ist bereits $(s_n)$ selbst konvergent (mit $\lim_{n \to \infty} s_n = s = s'$).
\end{proof}


\begin{question}
 Geben Sie eine Reihe an, die zwar konvergiert, aber nicht absolut konvergent ist.
\end{question}
\begin{solution}
 Die Folge $(1/n)_{n \geq 1}$ ist monoton fallend mit $\lim_{n \to \infty} a_n = 0$. Nach dem Leipnizkriterium konvergiert daher die Reihe $\sum_{k=1}^\infty \frac{(-1)^n}{n}$. Diese Reihe ist aber nicht absolut konvergent, da die harmonische Reihe divergiert.
\end{solution}


\begin{question}
 Es seien $\sum_{k=0}^\infty a_k$ und $\sum_{k=0}^\infty b_k$ konvergente Reihen. Konvergiert dann auch die Reihe $\sum_{k=0}^\infty a_k b_k$? Was ist, wenn eine der beiden Reihen absolut konvergent ist?
\end{question}
\begin{solution}
 Die Reihe $\sum_{k=1}^\infty (-1)^k/\sqrt{k}$ konvergiert nach dem Leibniz-Kriterium, die Reihe
 \[
  \sum_{k=1}^\infty \frac{(-1)^k}{\sqrt{k}} \frac{(-1)^k}{\sqrt{k}}
  = \sum_{k=1}^\infty \frac{1}{k}
 \]
 divergiert aber.
 
 Ist eine der beiden Reihen absolut konvergent, so können wir o.B.d.A.\ davon ausgehen, dass $\sum_{k=0}^\infty a_k$ absolut konvergent ist. Da die Reihe $\sum_{k=0}^\infty b_k$ zumindest konvergent ist (wenn auch nicht notwendigerweise absolut konvergent), ist die Folge $(b_k)$ eine Nullfolge und somit beschränkt (denn alle konvergenten Folgen sind beschränkt). Nach Übung \ref{qst: absolut konvergent und beschränkt} konvergiert die Reihe $\sum_{k=0}^\infty a_k b_k$ in diesem Fall absolut.
\end{solution}



\begin{question}\label{qst: Konvergenzradius}
Es sei $(a_n)_{n \in \N}$ eine Folge mit $\lim_{n \to \infty} |a_n|^{1/n} = 1/2$. Bestimmen Sie für möglichst viele $x \in \R$ das Konvergenzverhalten der Reihe $\sum_{k=0}^\infty a_n x^n$.
\end{question}
\begin{solution}
 Für alle $x \in \R$ gilt
 \[
  \limsup_{n \to \infty} |a_n x^n|^{1/n}
  = |x| \limsup_{n \to \infty} |a_n|^{1/n}
  = \frac{|x|}{2}.
 \]
 Aus dem Wurzelkriterium folgt, dass die Reihe $\sum_{k=0}^\infty a_n x^n$ für $|x| < 2$ absolut konvergiert und für $|x| > 2$ divergiert.
 
 Für den Fall $|x| = 2$, also $x = -2$ oder $x = 2$ gibt das Wurzelkriterium, und damit auch das Quotientenkriterium, keine Auskunft. Das Verhalten an diesen Stellen hängt tatsächlich von der Folge $(a_n)_{n \in \N}$ ab: Betrachten wir etwa die Folge $(a_n)$ mit
 \[
  a_n = \frac{1}{2^n n},
 \]
 so ist
 \[
  \lim_{n \to \infty} |a_n|^{1/n}
  \lim_{n \to \infty} \frac{1}{2} \frac{1}{n^{1/n}}
  = \frac{1}{2}.
 \]
 Für $x = 2$ konvergiert die Reihe $\sum_{k=1}^\infty a_k x^k$ nicht, da es sich dort die harmonische Reihe handelt; für $x = -2$ konvergiert sie, da es sich dort um die alternierende harmonische Reihe handelt. Betrachtet man hingegen die Reihe $(a'_n)_{n \in \N}$ mit
 \[
  a'_n = \frac{(-1)^n}{2^n n},
 \]
 so ergibt sich genau das umgekehrte Verhalten, d.h. die Reihe $\sum_{k=1}^\infty a'_k x^k$ konvergiert an $x = 2$ und divergiert an $x = -2$.
\end{solution}




\section{Potenzreihen}
Wir wollen hier noch einen besonderen Fall von Reihen ansprechen, sogennante Potenzreihen, und ihr Konvergenzverhalten mithilfe des Wurzelkriteriums charakterisieren.


\subsection{Definition}
\begin{defi}
 Eine \emph{Potenzreihe} ist eine Reihe der Form $\sum_{k=0}^\infty a_k x^k$, wobei $(a_k)$ eine Folge reeller Zahlen ist, und $x \in \R$ ein reeller Parameter. Man bezeichnet die Zahl $a_k$ als den $k$-ten Koeffizienten der Potenzreihe, und die Folge $(a_n)$ als die Folge der Koeffizienten der Potenzreihe.
 
 Allgemeiner bezeichnet man eine Reihe der Form $\sum_{k=0}^\infty a_k (x-x_0)^k$ als eine Potenzreihe mit \emph{Entwicklungsstelle} $x_0$.
\end{defi}


Die Konvergenz einer Potenzreihe hängt zum einen von der Folge der Koeffizienten $(a_n)$, als auch von der Stelle $x \in \R$. Für eine Potenzreihe $\sum_{k=0}^\infty a_k x^k$ können wir uns einige grundlegende Fragen stellen:
\begin{itemize}
 \item
  Für welche $x \in \R$ konvergiert die Potenzreihe $\sum_{k=0}^\infty a_k x^k$?
 \item
  Wenn eine Entwicklungsstelle $x_0 \in \R$ vorgegeben ist, an welchen Stellen $x \in \R$ konvergiert dann die Potenzreihe $\sum_{k=0}^\infty a_k (x-x_0)^k$?
 \item
  Inwiefern lässt sich eine Funktion $f \colon \R \to \R$ als Potenzreihe darstellen, d.h. inwiefern ist es möglich, eine Koeffizientenfolge $(a_n)$ zu finden, so dass $f(x) = \sum_{k=0}^\infty a_k x^k$, und für welche $x$ ist dies Darstellung gültig?
\end{itemize}


Wir wollen hier mithilfe des Wurzelkriteriums zumindest eine Antwort auf die ersten beiden Fragen geben.


\subsection{Konvergenzradius}
Zunächst bemerken wir, dass es genügt, das Konvergenzverhalten einer Potenzreihe der Form $\sum_{k=0}^\infty a_k x^k$, d.h. mit Entwicklungsstelle $x_0 = 0$ zu untersuchen:

Für eine Entwicklungsstelle $x_0 \in \R$ gilt für eine Koeffizientenfolge $(a_n)_{n \in \N}$ offenbar, dass
\begin{align*}
 &\sum_{k=0}^\infty a_k (x-x_0)^k \text{ konvergiert an der Stelle } y \\
 \Leftrightarrow\,
 &\sum_{k=0}^\infty a_k x^k \text{ konvergiert an der Stelle } y-x_0
\end{align*}
Es handelt sich bei dem Konvergenzverhalten von $\sum_{k=0}^\infty a_k (x-x_0)^k$ also nur um eine verschobe Versions des Konvergenverhaltens von $\sum_{k=0}^\infty a_k x^k$.


Um das Verhalten der Potenzreihe $\sum_{k=0}^\infty a_k x^k$ zu untersuchen, wenden wir auf die Reihe das Wurzelkriterium an. Dabei erhalten wir, dass
\[
 \limsup_{n \to \infty} |a_n x^n|^{1/n}
 = |x| \limsup_{n \to \infty} |a_n|^{1/n}.
\]
Ist also $|x| \limsup_{n \to \infty} |a_n|^{1/n} < 1$, so konvergiert die Potenzreihe $\sum_{k=0}^\infty a_k x^k$ absolut, und für $|x| \limsup_{n \to \infty} |a_n|^{1/n} > 1$ divergiert sie. Diese Beobachtung motiviert die folgende Definition:


\begin{defi}
 Der \emph{Konvergenzradius} einer Potenzreihe $\sum_{k=0}^\infty a_k (x-x_0)^k$ ist definiert als
 \[
  \rho \coloneqq \frac{1}{\limsup_{n \to \infty} |a_n|^{1/n}}.
 \]
 Dabei verstehen wir $1/0 = \infty$ und $1/\infty = 1$.
\end{defi}


Die obigen Beobachtungen lassen sich nun wie folgt formulieren:


\begin{prop}
 Es sei $\sum_{k=0}^\infty a_k (x-x_0)^k$ eine Potenzreihe mit Konvergenzradius $\rho$. Dann konvergiert die Potenzreihe für $|x-x_0| < \rho$ absolut und divergiert für $|x-x_0| > \rho$.
\end{prop}
\begin{proof}
 Wir können o.B.d.A.\ davon ausgehen, dass $x_0 = 0$.
 
 Im Falle $\rho = 0$ ist $\limsup_{n \to \infty} |a_n|^{1/n} = \infty$. Für alle $x \in \R$ mit $x \neq 0$ ist dann
 \[
  \limsup_{n \to \infty} |a_n x^n|^{1/n}
  = \limsup_{n \to \infty} |x| |a_n|^{1/n}
  = \infty,
 \]
 und die Reihe $\sum_{k=0}^\infty a_k x^k$ divergiert nach dem Wurzelkriterium.
 
 Im Falle $\rho = \infty$ ist $\limsup_{n \to \infty} |a_n|^{1/n} = 0$. Es ergibt sich dann für alle $x \in \R$, dass
 \[
  \limsup_{n \to \infty} |a_n x^n|^{1/n}
  = |x| \limsup_{n \to \infty} |a_n|^{1/n}
  = 0,
 \]
 so dass die Reihe $\sum_{k=0}^\infty a_k x^k$ nach dem Wurzelkriterium absolut konvergiert.
 
 Ist $\rho \in \R$ so ergibt sich, dass
 \[
  |x| < \rho
  \Leftrightarrow
  \limsup_{n \to \infty} |a_n x^n|^{1/n} < 1,
 \]
 weshalb die Reihe $\sum_{k=0}^\infty a_k x^k$ nach dem Wurzelkriterium für $|x| < \rho$ absolut konvergiert. Analog ergibt sich dass sie für $|x| > \rho$ divergiert.
\end{proof}


\subsection{Beispiele}


Wir betrachten die Potenreihe $\sum_{k=0}^\infty x^k/k!$. Da
\[
 \lim_{n \to \infty} \frac{1}{(n!)^{1/n}}
 = 0
\]
ist der Konvergenzradius der Potenzreihe $\rho = \infty$. Sie konvergiert also für jedes $x \in \R$ absolut. Man bezeichnet diese Potenreihe als die \emph{Exponentialreihe}. Ihr Grenzwert ist die \emph{Exponentialfunktion}.


\begin{defi}
 Die \emph{Exponentialfunktion} $\exp \colon \R \to \R$ ist definiert als
 \[
  \exp(x) \coloneqq \sum_{k=0}^\infty \frac{x^k}{k!} \quad \text{für alle $x \in \R$}.
 \]
\end{defi}


\begin{question}
 Zeigen Sie, dass die Exponentialfunktion die Funktionsgleichung
 \[
  \exp(x+y) = \exp(x) \cdot \exp(y) \quad \text{für alle $x,y \in \R$}
 \]
 erfüllt. (\emph{Hinweis}: Betrachten Sie das Cauchy-Product der Exponentialreihe mit sich selbst, d.h. die Reihe $\sum_{k=0}^\infty a_k$ mit
 \[
  a_k = \sum_{\ell=0}^k \frac{x^\ell}{\ell!} \frac{y^{k-\ell}}{(k-\ell)}!
 \]
 für alle $k \in \N$.)
\end{question}
\begin{solution}
 Es seien $x,y \in \R$ beliebig aber fest. Per Definition der Exponentialfunktion ist
 \[
  \exp(x+y)
  = \sum_{k=0}^\infty \frac{(x+y)^k}{k!}
  = \sum_{k=0}^\infty \frac{1}{k!} \sum_{\ell=0}^k \binom{k}{\ell} x^\ell y^{k-\ell}
  = \sum_{k=0}^\infty \sum_{\ell=0}^k \frac{x^\ell}{\ell!} \frac{y^{k-\ell}}{(k-\ell)!}.
 \]
 Andererseits haben wir
 \[
  \exp(x) \cdot \exp(y)
  = \sum_{n=0}^\infty \frac{x^n}{n!} \cdot \sum_{m=0}^\infty \frac{y^m}{m!}
  = \sum_{n=0}^\infty \sum_{m=0}^\infty \frac{x^n}{n!} \frac{y^m}{m!}.
 \]
 Da die Exponentialreihe absolut konvergiert erhalten wir, dass die Reihe
 \[
  \sum_{(n,m) \in \N \times \N} \frac{x^n}{n!} \frac{y^m}{m!}
 \]
 existiert und erhalten so, dass
 \[
  \sum_{k=0}^\infty \sum_{\ell=0}^k \frac{x^\ell}{\ell!} \frac{y^{k-\ell}}{(k-\ell)!}
  = \sum_{(n,m) \in \N \times \N} \frac{x^n}{n!} \frac{y^m}{m!}
  = \sum_{n=0}^\infty \sum_{m=0}^\infty \frac{x^n}{n!} \frac{y^m}{m!}.
 \]
\end{solution}


\begin{question}
 Bestimmen Sie die Konvergenzradien der Potenzreihen
 \begin{align*}
  \sum_{k=0}^\infty \frac{x^k}{k}
  \quad
  \text{und}
  \quad
  \sum_{k=0}^\infty \frac{(-1)^k}{(2k+1)!} x^{2k+1}.
 \end{align*}
\end{question}
\begin{solution}
 Es ist
 \[
  \limsup_{n \to \infty} \frac{1}{n^{1/n}} = 1,
 \]
 der Konvergenzradius der Potenzreihe $\sum_{k=0}^\infty x^k/k$ ist also $1$.
 
 Für die Reihe $\sum_{k=0}^\infty (-1)^k x^{2k+1}/(2k+1)!$ haben wir
 \[
  \limsup_{n \to \infty} \frac{1}{((2n+1)!)^{1/n}}
  = 0,
 \]
 also hat die Reihe einen Konvergenzradius von $\infty$.
\end{solution}


\begin{question}
Geben Sie eine Potenzreihe $\sum_{k=0}^\infty a_k x^k$ an, so dass
\[
 \sum_{k=0}^\infty a_k x^k = \frac{1}{1-x} \quad \text{für $|x| < 1$}.
\]
\end{question}
\begin{solution}
 Man wähle $a_k = 1$ für alle $k \in \N$, dann ergibt sich die Aussage aus dem Grenzwert geometrischen Reihe.
\end{solution}


\begin{question}
 Geben Sie eine Potenzreihe $\sum_{k=0}^\infty a_k (x-1)^k$ an, so dass
 \[
  \sum_{k=0}^\infty a_k (x-1)^k = \frac{1}{x} \quad \text{für alle $0 < x < 2$}.
 \]
\end{question}
\begin{solution}
 Für alle $|x-1| < 1$, also $0 < x < 2$, haben wir auch $|1-x| < 1$, und somit
 \[
  \frac{1}{x}
  = \frac{1}{1-(1-x)}
  = \sum_{k=0}^\infty (1-x)^k
  = \sum_{k=0}^\infty (-1)^k (x-1)^k.
 \]
\end{solution}















\newpage





\section{Lösungen der Übungen}

\printsolutions





\end{document}
