\documentclass[a4paper,10pt]{article}
%\documentclass[a4paper,10pt]{scrartcl}

\usepackage{../mystyle}

\setromanfont[Mapping=tex-text]{Linux Libertine O}
% \setsansfont[Mapping=tex-text]{DejaVu Sans}
% \setmonofont[Mapping=tex-text]{DejaVu Sans Mono}

\title{Bestimmungen des Grenzwertes}
\author{Jendrik Stelzner}
\date{\today}

\begin{document}
\maketitle

Auf dem achten Übungsblatt wurde sollte gezeigt werden, dass die Folge $(a_n)_{n \in \N}$ mit
\begin{align*}
 a_0 &\coloneqq 0 \\
 a_1 &\coloneqq 1 \\
 a_{n+2} &\coloneqq \frac{a_n + a_{n+1}}{2}
\end{align*}
konvergiert. Wir wollen uns nun damit beschäftigen, den entsprechenden Grenzwert zu ermitteln. Hierfür zeigen wir zunächst noch einmal, dass die Folge $(a_n)$ konvergiert.

\begin{proof}
 Für alle $n \geq 1$ ist
 \[
  |a_{n+1}-a_n|
  = \left| \frac{a_{n-1} + a_n}{2} - a_n \right|
  = \left| \frac{a_{n-1} - a_n}{2} \right|
  = \frac{1}{2} |a_n-a_{n-1}|.
 \]
 Durch wiederholten Anwenden dieser Gleichung ergibt sich mit $|a_1 - a_0| = 1$, dass für alle $n \in \N$
 \[
  |a_{n+1} - a_n| = \frac{1}{2^n}|a_1 - a_0| = \frac{1}{2^n}.
 \]
 Für alle $N' \in \N$ und $m, m' \geq N'$, wobei o.B.d.A. $m \geq m'$, gilt daher
 \begin{align*}
  |a_m - a_{m'}|
  &\leq |a_m - a_{m-1}| + |a_{m-1} - a_{m-2}| + \dotsb + |a_{m'+1} - a_{m'}| \\
  &= \frac{1}{2^{m-1}} + \frac{1}{2^{m-2}} + \dotsb + \frac{1}{2^{m'}}
  = \frac{1}{2^{m'}} \sum_{k=0}^{m-1-m'} \frac{1}{2^k} \\
  &\leq \frac{1}{2^{N'}} \sum_{k=0}^\infty \frac{1}{2^k}
  = \frac{1}{2^{N'}} \cdot 2
  = \frac{1}{2^{N'-1}}.
 \end{align*}
 Für beliebiges aber festes $\varepsilon > 0$ und $N \in \N$ mit $1/(2^{N-1}) < \varepsilon$ gilt daher für alle $m, m' \geq N$, dass $|a_m - a_{m'}| < \varepsilon$. Wegen der Beliebigkeit von $\varepsilon > 0$ zeigt dies, dass $(a_n)$ eine Cauchy-Folge ist. Also ist $(a_n)$ konvergent.
\end{proof}



\end{document}
